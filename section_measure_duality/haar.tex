\subsection{The Space \texorpdfstring{$C_c^{\infty}(G)$}{TEXT} and the Haar Measure}

Let $G$ be a locally profinite group. Then we define $\CG$ to be the space of functions $f:G\to\CC$ which are locally constant and that have compact support. Such a function $f$ can be equivalently characterized as a finite linear combination of characteristic functions of double cosets $KgK$ for some open compact subgroup $K$ of $G$. Hence, the space $\CG$ is a complex vector space and admits two natural actions by $G$ by left and right translation:
$$\lambda_g f:x\longmapsto f(g^{-1}x),\quad\text{and}\quad \rho_g f:x\longmapsto f(xg),$$
for $x,g\in G$ and $f\in\CG$. These actions define smooth representation since characteristic functions of double cosets of $K$ are invariant under translation by $K$. We are now ready to define the notion of a \textit{Haar integral} and \textit{Haar measure}.

\begin{defn}
    A \textit{left Haar integral} on $G$ is a non-zero linear functional 
    $$I:\CG\longrightarrow \CC$$
    such that
    \begin{enumerate}[(1)]
        \item $I(\lambda_g f)=I(f),\ g\in G,\ f\in\CG$, and
        \item $I(f)\geq 0$ for any $f\in\CG$ such that $f\geq 0$.
    \end{enumerate}
    A \textit{right Haar integral} is defined analogously by replacing $\lambda_g$ by $\rho_g$.
\end{defn}

The usefulness of the Haar integral relies on the fact that locally profinite groups possess essentially one unique left Haar integral.

\begin{prop}\label{prop:haar}
    There exists a left Haar integral $I:\CG\to\CC$. Moreover, a linear functional $I':\CG\to\CC$ is a left Haar integral if and only if $I'=cI$ for some constant $c>0$.
\end{prop}
\begin{proof}
    \cite[3.1 Proposition]{BH1}
\end{proof}

Whenever we have a \textit{left Haar integral} $I$, we can define the associated \textit{left Haar measure} as follows. Let $S\subset G$ and let $\Gamma_S$ be its characteristic function. Then $\Gamma_S\in\CG$ if and only if $S$ is open and compact. In that case, we define $$\mu_G(S)=I(\Gamma_S)$$ to be the Haar measure of $S$. We note that $\mu_G(S)>0$ and by left invariance, $\mu_G(gS)=\mu_G(S)$ for any $g\in G$. The relationship is commonly expressed by using the usual integral notation
$$I(f)=\int_G f(g)d\mu_G(g),\quad f\in\CG.$$
This choice of notation is motivated by the fact that since $f$ is locally constant and has constant support, the integral $I(f)$ is effectively a finite weighted sum of $\mu_G(S_i)$ where $S_i$ are open, compact and $f$ is constant on them. 

\begin{example}
    The notion of a left Haar measure is only determined up to a constant. In practice, to uniquely determine the measure, we associate a particular open compact subset with a value. For example if $G=F$ is a local field, one commonly chooses $\mu_F$ so that $\mu_F(R)=1$. Under this choice, we calculate that $\mu_F(\pp^n)=q^{-n}$.
\end{example}

Left Haar measures behave predictably under usual group constructions. For example, if $G_1,G_2$ are profinite groups, then $G=G_1\times G_2$ is also profinite group, and we have an isomorphism 
\begin{align*}
    C_c^{\infty}(G_1)\otimes C_c^{\infty}(G_1)&\longrightarrow \CG\\
    \sum_{i=1}^{r}f_i^1\otimes f_i^2 &\longmapsto \left((g_1,g_2)\mapsto \sum_{i=1}^{r}f_i^1(g_1)f_i^2(g_2)\right).
\end{align*}
If $\mu_i$ is a left Haar measure in $G_i$ for $i=1,2$, then there is a unique left Haar measure $\mu_G$ on $G$ such that 
$$\int_G f_1\otimes f_2(g)d\mu_G(g)=\int_{G_1}f_1(g_1)d\mu_1(g_1)\int_{G_2}f_2(g_2)d\mu_2(g_2),$$
usually dentoted as $\mu_G=\mu_1\otimes\mu_2$.



