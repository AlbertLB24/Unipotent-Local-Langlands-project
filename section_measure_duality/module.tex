\subsection{The Module of a Group}

Of course, the discussion from the previous subsection holds if we replace `right' by `left' throughout. At this point it is therefore natural to ask whether a left Haar integral $I$ on $G$ is also a right Haar integral. This important consideration motivates the following definition.

\begin{defn}
    The group $G$ is \textit{unimodular} if any left Haar integral on $G$ is also a right Haar integral. 
\end{defn}

As a first observation, we note that if the group $G$ is abelian, then $\lambda_g f=\rho_{g^{-1}} f$ and therefore $G$ is unimodular. However, for general groups this is not the case. 

To investigate this, choose some left Haar measure $\mu_G$ on $G$, and consider the functional
\begin{align*}
    I_g:\CG&\longrightarrow\CC\\
    f\longmapsto\int_G& f(xg)d\mu_G(x).
\end{align*}
In other words, if $I$ is the associated left Haar integral of $\mu_G$, then $I_g(f)=I(\rho_g f)$. Since the actions of $G$ on $\CG$ by left and right translation commute, $$I_g(\lambda_h f)=I(\rho_g\lambda_h f)=I(\lambda_h\rho_g f)=I(\rho_g f)=I_g(f)$$
and so $I_g$ is also a left Haar integral. Therefore, there is a unique $\delta_G(g)\in \RR_+^{\times}$ such that $\delta_G(g)I_g(f)=I(f)$ for all $f\in\CG$. In the integral notation, this means that 
$$\delta_G(g)\int_G f(xg)d\mu_G(x)=\int_G f(x)d\mu_G(x)$$
for all $f\in\CG$. Moreover, the map $\delta_G$ also interacts predictably with the left Haar measure. If $S$ is an open compact subset of $G$ and $f=\Gamma_S$ is its characteristic function then one obtains that 
$$\delta_G(g)\mu_G(Sg)=\mu_G(S),$$
which also uniquely identifies $\delta_G(g)$.

\begin{lemma}
    The map $\delta_G:G\to\RR_+^{\times}$ is a homomorphism independent of the choice of left Haar integral $I$ and it is trivial on any open compact subgroup of $G$. In particular, $\delta_G$ is a character of $G$.
\end{lemma}
\begin{proof}
    By above, we have that 
    $$\delta_G(gh)I(\rho_{gh}f)=I(f)=\delta_G(g)I(\rho_g f)=\delta_G(g)\delta_G(h)I(\rho_h\rho_g f)$$
    for any $g,h\in G$ and $f\in\CG$.
    By uniqueness of $\delta_G$ and the fact that $\rho_{gh}=\rho_g\rho_h$, it follows that $\delta_G$ is a homomorphism. The fact that it is independent of the left Haar measure follows immediately from its definition and Prposition \ref{prop:haar}. 
    If $K$ is an open compact subgroup of $G$ and $k\in K$, then by choosing $f=\Gamma_K$ to be the characteristic function of $K$, it follows that $\rho_k f= f$ and therefore $\delta_G(k)=1$.
\end{proof}

The character $\delta_G:G\to\CC$ is denoted as the \textit{module} of $G$, and its importance relies of the following result.

\begin{lemma}
    Let $G$ be a locally profinite group let and $\delta_G:G\to\CC$ be its module. Then $G$ is unimodular if and only if $\delta_G$ is trivial. 
\end{lemma}
\begin{proof}
    Let $I$ be a left Haar integral on $G$. Then $G$ is unimodular if and only if $I$ is a right Haar integral. This is equivalent to $I(f)=I(\rho_g f)=I_g(f)=\delta(g)^{-1}I(f)$ for every $g\in G$. But this is clearly equivalent to $\delta_G$ being trivial. 
\end{proof}

Finally, when the group $G$ is not unimodular, the module $\delta_G$ gives a canonical relationship between left and right Haar integrals.

\begin{lemma}
    Let $I$ be a left Haar integral on $G$ with associated left Haar measure $\mu_G$. If $\delta_G$ is the module of $G$, then the functional
    \begin{align*}
        J:\CG&\longrightarrow\CC\\
        f\longmapsto\int_G& \delta_G(x)^{-1}f(x)d\mu_G(x)
    \end{align*}
    is a right Haar integral for $G$.
\end{lemma}
\begin{proof}
    The functional $J$ can also be expressed as $J(f)=I(\delta_G^{-1}f)$. We note that $\delta_G^{-1}\rho_g(f)=\delta_G(g)\rho_g\delta_G^{-1}f$ as elements of $\CG$ for all $g\in G$ and $f\in\CG$.  Hence, 
    $$J(\rho_g f)=I(\delta_G^{-1}\rho_g f)=\delta_G(g)I(\rho_g\delta_G^{-1}f)=I(\delta^{-1}_G f)=J(f)$$
    for every $g\in G$ and $f\in\CG$, as desired.
\end{proof}