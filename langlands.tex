We are finally ready to formally state the \textit{local Langlands correspondence for $\GL_2$}, the main result of this project. This central theorem gives a bijection between the set of $2$-dimensional semisimple Weil--Deligne representations of $\mathcal{W}_F$, denoted $\mathcal{G}_2(F)$, and the set of irreducible smooth representations of $G=\GL_2(F)$. As we shall see, the bijection uses the language of $L$-functions that we developed in Chapters \ref{sec:func_equation} and \ref{chap:LfuncGL2}. 

More generally, for $n\geq 1$, the local Langlands correspondence for $\GL_n$ relates in a formal way (we will not cover this) the $n$-dimensional semisimple Weil--Deligne representations of $\mathcal{W}_F$ and irreducible smooth representations of $\GL_n(F)$. We remark that, for $n=1$, the correspondence relates characters of $\mathcal{W}_F$ with characters of $F^\times$. As we say in Section \ref{sec:CFT}, this is precisely local class field theory! Therefore, one can view the local Langlands correspondence for $\GL_n$ as a generalization of local class field theory.

To state the correspondence for $\GL_2$, it is convenient to uniformize notation, so we define $\mathcal{A}_2(F)$ to be the equivalence classes of irreducible smooth representations of $\GL_2(F)$. In the following statement, characters of $F^\times$ are viewed as characters of $\mathcal{W}_F$ via the isomorphism $\chi\mapsto\chi\circ a_F$, where $\mathbf{a}_F$ is the Artin map from Theorem \ref{thm:lcf}.


\begin{thm}[Local Langlands Correspondence for $\GL_2$]\label{thm:langcorr}
    Let $\psi\in\hat{F},\psi\neq 1$. There is a unique map 
    $$\bm\pi:\mathcal{G}_2(F)\longrightarrow\mathcal{A}_2(F)$$
    called the \textit{Langlands correspondence}, such that 
    \begin{equation}
        \begin{split}
            L(\chi\bm\pi(\rho),s)&=L(\chi\otimes\rho,s),\\
            \varepsilon(\chi\bm\pi(\rho),s,\psi)&=\varepsilon(\chi\otimes\rho,s,\psi),
        \end{split}
        \label{eqn:langcorr}\tag{\ref{thm:langcorr}.1}
    \end{equation}
    for all $\rho\in\mathcal{G}_2(F)$ and all characters $\chi$ of $F^\times$.

    The map $\bm\pi$ is a bijection, independent of $\psi\in\hat{F}$, and \eqref{eqn:langcorr} holds for any $\psi\in\hat{F}, \psi\neq1$.
\end{thm}

In this report we have only studied and classified the principal series representations of $\GL_2(F)$, so we will not be able to prove Theorem \ref{thm:langcorr} in full generality. In this section, we will show that the Langlands correspondence $\bm\pi$ sets a bijection satisfying \eqref{eqn:langcorr} between the principal series representations and \textit{reducible} $2$-dimensional semisimple Weil--Deligne representations of $\mathcal{W}_F$ . To prove the correspondence in full generality, one needs to study cuspidal representations of $\GL_2(F)$ in depth, and their associated local constants. 

To state the version of the theorem we will prove, it is convenient to introduce some further notation. We partition 
$$\mathcal{G}_2(F)=\mathcal{G}_2^1(F)\cup\mathcal{G}_2^0(F),$$
where $\mathcal{G}_2^1(F)$ is the set of equivalence classes of \textit{reducible} $2$-dimensional semisimple Weil--Deligne representations of $\mathcal{W}_F$ and $\mathcal{G}_2^0(F)$ consists of those classes of Weil--Deligne representations that are \textit{irreducible}. We remark that, necessarily, if $(\rho,V,\nn)\in\mathcal{G}_2^0(F)$, then $\nn=0$. Similarly, we partition
$$\mathcal{A}_2(F)=\mathcal{A}_2^1(F)\cup\mathcal{A}_2^0(F),$$ 
where $\mathcal{A}_2^1$ are the classes of principal series representations and $\mathcal{A}_2^0$ are the classes of cuspidal representations.

\begin{thm}\label{thm:langcorr2}
    There is a unique map
    $$\bm\pi^1:\mathcal{G}_2^1(F)\longrightarrow\mathcal{A}_2^1(F)$$
    such that 
    \begin{equation}
        L(\chi\bm\pi^1(\rho),s)=L(\chi\otimes\rho,s),
        \label{eqn:langcorrLfunc}\tag{\ref{thm:langcorr2}.1}
    \end{equation}
    for all $\rho\in\mathcal{G}_2^1(F)$ and all characters $\chi$ of $F^\times$. Moreover, the map $\bm\pi^1$ is a bijection and it satisfies
    \begin{equation}
        \begin{split}
            \bm\pi^1(\chi\otimes\rho)&=\chi\bm\pi^1(\rho),\\
            \varepsilon(\chi\bm\pi(\rho),s,\psi)&=\varepsilon(\chi\otimes\rho,s,\psi),
        \end{split}
        \label{eqn:langcorreps}\tag{\ref{thm:langcorr2}.2}
    \end{equation}
    for all $\rho\in\mathcal{G}_2^1(F)$, $\psi\in\hat{F}$ and $\chi\in\hat{F^\times}$.
\end{thm}

By uniqueness of the maps, the map $\bm\pi^1$ is simply the restriction of the Langlands correspondence $\bm\pi$ at the subset $\mathcal{G}_2^1(F)\subset\mathcal{G}_2(F)$. 

\begin{rem}
    The statement of Theorem \ref{thm:langcorr} has an important implication. The Langlands correspondence $\bm\pi^1$ restricted to $\mathcal{G}_2^1(F)$ is uniquely determined by the condition \eqref{eqn:langcorrLfunc} on $L$-functions, and the condition \eqref{eqn:langcorreps} on local factors is simply an additional property. This is not the case at all for the Langlands correspondence $\bm\pi$, since $L$-functions associated to cuspidal representations of $\GL_2(F)$ and irreducible Weil--Deligne representations of $\mathcal{W}_F$ provide no information. This idea is formalized with the following result.
    \begin{prop}
        The following holds:
        \begin{enumerate}[(1)]
            \item If $\pi\in\mathcal{A}_2(F)$, then $\pi\in\mathcal{A}_2^0(F)$ if and only if $L(\chi\pi,s)=1$ for all characters $\chi$ of $F^\times$.
            \item If $\rho\in\mathcal{G}_2(F)$, then $\rho\in\mathcal{G}_2^0(F)$ if and only if $L(\chi\otimes\rho,s)=1$ for all characters $\chi$ of $F^\times$.
        \end{enumerate}
    \end{prop}
    \begin{proof}
        We have already proven $(1)$ in Proposition \ref*{prop:twistL} and $(2)$ follows immediately from Definition \ref{defn:LfuncWeil} and Equations \eqref{eqn:Lweilirrep} and \eqref{eqn:Lweilsum}.
    \end{proof}
    Hence any map $\bm\pi:\mathcal{G}_2(F)\longrightarrow\mathcal{A}_2(F)$ satisfying \eqref{eqn:langcorrLfunc} must take $\mathcal{G}_2^i(F)$ to $\mathcal{A}^i_2(F)$ for $i=0,1$. 
\end{rem}

We are now ready to prove the existence of the map $\bm\pi^1$.
\begin{proof}[Proof of Theorem \ref{thm:langcorr2}.]
    Firstly, we note that the map $\bm\pi^{1}$, if it exists, is necessarily unique by the Converse Theorem \ref{thm:converse} as principal series representations are uniquely identified by its $L$-function. We now show it exists. Any $(\rho,V,\nn)\in\mathcal{G}_2^1(F)$ is $2$-dimensional, semisimple and reducible. Hence, $(\rho,V)=(\chi_1,V_1)\oplus(\chi_2,V_2)$ as smooth representations of $\mathcal{W}_F$ (not as Weil--Deligne representations!) for some \textit{unique} characters $\chi_1,\chi_2$ of $F^\times$. Naturally, we form the parabolically induced representation $\pi=\iota_B^G(\chi_1\otimes\chi_2)$, where $\chi_1\otimes\chi_2$ is viewed as a character of $T$. At this stage, we have two cases depending on whether $\pi$ is irreducible. To study them, we first need to prove the following lemma:
    \begin{lemma}
        If $\pi=\iota_B^G(\chi_1\otimes\chi_2)$ is irreducible, then $\nn=0$.
    \end{lemma}
    \begin{proof}
        Suppose that $\nn\neq 0$. Since $\nn$ is nilpotent and $V$ is $2$-dimensional, $\nn^2=0$ and therefore $\dim_\CC\ker\nn=1$ and $\nn(V)=\ker\nn$. Moreover, we have seen in \eqref{eqn:Vnsubspace} that $\ker\nn$ carries a smooth representation $(\ker\nn,\rho_\nn)$ of $\mathcal{W}_F$. Since $\rho$ is semisimple, there is a $\mathcal{W}_F$-invariant subspace $W\leq V$ carrying a smooth representation $(W,\rho_W)$ such that $(V,\rho)=(\ker\nn,\rho_nn)\oplus(W,\rho_W)$. Both $\ker\nn$ and $W$ are $1$-dimensional, and therefore we may assume without loss of generality that $\rho_\nn=\chi_1$ and $\rho_W=\chi_2$. Under these assumptions, we note that for any $w\in W$, $\nn w\in\ker\nn$ and therefore
        $$\rho(x)\nn\rho(x)^{-1}w=\rho(x)\nn\chi^{-1}_2(x)w=\chi_1(x)\chi_2^{-1}(x)\nn w.$$
        By Definition \ref{defn:WeilDeligne} of a Weil--Deligne representation, it follows that $\chi_1(x)\chi_2^{-1}(x)=|x|$. By Theorem \ref{classify} on the classification of principal series representations, this means that $\pi$ is reducible, as desired.
    \end{proof}
    We are now ready to cover both cases.
    \begin{enumerate}[(1)]
        \item If $\pi$ is irreducible, then $\nn=0$ by the previous lemma. In that case, we set 
        $$\bm\pi^{-1}((\rho,V,\nn))=\pi=\iota_B^G(\chi_1\otimes\chi_2).$$
        \item If $\pi$ is reducible, then by Theorem \ref{classify}, we know that $\chi_1\chi_2^{-1}$ is one of the characters $x\mapsto|x|^{\pm1}$. By swapping $\chi_1$ with $\chi_2$ if necessary, we may assume that $\chi_1(x)\chi_2^{-1}(x)=|x|$ and thus there is a unique character $\phi$ of $F^\times$ such that $\chi_1(x)=\phi(x)|x|^{1/2}$ and $\chi_2(x)=\phi(x)|x|^{-1/2}$. If $\nn=0$, then we let 
        $$\bm\pi^1(\rho,V,0)=\phi\circ\det.$$
        If $\nn\neq0$, then the previous lemma shows that $\rho$ acts on $\ker\nn$ by $\chi_1$ and we set 
        $$\bm\pi^1(\rho,V,\nn)=\phi\cdot\St_G,\quad \nn\neq0.$$
    \end{enumerate}
    To finish the proof, we need to show that $\bm\pi^1$ does indeed satisfy the conditions of the theorem. Let $\rho=\chi_1\otimes\chi_2$ be a Weil--Deligne representation as above with nilpotent $\nn$ and let $\chi$ be any character of $F^\times$. Then , we note that $\chi\otimes\rho=(\chi\otimes\chi_1)\oplus(\chi\otimes\chi_2)$ and, in particular, $\pi=\iota_B^G(\chi_1\otimes\chi_2)$ is irreducible if and only if $\chi\pi=\iota_B^G((\chi\otimes\chi_1)\otimes(\chi\otimes\chi_2))$ is irreducible. Hence, we may consider again the same two cases.
    \begin{enumerate}[(1)]
        \item If $\pi$ is irreducible, then $\bm\pi^1(\chi\otimes\rho)=\chi\pi=\chi\bm\pi^1(\rho)$ and
        $$L(\chi\bm\pi^1(\rho),s)=L(\chi\pi,s)=L(\chi\chi_1,s)L(\chi\chi_2,s)=^{(\dagger)}L(\chi\otimes\chi_1,s)L(\chi\otimes\chi_2,s)=L(\chi\otimes\rho,s),$$
        where the representations on the right hand side of $(\dagger)$ are viewed as representations of $\mathcal{W}_F$.
        \item If $\pi$ is reducible and $\nn=0$, then $\bm\pi^1(\chi\otimes\rho)=\chi\phi\circ\det=\chi\bm\pi^1(\rho)$ and
        $$L(\chi\bm\pi^1(\rho),s)=L(\chi\phi\circ\det,s)=L\left(\chi\phi,s+\frac{1}{2}\right)L\left(\chi\phi,s-\frac{1}{2}\right)=L\left(\chi\chi_1,s\right)L\left(\chi\chi_2,s\right)=L(\chi\otimes\rho,s),$$
        where $\chi\phi$ are simultaneously viewed as characters of $F^\times$ and $\mathcal{W}_F$ as usual, and $\chi_1(x)=\phi(x)|x|^{1/2}$ and $\chi_2(x)=\phi(x)|x|^{-1/2}$ as above.
        \item Finally, if $\pi$ is reducible and $\nn\neq0$, then $\bm\pi^1(\chi\otimes\rho)=\chi\phi\cdot\St_G=\chi\bm\pi^1(\rho)$ and
        $$L(\chi\bm\pi^1(\rho),s)=L(\chi\phi\cdot\St_G,s)=L\left(\chi\phi,s+\frac{1}{2}\right)=L\left(\chi\chi_1,s\right)=L(\chi\otimes\rho_\nn,s)=L(\chi\otimes(\rho,V,\nn),s),$$
        where in this last case we are keeping track of the nilpotent $\nn$.
    \end{enumerate}
    To prove that the $\bm\pi^1$ satisfies the property of the local constants, one needs considerable amount of work. Firstly, one needs to use the information from Figure \ref{table:Lfunc} and a better understanding of local constants associated to Weil--Deligne representations. We shall not do so here, so this concludes the proof.
\end{proof}








\newpage