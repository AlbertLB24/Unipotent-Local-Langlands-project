The aim of this first section is to motivate the abstract notions of a \textit{locally profinite group} and of a \textit{smooth representation}, which will be our main object of study during the later subsections. To do so, we briefly recall some basic facts about non-Archimedean fields and we introduce a few important algebraic objects related to them. For the sake of brevity, we will omit proofs, and therefore we assume familiarity with the subject. For the unfamiliar reader, we encourage them to read \textbf{(insert here Gouvea reference, or others)}, where a detailed explanation is provided. 


\subsection{Local Fields and Locally Profinite Groups}
We begin by recalling some basic objects from algebraic number theory. Given a field $F$, a \textbf{discrete valuation} on $F$ is a surjective function $\nu: F\to\ZZ\cup\{\infty\}$ satisfying the three conditions

\begin{enumerate}
    \item $\nu(xy)=\nu(x)+\nu(y)$ for any $x,y\in F$ 
    \item $\nu(x+y)\geq\min\{\nu(x),\nu(y)\}$ for any $x,y\in F$.
    \item $\nu(x)=\infty$ if and only if $x=0$.
\end{enumerate}

Any discrete valuation $\nu$ induces an absolute value given by the formula 
$$|x|=c^{\nu(x)}$$ 
for any $c\in(0,1)$, and therefore it also induces a topology which is independent of the choice of $c$. One easily checks that this absolute value satisfies $|x+y|\leq\max\{|x|,|y|\}$ for any $x,y\in K$. Absolute values with this property are denoted as \textit{non-Archimedean}. 

A field $F$ with an absolute value $|\cdot|$ induced by a discrete valuation $\nu$ is the fraction field of the \textbf{valuation ring}
$$R:=\{x\in F:v(x)\geq 0\}=\{x\in K: |x|\leq1\},$$ 
which contains a unique maximal ideal
$$\pp:=\{x\in F:v(x)> 0\}=\{x\in K: |x|<1\},$$
denoted as the \textbf{valuation ideal} or the \textbf{ring of integers of $F$}. The valuation idea is principal, and it is generated by any $\varpi\in K$ with $\nu(\varpi)=1$. Such an element is called a uniformizer of $F$. Finally, the residue field $\kappa$ of $F$ is the quotient $R/\pp$. This motivates the following important definition.

\begin{defn}
    A field $F$ is a (non-Archimedean) \textit{local field} if it is complete with respect to a topology induced by a discrete valuation and with finite residue field.
\end{defn}

\begin{rem}
    When the residue field is finite, it is conventional to write 
    $$|x|=q^{-\nu(x)},$$ 
    where $q=|\kappa|$. From here onwards, we will follow this convention.
\end{rem}
\begin{rem}
    Local fields are ubiquitous in number theory. They arise as completions of number fields at non-archimedean places, if they have $0$ characteristic, or as completions of finite extensions of $\FF_p(t)$ at non-archimedean places, if the characteristic is positive.
\end{rem}

As discussed above, the valuation ring $R$ of a local field $F$ is a local ring with unique principal ideal $\pp$. The ideals 
$$\pp^n=\{x\in F:\nu(x)\geq n\}=\{x\in F: |x|\leq q^{-n}\}=\varpi^n R,\quad n\in\NN$$
are a complete set of ideals of $R$ and under the topology induced by the discrete valuation, they are also a fundamental system of neighbourhoods of the identity. Moreover, the field $F$ (and therefore also $R$) is totally disconnected, and we also have a topological isomorphism
$$R\longrightarrow\varprojlim_{n\geq 1} R/\pp^n\quad x\mapsto (x\ (\textrm{mod }{\pp^n}))_{n\geq 1}$$
where the maps implicit in the right hand side are the obvious ones.
In particular, since the residue field is finite and $\pp^{n}/\pp^{n+1}\cong\kappa$, all rings $R/\pp^n$ are finite and induced with the discrete topology. 
%Hence, their inverse limit, being a closed subset of the product of compact sets, is a compact set. 
This shows that $R$, and also any $\pp^n$, is a profinite group, and in particular it is compact subring of $F$. We have therefore shown that $F$ has the important property that any open subset of $F$ contains an open compact subgroup (namely $\pp^n$ for a sufficiently large $n$). 

We also remark that $F$ satisfies the rather special property of being the union of its open compact subgroups, even though $F$ itself is not compact. This fact has relevant consequences that will be discuss later.

We are now ready to give the main definition of this section.

\begin{defn}\label{loc_prof_grp}
    A topological group $G$ (which we always assume to be Hausdorff) is a \textit{locally profinite group} if every open neighbourhood of the identity contains a compact open subgroup. 
\end{defn}

In this document we will be interested in studying the representation theory of many important groups and rings related to the local field $F$. The notion of a locally profinite group is an abstract one, but it has the great advantage of accomodating many important groups and rings associated to non-Archimedean local fields and their representation theory.

\begin{examples}

    \begin{enumerate}
        \item In the preceding discussion, we have shown that $F$ is a locally profinite group, where $\pp^n$ for $n\geq1$ is a fundamental system of open compact subgrups.
        \item The multiplicative group $F^{\times}$ is also a locally profinite group, where the congruence unit groups $U_F^n=1+\pp^n$ for $n\geq1$ is a fundamental system of open compact subgroups. We remark that unlike $F$, the group $F^{\times}$ is not the union of its open compact subgroups.
        \item Given $m\geq1$ an integer, the additive group $F^m=F\times\dots\times F$ is also a locally profinite group endowed with the product topology. A fundamental system of open compact subgroups is given by $\pp^{n}\times\dots\times\pp^{n}$ for $n\geq1$. More generally, any product of locally profinite groups is locally profinite.
        \item The matrix ring $M_m(F)$ is also locally profinite since it is isomorphic to $F^{m^2}$ as additive groups. The open compact subgroups $\pp^n M_m(R)$ are a fundamental system of neighbourhood of the identity.
        \item The group $\GL_m(F)$ of invertible matrices is an open subset of $M_m(F)$ since $\det:M_m(F)\rightarrow F$ is continuous and $F^{\times}$ is an open subset of $F$. Furthermore, mutiplication by a matrix $A\in M_m(F)$ and inversion of matrices are continous maps in $M_m(F)$ and therefore $\GL_m(F)$ is also a topological group and the subgroups
        $$K=\GL_m(R),\quad K_n=1+\pp^nM_m(R),\quad n\geq 1,$$
        are compact open, and a fundamental neighbourhood of the identity.
    \end{enumerate}
\end{examples}

We give some further insight into the terminology used. If $G$ is a locally profinite group, any open subgroup $K$ of $G$ is also a locally profinite group under the subspace topology and if $H$ is a closed normal subgroup of $G$, then $G/H$ is also locally profinite under the quotient topology. 

Moreover, it is an easy exercise to prove that a profinite group is a compact locally profinite group. Rather strickingly, using a topological argument one can also show that the converse also holds. That is, if $K$ is a compact locally profinite group, then
$$K\longrightarrow\varprojlim K/N$$
is a topological isomorphism where $N$ ranges over the normal open subgroups, and the implicit maps are the obvious ones. Since $K$ is compact and $N$ is open, $K/N$ must be finite and discrete, showing that $K$ is profinite.

\subsection{Characters of Local Fields}

Now we turn our attention to the representation theory of locally profinite groups. We begin by recalling the notion of a representation of a group.

\begin{defn}
    A representation of a group $G$ over a field $k$ is a pair $(\pi,V)$ where $V$ is a $k$-vector space and $\pi:G\rightarrow\GL_n(V)$ is a group homomorphism. We say that the $\dim V$ is the dimension of the representation.
\end{defn}

Equivalently, a representation of $G$ is a $k$-vector space $V$ equipped with a $k$-linear $G$ action. For $g\in G, v\in V$, we will often write $g\cdot v$ for $\pi(g)v$. Throughout this document, we will only be interested in complex representations, so for now on we will assume that $k=\CC$, unless otherwise stated.

Naturally, we also define the notion of a homomorphism between representations.

\begin{defn}
    A morphism between two representations $(\pi,V)$, $(\sigma,W)$ of a group $G$ is a linear map $\phi:V\rightarrow W$ compatible with the $G$ action. That is, $\phi$ satisfies 
    $$\phi(\pi(g)v)=\sigma(g)\phi(v)\ \text{for all } g\in G,\ v\in V.$$
\end{defn}

In studying the representation theory of a locally profinite group $G$, it turns out that the entire set of representations of $G$ is too big, so we need to restrict our attention to those representations satisfying a certain ``smoothness" condition. To motivate this condition, we will first describe the simplest (yet very relevant!) case: one-dimensional representations of a local field $F$: that is, group homomorphisms $\phi:F\rightarrow\CC^{\times}$. Later in this section we will also study the one-dimensional representations of $F^{\times}$.

As we have discussed in the previous section, locally profinite groups carry a certain topology, so a natural condition to impose is \textbf{continuity} with respect to the usual topologies in $\CC^{\times}$ and $G$. A continuous homomorphism $\psi:G\rightarrow\CC^{\times}$ will be denoted as a \textit{character} of $G$.

Characters of a locally profinite group $G$ are a group under multiplication, denoted by $\hat{G}$. It turns out that for one-dimensional representations, continuity coincides with the smoothness condition we require. 

Firstly, we state this rather surprising result which we will use later. 

\begin{lemma}
    Let $G$ be a locally profinite group and $\psi: G\rightarrow\CC^{\times}$ a homomorphism. Then $\psi$ is continous if and only if $\ker\psi$ is open in $G$. Furthermore, if $G$ is the union of its compact open subgroups, then\footnote{Characters satisfying this property are called \textit{unitary}} $$\psi(G)\subseteq\{z\in\CC^{\times}:|z|=1\}=S^1.$$
\end{lemma}
\begin{proof}
    If $\ker\psi=\psi^{-1}(1)$ is open in $G$, then for any $z\in\Ima\psi$, then $\psi^{-1}(z)=g\ker\psi$ is also open, where $\psi(g)=z$. So in fact, \textbf{for any} $U\subseteq\CC^{\times}$, 
    $$\psi^{-1}(U)=\bigcup_{z\in U\cap\Ima\psi}\psi^{-1}(z),$$
    and so in particular it is continous.
    Conversely, if $\psi$ is continous, then for any open neighbourhood $\mathcal{N}$ of $1$, $\psi^{-1}(\mathcal{N})$ contains an open compact subgroup $K$ of $G$. But $\mathcal{N}$ can be chosen sufficiently small so that it does not contain any non-trivial subgroup of $\CC^{\times}$. Hence, $\psi(K)=1$ so $K\subseteq\ker\psi$, and since $K$ is open, so is $\ker\psi$.
    The last assertion is a direct consequence of the fact that the continuous image of a compact set is compact and $S^1$ is the unique maximal compact subgroup of $\CC^{\times}$.
\end{proof}

By the remark above Definition \ref{loc_prof_grp}, all characters of $F$ are unitary. However, this is not the case for $F^{\times}$. Indeed, the map $x\mapsto|x|$ is a character of $F^{\times}$, yet it is clearly not unitary. 

Before stating the classification theorem for   characters of $F$, we need one last definition. 

\begin{defn}
    Let $\psi$ be a non-trivial character of $F$ (resp. of $F^{\times}$). The \textbf{level} of $\psi$ is defined as be the least $d\geq0$ such that $\pp^d\subseteq\ker\psi$ (resp. $U_F^{d+1}\subseteq\ker\psi$).
\end{defn}

We are now ready to give the classfication theorem for $\hat{F}$.

\begin{thm}[Additive duality]
    Let $\psi\in\hat{F}$ be a non-trivial character with level $d$. 
    \begin{enumerate}
        \item Let $a\in F$. Then the map $a\psi:x\mapsto\psi(ax)$ is a character of $F$ and if $a\neq0$ then $a\psi$ has level $d-\nu_F(a)$.
        \item The map $a\mapsto a\psi$ induces an isomorphism $F\cong\hat{F}$.
    \end{enumerate}
\end{thm}
\begin{proof}
    For $1.$, it is clear that $a\psi$ is a character since if $x\in\pp^{d-\nu_F(a)}$, then $ax\in\pp^d$ so $a\psi(x)=1$ so $\pp^{d-\nu_F(a)}\subseteq\ker(a\psi)$. Furthermore, there is some $y\in\pp^{d-1}$ such that $\psi(y)\neq1$ so $a\psi(a^{-1}y)\neq1$. Since $a^{-1}y\in\pp^{d-1-\nu_F(a)}$, this indeed shows that the level of $a\psi$ is $d-\nu_F(a)$.
    For $2.$ the map $a\mapsto a\psi$ is clearly a homomorphism. To prove injectivity, suppose that $a\neq b$ but $a\psi=b\psi$. Then it follows that $x(a-b)\in\ker\psi$ for all $x\in F$. But since $a-b\neq 0$, then $\ker\psi=F$, a contradiction.

    The proof of surjectivity is more involved. Let $\theta\in\hat{F}$ non-trivial, and let $l$ be the level of $\theta$. The idea of the proof is to construct inductively a sequence some $u_0,u_1,\ldots\in F$ such that $u_n\psi$ agrees with $\theta$ on $\pp^{l-n}$ and $u_{n+1}\equiv u_n\ (\mathrm{mod }\pp)$. More concretely, for any $u\in U_F$, the character $u\varpi^{d-l}\psi$ has level $l$, thus agreeing with $\theta$ at $\pp^l$, and any $u_0=u\varpi^{d-l}$ will do. To construct $u_1$, we note that given $u,u'\in U_F$, then $u\varpi^{d-l}\psi$ and $u\varpi^{d-l}\psi$ agree on $\pp^{l-1}$ if and only if $u\equiv u'\pmod{\pp}$, i.e. when $u'u^{-1}\in U_F^1$. Since $\kappa^{\times}\cong(\pp^{l-1}/\pp^l)^{\times}\cong U_F/U_F^1$ are cyclic of order $q-1$, there are $q-1$ characters of $\pp^{l-1}$ trivial on $\pp^l$ and also $q-1$ characters of $\pp^{l-1}$ of the form $u\varpi^{d-l}\psi$ for $u\in U_F$. Then one of them, say $u_1\varpi^{d-l}\psi|_{\pp^{l-1}}$ equals $\theta|_{\pp^{l-1}}$.

    This same idea can be iterated to find the sequence $\{u_n\}_{n\geq1}$ described above. Since this sequence is Cauchy and $F$ is complete, it converges to some $u\in F$ such that $u\varpi^{d-l}\psi=\theta$.
\end{proof}


\subsection{Smooth representations of locally profinite groups}

Let $G$ be a locally profinite group. A representation of $G$ is a pair $(\pi, V)$ where $V$ is a complex vector space (not necessarily finite-dimensional) and $\pi: G \to \GL(V)$ is a homomorphism.
\begin{defn}
	A representation $V$ of $G$ is smooth if for $v\in V$ there exists a compact-open subgroup $K\subseteq G$ such that $kv = v$ for all $k\in K$ (i.e. $v\in V^K$). We say $V$ is admissible if $V^K$ is finite-dimensional for all compact-open $K$.
\end{defn}

Let us define induced representations:
\begin{defn}\label{induction}
	Let $G$ be a locally profinite group, $H\subseteq G$ a closed subgroup, and $W$ a smooth representation of $H$. We define the induced representation as the space of functions $f: G\to W$ satisfying
	\begin{enumerate}
		\item $f(hg) = h\cdot f(g)$ for all $h\in H, g\in G$
		\item there is a compact open subgroup $K\subseteq G$ (depending on $f$) such that $f(gk) = f(g)$ for all $g\in G, k\in K$.
	\end{enumerate}
	We let $G$ act on this space via $(g\cdot f)(x) = f(xg)$. This is a smooth representation of $G$, denoted by $\Ind_H^G W$.
\end{defn}
This construction satisfies the following important universal property:
\begin{thm}[Frobenius reciprocity]
	Let $V$ be a smooth representation of $G$, and $W$ a smooth representation of $H$. Then there's a natural bijection
	\begin{align*}
		\Hom_G(V, \Ind_H^G W)&\cong \Hom_H(V, W)\\
		\varphi &\mapsto \alpha_W \circ \varphi
	\end{align*}
	where $\alpha_W: \Ind_H^G W \to W$ is the canonical map $\alpha_W(f) = f(1)$.
\end{thm}

There is also a different variant of induction, called compact induction:
\begin{defn}
	Let $G$ be a locally profinite group, $H$ a closed subgroup, and $W$ a smooth representation of $H$. Consider the space of functions $f: G\to W$ that satisfy (1) and (2) in Definition \ref{induction}, and are also compactly supported mod $H$, i.e. the support $\mathrm{supp} f\subseteq H\backslash G$ is compact. The group $G$ also acts on this space by right translations, so we get a representation of $G$, denoted by $c-\Ind_H^G W$.
\end{defn}
This construction is mainly of interest in the case when $H$ is open in $G$. In this case, it satisfies a version of Frobenius reciprocity:
\begin{thm}
	Let $H\subseteq G$ be open, $W$ a smooth representation of $H$ and $V$ a smooth representation of $G$. We have a natural bijection
	\begin{align*}
		\Hom_G(c-\Ind_H^G W, V)&\cong \Hom_H(W, V)\\
		\varphi &\mapsto \varphi \circ \alpha^c_W 
	\end{align*}
	Where $\alpha^c_W: W\to c-\Ind_H^G W$ is the map $w\mapsto f_w$ where $f_w$ is supported in $H$ and satisfies $f_w(h) = hw$.
\end{thm}