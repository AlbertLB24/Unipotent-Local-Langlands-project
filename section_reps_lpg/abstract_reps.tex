\subsection{Abstract Representations of Groups} \label{Abstract_Reps}
Before discussing the representation theory of locally profinite groups, we first review some general results and constructions of representations of arbitrary groups $G$. We begin by recalling the notion of a representation.

\begin{defn}
    A \textit{representation} of a group $G$ over a field $k$ is a pair $(\pi,V)$ where $V$ is a $k$-vector space and $\pi:G\rightarrow\GL(V)$ is a group homomorphism. We say that $\dim V$ is the \textit{dimension} of the representation.
\end{defn}

Equivalently, a representation of $G$ is a $k$-vector space $V$ equipped with a $k$-linear $G$-action. Whenever the representation is clear from the context, we will omit $\pi$ from the notation and write $g\cdot v$ for $\pi(g)v$. 

Throughout this document we will mostly be interested in complex representations, so from now on we will assume that $k=\CC$ unless otherwise stated.

We say that $U\leq V$ is a \textit{$G$-subspace} if $U$ is closed under the $G$-action; i.e., if $g\cdot U\subseteq U$ for every $g\in G$. When this happens, both $U$ and $V/U$ are naturally $G$-representations. We say that that a representation $(\pi,V)$ is \textit{irreducible} (or \textit{simple}) if $V$ has no non-trivial $G$-subspaces. These are the building blocks of more complicated representations, and thus we are often interested in classifying them.


\begin{defn}
    A representation $(\pi,V)$ of a group $G$ is \textit{semisimple} if it is the direct sum of simple subrepresentations. 
\end{defn}

\begin{rem}\label{rem_semisimple}
    If $G$ is a finite group, Maschke's Theorem shows that all finite dimensional complex representations of $G$ are semisimple. As a consequence, one can show that any complex irreducible representation of $G$ is finite dimensional, appearing as a subrepresentation of the regular representation $\CC G$. 
    
    As we shall see later in this chapter, continuous finite-dimensional representations of profinite groups also share these properties. However, it is easy to construct representations of locally profinite groups which are continuous yet not semisimple. For example,
    \begin{align*}
        \phi:\ZZ\longrightarrow \GL_2(\CC)\\
        n\mapsto 
        \begin{pmatrix}
            1 & n\\
            0 & 1\\
        \end{pmatrix}
    \end{align*}
    has a single one-dimensional invariant subspace. One can also construct irreducible representations that are infinite dimensional; we will meet some examples in Section \ref{sec:principal}.
\end{rem}


Naturally, we also define the notion of a morphism of representations.

\begin{defn}
    A morphism between two complex representations $(\pi,V)$, $(\sigma,W)$ of a group $G$ is a linear map $\phi:V\rightarrow W$ compatible with the $G$ action. That is, 
    $$\phi(\pi(g)v)=\sigma(g)\phi(v)\ \text{for all } g\in G,\ v\in V.$$
\end{defn}

This turns the set of complex representations of $G$ into a category, denoted $\Rep(G)$, which is an \textit{abelian category}.

We finish this subsection by introducing important constructions and functors between these categories that allow us to obtain new representations from old ones, which we will use heavily later on.

\begin{defn}
    Given $(\pi,V)\in\mathrm{Rep}_G$, define the dual space $V^*=\Hom(V,\CC)$, and denote by 
    \begin{align*}
        V^*\times V\longrightarrow \CC,\\
        (v^*,v)\longmapsto\langle v^*,v\rangle,
    \end{align*}
    the canonical evaluation homomorphism. Then $V^*$ carries a natural representation of $G$ defined by
    $$\langle\pi^*(g)v^*,v\rangle=\langle v^*,\pi(g^{-1})v\rangle.$$
    This is the \textit{dual representation} of $V$, and the functor 
    \begin{align*}
        (-)^*:\Rep(G)&\longrightarrow\Rep(G)\\
        (\pi,V)&\longrightarrow(\pi^*,V^*)
    \end{align*}
    is an additive and exact contravariant functor.
\end{defn}

One can also consider the composition of this functor with itself to obtain the \textit{double dual} $(\pi^{**},V^{**})$. There is a canonical $G$-homomorphism $\delta:V\rightarrow V^{**}$ such that $$\langle\delta(v),v^*\rangle_{V^*}=\langle v^*,v\rangle_{V}.$$
When $V$ is finite dimensional, $\delta$ is a $G$-isomorphism. For general representations of locally profinite groups, this is not always the case, but under additional assumptions it is possible to give a precise criterion that determines when $\delta$ is bijective (\cite[Corollary 2.8, Proposition 2.9]{BH1}).


\begin{defn}\label{def:absresind}
    Let $H\leq G$ be groups and let $(\pi,V)$ and $(\sigma,W)$ be representations of $G$ and $H$ repectively. The restriction of $\pi$ to $H$ gives a \textit{restriction} functor
    \begin{align*}
        \Res_H^G:\Rep(G)&\longrightarrow\Rep(H)\\
        (\pi,V)&\longmapsto(\pi|_H,V)
    \end{align*}
    On the other hand, given $(\sigma,W)\in\Rep(H)$, one can define the vector space
    $$X=\{f:G\to W:f(hg)=\sigma(h)f(g)\text{ for all }h\in H, g\in G\},$$
    equipped with the $G$-action $\Sigma:G\longrightarrow\Aut_\CC(X)$ defined by right translation:
    $$\Sigma(g)f:x\longmapsto f(xg),\ x,g\in G.$$
    This defines the \textit{induction} functor
    \begin{align*}
        \Ind_H^G:\Rep(H)&\longrightarrow\Rep(G)\\
        (\pi,V)&\longmapsto(\Sigma,X).
    \end{align*}
\end{defn}

As with the dual functor, both the restriction and induction functors are additive and exact, but are now covariant functors. To simplify notation, we will write $\Ind_H^G\sigma$ instead of $\Ind_H^G(\sigma,W)$, which is the usual convention in the literature.

We remark that one can construct the following canonical $H$-homomorphisms 
\begin{align*}
    a_\sigma: \Ind_H^G\sigma&\longrightarrow W\\
    f&\longmapsto f(1)
\end{align*}
and 
\begin{align*}
    a_\sigma^c: W&\longrightarrow \Ind_H^G\sigma,\\
    w&\longmapsto f_w
\end{align*}
where $f_w$ is supported in $H$ and $f_w(h)=\sigma(h)w$ for $h\in H$. The choice of notation will be understood later. These, in turn, induce the maps 
\begin{align*}
    \Psi:\Hom_G(\pi,\Ind_H^G\sigma)&\longrightarrow\Hom_H(\Res_H^G\pi,\sigma),\\
    \phi&\longmapsto a_\sigma\circ\phi,
\end{align*}
and
\begin{align*}
    \Psi^c:\Hom_G(\Ind_H^G\sigma,\pi)&\longrightarrow\Hom_H(\sigma,\Res_H^G\pi),\\
    f&\longmapsto f\circ a_\sigma^c.
\end{align*}
When $G$ is a finite group, we have the following result.
\begin{thm}[Frobenius reciprocity]
    Let $G$ be a finite group. Then the maps $\Psi$ and $\Psi^c$ are bijections that are functorial in both variables $\sigma$ and $\pi$. In categorical terms, we have the adjunctions $$\Ind_H^G\dashv\Res_H^G\dashv\Ind_H^G.$$
\end{thm}

There is an analogue of Frobenius reciprocity for locally profinite groups; see Theorem \ref{thm:frob} and Theorem \ref{thm:frob2}.