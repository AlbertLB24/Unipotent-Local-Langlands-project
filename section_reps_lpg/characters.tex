\subsection{Characters of Local Fields}

Now we turn our attention to the representation theory of locally profinite groups. In studying their representations, it turns out that the entire set is too big, so we need to restrict our attention to those representations satisfying a certain ``smoothness" condition. To motivate this condition, we will first describe the simplest (yet very relevant!) case: one-dimensional representations of a local field $F$: that is, group homomorphisms $\phi:F\rightarrow\CC^{\times}$. Later in this section we will also study the one-dimensional representations of $F^{\times}$.

As we have discussed in the previous section, locally profinite groups carry a certain topology, so a natural condition to impose is \textbf{continuity} with respect to the usual topologies in $\CC^{\times}$ and $G$. A continuous homomorphism $\psi:G\rightarrow\CC^{\times}$ will be denoted as a \textit{character} of $G$.

Characters of a locally profinite group $G$ are a group under multiplication, denoted by $\hat{G}$. It turns out that for one-dimensional representations, continuity coincides with the smoothness condition we require which will be introduced later. 

When $G$ is a finite group with discrete topology, then any one-dimensional representation is a character, and we have the following simple description.

\begin{prop}
    If $G$ is a finite group with the discrete topology, then $\hat{G}\cong G^{ab}$. In particular, if $G$ is abelian then $\hat{G}\cong G$.
\end{prop}
\begin{proof}
    \textbf{Insert reference here}
\end{proof}

For general locally profinite results, we have this rather surprising result. 
\begin{lemma}\label{lem_cont_chars}
    Let $G$ be a locally profinite group and $\psi: G\rightarrow\CC^{\times}$ a homomorphism. Then $\psi$ is continous if and only if $\ker\psi$ is open in $G$. Furthermore, if $G$ is the union of its compact open subgroups, then\footnote{Characters satisfying this property are called \textit{unitary}} $$\psi(G)\subseteq\{z\in\CC^{\times}:|z|=1\}=S^1.$$
\end{lemma}
\begin{proof}
    If $\ker\psi=\psi^{-1}(1)$ is open in $G$, then for any $z\in\Ima\psi$, then $\psi^{-1}(z)=g\ker\psi$ is also open, where $\psi(g)=z$. So in fact, \textbf{for any} $U\subseteq\CC^{\times}$, 
    $$\psi^{-1}(U)=\bigcup_{z\in U\cap\Ima\psi}\psi^{-1}(z),$$
    and so in particular it is continous.
    Conversely, if $\psi$ is continous, then for any open neighbourhood $\mathcal{N}$ of $1$, $\psi^{-1}(\mathcal{N})$ contains an open compact subgroup $K$ of $G$. But $\mathcal{N}$ can be chosen sufficiently small so that it does not contain any non-trivial subgroup of $\CC^{\times}$. Hence, $\psi(K)=1$ so $K\subseteq\ker\psi$, and since $K$ is open, so is $\ker\psi$.
    The last assertion is a direct consequence of the fact that the continuous image of a compact set is compact and $S^1$ is the unique maximal compact subgroup of $\CC^{\times}$.
\end{proof}

Since $F$ is the union of its open compact subgroups, all characters of $F$ are unitary. However, this is not the case for $F^{\times}$. Indeed, the map $x\mapsto|x|$ is a character of $F^{\times}$, yet it is clearly not unitary. 

Before stating the classification theorem for characters of $F$, we need one last definition. 

\begin{defn}
    Let $\psi$ be a non-trivial character of $F$ (resp. of $F^{\times}$). The \textbf{level} of $\psi$ is defined as be the least $d\geq0$ such that $\pp^d\subseteq\ker\psi$ (resp. $U_F^{d+1}\subseteq\ker\psi$).
\end{defn}

\begin{lemma}
    Let $\psi\in\hat{F}$ be a character of level $d$ and let $a\in F$. Then the map $a\psi:x\mapsto\psi(ax)$ is a character of $F$ and if $a\neq0$ then $a\psi$ has level $d-\nu_F(a)$.
\end{lemma}
\begin{proof}
    It is clear that $a\psi$ is a character since if $x\in\pp^{d-\nu_F(a)}$, then $ax\in\pp^d$ so $a\psi(x)=1$ so $\pp^{d-\nu_F(a)}\subseteq\ker(a\psi)$ and the kernel is open. Furthermore, there is some $y\in\pp^{d-1}$ such that $\psi(y)\neq1$ so $a\psi(a^{-1}y)\neq1$. Since $a^{-1}y\in\pp^{d-1-\nu_F(a)}$, this indeed shows that the level of $a\psi$ is $d-\nu_F(a)$. 
\end{proof}

We are now ready to give the classfication theorem for $\hat{F}$.

\begin{thm}[Additive duality]\label{add_dual}
    Let $\psi\in\hat{F}$ be character with level $d$. The map $a\mapsto a\psi$ induces an isomorphism $F\cong\hat{F}$. 
\end{thm}

The proof of surjectivity of the theorem requires an inductive step, which heavily relies on the following results.

\begin{lemma}\label{lem_congruence}
    Let $\psi\in\hat{F}$ be a character of level $d$ and let $u,u'\in U_F$ be two units of $F$. Then $u\psi$ coincides with $u'\psi$ on $\pp^{d-n}$ if and only if $u'u^{-1}\in U_F^{n}$.
\end{lemma}
\begin{proof}
    Let $\alpha=\nu_F(u-u')$. A simple definition chase shows that $u\psi$ and $u'\psi$ agree on $\pp^{d-n}$ if and only if $\pp^{d-n+\alpha}=(u-u')\pp^{d-n}\subseteq\ker\psi$. By definition of level, this is the case if and only if $\alpha\geq n$; that is, if $u\equiv u'\pmod{\pp^n}$ or $u'u^{-1}\in U_F^{n}$.
\end{proof}

\begin{lemma}\label{lem_chars}
    Let $\theta:\pp^{n}\rightarrow\CC^{\times}$ be a character. Then there are exacty $q$ characters $\Theta$ of $\pp^{n-1}$ such that $\Theta|_{\pp^n}=\theta$.
\end{lemma}

\begin{proof}
    Since $\hat\kappa\cong\kappa$, it is enough to construct a bijection between $\mathcal{A}:=\{\Theta\in\widehat{\pp^{n-1}}:\Theta|_{\pp^n}=\theta\}$ and $\hat{\kappa}$. Let $\phi=\theta^{-1}$ and let $\Phi$ be \textbf{any} lift of $\phi$ as a character of $\pp^{n-1}$. Now given $\Theta\in\mathcal{A}$, the character $\Theta\cdot\Phi$ is trivial on $\pp^{n}$ and thus it descends to a map 
    $$\overline{\Theta\cdot\Phi}:\kappa\cong\pp^{n-1}/\pp^n\longrightarrow\CC^{\times}.$$

    To construct an inverse to the map $\Theta\mapsto\overline{\Theta\cdot\Phi}$, choose some $\chi\in\hat\kappa$, view it as a character of $\pp^{n-1}/\pp^{n}$ and consider the map $\tilde\chi:\pp^{n-1}\rightarrow\CC^{\times}$ given by $\tilde{\chi}(u)=\chi(u+\pp^n)$. Then the map $\chi\mapsto\Phi^{-1}\cdot\tilde\chi$ is the required inverse map.
\end{proof}

We are now ready for the proof of Additive duality.

\begin{proof}[Proof of Theorem \ref{add_dual}]
    The map $a\mapsto a\psi$ is clearly a homomorphism. To prove injectivity, suppose that $a\neq b$ but $a\psi=b\psi$. Then it follows that $x(a-b)\in\ker\psi$ for all $x\in F$. But since $a-b\neq 0$, then $\ker\psi=F$, a contradiction.

    Let $\theta\in\hat{F}$ be any non-trivial character (if $\theta$ were trivial, then $0\psi=\theta$), and let $l$ be the level of $\theta$. By replacing $\theta$ with $\varpi^{l-d}\theta$, which has level $d$, we may assume without loss of generality that $\theta$ and $\psi$ have the same level $d$, and therefore they both agree on $\pp^d$. To show there is some $u\in F$ (in fact, $u\in U_F$ necessarily) such that $u\psi=\theta$,   we construct a sequence $\{u_n\}_{n\geq0}$ inductively such that $u_n\psi|_{\pp^{d-n}}=\theta|_{\pp^{d-n}}$ and $u_{n+1}\equiv u_n\pmod{\pp^n}$. Such a sequence is clearly Cauchy, and since $F$ is complete, it converges to some $u\in U_F$ such that $u\equiv u_n\pmod{\pp^n}$ for all $n\geq 1$ and thus $u\psi$ agrees with $\theta$ on $\cup_{n\in\ZZ}\ \pp^n=F$, which concludes the proof.

    Thus, it remains to construct the sequence above. To construct $u_1$ we note that by Lemma \ref{lem_chars}, there are exactly $q-1$ non-trivial characters on $\pp^{d-1}$ that are trivial on $\pp^d$. In addition, by Lemma \ref{lem_congruence}, as $u$ ranges over the cossets of $U_F/U_F^1$, the characters $u\psi|_{\pp^{d-1}}$ are distinct. Since $|U_F/U_F^1|=|\kappa^{\times}|=q-1$, there is some $u_1\in U_F$ such that $u_1\psi$ agrees with $\theta$ on $\pp^{d-1}$. 
    
    Assuming now we have constructed $u_1,\ldots,u_n$ in $U_F$ with the desired conditions, we note that by Lemma \ref*{lem_chars}, there are exactly $q$ characters of $\pp^{d-n-1}$ that coincide with $\theta|_{\pp^{d-n}}$ when they are restricted. Again by Lemma \ref*{lem_congruence}, as $\alpha$ ranges over the cossets of $U_F^n/U_F^{n+1}$ the characters $\alpha u_n\psi$ are distinct on $\pp^{d-n-1}$ but they all coincide on $\pp^{d-n}$. Since $|U_F^n/U_F^{n+1}|=|\kappa|=q$, there is some $\alpha_n$ such that $\alpha_n u_n\psi$ coincides with $\theta$ on $\pp^{d-n-1}$. Since $\alpha_n\in U_F^n$, $\alpha_n u_n\equiv u_n\pmod{\pp^n}$. Hence $u_{n+1}:=\alpha_n u_n$ has the requried properties.
\end{proof}