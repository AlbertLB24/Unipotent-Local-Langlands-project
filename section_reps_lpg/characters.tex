\subsection{Characters of Local Fields}\label{sec:chars}

Now we turn our attention to the representation theory of a locally profinite group $G$. The category $\Rep(G)$ of abstract representations introduced in the previous section only takes into consideration the group structure of $G$, but it completely ignores its topology. In the previous section, we have seen that locally profinite groups have a rather special topology and, as it will become apparent as we develop the theory, this topology is a crucial piece of information associated to the group. Consequently, instead of working with $\Rep(G)$, we will work with a full subcategory of representations that satisfy an additional smoothness condition, which will be denoted as $\Smo(G)$ and its elements called \textit{smooth representations}. This condition, as the name suggests, requires the representation to be continuous with respect to the topology on $G$ and the complex topology of $\Aut_\CC(V)$. In sections \ref{sec:chars} and \ref{sec:smoothrep}, we will see that these two notions coincide when the representation is finite-dimensional, but not in general. To motivate this condition, we will first describe the simplest case: one-dimensional representations of a local field $F$: that is, group homomorphisms $\phi:F\rightarrow\CC^{\times}$. Later in this section we will also study the one-dimensional representations of $F^{\times}$. We now follow closely the development from \cite[\S 2]{BH1}, while stressing the motivation for the terms introduced through examples.

\begin{defn}
    A \textit{character} of a locally profinite group $G$ is a continuous homomorphism $\psi:G\rightarrow\CC^{\times}$.
\end{defn}

Characters of a locally profinite group $G$ form an abelian group $\hat{G}$ under multiplication, denoted as the \textit{dual group} of $G$. 

\begin{example}
    Let $G$ be a finite group with the discrete topology. Then any one-dimensional representation is a character, and we have the simple description $\hat{G}\cong G^{ab}$. In particular, if $G$ is abelian then $\hat{G}\cong G$.
\end{example}

For general locally profinite results, we have this rather surprising result
\begin{lemma}\label{lem_cont_chars}
    Let $G$ be a locally profinite group and $\psi: G\rightarrow\CC^{\times}$ a homomorphism. Then $\psi$ is continuous if and only if $\ker\psi$ is open in $G$. Furthermore, if $G$ is the union of its compact open subgroups, then $$\psi(G)\subseteq\{z\in\CC^{\times}:|z|=1\}=S^1.$$
\end{lemma}
\begin{rem}
    Characters of locally profinite groups that have image in $S^1$ are called \textit{unitary}.
\end{rem}
\begin{proof}
    If $\ker\psi=\psi^{-1}(1)$ is open in $G$, then for any $z\in\Ima\psi$, the preimage $\psi^{-1}(z)=g\ker\psi$ is also open, for any $g \in G$ satisfying $\psi(g)=z$. Then for any $U\subseteq\CC^{\times}$, 
    $$\psi^{-1}(U)=\bigcup_{z\in U\cap\Ima\psi}\psi^{-1}(z),$$
    so that $\psi$ is continuous.
    Conversely, if $\psi$ is continuous, then for any open neighbourhood $\mathcal{N}$ of $1$, $\psi^{-1}(\mathcal{N})$ contains an open compact subgroup $K$ of $G$. But $\mathcal{N}$ can be chosen sufficiently small so that it does not contain any non-trivial subgroup of $\CC^{\times}$. Hence, $\psi(K)=1$, so $K\subseteq\ker\psi$, and since $K$ is open, so is $\ker\psi$.
    The last assertion is a direct consequence of the fact that the continuous image of a compact set is compact, and $S^1$ is the unique maximal compact subgroup of $\CC^{\times}$.
\end{proof}

\begin{example}
    The local field $F$ is the union of its open compact subgroups, so all characters of $F$ are unitary. This can also be checked directly as follows. Let $\psi:F\to\CC^\times$ be a character. By Lemma \ref{lem_cont_chars}, $\ker\psi$ is open in $F$ and therefore it contains $\pp^N$ for some $N$ large enough. Assume, for example, that $\psi$ is trivial on $R=\pp^0$. We will describe such characters inductively for each $\pp^n, n<0$. Fix some $n<0$ and assume that $\psi(\pp^n)\subset S^1$. Then $\psi(\varpi^{n-1})^q=\psi(q\varpi^{n-1})\in S^1$ since $q\varpi^{n-1}\in\pp^n$ and therefore $\psi(\varpi^{n-1})\in S^1$. Since any $x\in\pp^{n-1}$ can be expressed uniquely as $x=a\varpi^{n-1}+y$ for $a\in\{0,1,\ldots,q-1\}$ and $y\in \pp^n$, it follows that $\psi(x)=\psi(\varpi^{n-1})^a\psi(y)\in S^{1}$, so $\psi(\pp^{n-1})\subset S^1$.
    
    
    We remark that, for each $n<0$, there are exactly $q$ choices for $\psi(\varpi^n)$, since it is a $q$th root of $\psi(q\varpi^{n})$ and $q\varpi^{n}\in\pp^{n+1}$. Once this choice is made $\psi$ is completely determined on $\pp^n$. We have shown that all characters of $F$ trivial on $R$ are constructed this way.

    It is also worth mentioning that if the uniformizer is chosen appropiately, the above construction can be made explicit. For example, if $F=\QQ_p$ and $\varpi=p$, then $\psi(p^{n-1})^p=\psi(p^{n})$ for any character $\psi$ of $\QQ_p$ and $n\in\ZZ$. This means that if $\psi$ is trivial on $R$ (in particular, $\psi(1)=1$), then $\psi$ is determined by a sequence $(\zeta_1,\zeta_2,\zeta_3,\ldots)$ where $\zeta_n$ is a $p^n$th root of unity and $\zeta_n^p=\zeta_{n-1}$. For example,
    \begin{align*}
        \psi:\QQ_p&\longrightarrow \CC^\times\\
        \sum_{k\geq m} a_kp^k&\longmapsto
        \begin{cases}
            1 \text{ if } m\geq 0,\\
            \prod_{k=m}^{-1}e^{2\pi i a_kp^k} \text{ if } m < 0.\\
        \end{cases}
    \end{align*}
    is a non-trivial character of $\QQ_p$.
    
    
    From this perspective, it is clear that
    $$\{\psi\in\widehat{\QQ_p}\text{ trivial on } p^n\ZZ_p \text{ for some }n\in\ZZ \}\cong\{\psi\in\widehat{\QQ_p}\text{ trivial on } \ZZ_p \}\cong \varprojlim_{n\geq 0} \ZZ/p^n\ZZ\cong \ZZ_p,$$
    where the first isomorphism follows by replacing some character $\psi$ trivial on $p^n\ZZ_p$ for the character $x\mapsto\psi(p^n x)$, trivial on $\ZZ_p$. The more general statement 
    $$\{\psi\in\hat{F}\text{ trivial on } \pp^n \text{ for some }n\in\ZZ \}\cong R$$
    also holds for any local field $F$, but this takes some more work. We prove this fact in Theorem \ref{thm:duality} (Additive Duality), together with the important isomorphism $\hat{F}\cong F$.

\end{example}

\begin{example}
    In contrast, the multiplicative group $F^\times$ is not the union of its open compact subgroups. For instance, no open compact subgroup contains $2$ assuming that $\mathrm{char}\kappa\geq 3$. Moreover, it is not the case that all characters of $F^\times$ are unitary. Indeed, the map $\chi:x\mapsto|x|$ is a character of $F^{\times}$ since $\ker\chi=R^\times$ is an open subgroup of $F^\times$, and it is not unitary.

    This example hints at the fact that the group structure of $\hat{F^\times}$ is quite subtle and we will not cover its description here. The interested reader can find a partial description in \cite[\S 1.8]{BH1}.
\end{example}


Before stating Additive Duality, the main result of this section, we need one last definition.

\begin{defn}\label{def:addlevel}
    Let $\psi$ be a non-trivial character of $F$. The \textit{level} of $\psi$ is the least integer $d$ such that $\pp^d\subseteq\ker\psi$.
\end{defn}

The following is a simple property of the level of a character.

\begin{lemma}
    Let $\psi\in\hat{F}$ be a character of level $d$ and let $a\in F$. Then the map $a\psi:x\mapsto\psi(ax)$ is a character of $F$, and if $a\neq0$ then $a\psi$ has level $d-\nu(a)$.
\end{lemma}
\begin{proof}
    The map $a\psi$ is clearly a homomorphism. It is also a character since if $x\in\pp^{d-\nu(a)}$, then $ax\in\pp^d$, so $a\psi(x)=1$, and therefore $\pp^{d-\nu(a)}\subseteq\ker(a\psi)$ and the kernel of $a\psi$ is open. Furthermore, there is some $y\in\pp^{d-1}$ such that $\psi(y)\neq1$, and so $a\psi(a^{-1}y)\neq1$. Since $a^{-1}y\in\pp^{d-1-\nu(a)}$, this indeed shows that the level of $a\psi$ is $d-\nu(a)$. 
\end{proof}

We are now ready to give the classfication theorem for $\hat{F}$.

\begin{thm}[Additive Duality]\label{add_dual}
    Let $\psi\in\hat{F}$ be a character of level $d$. The map $a\mapsto a\psi$ induces the isomorphisms 
    $$F\cong\hat{F} \quad\text{ and }\quad R\cong\{\psi\in\hat{F}:\pp^d\subseteq\ker\psi\}.$$  
\end{thm}

The proof of surjectivity of the theorem requires an inductive step, which relies on the following results.

\begin{lemma}\label{lem_congruence}
    Let $\psi\in\hat{F}$ be a character of level $d$ and let $u,u'\in U_F$ be two units of $F$. Then $u\psi$ coincides with $u'\psi$ on $\pp^{d-n}$ if and only if $u'u^{-1}\in U_F^{n}$.
\end{lemma}
\begin{proof}
    Let $\alpha=\nu(u-u')$. A simple definition chase shows that $u\psi$ and $u'\psi$ agree on $\pp^{d-n}$ if and only if $\pp^{d-n+\alpha}=(u-u')\pp^{d-n}\subseteq\ker\psi$. By definition of level, this is the case if and only if $\alpha\geq n$; that is, if $u\equiv u'\pmod{\pp^n}$ or equivalently $u'u^{-1}\in U_F^{n}$.
\end{proof}

\begin{lemma}\label{lem_chars}
    Let $\theta:\pp^{n}\rightarrow\CC^{\times}$ be a character. Then there are exacty $q$ characters $\Theta$ of $\pp^{n-1}$ such that $\Theta|_{\pp^n}=\theta$.
\end{lemma}

\begin{proof}
    Since $\hat\kappa\cong\kappa$, where $\kappa$ is the residue field of $F$, it is enough to construct a bijection between $\mathcal{A}:=\{\Theta\in\widehat{\pp^{n-1}}:\Theta|_{\pp^n}=\theta\}$ and $\hat{\kappa}$. Let $\phi=\theta^{-1}$ and let $\Phi$ be \textit{any} lift of $\phi$ as a character of $\pp^{n-1}$. Now given $\Theta\in\mathcal{A}$, the character $\Theta\cdot\Phi$ is trivial on $\pp^{n}$ and thus it descends to a map 
    $$\overline{\Theta\cdot\Phi}:\kappa\cong\pp^{n-1}/\pp^n\longrightarrow\CC^{\times}.$$

    To construct an inverse to the map $\Theta\mapsto\overline{\Theta\cdot\Phi}$, choose some $\chi\in\hat\kappa$, view it as a character of $\pp^{n-1}/\pp^{n}$ and consider the map $\tilde\chi:\pp^{n-1}\rightarrow\CC^{\times}$ given by $\tilde{\chi}(u)=\chi(u+\pp^n)$. Then the map $\chi\mapsto\Phi^{-1}\cdot\tilde\chi$ is the required inverse map.
\end{proof}

We are now ready for the proof of Additive Duality.

\begin{proof}[Proof of Theorem \ref{add_dual}]
    The map $a\mapsto a\psi$ is clearly a homomorphism. To prove injectivity, suppose that $a\neq b$ but $a\psi=b\psi$. It follows that $x(a-b)\in\ker\psi$ for all $x\in F$. But since $a-b\neq 0$, we have $\ker\psi=F$, contradicting our assumption that $\psi$ is non-trivial.

    Let $\theta\in\hat{F}$ be any non-trivial character (if $\theta$ were trivial, then $0\psi=\theta$), and let $l$ be the level of $\theta$. By replacing $\theta$ with $\varpi^{l-d}\theta$, which has level $d$, we may assume without loss of generality that $\theta$ and $\psi$ have the same level $d$, and therefore they both agree on $\pp^d$. To show there is some $u\in F$ (in fact, $u\in U_F$ necessarily) such that $u\psi=\theta$,   we construct a sequence $\{u_n\}_{n\geq0}$ inductively such that $u_n\psi|_{\pp^{d-n}}=\theta|_{\pp^{d-n}}$ and $u_{n+1}\equiv u_n\pmod{\pp^n}$. Such a sequence is clearly Cauchy, and since $F$ is complete, it converges to some $u\in U_F$ such that $u\equiv u_n\pmod{\pp^n}$ for all $n\geq 1$ and thus $u\psi$ agrees with $\theta$ on $\cup_{n\in\ZZ}\ \pp^n=F$, which concludes the proof.

    Thus, it remains to construct the sequence above. To construct $u_1$ we note that by Lemma \ref{lem_chars}, there are exactly $q-1$ non-trivial characters on $\pp^{d-1}$ that are trivial on $\pp^d$. In addition, by Lemma \ref{lem_congruence}, as $u$ ranges over the cossets of $U_F/U_F^1$, the characters $u\psi|_{\pp^{d-1}}$ are distinct. Since $|U_F/U_F^1|=|\kappa^{\times}|=q-1$, there is some $u_1\in U_F$ such that $u_1\psi$ agrees with $\theta$ on $\pp^{d-1}$. 
    
    Assuming now we have constructed $u_1,\ldots,u_n$ in $U_F$ with the desired conditions, we note that by Lemma \ref*{lem_chars}, there are exactly $q$ characters of $\pp^{d-n-1}$ that coincide with $\theta|_{\pp^{d-n}}$ when they are restricted. Again by Lemma \ref*{lem_congruence}, as $\alpha$ ranges over the cosets of $U_F^n/U_F^{n+1}$ the characters $\alpha u_n\psi$ are distinct on $\pp^{d-n-1}$ but they all coincide on $\pp^{d-n}$. Since $|U_F^n/U_F^{n+1}|=|\kappa|=q$, there is some $\alpha_n$ such that $\alpha_n u_n\psi$ coincides with $\theta$ on $\pp^{d-n-1}$. Since $\alpha_n\in U_F^n$, $\alpha_n u_n\equiv u_n\pmod{\pp^n}$. Hence $u_{n+1}:=\alpha_n u_n$ has the required properties.

    Finally, it follows immediately from the definition of level that. under the above isomorphism, the elements $a\in R$ correspond to the characters $\psi\in\hat{F}$ that are trivial on $\pp^d$. This concludes the proof.
\end{proof}