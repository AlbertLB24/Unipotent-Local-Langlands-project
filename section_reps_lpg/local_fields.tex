

\subsection{Local Fields and Locally Profinite Groups}
We begin by recalling some basic objects from algebraic number theory. Given a field $F$, a \textit{discrete valuation} on $F$ is a surjective function $\nu: F\to\ZZ\cup\{\infty\}$ satisfying the conditions

\begin{enumerate}
    \item $\nu(xy)=\nu(x)+\nu(y)$ for any $x,y\in F$ 
    \item $\nu(x+y)\geq\min\{\nu(x),\nu(y)\}$ for any $x,y\in F$.
    \item $\nu(x)=\infty$ if and only if $x=0$.
\end{enumerate}

Any discrete valuation $\nu$ induces an absolute value on $F$ given by the formula 
$$|x|=c^{\nu(x)}$$ 
for any $c\in(0,1)$, and therefore it also induces a topology on $F$. This topology is independent of the choice of $c$. One easily checks that this absolute value satisfies $|x+y|\leq\max\{|x|,|y|\}$ for any $x,y\in F$. Absolute values with this property are called \textit{non-archimedean}. 

A field $F$ with an absolute value $|\cdot|$ induced by a discrete valuation $\nu$ is the fraction field of the \textit{valuation ring}
$$R:=\{x\in F:v(x)\geq 0\}=\{x\in F: |x|\leq1\},$$ 
which contains a unique maximal ideal
$$\pp:=\{x\in F:v(x)> 0\}=\{x\in F: |x|<1\},$$
the \textit{valuation ideal} or the \textit{ring of integers of $F$}. The valuation ideal is principal, and it is generated by any $\varpi\in F$ with $\nu(\varpi)=1$. Such an element is called a \textit{uniformiser} of $F$. Finally, the \textit{residue field} $\kappa$ of $F$ is the quotient $R/\pp$. This motivates the following important definition.

\begin{defn}
    A field $F$ is a \textit{non-archimedean local field} if it is complete with respect to a topology induced by a discrete valuation and the residue field is finite.
\end{defn}

\begin{rem}
    When the residue field is finite, it is conventional to define the absolute value on $F$ by 
    $|x|=q^{-\nu(x)},$
    where $q=|\kappa|$. From here onwards, we will follow this convention.
\end{rem}
\begin{rem}
    Local fields are ubiquitous in number theory. They arise as completions of number fields at non-archimedean places in characteristic 0, or as completions of finite extensions of $\FF_p(t)$ at non-archimedean places in positive characteristic.
\end{rem}

Let us now discuss important aspects of the topology on $F$ and $R$ induced by the discrete valuation $\nu$. We have already seen that $R$ is a local ring with maximal ideal $\pp$ and therefore $U_F:=R\setminus\pp$ is the set of units of $R$. The ideals 
$$\pp^n=\{x\in F:\nu(x)\geq n\}=\{x\in F: |x|\leq q^{-n}\}=\varpi^n R,\quad n\in\ZZ$$
are a complete set of fractional ideals of $R$ and, since the valuation is assumed to be discrete, they are also open subsets of $F$.
Therefore, they are a fundamental system of neighbourhoods of the identity. A direct consequence of this fact implies that $F$ (and therefore $R$) are totally disconnected topological rings.

Furthermore, the ring $R$ is a closed subring of $F$, which is assumed to be complete. Hence, $R$ is also complete, and a stardard topological argument shows that $R$ is in fact compact. This proves that $R$ (and therefore any $\pp^n$) is in fact a profinite group, and we have a topological isomorphism 
$$R\longrightarrow\varprojlim_{n\geq 1} R/\pp^n\quad x\mapsto (x\ (\textrm{mod }{\pp^n}))_{n\geq 1}$$
where the maps implicit in the right hand side are the obvious ones.

However, $F$ itself is clearly not compact, and therefore it is not profinite. Nevertheless, $F$ has the important property that any open neighbourhood of the identity contains an open compact (and therefore profinite) subgroup - some $\pp^n$ for a sufficiently large $n$.

We are now ready to give the main definition of this section, which encapsulates this last property in greater generality.

\begin{defn}\label{loc_prof_grp}
    A topological group $G$ (which we always assume to be Hausdorff) is a \textit{locally profinite group} if every open neighbourhood of the identity contains a compact open subgroup. 
\end{defn}

In this report we will be interested in studying the representation theory of many important groups and rings related to the local field $F$. The notion of a locally profinite group is an abstract one, but it has the great advantage of accomodating many important groups and rings associated to non-Archimedean local fields and their representation theory.

\begin{examples} \label{example_prof_groups}

    \begin{enumerate}[(1)]
        \item Trivially, any group equipped with the discrete topology is profinite, where $\{e\}$ is the fundamental neighbourhood.
        \item In the preceding discussion, we have shown that the local field $F$ is a locally profinite group, where $\pp^n$ for $n\geq1$ is a fundamental system of open compact subgrups. We remark that $F$ satisfies the rather special property of being the union of its open compact subgroups. %This fact has relevant consequences that will be discuss later.
        \item The multiplicative group $F^{\times}$ is also a locally profinite group, where the congruence unit groups $U_F^n=1+\pp^{n+1}$ for $n\geq1$ is a fundamental system of open compact subgroups. Unlike $F$, the group $F^{\times}$ is not the union of its open compact subgroups.
        \item Given $m\geq1$ an integer, the additive group $F^m=F\times\dots\times F$ is also a locally profinite group endowed with the product topology. A fundamental system of open compact subgroups is given by $\pp^{n}\times\dots\times\pp^{n}$ for $n\geq0$. More generally, any product of locally profinite groups is locally profinite.
        \item The matrix ring $M_m(F)$ is also locally profinite since it is isomorphic to $F^{m^2}$ as additive groups. The open compact subgroups $\pp^n M_m(R)$ are a fundamental system of neighbourhood of the identity.
        \item The group $\GL_m(F)$ of invertible matrices is an open subset of $M_m(F)$ since $\det:M_m(F)\rightarrow F$ is continuous and $F^{\times}$ is an open subset of $F$. Furthermore, mutiplication by a matrix $A\in M_m(F)$ and inversion of matrices are continuous maps in $M_m(F)$, and therefore $\GL_m(F)$ is also a topological group. The subgroups
        $$K=\GL_m(R),\quad K_n=1+\pp^{n+1}M_m(R),\quad n\geq 0,$$
        are compact open, and a fundamental neighbourhood of the identity.
        \item Let $G$ be a locally profinite group and $H\leq G$ be a closed subgroup. Then $H$ is also a locally profinite group. If in addition $H$ is assumed to be normal in $G$, then $G/H$ is locally profinite. 
        
        %If $U\subseteq H$ is a neighbourhood of the identity on $H$, then there is some $V$ open in $G$ such that $U=H\cap V$. Let $K\subseteq V$ be some open compact subgroup of $G$. Then $K\cap H$ is an open subgroup of $H$ and a closed subgroup of $K$. But since $K$ is compact and Hausdorff, $K\cap H$ is also compact. This shows that $H$ is also a locally profinite subgroup.
    \end{enumerate}
\end{examples}

We give some further insight into the terminology used. It is an easy exercise to prove that a profinite group is compact and locally profinite. Rather strikingly, the converse also holds. That is, if $K$ is a compact locally profinite group, then
$$K\longrightarrow\varprojlim_N K/N$$
is a topological isomorphism, where $N$ ranges over the normal open subgroups. Since $K$ is compact and $N$ is open, $K/N$ must be finite and discrete, showing that $K$ is profinite.

