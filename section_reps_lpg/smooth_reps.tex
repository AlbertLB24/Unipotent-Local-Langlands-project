\subsection{Smooth Representations of Locally Profinite Groups} \label{sec:smoothrep}

We now turn our attention to representations of arbitrary dimension of locally profinite groups and we introduce the notion of \textit{smooth representations}, which form a full subcategory of $\Rep(G)$. For one-dimensional representations, we imposed a natural continuity condition, and Lemma \ref{lem_cont_chars} showed that characters have open kernel. This is a remarkable result, since this means that the homomorphism is continuous with respect to \textbf{any} topology on $\CC^{\times}$, not just the usual one. 

If $V$ is a finite-dimensional representation of a locally profinite group $G$, the group $\GL_\CC(V)$ has a natural topology as an open subspace of $M_n(\CC)\cong\CC^{n^2}$. Again, it is a natural requirement that finite dimensional representations should be continous with respect to these topologies. It is a fact, analogous to $\CC^{\times}$, that small neighbourhoods of the identity of $\GL_\CC(V)$ do not contain any non-trivial subgroups. Therefore, the same reasoning as in Lemma \ref{lem_cont_chars} shows that continuous finite-dimensional representations of $G$ have open kernel too. That is, the homomorphism is continuous with respect to any topology on $\GL_\CC(V)$.

However, for infinite-dimensional representations $V$, equipping $\GL_\CC(V)$ with a topology is not as straightforward, and the requirement of having an open kernel is too restrictive. Here is where the notion of smooth representation becomes relevant, for which we must first introduce the module of invariants and coinvariants.

\begin{defn}
    Let $H\leq G$ be groups and $(\pi,V)$ a representation of $G$. We define the $H$-invariants of $V$ to be 
    $$V^{H}:=\{v\in V:\pi(h)v=v\text{ for all }h\in H\},$$
    and the $H$-coinvariants to be 
    $$V_H:=V/V(H)\text{ where } V(H)=\textrm{Span}_\CC\{v-\pi(h)v:v\in V,h\in H\}.$$
    That is, $V^H$ (resp. $V_H$) is the largest subspace (resp. quotient) on which $H$ acts trivially.
\end{defn}


\begin{defn}
	A representation $V$ of $G$ is \textit{smooth} if for all $v\in V$ there exists a compact open subgroup $K\leq G$ such that $v\in V^K$. In other words,
    $$V=\bigcup_K V^K$$ as we range over all compact open subgroups $K$ of $G$. We say that $V$ is \textit{admissible} if $V^K$ is finite dimensional for all compact open $K$.
\end{defn}

Smooth representations of $G$ are a full abelian subcategory of $\Rep(G)$ denoted by $\Smo(G)$. 


\begin{rem}\label{rem:contsmooth}
    If $(\pi,V)$ is a finite-dimensional smooth representation and $\{v_1,\ldots,v_n\}$ is a $\CC$-basis such that $v_i\in V^{K_i}$ for some open compact subgroups $K_i$, then 
    $$K:=\bigcap_{i=1}^n K_i\subseteq\ker\pi$$
    is open and compact too, so the kernel is open. 
    Conversely, if $\ker\pi$ is open, then there is some open compact subgroup $K$ fixing all of $V$, so in this case smooth and continuous coincide. 
\end{rem}


As we hinted in Remark \ref{rem_semisimple}, smooth representations of locally profinite groups have remarkable algebraic structures, and they share many properties with representations of finite groups, particularly if the group is compact (and thus profinite). A direct application of Zorn's Lemma provides the following useful criterion to determine whether a representation is semisimple. 

\begin{prop}\label{prop_semisimple}
    Let $(\pi,V)$ be a smooth representation of a locally profinite group $G$. The following are equivalent:
    \begin{enumerate}
        \item $V$ is the sum of its irreducible $G$-subspaces.
        \item $V$ is the direct sum of a family of irreducible $G$-subspaces (i.e. $V$ is semisimple)
        \item any $G$-subspace of $V$ has a $G$-complement in $V$.
    \end{enumerate}
\end{prop}

\begin{proof}
    \cite[Lemma 2.2]{BH1}
\end{proof}

Using this proposition, we can now prove that smooth representations of profinite groups behave in a similar way to those of finite groups. We note that any open compact subgroup $K$ of a locally profinite group $G$ is profinite, and that any smooth $G$-representation is natually a smooth $K$-representation by restriction. Therefore, the following results apply for any open compact subgroup of $G$.

\begin{prop}\label{lem_profinite_smooth}
    Let $(\pi,V)$ be a representation of a profinite group $K$. If $V$ is irreducible then it is finite dimensional. If $V$ is finite dimensional, then it is semisimple.
\end{prop}

\begin{proof}
    The first statement is a matter of following the definitons. Fix any non-zero $v\in V$, and suppose $v\in V^{K_0}$ for some open compact $K_0 \leq K$. Then the subspace 
    $$U=\Span\{\pi(k)v:k\in K\}=\Span\{\pi(k)v:k\in K/K_0\}$$
    is clearly a $K$-subspace and it is also finite dimensional since $K_0$ is open and $K$ is compact, so $[K:K_0]$ is finite.   

    To prove the second statement, let $v$ and $K_0$ be as above. By replacing $K_0$ by $\cap_{g\in K/K_0}gK_0g^{-1}$ if needed, we may assume that $K_0$ is normal in $K$. As above, the subspace 
    $$W=\Span\{\pi(k)v:k\in K\}$$
    is finite dimensional and $K_0$ acts trivially on it.
    Thus $W$ factors through a finite dimensional representation of the finite group $K/K_0$. By Maschke's Theorem, $W$ is then the sum of its irreducible $K$ subspaces. Since $v$ was arbitrary this shows that condition $1.$ of Proposition \ref{prop_semisimple} is satisfied, so $V$ is semisimple.

\end{proof}

This proposition has important structural results. Let $\hat{K}$ denote the set of equivalence classes of irreducible smooth representations of $K$. As we shall see, this notation is consistent with $\hat{F}$ since all irreducible smooth representations of $F$ are one-dimensional.

Let $(\pi,V)$ be a smooth representation of a locally profinite group $G$ and let $K$ be an open compact subgroup. For each $\rho\in\hat{K}$, let $V^\rho$ be the sum of all irreducible $K$-subspaces of $V$ isomorphic to $\rho$, the \textit{$\rho$-isotypic component} of $V$. In particular, $V^{1_K}=V^K$.

\begin{prop}
    Let $G$ be a locally profinite group and $K$ a compact open subgroup of $G$. Let $(\tau,U),(\pi,V)$, $(\sigma,W)\in\Smo(G)$ and $a:U\rightarrow V$ and $b:V\rightarrow W$ be $G$-homomorphisms. 
    \begin{enumerate}
        \item The space $V$ is the sum of the $K$-isotypic components:
        $$V=\bigoplus_{\rho\in\hat{K}}V^\rho.$$
        \item The following holds:
        $$W^\rho\cap b(V)=b(V^\rho).$$
        \item The sequence
        $$U\xlongrightarrow{a} V\xlongrightarrow{b} W$$
        is exact if and only if 
        $$U^K\xlongrightarrow{a} V^K \xlongrightarrow{b} W^K$$
        is exact for every compact open subgroup $K$ of $G$.
        \item Denoting by $V(K)$ the span of the elements $v-\pi(k)v$ for $v\in V, k\in K$,
        $$V(K)=\bigoplus_{\substack{\rho\in\hat{K}\\\rho\neq 1}}V^\rho \text{ and } V=V^K\oplus V(K)$$
        and $V(K)$ is the unique $K$-complement of $V^K$ in $V$. 
    \end{enumerate}
\end{prop}

\begin{proof}
    \cite[Proposition 2.3 and Corollary 1.2]{BH1}
\end{proof}

As promised in \S\ref{Abstract_Reps}, we now discuss the dual, restriction and induction functors in the context of smooth representations of locally profinite groups. From our previous discussion, two major problems arise in this context. Firstly, given a locally profinite group $G$ and a subgroup $H$, there is no guarantee that $H$ is locally profinite, and thus $\Smo(H)$ may not be well-defined. Secondly, when we perform some construction on a smooth representation (e.g., constructing its dual, inducing to a bigger group) there is no gurarantee that the resulting representation is smooth. Thankfully, both of these problems can be resolved in a straightforward way.

To ensure that $H$ is locally profinite, we must add a condition on the topology of $H$. Based on Example \ref{example_prof_groups}(7), we just need to assume that $H$ is a closed subgroup of $G$. In some cases, we will need to assume that $H$ is also open, which is a more restrictive condition. To resolve the second problem, we construct a functor that associates, to each abstract representation, a smooth representation in a canonical way.

\begin{defn}
    Let $G$ be a locally profinite group. Define the \textit{smoothness functor}
    \begin{align*}
        (-)^\infty:\Rep(G)&\longrightarrow\Smo(G),\\
        (\pi,V)&\longmapsto(\pi^\infty,V^\infty)
    \end{align*}
    by defining 
    $$V^\infty:=\bigcup_K V^K \text{  and  } \pi^\infty(g):=\pi(g)|_{V^\infty} \text{  for each  } g\in G,$$ where $K$ ranges over the compact open subgroups of $G$.
\end{defn}

\begin{rem}
    One should check that the smoothness functor is well-defined. In other words, we should check that the space $V^\infty$ is preserved under the $G$-action, making it a $G$-representation. Let $v\in V^\infty$ and choose some open compact subgroup $K$ such that $v\in V^K$. For any $g\in G$, we have that $\pi^\infty(g)v=\pi(g)v\in V^{gKg^{-1}}\subseteq V^\infty$ since $\pi(gkg^{-1})\pi(g)v=\pi(gk)v=\pi(g)v$ for any $k\in K$.
\end{rem}

Furthermore, the functor $(-)^\infty$ is left-exact and it satisfies that
$$\Hom_G(V,W)=\Hom_G(V,W^\infty) \text{ for all } V\in\Smo(G), W\in\Rep(G).$$

Using these constructions, we can define the smooth dual, restriction and induction functors. If $H\leq G$ is a closed subgroup, the restriction functor $\Res_H^G:\Rep(G)\to\Rep(H)$ (Definition \ref{def:absresind}) sends smooth representations of $G$ to smooth representations of $H$. This is because the intersection of an open compact subgroup of $G$ with $H$ is still open compact in the subspace topology of $H$. The analogous statement does not hold for the dual and induction functors, so we must compose with the smoothness functor. 

\begin{defn}
    If $G$ is a locally profinite group, define the \textit{smooth dual functor} 
    \begin{align*}
        \check{(-)}:\Smo(G)&\longrightarrow\Smo(G),\\
        (\pi,V)&\longmapsto(\check{\pi},\check{V})
    \end{align*}
    by $(\check{\pi},\check{V})=(\pi^*,V^*)^\infty$.
\end{defn}

The smooth dual satisfies an important property: if $V$ is a smooth representation of $G$ and $v\in V, v\neq 0$, then there is some $\check{v}\in\check{V}$ such that $\langle\check{v},v\rangle\neq 0$. Consequently, the map $\delta:V\rightarrow\check{\check{V}}$ is injective, and the following proposition gives a criterion for surjectivity.

\begin{prop}
    If $G$ is a locally profinite group, and $V$ is a smooth representation of $G$, the canonical map $\delta:V\longrightarrow\check{\check{V}}$ is an isomorphism if and only if $(\pi,V)$ is admissible.
\end{prop}
\begin{proof}
    \cite[Proposition 2.9]{BH1}
\end{proof}

We also define the smooth induction functor as the composition of the induction and smoothness functor.

\begin{defn}\label{induction}
    Let $G$ be a locally profinite group and $H\leq G$ a closed subgroup. Define the \textit{smooth induction functor}
    \begin{align*}
        (\Ind_H^G(-))^\infty:\Smo(H)&\longrightarrow\Smo(G),\\
        (\sigma,W)&\longmapsto(\Sigma,X)^\infty
    \end{align*}
    where we recall that $X$ is the space of functions $f: G\to W$ satisfying $f(hg) = \sigma(h)f(g)$ for all $h\in H, g\in G$ and the action of $\Sigma$ on $X$ is given by right translation $\Sigma(g)f:x\mapsto f(xg)$.
\end{defn}

\begin{rem}
    Throghout this document, we will only be interested in studying the smooth induction of smooth representations. The idea is that smooth induction is the `right' construction in this setting, which coincides with the abstract induction from Definition \ref{def:absresind} when the group is finite with the discrete topology. Therefore, as it is common in the literature, we will use a slight abuse of notation and denote the smooth induction functor as $\Ind_H^G$. We will write 
    \begin{align*}
        \Ind_H^G:\Smo(H)&\longrightarrow\Smo(G),\\
        (\sigma,W)&\longmapsto(\Sigma,X)
    \end{align*}
    where $\Sigma$ is now the space of functions $f:G\to W$ satisfying:
    \begin{enumerate}
        \item For all $h\in H, g\in G$, we have $f(hg) = \sigma(h)f(g)$.
        \item There is some open compact subgroup $K$ of $G$ such that $f(xg)=f(x)$ for all $x\in G$ and $g\in K$,
    \end{enumerate}
    and $\Sigma$ is the action on $X$ by right translation.

    The second condition is precisely the smoothness condition that appears after composing the abstract induction functor with the smoothness functor.
    
\end{rem}

Since the action $\Sigma$ on $X$ is given by $\Sigma(g)f:x\mapsto f(xg)$, condition $2.$ is precisely the smoothness condition that $f\in X^K$ for some open compact subgroup $K$. As above, we will denote this representation of $G$ by $\Ind_H^G\sigma$. Under these conditions, the first half of Frobenius reciprocity holds:

\begin{thm}[Frobenius reciprocity]\label{thm:frob}
	Let $(\pi,V)$ be a smooth representation of $G$, and $(\sigma,W)$ a smooth representation of a closed subgroup $H$. Then the map
	\begin{align*}
		\Psi:\Hom_G(\pi, \Ind_H^G\sigma)&\longrightarrow \Hom_H(\Res_H^G \pi, \sigma),\\
		\varphi &\longmapsto a_\sigma \circ \varphi,
	\end{align*}
    is a bijection that is functorial in both variables $\pi,\sigma$. Here $a_\sigma:\Ind_H^G\sigma \to W$ is the canonical map $a_\sigma(f) = f(1)$. In categorical terms,
    $$\Res_H^G\dashv\Ind_H^G.$$
\end{thm}
\begin{proof}
    \cite[2.4 Frobenius Reciprocity]{BH1}
\end{proof}

However, in this context, it is not the case that $\Ind_H^G$ is left adjoint to $\Res_H^G$. With a small modification we can recover left exactness. Firstly, we note that to ensure that $a_\sigma^c$ (to be defined shortly) is a $H$-homomorphism, we need the stronger assumption that $H$ is open in $G$. Secondly, we observe that given representations $(\pi,V)$ and $(\sigma,W)$, of $G$ and $H$ respectively, $a_\sigma^c(w)$ is supported only in $H$ for any $w\in W$. Hence, one should not consider the entire representation $\Ind_H^G\sigma$, but rather a subrepresentation of it. Here is the precise construction.

\begin{defn}
	Let $G$ be a locally profinite group, $H$ a closed subgroup, and $(\sigma,W)$ a smooth representation of $H$. Define the \textit{compact induction functor} 
    \begin{align*}
        \cInd_H^G:\Smo(H)&\longrightarrow\Smo(G),\\
        (\sigma,W)&\longmapsto(\Sigma_c,X_c)
    \end{align*}
    where, if $\mathrm{Ind}_H^G(\sigma,W) = (\Sigma,X)$, then
    $$X_c:=\{f\in X: \supp f \text{ in } H\backslash G \text{ is compact}\},$$
    and $\Sigma_c$ acts on $X_c$ by right translation.
    We say that functions satisfying the later condition are \textit{compactly supported modulo $H$}, and this condition is equivalent to $\supp f\subseteq HC$ for some compact subset $C$ of $G$.
    The space $X_c$ is closed under the action by $\Sigma$, so the functor is well-defined.
\end{defn}

This construction is mainly of interest in the case when $H$ is open in $G$, in which case $a_\sigma^c$ is a $H$-homomorphism. This construction satisfies the second half of Frobenius reciprocity.

\begin{thm}\label{thm:frob2}
	Let $(\pi,V)$ be a smooth representation of $G$, and $(\sigma,W)$ a smooth representation of an open subgroup $H$. Then the map 
	\begin{align*}
		\Psi^c:\Hom_G(\cInd_H^G \sigma, \pi)&\longrightarrow \Hom_H(\sigma, \Res_H^G\pi)\\
		\varphi &\longmapsto \varphi \circ a^c_\sigma 
	\end{align*}
    is a bijection that is functorial in both variables $\pi,\sigma$. Here $a^c_\sigma: W\to c-\Ind_H^G \sigma$ is the map $w\mapsto f_w$, where $f_w$ is supported in $H$ and defined by $f_w(h) = hw$.
\end{thm}
\begin{proof}
    \cite[2.5 Theorem]{BH1}
\end{proof}
In categorical terms, under the assumptions of this theorem we have
$$\cInd_H^G\dashv\Res_H^G\dashv\Ind_H^G.$$