In this section, we turn to the other side of the Langlands correspondence, the Galois side. We shall be interested in certain finite-dimensional smooth representations of the Weil group $\mathcal{W}_F$. We will relate the appropriate representations of $\mathcal{W}_F$ with representations of $G$ in terms of the $L$-functions and local constants of the two types of representation. Hence our main task in this chapter is to define the appropriate class of representations, and to present the theory of $L$-functions and local constants attached to these.

First we define the Weil group $\mathcal{W}_F$, for which we will need to give a review of the Galois theory of the local field $F$ and the structure of its separable extensions. We also review the general theory of the smooth representations of $\mathcal{W}_F$. We then define the $L$-function and local constant for \emph{characters} of $\mathcal{W}_F$, which will be a consequence of local class field theory and the results in Section \ref{sec:func_equation}. Then we extend the definition to all finite-dimensional semisimple representations.

One final subtlety arises: in order to get an exact correspondence between smooth representations of $G = \GL_2(F)$ and 2-dimensional representations of $\mathcal{W}_F$, one has to expand the framework by introducing \emph{Weil-Deligne representations}, which are finite-dimensional representations with some additional structure. We finish this chapter by defining these representations, along with their $L$-functions and local constants.

\subsection{Definition of the Weil Group}
We begin by defining the Weil group $\mathcal{W}_F$, and related objects in the Galois theory of $F$. The proofs of most statements in this subsection are rather standard, so we have omitted them.

Fix a separable closure $\overline{F}$ of $F$. Then the absolute Galois group of $F$ is
\[\Omega_F = \varprojlim\Gal(E/F),\]
where the limit is taken over all finite Galois extensions $E/F$ contained in $\overline{F}$. This is naturally a topological group with its profinite topology. If $K/F$ is a finite separable extension, then $\overline{F}$ is also a separable closure of $K$, and $\Omega_K = \Gal(\overline{F}/K)$ is an open subgroup of $\Omega_F$.

For any positive integer $m$, there is a unique unramified extension $F_m/F$ of degree $m$ contained in $\overline{F}$. It is Galois, and the natural restriction map
\[\Gal(F_m/F)\to \Gal(k_{F_m}/k_F)\]
is an isomorphism. The extension of residue fields is Galois with cyclic Galois group, generated by the Frobenius automorphism $x\mapsto x^q$, where $q = |k|$. Hence there is a unique element $\phi_m\in \Gal(F_m/F)$ restricting to the Frobenius on the residue fields. We set $\Phi_m = \phi_m^{-1}$. Hence there is a canonical isomorphism
\begin{align*}
	\Gal(F_m/F)&\to \ZZ/m\ZZ\\
	\Phi_m&\mapsto 1
\end{align*}

The field $F_\infty = \bigcup_{m\ge1} F_m$ is the largest unramified extension of $F$. Its Galois group is obtained as a limit
\[\Gal(F_\infty/F) = \varprojlim_{m\ge 1}\Gal(F_m/F).\]
Since the isomorphisms $\Gal(F_m/F)\cong \ZZ/m\ZZ$, $\Phi_m\mapsto 1$ are compatible with the restriction maps $\Gal(F_m/F)\to\Gal(F_l/F)$ for $l\mid m$, we can write this as
\[\Gal(F_\infty/F)\cong \varprojlim_{m\ge 1} \ZZ/m\ZZ\]
The group on the right is the group $\widehat{\ZZ}$ of profinite integers. Under this isomorphism, $1\in \widehat{\ZZ}$ corresponds to a unique element $\Phi_F\in \Gal(F_\infty/F)$ called the \emph{geometric Frobenius substitution}. We will call an element of $\Omega_F$ a \emph{Frobenius element} if its restriction to $\Gal(F_\infty/F)$ is $\Phi_F$. 

We set $\mathcal{I}_F = \Gal(\overline{F}/F_\infty)$, called the \emph{inertia group} of $F$. Then we have a short exact sequence of topological groups
\[0 \to \mathcal{I}_F\to \Omega_F \xrightarrow{\text{res}} \widehat{\ZZ}\to 0\]
where the map $\Omega_F\to \widehat{\ZZ}$ comes from the restriction to $F_\infty$ as above.

\begin{defn}
	The \emph{Weil group} of $F$ is the topological group whose underlying abstract group is 
	\[\mathcal{W}_F = \text{res}^{-1}(\ZZ)\subseteq \Omega_F,\]
	and which is topologized as follows: a basis of open sets is given by the collection $\{\sigma U\}$, where $\sigma \in \mathcal{W}_F$ and $U\subseteq \mathcal{I}_F$ is an open set in the topology of $\mathcal{I}_F$ as a subspace of $\Omega_F$.
\end{defn}
One easily checks that the given collection is indeed a basis for a topology on $\mathcal{W}_F$, that $\mathcal{W}_F$ becomes a topological group, $\mathcal{I}_F\subseteq \mathcal{W}_F$ is open, and the subspace topology on $\mathcal{I}_F$ from $\mathcal{W}_F$ is the same as its natural topology as a subspace of $\Omega_F$. It also follows easily that $\mathcal{W}_F$ is locally profinite, as it is covered by translates of the profinite group $\mathcal{I}_F$.

In particular, the topology on $\mathcal{W}_F$ is \emph{not} the subspace topology inherited from $\Omega_F$; it is a finer topology. Hence the canonical injection
\[\iota_F: \mathcal{W}_F\to \Omega_F\]
is continuous. One also shows that the image of $\iota_F$ is dense in $\Omega_F$.
\begin{rem}
	It is natural to wonder why we have to consider $\mathcal{W}_F$ rather than all of $\Omega_F$, and why we have decided to put this particular topology on it. One reason is that the quotient $\widehat{\ZZ}$ is rather unwieldy, so we choose to focus on the set of elements of $\Omega_F$ mapping to the dense infinite cyclic subgroup $\ZZ\subseteq \widehat{\ZZ}$, which is exactly $\mathcal{W}_F$. This gives rise to the short exact sequence of abstract groups
	\[0\to \mathcal{I}_F\to \mathcal{W}_F\to \ZZ\to 0\]
	as above. However, if we equip $\mathcal{I}_F$ and $\mathcal{W}_F$ with their natural topologies as subspaces of $\Omega_F$, then the quotient topology on $\ZZ = \mathcal{W}_F/\mathcal{I}_F$ is \emph{not} the discrete topology; rather it's the subspace topology coming from the inclusion $\ZZ\hookrightarrow \widehat{\ZZ}$. If we insists that $\ZZ$ should have the discrete topology, while keeping the usual topology of $\mathcal{I}_F$, we need to make $\mathcal{I}_F\subseteq \mathcal{W}_F$ open. Hence the topology we have defined is the coarsest one that makes $\mathcal{W}_F$ into a topological group and the quotient $\ZZ = \mathcal{W}_F/\mathcal{I}_F$ discrete.
\end{rem}

 We therefore have a short exact sequence
\[0 \to \mathcal{I}_F \to \mathcal{W}_F \xrightarrow{v_F} \ZZ\to 0\]
where $v_F$ takes a geometric Frobenius element to 1, and we have the norm
\[|x| = q^{-v_F(x)}\]
for $x\in \mathcal{W}_F$.

If $E/F$ is a finite separable extension, we have the inclusion $\Omega_E\subseteq \Omega_F$. This interacts well with restriction to Weil groups:
\begin{prop}
	Let $E/F$ be a finite separable extension. Then
	\begin{enumerate}
		\item We have that
		\[\mathcal{W}_E = \Omega_E\cap \mathcal{W}_F\]
		as abstract subgroups of $\Omega_F$.
		\item We have $[E:K] = [\Omega_F:\Omega_E] = [\mathcal{W}_F:\mathcal{W}_E]$.
		\item If $E/F$ is Galois, then the restriction map
		\[\mathcal{W}_F \hookrightarrow \Omega_F \to \Gal(E/F)\]
		is surjective with kernel $\mathcal{W}_E$, and in particular $\mathcal{W}_F/\mathcal{W}_E\cong \Gal(E/F)$.
		\item We have the Galois correspondence
		\begin{align*}
			\{\text{finite separable extensions }E/F\} &\leftrightarrow \{\text{finite index open subgroups of }\mathcal{W}_F\}\\
			 E &\mapsto \mathcal{W}_E
		\end{align*}
	\end{enumerate}
\end{prop}
In particular, $(3)$ above shows that keeping track of $\mathcal{W}_E$ for each finite separable $E/F$, we can recover each finite Galois group $\Gal(E/F)$, and hence the full group $\Omega_F$. Hence passing to Weil groups loses no information.

\subsection{Representations of the Weil Group}
Here we collect the basic facts relating to the representation theory of the locally profinite group $\mathcal{W}_F$. We care about semisimple smooth finite-dimensional representations. We first investigate semisimplicity.

Of course $\Omega_F$ is profinite, hence every smooth representation of it is semisimple (Proposition \ref{lem_profinite_smooth}). However $\mathcal{W}_F$ has the discrete group $\ZZ$ as a quotient, and therefore has finite-dimensional representations which are indecomposable but not irreducible, such as the representation $\mathcal{W}_F\to \GL_2(\CC)$ given by

\[x\mapsto \begin{pmatrix}
		1&v_F(x)\\
		0&1
	\end{pmatrix}.\]
(cf. Remark \ref{rem_semisimple}). Here one can see that any geometric Frobenius element is sent to the non-semisimple matrix $\left(\begin{smallmatrix}
	1&1\\0&1
\end{smallmatrix}\right)$.  This happens in general: for any smooth representation $\rho$ of $\mathcal{W}_F$, its restriction to $\mathcal{I}_F$ is semisimple, and $\rho(\mathcal{I}_F)$ is finite. So all complications arise from the Frobenius elements of $\mathcal{W}_F$. Indeed, we have the following criterion for recognizing semisimple representations in terms of Frobenius elements.
\begin{prop}
	Let $(\rho, V)$ be a smooth representation of $\mathcal{W}_F$ of finite dimension, and let $\Phi\in \mathcal{W}_F$ be a Frobenius element. Then the following are equivalent:
	\begin{enumerate}
		\item The representation $\rho$ is semisimple;
		\item $\rho(\Phi)\in \Aut_{\CC}(V)$ is semisimple;
		\item $\rho(\Psi)\in \Aut_{\CC}(V)$ is semisimple for all $\Psi\in \mathcal{W}_F$.
	\end{enumerate}
\end{prop}
\begin{proof}
	\cite[Proposition 28.7]{BH1}
\end{proof}


Now consider a finite separable extension $E/F$. We have two ways of relating the smooth representations of $F$ with those of $E$:
\begin{itemize}
	\item Given a smooth representation $\rho$ of $\mathcal{W}_F$, we can use the inclusion $\mathcal{W}_E\subseteq \mathcal{W}_F$ to restrict $\rho$ to a representation of $\mathcal{W}_E$. We denote this representation by 
	\[\rho\mid_{\mathcal{W}_E} = \Res_{E/F} \;\rho = \rho_E\]
	\item Given instead a smooth representation $\tau$ of $\mathcal{W}_E$, smooth induction gives a representation of $\mathcal{W}_F$:
	\[\Ind_{\mathcal{W}_E}^{\mathcal{W}_F}\tau = \Ind_{E/F}\, \tau\]
\end{itemize}
We investigate semisimplicity of representations of $\mathcal{W}_E$ and $\mathcal{W}_F$ with respect to these constructions.

\begin{lemma}
	Let $E/F$ be a finite separable extension. Then the following hold:
	\begin{enumerate}
		\item Let $\rho$ be a smooth representation of $\mathcal{W}_F$. Then $\rho$ is semisimple if and only if $\rho_E$ is semisimple.
		\item Let $\tau$ be a smooth representation of $\mathcal{W}_E$. Then $\tau$ is semisimple if and only if $\Ind_{E/F}\, \tau$ is semisimple.
	\end{enumerate}
\end{lemma}
\begin{proof}
	In fact, this lemma holds with $\mathcal{W}_F, \mathcal{W}_E$ replaced by an arbitrary locally profinite group $G$ and a finite index open subgroup $H$. See \cite[Lemma 2.7]{BH1}.
\end{proof}
\begin{notn}
	We denote by $\mathcal{G}^{ss}_n(F)$ the set of isomorphism classes of $n$-dimensional semisimple smooth representations of $\mathcal{W}_F$, and by $\mathcal{G}^0_n(F)$ the set of isomorphism classes of $n$-dimensional irreducible smooth representations of $\mathcal{W}_F$.
\end{notn}
In this notation, for a finite extension $E/F$ of degree $d$, we have restriction and induction maps
\begin{align*}
	\Ind_{E/F}: \mathcal{G}^{ss}_n(E) &\to \mathcal{G}^{ss}_{nd}(F)\\
	\Res_{E/F}: \mathcal{G}^{ss}_n(F) &\to \mathcal{G}^{ss}_n(E)
\end{align*}
\begin{notn}
	We let $1_E$ be the trivial character of $\mathcal{W}_E$. If $E/F$ is a finite separable extension of degree $d$, then we form the \emph{regular representation}
	\[R_{E/F} = \Ind_{E/F}\, 1_E\in \mathcal{G}_{d}^{ss}(F).\]
\end{notn}
\subsection{Local Class Field Theory}\label{sec:CFT}
In this subsection we summarize local class field theory, giving an axiomatic account. We then use it to relate characters of the Weil group to multiplicative characters of $F$.

\begin{thm}[Local class field theory]
	There is a unique continuous group homomorphism
	\[\mathbf{a}_F: \mathcal{W}_F\to F^\times\]
	with the following properties:
	\begin{enumerate}
		\item $\mathbf{a}_F$ induces a topological isomorphism $\mathcal{W}_F^{ab}\cong F^\times$, where $\mathcal{W}_F^{ab} = \mathcal{W}_F/\overline{[\mathcal{W}_F, \mathcal{W}_F]}$ is the quotient by the closure of the commutator subgroup.
		\item An element $x\in \mathcal{W}_F$ is a geometric Frobenius if and only if $\mathbf{a}_F(x)$ is a uniformizer in $F$;
		\item We have $\mathbf{a}_F(\mathcal{I}_F) = \mathcal{O}_F^\times$;
		\item For any finite separable extension $E/F$, the diagram 
		$$\xymatrix{
			\mathcal W_E \ar[r]^{\mathbf a_E} \ar[d]_{\Res_{E/F}} & E^\times \ar[d]^{N_{E/F}} \\
			\mathcal W_F \ar[r]^{\mathbf a_F} & F^\times
		}$$
		
		commutes.
	\end{enumerate}
	The map $\mathbf{a}_F$ is called the \emph{Artin reciprocity map}, or just the \emph{Artin map}.
\end{thm}
Note that any character $\chi: \mathcal{W}_F\to \CC^\times$ of $\mathcal{W}_F$ must factor through the abelianization $\mathcal{W}_F^{ab}$, (because $\chi$ is continuous, so its kernel is closed), which is isomorphic via the Artin map to $F^\times$. So we have an induced isomorphism of character groups
\begin{align*}
	\{\text{smooth characters of }F^\times\} &\cong \{\text{smooth characters of }\mathcal{W}_F\}\\
	\chi &\mapsto \chi\circ\mathbf{a}_F
\end{align*}
\subsection{\texorpdfstring{$L$}{TEXT}-function and Local Constant}
In this subsection, we define the quantities $L(\sigma, s)$ (the $L$-function) and $\varepsilon(\sigma, s, \psi)$ (the local constant) for smooth semisimple finite-dimensional representations $\sigma$ of $\mathcal{W}_F$. The corresponding quantities for characters of $F^\times$ were defined in Section \ref{sec:func_equation}. We transfer these to characters of $\mathcal{W}_F$ via the Artin map.
\begin{defn}\label{defn:LfuncWeil}
	If $\chi: F^\times \to \CC^\times$ is a character and $\psi: F\to \CC^\times$ is an additive character, we define
	\begin{align*}
		L(\chi\circ\mathbf{a}_F, s) &= L(\chi, s)\\
		\varepsilon(\chi\circ\mathbf{a}_F, s, \psi) &= \varepsilon(\chi, s, \psi).
	\end{align*}
	This defines the $L$-function and the local constant for all characters of $\mathcal{W}_F$.
\end{defn}
We wish to extend this definition to all finite-dimensional semisimple representations. For the $L$-function this is simple: first we set
\[L(\sigma, s) = 1\]
for all irreducible smooth representations $\sigma$ of dimension $n\ge 2$, which defines the $L$-function for all irreducible representations. Then we extend this to all semisimple representations by setting
\[L(\sigma_1\oplus \sigma_2, s) = L(\sigma_1, s)L(\sigma_2, s)\]
for all finite-dimensional smooth representations $\sigma_1, \sigma_2$.
\begin{rem}
	There is a more uniform way of defining the $L(\sigma, s)$: if $(\sigma, V)$ is a finite-dimensional, semisimple, smooth representation of $\mathcal{W}_F$, then the space $V^{\mathcal{I}_F}$ of $\mathcal{I}_F$-fixed vectors carries a natural representation $\sigma_{I}$ of $\mathcal{W}_F$. If $\Phi$ is a geometric Frobenius element, we have
	\[L(\sigma, s) = \det(1 - \sigma_I(\Phi)q^{-s})^{-1}.\]
	This is the standard definition of an \emph{Artin $L$-function}. 
\end{rem}
Extending the definition of the local constant $\varepsilon(\sigma, s, \psi)$ is considerably harder. We shall just quote a result asserting its existence. In what follows if $\psi: F\to \CC^\times$ is an additive character and $E/F$ is a finite extension, we set $\psi_E = \psi \circ \mathrm{Tr}_{E/F}\in \widehat{E}$. Recall that $1_E$ denotes the trivial character of $\mathcal{W}_E$ and $R_{E/K}$ is the regular representation $\Ind_{E/K} 1_E$. Also write $\mathcal{G}^{ss}(F) = \bigcup_{n\ge 1} \mathcal{G}^{ss}_n(F)$, the set of isomorphism classes of all finite-dimensional semisimple smooth representations of $F$.
\begin{thm}
	Let $\psi\in \widehat{F}$, $\psi\not=1$, and let $E/F$ range over the finite separable extensions of $F$. There is a unique family of functions
	\begin{align*}
		\mathcal{G}^{ss}(E)&\to \CC[q^s, q^{-s}]^\times\\
		\rho &\mapsto \varepsilon(\rho, s, \psi_E)
	\end{align*}
	satisfying the following properties:
	\begin{enumerate}
		\item If $\chi$ is a character of $E^\times$, then
		\[\varepsilon(\chi\circ \mathbf{a}_F, s, \psi_E) = \varepsilon(\chi, s, \psi_E)\]
		\item If $\rho_1, \rho_2\in \mathcal{G}^{ss}(E)$, then
		\[\varepsilon(\rho_1\oplus \rho_2, s, \psi_E) = \varepsilon(\rho_1, s, \psi_E)\varepsilon(\rho_2, s, \psi_E)\] 
		\item If $\rho \in \mathcal{G}^{ss}_n(E)$ and $F\subseteq K\subseteq E$ then
		\[\frac{\varepsilon(\Ind_{E/K}\rho, s, \psi_K)}{\varepsilon(\rho, s, \psi_E)} = \frac{\varepsilon(R_{E/K}, s, \psi_K)^n}{\varepsilon(1_E, s, \psi_E)^n}\]
	\end{enumerate}
\end{thm}
\begin{proof}
	\cite[Theorem 29.4]{BH1}
\end{proof}
The quantity $\varepsilon(\rho, s, \psi)$ is called the \emph{Langlands--Deligne local constant} of $\rho$, relative to the character $\psi\in \widehat{F}$ and the complex variable $s$.

%\begin{prop}
%	Let $\psi\in \widehat{F}, \psi\not=1$ and $\rho \in \mathcal{G}^{ss}(F)$. 
%	\begin{itemize}
%		\item There is an integer $n(\rho, \psi)$ such that
%		\[\varepsilon(\rho, s, \psi) = q^{n(\rho, \psi)(\frac{1}{2}-s)}\varepsilon\left(\rho, \frac{1}{2}, \psi\right)\]
%		\item Let $a\in F^\times$. Then
%		\[\varepsilon(\rho, s, a\psi) = \det \rho(a) |a|^{\dim(\rho)(s-\frac{1}{2})}\varepsilon(\rho, s, \psi)\]
%		and
%		\[n(\rho, a\psi) = n(\rho, \psi) + v_F(a)\dim(\rho)\]
%		In particular, $n(\rho, \psi)$ depends only on $\rho$ and the level of $\psi$. (Recall that $(a\psi)(x) = \psi(ax)$ for $x\in F$.)
%		\item The local constants satisfy the functional equation 
%		\[\varepsilon(\rho, s, \psi)\varepsilon(\check{\rho}, 1-s, \psi) = \det \rho(-1)\]
%		\item There is an integer $n_\rho$ such that if $\chi$ is a character of $F^\times$ of level $k\ge n_\rho$ then 
%		\[\varepsilon(\chi\otimes \rho, s, \psi) = \det \rho(c(\chi))^{-1}\varepsilon(\chi, s, \psi)^{\dim(\rho)}\]
%		for any $c(\chi)\in F^\times$ such that $\chi(1 + x) = \psi(c(\chi)x)$, $x\in \mathfrak{p}^{\left\lfloor \frac{k}{2}\right\rfloor+1}$.
%	\end{itemize}
%\end{prop}



