\subsection{Definition of the Weil Group}
We begin by defining the Weil group of the local field $F$, and an appropriate class of its representations.

Fix a separable closure $\overline{F}$ of $F$. Then the absolute Galois group of $F$ is
\[\Omega_F = \varprojlim\Gal(E/F),\]
where the limit is taken over all finite Galois extensions $E/F$ contained in $\overline{F}$. This is naturally a topological group with its profinite topology. If $K/F$ is a finite extensions, then $\overline{F}$ is also a separable closure of $K$, and $\Omega_K = \Gal(\overline{F}/K)$ is an open subgroup of $\Omega_F$.

For any positive integer $m$, there is a unique unramified extension $F_m/F$ of degree $m$ contained in $\overline{F}$. It is Galois, and the natural restriction map
\[\Gal(F_m/F)\to \Gal(k_{F_m}/k_F)\]
is an isomorphism. The extension of residue fields is Galois with cyclic Galois group, generated by the Frobenius automorphism $x\mapsto x^q$, where $q = |k|$. Hence there is a unique element $\phi_m\in \Gal(F_m/F)$ restricting to the Frobenius on the residue fields. We set $\Phi_m = \phi_m^{-1}$.

The field $F_\infty = \bigcup_{m\ge1} F_m$ is the largest unramified extension of $F$. Its Galois group is obtained as a limit
\[\Gal(F_\infty/F) = \varprojlim_{m\ge 1}\Gal(F_m/F).\]
Since the isomorphisms $\Gal(F_m/F)\cong \ZZ/m\ZZ$, $\Phi_m\mapsto 1$ are compatible with the restriction maps $\Gal(F_m/F)\to\Gal(F_l/F)$ for $l\mid m$, we can write this as
\[\Gal(F_\infty/F)\cong \varprojlim_{m\ge 1} \ZZ/m\ZZ\]
The group on the right is the group $\widehat{\ZZ}$ of profinite integers. Under this isomorphism, $1\in \widehat{\ZZ}$ corresponds to a unique element $\Phi_F\in \Gal(F_\infty/F)$ called the \emph{geometric Frobenius substitution}. We will call an element of $\Omega_F$ a \emph{Frobenius element} if its restriction to $\Gal(F_\infty/F)$ is $\Phi_F$. 

We set $\mathcal{I}_F = \Gal(\overline{F}/F_\infty)$, called the \emph{inertia group} of $F$. Then we have a short exact sequence
\[0 \to \mathcal{I}_F\to \Omega_F \xrightarrow{\text{res}} \widehat{\ZZ}\to 0\]
where the map $\Omega_F\to \widehat{\ZZ}$ comes from the restriction to $F_\infty$ as above.

\begin{defn}
	The \emph{Weil group} of $F$ is the topological group whose underlying abstract group is 
	\[\mathcal{W}_F = \text{res}^{-1}(\ZZ)\subseteq \Omega_F,\]
	and which is topologized as follows: a basis of open sets is given by the collection $\{\sigma U\}$, where $\sigma \in \mathcal{W}_F$ and $U\subseteq \mathcal{I}_F$ is an open set in the topology of $\mathcal{I}_F$ as a subspace of $\Omega_F$.
\end{defn}
One easily checks that the given collection is indeed a basis for a topology on $\mathcal{W}_F$, that $\mathcal{W}_F$ becomes a topological group, $\mathcal{I}_F\subseteq \mathcal{W}_F$ is open, and the subspace topology on $\mathcal{I}_F$ from $\mathcal{W}_F$ is the same as its natural topology as a subspace of $\Omega_F$. It also follows easily that $\mathcal{W}_F$ is locally profinite, as it is covered by translates of the profinite group $\mathcal{I}_F$.

In particular, the topology on $\mathcal{W}_F$ is not the subspace topology inherited from $\Omega_F$; it is a finer topology. Hence the canonical injection
\[\iota_F: \mathcal{W}_F\to \Omega_F\]
is continuous. Its image is dense in $\Omega_F$.

 We therefore have a short exact sequence
\[0 \to \mathcal{I}_F \to \mathcal{W}_F \xrightarrow{v_F} \ZZ\to 0\]
where $v_F$ takes a geometric Frobenius element to 1, and we have the norm
\[|x| = q^{-v_F(x)}\]
for $x\in \mathcal{W}_F$. 

If $E/F$ is a finite extension, we have the inclusion $\Omega_E\subseteq \Omega_F$, and under this inclusion we have that 
\[\mathcal{W}_E = \mathcal{W}_F \cap \Omega_E\]
so $\mathcal{W}_E\subseteq \mathcal{W}_F$ is an open subgroup. We have $[\mathcal{W}_F:\mathcal{W}_E] = [E:F]$, and if $E/F$ is Galois, then $\mathcal{W}_E\subseteq \mathcal{W}_F$ is normal and $\mathcal{W}_F/\mathcal{W}_E\cong \Gal(E/F)$. Hence finite extensions of $F$ correspond to finite index open subgroups of $\mathcal{W}_F$.

\subsection{Representations of the Weil Group}
Here we collect the basic facts relating to the representation theory of the locally profinite group $\mathcal{W}_F$. We have the following basic fact:
\begin{lemma}
	Let $V$ be an irreducible smooth representation of $\mathcal{W}_F$. Then $V$ has finite dimension.
\end{lemma}
As above, we have a canonical injection
\[\iota_F: \mathcal{W}_F\to \Omega_F\]
hence if $\rho$ is a smooth representation of $\Omega_F$, then $\rho\circ \iota_F$ is a smooth representation of $\mathcal{W}_F$, which we think of as the restriction of $\rho$ to $\mathcal{W}_F$. 
\begin{lemma}
\begin{enumerate}
	\item Let $\rho$ be an irreducible smooth representation of $\Omega_F$. Then $\rho\circ \iota_F$ is an irreducible smooth representation of $\mathcal{W}_F$.
	\item Let $\rho_1, \rho_2$ be irreducible smooth representations of $\Omega_F$. Then $\rho_1\cong \rho_2$ if and only if $\rho_1\circ \iota_F\cong \rho_2\circ\iota_F$. 
\end{enumerate}
\end{lemma}
We say that a character $\chi$ of $\mathcal{W}_F$ is \emph{unramified} if it's trivial on $\mathcal{I}_F$. 
\begin{prop}
	Let $\tau$ be an irreducible smooth representation of $\mathcal{W}_F$. The following are equivalent:
	\begin{enumerate}
		\item The group $\tau(\mathcal{W}_F)$ is finite;
		\item $\tau \cong \rho\circ\iota_F $ for an irreducible smooth representation $\rho$ of $\Omega_F$;
		\item The character $\det \circ \tau$ has finite order.
	\end{enumerate}
	Moreover, for any irreducible smooth representation $\tau$ of $\mathcal{W}_F$, there is an unramified character $\chi$ of $\mathcal{W}_F$ such that $\chi\otimes \tau$ satisfies 1.-3. above.
\end{prop}
In what follows, we consider a finite separable extension $E/F$. We have two ways of relating the smooth representations of $F$ with those of $E$:
\begin{itemize}
	\item Given a smooth representation $\rho$ of $\mathcal{W}_F$, we can use the inclusion $\mathcal{W}_E\subseteq \mathcal{W}_F$ to restrict $\rho$ to a representation of $\mathcal{W}_E$. We denote this representation by 
	\[\rho\mid_{\mathcal{W}_E} = \Res_{E/F} \;\rho = \rho_E\]
	\item Given instead a smooth representation $\tau$ of $\mathcal{W}_E$, smooth induction gives a representation of $\mathcal{W}_F$:
	\[\Ind_{\mathcal{W}_E}^{\mathcal{W}_F}\tau = \Ind_{E/F} \tau\]
\end{itemize}
We investigate semisimplicity of representations of $\mathcal{W}_E$ and $\mathcal{W}_F$ with respect to these constructions.

\begin{lemma}
	Let $E/F$ be a finite separable extension. Then the following hold:
	\begin{enumerate}
		\item Let $\rho$ be a smooth representation of $\mathcal{W}_F$. Then $\rho$ is semisimple if and only if $\rho_E$ is semisimple.
		\item Let $\tau$ be a smooth representation of $\mathcal{W}_E$. Then $\tau$ is semisimple if and only if $\Ind_{E/F} \tau$ is semisimple.
	\end{enumerate}
\end{lemma}
\begin{notn}
	We denote by $\mathcal{G}^{ss}_n(F)$ the set of isomorphism classes of $n$-dimensional semisimple smooth representations of $\mathcal{W}_F$, and by $\mathcal{G}^0_n(F)$ the set of isomorphism classes of $n$-dimensional irreducible smooth representations of $\mathcal{W}_F$.
\end{notn}
In this notation, for a finite extension $E/F$ of degree $d$, we have restriction and induction maps
\begin{align*}
	\Ind_{E/F}: \mathcal{G}^{ss}_n(E) &\to \mathcal{G}^{ss}_{nd}(F)\\
	\Res_{E/F}: \mathcal{G}^{ss}_n(F) &\to \mathcal{G}^{ss}_n(E)
\end{align*}
We have the following criterion for recognizing semisimple representations in terms of Frobenius elements.
\begin{prop}
	Let $(\rho, V)$ be a smooth representation of $\mathcal{W}_F$ of finite dimension, and let $\Phi\in \mathcal{W}_F$ be a Frobenius element. Then the following are equivalent:
	\begin{enumerate}
		\item The representation $\rho$ is semisimple;
		\item $\rho(\Phi)\in \Aut_{\CC}(V)$ is semisimple;
		\item $\rho(\Psi)\in \Aut_{\CC}(V)$ is semisimple for all $\Psi\in \mathcal{W}_F$.
	\end{enumerate}
\end{prop}
\subsection{Local Class Field Theory}\label{sec:CFT}
In this subsection we summarize local class field theory, giving an axiomatic account. We then use it to relate characters of the Weil group to multiplicative characters of $F$.

\begin{thm}[Local class field theory]
	There is a unique continuous group homomorphism
	\[\mathbf{a}_F: \mathcal{W}_F\to F^\times\]
	with the following properties:
	\begin{enumerate}
		\item $\mathbf{a}_F$ induces a topological isomorphism $\mathcal{W}_F^{ab}\cong F^\times$;
		\item An element $x\in \mathcal{W}_F$ is a geometric Frobenius if and only if $\mathbf{a}_F(x)$ is a uniformizer in $F$;
		\item We have $\mathbf{a}_F(\mathcal{I}_F) = \mathcal{O}_F^\times$;
		\item For any finite separable extension $E/F$, the diagram 
		\[\begin{tikzcd}
			{\mathcal{W}_E} && {E^\times} \\
			\\
			{\mathcal{W}_F} && {F^\times}
			\arrow["{\mathbf{a}_E}", from=1-1, to=1-3]
			\arrow["{\Res_{E/F}}"', from=1-1, to=3-1]
			\arrow["{N_{E/F}}", from=1-3, to=3-3]
			\arrow["{\mathbf{a}_F}", from=3-1, to=3-3]
		\end{tikzcd}\]
		commutes.
	\end{enumerate}
	The map $\mathbf{a}_F$ is called the \emph{Artin reciprocity map}.
\end{thm}
We also mention some consequences of local class field theory.
\begin{cor}
	\begin{enumerate}
	\item There is a bijection
		\begin{align*}
			\{\text{finite abelian extensions of $F$}\} &\leftrightarrow \{\text{finite index open subgroups of $F^\times$}\}\\
				E/F &\mapsto N_{E/F}(E^\times)\\
		\end{align*}
		which sends unramified extensions to subgroups of $\mathcal{O}_F^\times$.
	\item For a finite abelian extension $E/F$, the Artin map $\mathbf{a}_F$ induces an isomorphism
		\[\Gal(E/F)\cong F^\times/N_{E/F}(E^\times)\]
	\end{enumerate}
\end{cor}
Note that any character $\chi: \mathcal{W}_F\to \CC^\times$ of $\mathcal{W}_F$ must factor through the abelianization $\mathcal{W}_F^{ab}$, which is isomorphic via the Artin reciprocity map to $F^\times$. So we have an induced isomorphism of character groups
\begin{align*}
	\{\text{smooth characters of }F^\times\} &\cong \{\text{smooth characters of }\mathcal{W}_F\}\\
	\chi &\mapsto \chi\circ\mathbf{a}_F
\end{align*}
\subsection{\texorpdfstring{$L$}{TEXT}-function and Local Constant}
In this subsection, we define the quantities $L(\sigma, s)$ (the $L$-function) and $\varepsilon(\sigma, s, \psi)$ (the local constant) for smooth semisimple finite-dimensional representations $\sigma$ of $\mathcal{W}_F$. The corresponding quantities for characters of $F^\times$ were defined in Section \ref{sec:func_equation}. We transfer these to characters of $\mathcal{W}_F$ via the Artin map.
\begin{defn}\label{defn:LfuncWeil}
	If $\chi: F^\times \to \CC^\times$ is a character and $\psi: F\to \CC^\times$ is an additive character, we define
	\begin{align*}
		L(\chi\circ\mathbf{a}_F, s) &= L(\chi, s)\\
		\varepsilon(\chi\circ\mathbf{a}_F, s, \psi) &= \varepsilon(\chi, s, \psi).
	\end{align*}
	This defines the $L$-function and the local constant for all characters of $\mathcal{W}_F$.
\end{defn}
We wish to extend this definition to all finite-dimensional semisimple representations. For the $L$-function this is simple: first we set
\[L(\sigma, s) = 1\]
for all irreducible smooth representations $\sigma$ of dimension $n\ge 2$, which defines the $L$-function for all irreducible representations. Then we extend this to all semisimple representations by setting
\[L(\sigma_1\oplus \sigma_2, s) = L(\sigma_1, s)L(\sigma_2, s)\]
\begin{rem}
	There is a more uniform way of defining the $L(\sigma, s)$: if $(\sigma, V)$ is a finite-dimensional, semisimple, smooth representation of $\mathcal{W}_F$, then the space $V^{\mathcal{I}_F}$ of $\mathcal{I}_F$-fixed vectors carries a natural representation $\sigma_{I}$ of $\mathcal{W}_F$. If $\Phi$ is a geometric Frobenius element, we have
	\[L(\sigma, s) = \det(1 - \sigma_I(\Phi)q^{-s})^{-1}.\] 
\end{rem}
Extending the definition of the local constant $\varepsilon(\sigma, s, \psi)$ is considerably harder. We shall just quote a result asserting its existence. In what follows if $\psi: F\to \CC^\times$ is an additive character and $E/F$ is a finite extension, we set $\psi_E = \psi \circ \mathrm{Tr}_{E/F}\in \widehat{E}$. Recall that $1_E$ denotes the trivial character of $\mathcal{W}_E$ and $R_{E/K}$ is the regular representation $\Ind_{E/K} 1_K$. Also write $\mathcal{G}^{ss}(F) = \bigcup_{n\ge 1} \mathcal{G}^{ss}_n(F)$, the set of isomorphism classes of all finite-dimensional semisimple smooth representations of $F$.
\begin{thm}
	Let $\psi\in \widehat{F}$, $\psi\not=1$, and let $E/F$ range over the finite separable extensions of $F$. There is a unique family of functions
	\begin{align*}
		\mathcal{G}^{ss}(E)&\to \CC[q^s, q^{-s}]^\times\\
		\rho &\mapsto \varepsilon(\rho, s, \psi_E)
	\end{align*}
	satisfying the following properties:
	\begin{enumerate}
		\item If $\chi$ is a character of $E^\times$, then
		\[\varepsilon(\chi\circ \mathbf{a}_F, s, \psi_E) = \varepsilon(\chi, s, \psi_E)\]
		\item If $\rho_1, \rho_2\in \mathcal{G}^{ss}(E)$, then
		\[\varepsilon(\rho_1\oplus \rho_2, s, \psi_E) = \varepsilon(\rho_1, s, \psi_E)\varepsilon(\rho_2, s, \psi_E)\] 
		\item If $\rho \in \mathcal{G}^{ss}_n(E)$ and $F\subseteq K\subseteq E$ then
		\[\frac{\varepsilon(\Ind_{E/K}\rho, s, \psi_K)}{\varepsilon(\rho, s, \psi_E)} = \frac{\varepsilon(R_{E/K}, s, \psi_K)^n}{\varepsilon(1_E, s, \psi_E)^n}\]
	\end{enumerate}
\end{thm}
The quantity $\varepsilon(\rho, s, \psi)$ is called the \emph{Langlands--Deligne local constant} of $\rho$, relative to the character $\psi\in \widehat{F}$ and the complex variable $s$. We also list some of its properties:

\begin{prop}
	Let $\psi\in \widehat{F}, \psi\not=1$ and $\rho \in \mathcal{G}^{ss}(F)$. 
	\begin{itemize}
		\item There is an integer $n(\rho, \psi)$ such that
		\[\varepsilon(\rho, s, \psi) = q^{n(\rho, \psi)(\frac{1}{2}-s)}\varepsilon\left(\rho, \frac{1}{2}, \psi\right)\]
		\item Let $a\in F^\times$. Then
		\[\varepsilon(\rho, s, a\psi) = \det \rho(a) |a|^{\dim(\rho)(s-\frac{1}{2})}\varepsilon(\rho, s, \psi)\]
		and
		\[n(\rho, a\psi) = n(\rho, \psi) + v_F(a)\dim(\rho)\]
		In particular, $n(\rho, \psi)$ depends only on $\rho$ and the level of $\psi$. (Recall that $(a\psi)(x) = \psi(ax)$ for $x\in F$.)
		\item The local constants satisfy the functional equation 
		\[\varepsilon(\rho, s, \psi)\varepsilon(\check{\rho}, 1-s, \psi) = \det \rho(-1)\]
		\item There is an integer $n_\rho$ such that if $\chi$ is a character of $F^\times$ of level $k\ge n_\rho$ then 
		\[\varepsilon(\chi\otimes \rho, s, \psi) = \det \rho(c(\chi))^{-1}\varepsilon(\chi, s, \psi)^{\dim(\rho)}\]
		for any $c(\chi)\in F^\times$ such that $\chi(1 + x) = \psi(c(\chi)x)$, $x\in \mathfrak{p}^{\left\lfloor \frac{k}{2}\right\rfloor+1}$.
	\end{itemize}
\end{prop}



