In the previous section, we classified the prinicipal series representations of $G=\GL_2(F)$ over a nonarchimedean local field $F$. For characters $\chi$ of $\GL_1(F)$, Tate's thesis \cite{Tate} associates a space $\mathcal Z(\chi)$ of zeta functions in a complex variable $s$. This space will, in a sense to be made precise, be generated by a single element, the $L$-function $L(\chi,s)$. The zeta functions will also satisfy a functional equation depending on the `local constant' $\epsilon(\chi,s,\psi)$. Here $\psi :F \to \CC^\times$ is a character whose purpose is to fix a form of Fourier transform on $F$. These definitions and results in Tate's thesis are intended to mimic the classical theory of $L$-functions due largely to Hecke, which encompass the Riemann zeta function. The $L$-function and local constant of a character $\chi:F^\times \to \CC^\times$ will turn out to carry the essential information of $\chi$. In the classical setting see, for example, the converse theorem of Weil reproduced in \cite[Theorem 1.5.1]{Bump}.

In the setting of irreducible smooth representations $\pi$ of $G$, in particular the principal series representations $\pi$, we want to again associate a space $\mathcal Z(\pi)$ of zeta functions, an $L$-function $L(\pi,s)$ and a local constant $\epsilon(\pi,s,\psi)$ determining a functional equation. 

We begin this section with a brief review of harmonic and Fourier analysis and the role it plays in representation theory. For more details, see \cite[Chapter 3.1]{Bump}. Following the presentation in \cite{GH1}, we define the $L$-functions and local constants of characters of $F^\times$. We explain how this theory generalises to irreducible smooth representations $\pi$ of $G$, culminating in the Theorems \ref{BHThm1} and \ref{BHThm2}, which determine the functional equations satisfied by the zeta functions associated to $\pi$. Propositions \ref{prop:gl2factor} and \ref{prop:gl2gamma} prove these in the case where $\pi = \iota_B^G \chi$ is a principal series representation. The case where $\iota_B^G \chi$ is reducible, so that $\pi$ is only a subquotient, requires slightly more work. The results are summarised in Table 1. Finally, we prove a converse theorem for principal series representations of $G$.


\subsection{Review of harmonic analysis}

Take as motivation the representation theory of a finite group $H$. Every irreducible representation of $H$ appears as a direct summand of the regular representation $\CC[H]$, with some multiplicity. For a locally compact topological group $\mathbb G$ with Haar measure $dg$, the correct generalisation of $\CC[H]$ is the space $L^2(\mathbb G)$ of measurable functions $f:\mathbb G \to \CC$ for which 
$$\int_{\mathbb G} |f(g)|^2 dg < \infty.$$
The action of $\mathbb G$ is by right translation. If $\mathbb G$ is additionally abelian, the group $\hat{\mathbb G}$ of (unitary) characters of $\mathbb G$ is also a locally compact abelian group, the Pontryagin dual of $\mathbb G$. 

\begin{example}
    The Pontryagin duals of $\mathbb G = \RR, \ZZ, \RR/\ZZ$ are $\RR, \RR/\ZZ, \ZZ$ respectively. The characters of $\RR$ are of the form $x \mapsto e^{-2\pi i xy}$ for $y \in \RR$. The characters of $\ZZ$ are of the form $n \mapsto e^{-2\pi i nx}$ for $x \in \RR/\ZZ \cong S^1$. The characters of $\RR/\ZZ$ are of the form $x \mapsto e^{-2\pi i nx}$ for $n \in \ZZ$. In particular, $\RR$ is self-dual.
\end{example}

On a suitable dense subset of $L^2(\mathbb G)$ (the Schwartz space), one can define the Fourier transform $\hat{f} \in L^2(\hat{\mathbb G})$ of $f$ by
$$\hat{f}(\chi) = \int_{\mathbb G} f(g)\chi(g) dg.$$
The Fourier transform uniquely extends to a map $L^2(\mathbb G) \to L^2(\hat{\mathbb G})$. For suitable choices of Haar measures there is then a Fourier inversion formula 
$$\hat{\hat{f}}(g)=f(-g),$$ so that the above map is a bijection.

\begin{example}
    For $\mathbb G=\RR$, the Fourier transform of $f$ is 
    $$\hat{f}(x) = \int_{\RR} f(y)e^{-2\pi i xy} dy$$
    which is the classical Fourier transform. Identifying $\hat{\RR} = \RR$, the Fourier transform gives an invertible map $L^2(\RR) \to L^2(\RR)$, so that any element of $L^2(\RR)$ can be expressed as an integral of elements of $\hat{\RR}$. 

    Inside the representation $L^2(\RR)$ of $\RR$ we therefore see this `continuous spectrum' of the irreducible unitary representations (characters) of $\RR$, parametrised by $\RR$. Note, however, that each such character can not be realised as a subrepresentation of $L^2(\RR)$; for $y \in \RR$ the character $x \mapsto e^{-2\pi i xy}$ is realised as the Fourier transform of a function on $\RR$ supported only at $y$, but such a function is not in $L^2(\RR)$.
\end{example}

\begin{example}
    For $\mathbb G = \ZZ$, the Fourier transform of $f$ is 
    $$\hat{f}(x) = \sum_{\ZZ} f(n)e^{-2\pi i nx}.$$
    So any element of $L^2(\RR/\ZZ)$ can be expressed as a sum of unitary characters of $\ZZ$; we have a `discrete spectrum'. 
\end{example}

\begin{rem}
    The terminology of discrete and continuous spectra comes from the analogy with the spectral theory of the Laplacian. Over $\RR$, the Laplacian is $\Delta = \frac{\partial^2}{\partial x^2}$, and the characters $x \mapsto e^{-2\pi i xy}$ are eigenfunctions. 
\end{rem}

The dichotomy in the above examples is reflected in the compactness of $S^1$ and non compactness of $\RR$. More generally,

\begin{thm}[Peter-Weyl]
    Let $K$ be a compact Hausdorff topological group. Any unitary representation of $K$ decomposes into a completed Hilbert space direct sum of irreducible unitary subrepresentations. There is a unitary equivalence
    $$L^2(K) \cong \widehat{\bigoplus}_{\pi \in \hat{K}} \mathrm{End}(V_\pi)$$
    of representations of $K\times K$, where $(\pi,V_\pi)$ ranges over the set $\hat{K}$ of equivalence classes of irreducible representations of $K$, and $\hat\oplus$ denotes the completed Hilbert space direct sum.
\end{thm}
\begin{proof}
    \cite[Theorem 7.3.2]{DE} and \cite[Theorem 7.2.3]{DE}.
\end{proof}

Even more generally, for so-called Type I groups one can decompose unitary representations through a combination of integrals and Hilbert space direct sums. See \cite[Section 3.10]{GH1} for further details.

Returning to $G=\GL_2(F)$, as this is not compact we would expect the regular representation $L^2(G)$ to decompose according to both a continuous spectra and a discrete spectra. This continous spectra is provided by the parabolically induced representations $\iota_B^G \chi$, where $\chi$ ranges over the characters of $T \cong F^\times \times F^\times$.

In order to compare representations of $G$ and Galois representations through the local Langlands correspondence, we would like to classify them according to some common language. This is provided by the zeta functions, $L$-functions and functional equations discussed in this section. 

The prototypical example of an $L$-function is the Riemann zeta function $\zeta(s) = \sum_{n \geq 1} n^{-s}$.

\begin{prop}
    The function $\zeta(s) = \sum_{n \geq 1} n^{-s}$ satisfies the following properties:
    \begin{itemize}
        \item (Analytic continuation) The Riemann zeta functions converges absolutely to a holomorphic function on $\mathrm{Re}(s)>1$. It has a unique analytic continuation to the complex plane, except the point $s=1$ where $\zeta(s)$ has a simple pole.
        \item (Euler product) We have the identity $$\sum\limits_{n=1}^\infty n^{-s} = \prod\limits_{p \text{ prime}} \frac{1}{1-p^{-s}},$$ convergent for $\mathrm{Re}(s)>1$.
        \item (Functional equation) There is an explicit function $\gamma(s)$ such that $\zeta(1-s)=\gamma(s)\zeta(s)$.
    \end{itemize}
\end{prop}

The approach of Tate in his thesis was to view the Riemann (And Dedekind) zeta functions from an adelic perspective. There the Euler product formulation is immediate and we only need to study the zeta functions locally. Attached to any character $\chi:F^\times \to \CC^\times$ there is an associated space $\mathcal Z(\chi)$ of zeta functions $\zeta(\Phi,\chi,s)$, where $\Phi \in C_c^\infty(F)$. The factor at the prime $p$ of the Riemann zeta function corresponds to the trivial character of $\QQ_p^\times$ and the function $\mathbbm{1}_{\ZZ_p} \in C_c^\infty(\QQ_p)$. A key ingredient in the proof of the functional equation of the Riemann zeta function is the Fourier transform over $\CC$. In general, the functional equation associated to $\chi$ relates zeta functions $\zeta(\hat{Phi},\chi^{-1},1-s)$ and $\zeta(\Phi,\chi,s)$, where $\hat{\Phi}$ is the Fourier transform of $\Phi$ in $C_c^\infty(F)$. 







\subsection{Functional equation for \texorpdfstring{$\GL_1$}{TEXT}}

Let $F$ be a nonarchimedean local field, $\varpi$ be a uniformiser and $q$ be the size of the residue field. We will later define $L$-functions attached to an irreducible smooth representation of $\GL_2(F)$ and determine a functional equation they satisfy. First we explain this in the context of irreducible smooth representations $\chi$ of $\GL_1(F)$, necessarily a character $\chi: F^\times \to \CC^\times$.

Taking from the classical study of the Riemann zeta function and its functional equation, we want to introduce an analogue of the Fourier transform over $F$. We replace the additive character $e^{2\pi i -}: \RR \to \CC^\times$ with any choice of additive character $\psi: F \to \CC\times$ with $\psi \neq 1$. In this way, all characters of $F$ are of the form $\psi(-y)$ for $y \in F$, by Additive Duality. The functions we will apply the Fourier transform to will be the algebra $C_c^\infty(F)$ of locally constant compactly supported functions $F \to \CC$. For any choice of Haar measure $\mu$ on $F$, we now define the Fourier transform.

\begin{defn}
    Let $\Phi \in C_c^\infty(F)$, $\psi:F \to \CC^\times$ be an additive character of $F$, and $\mu$ be a Haar measure on $F$. The Fourier transform of $\Phi$ (with respect to $\psi$ and $\mu$) is 
    $$\hat{\Phi}(x) := \int_F \Phi(y)\psi(xy) d\mu(y).$$
\end{defn}

To match the classical definition over $\RR$, we would like the Fourier transform to preserve $C_c^\infty(F)$, and to have a Fourier inversion formula. Indeed:

\begin{prop}
    
    \begin{itemize}
        \item For any $\Phi \in C_c^\infty(F)$, we have $\hat{\Phi} \in C_c^\infty(F)$.
        \item For any $\psi: F \to \CC^\times$ with $\psi \neq 1$, there is a unique Haar measure $\mu_\psi$ on $F$ such that for the associated Fourier transform we have $$\hat{\hat{\Phi}}(x) = \Phi(-x)$$ for any $\Phi \in C_c^\infty(F)$ and $x \in F$.
    \end{itemize}
    
\end{prop}
\begin{proof}
    \cite[Proposition 23.1]{BH1}
\end{proof}

\begin{notn}
    For the remainder of this subsection, $\psi \neq 1$ will be an additive character of $F$, and $\mu= \mu_\psi$ will denote the associated self-dual Haar measure on $F$.
\end{notn}


Now let $\chi: F^\times \to \CC^\times$ be a smooth character of $F^\times$. We want to attach to this character an $L$-function $L(\chi,s)$ in the formal variable $s$. This is defined to be $(1-\chi(\varpi)q^{-s})^{-1}$ when $\chi$ is unramified, and 1 otherwise. In order to generalise to $\GL_2$ it would be preferable to have a more intrinsic definition.

\begin{defn}
    For $\Phi \in C_c^\infty(F)$ and $\chi :F^\times \to \CC^\times$, define the zeta function $\zeta(\Phi,\chi,s)$ to be
    $$\zeta(\Phi,\chi,s) := \int_{F^\times} \Phi(x)\chi(x)|x|^s d^*x,$$ in the formal variable $s$, where $d\mu^*(x) = d^*x$ denotes any choice of Haar measure on $F^\times$.
\end{defn}

Equivalently, we have
$$\zeta(\Phi,\chi,s) = \sum\limits_{m \in \ZZ} z_m q^{-ms}$$
for $$z_m = \int\limits_{\varpi^m \mathcal O_F^\times} \Phi(x)\chi(x)d^*x.$$ In this way it is clear that $\zeta(\Phi,\chi,s) \in \CC((q^{-s}))$. The $z_m=z_m(\Phi,\chi)$ vanish for $m <<0$ because $\Phi$ is compactly supported on $F$.

The zeta function $\zeta(\Phi,\chi,s)$ only depends on $d^*x$ up to scaling. To remove this dependence we define the following notation.

\begin{notn}
    Let $$\mathcal Z(\chi) = \{\zeta(\Phi,\chi,s) \mid \Phi \in C_c^\infty(F)\}.$$
\end{notn}

\begin{notn}
    For $a \in F^\times$ and $\Phi \in C_c^\infty(F)$, denote by $a\Phi$ the function $x \mapsto \Phi(a^{-1}x)$.
\end{notn}

\begin{lemma}
    The space $\mathcal Z(\chi)$ is a $\CC[q^{-s},q^s]$-module, containing $\CC[q^{-s},q^s]$.
\end{lemma}
\begin{proof}
    Let $a \in F^\times$ of valuation $v_F(a)$. Then 
    $$\zeta(a\Phi,\chi,s) = \chi(a)q^{-v_f(a)s}\zeta(\Phi,\chi,s),$$ giving the desired module structure. To establish the containment we show that $\mathcal Z(\chi)$ contains a nonzero constant. Let $d$ be such that $\chi \mid_{U_F^{d+1}} = 1$. Taking $\Phi=\mathbbm{1}_{U_F^{d+1}}$, we see that 
    $$Z(\Phi,\chi,s) = \mu^*(U_F^{d+1}) \neq 0.$$
\end{proof}

\begin{prop}\label{prop:gl1factor}
    Let $\chi:F^\times \to \CC^\times$. There exists a unique polynomial $P_\chi \in \CC[X]$ with $P_\chi(0)=1$ such that
    $$\mathcal Z(\chi) = P_\chi(q^{-s})^{-1}\cdot \CC[q^{-s},q^s].$$
    Moreover, we have
    $$
    P_\chi(X) =
    \begin{cases}
        1-\chi(\varpi)X & \text{if $\chi$ is unramified} \\
        1 & \text{otherwise}
    \end{cases}
    $$
\end{prop}
\begin{proof}
    Suppose $\Phi(0)=0$. Then $\Phi|_{F^\times} \in C_c^\infty(F^\times)$, and so $\Phi$ is identically zero on $\varpi^m\mathcal O_F^\times$ for $|m| >>0$. Thus only finitely many of the coefficients $z_m$ are nonzero, so that $\Phi \in \CC[q^{-s},q^s]$.

    The space $C_c^\infty(F)$ is spanned by $C_c^\infty(F^\times)$ and $\mathbbm{1}_{\mathcal O_F}$. We compute
    $$\zeta(\mathbbm{1}_{\mathcal O_F},\chi,s) = \sum\limits_{m \geq 0} \chi(\varpi^m)q^{-ms} \int_{\mathcal O_F^\times} \chi(x)d^*x.$$
    If $\chi$ is unramified (trivial on $\mathcal O_F^\times$), this gives us 
    $$\sum\limits_{m \geq 0} \chi(\varpi)^mq^{-ms} \mu^*(\mathcal O_F^\times) = (1-\chi(\varpi)q^{-s})^{-1} \mu^*(\mathcal O_F^\times).$$
    When $\chi$ is ramified the integral is zero. Indeed, translation invariance of $d^*x$ implies
    $$\int_{\mathcal O_F^\times} \chi(x)d^*x = \int_{\mathcal O_F^\times} \chi(xy)d^*x = \chi(y)\int_{\cO_F^\times} \chi(x) d^*x$$ for any $y \in \cO_F^\times$, so that this is zero if there is some $y$ with $\chi(y) \neq 1$. This computation, together with the previous lemma, establish the result. 
\end{proof}

\begin{rem}
    The computation in the proof above shows, in the case $\chi = 1$, that $\zeta(\mathbbm{1}_{\cO_F},1,s) = (1-q^{-s})^{-1}$, provided we normalise $d^*x$ appropriately. If $F=K_v$ is the completion of a number field $K$ at a nonarchimedean place $v$, we recover the Euler factor of the Dedekind zeta function $\zeta_K(s)$ at the place $v$. This explains the naming of our zeta functions. 
\end{rem}

\begin{rem}
    The computations of Proposition \ref{prop:gl1factor} show that each $\zeta(\Phi,\chi,s)$ converges absolutely and uniformly in vertical strips in some right half plane, and admit analytic continuation to a rational function in $q^{-s}$.
\end{rem}

\begin{defn}
    Define the $L$-function attached to $\chi$ to be $L(\chi,s)=P_\chi(q^{-s})^{-1}$.
\end{defn}

As with the Riemann zeta function, we have functional equations for the zeta functions.

\begin{thm}\label{thm:gl1gamma}
    Let $\chi: F^\times \to \CC^\times$. There is a unique $\gamma(\chi,s,\psi) \in \CC(q^{-s})$ such that 
    $$\zeta(\hat{\Phi}, \check{\chi},1-s) = \gamma(\chi,s,\psi) \zeta(\Phi,\chi,s)$$ for all $\Phi \in C_c^\infty(F)$, where $\check{\chi}=1/\chi : F^\times \to \CC^\times$.
\end{thm}
\begin{proof}
    \cite[Theorem 23.3]{BH1}.
\end{proof}

Since $\mathcal Z(\chi) = L(\chi,s)\cdot \CC[q^{-s},q^s]$, it is natural to consider the terms $\frac{\zeta(\Phi,\chi,s)}{L(\chi,s)} \in \CC[q^{-s},q^s]$. This allows us to treat the case of $\chi$ ramified and unramified evenly. 

\begin{defn}
    Let $$\epsilon(\chi,s,\psi) := \gamma(\chi,s,\psi) \cdot \frac{L(\chi,s)}{L(\check{\chi},s)}.$$
\end{defn}
This is known as Tate's local constant.

The functional equation for $\zeta$ can be rewritten as
$$\frac{\zeta(\hat{\Phi},\check{\chi},1-s)}{L(\check{\chi},1-s)} = \epsilon(\chi,s,\psi) \frac{\zeta(\Phi,\chi,s)}{L(\chi,s)}.$$

\begin{cor}
    The local constant satisfies the functional equation
    $$\epsilon(\chi,s,\psi)\epsilon(\check{\chi},1-s,\psi) = \chi(-1).$$
    The local constant is of the form $$\epsilon(\chi,s,\psi) = aq^{bs}$$ for some $a \in \CC^\times$, $b \in \ZZ$.
\end{cor}
\begin{proof}
    The first statement comes from the Fourier inversion formula, where the $\chi(-1)$ term comes from the minus sign in $\hat{\hat{\Phi}}(x) = \Phi(-x)$. The functional equation implies that $\epsilon$ is a unit in $\CC[q^{-s},q^s]$, and the units are precisely the elements of the form $aq^{bs}$ for $b \in \ZZ$.
\end{proof}


\subsection{Functional equation for \texorpdfstring{$\GL_2$}{TEXT}}

We turn now to smooth representations $\pi$ of $G=\GL_2(F)$ and define the $L$-functions and local constants in an analogous manner to the characters $\chi :F^\times \to \CC^\times$.

In this context, we need an additive character of $A=M_2(F)$, which we will take to be $\psi_A = \psi \circ \mathrm{tr}$ for $\psi : F \to \CC^\times$ any nontrivial additive character of $F$. We will apply the Fourier transform to the $F$-algebra $\Phi \in C_c^\infty(A)$ of locally constant compactly supported functions on $M_2(F)$.

\begin{defn}
    With respect to a Haar measure $\mu$ in $A$, and $\psi_A=\psi \circ \mathrm{tr}$ an additive character of $A$, define for any $\Phi \in C_c^\infty(A)$
    $$\hat{\Phi}(x) = \int_A\Phi(y) \psi_A(xy)d\mu(y).$$
\end{defn}

\begin{prop}
    
    \begin{itemize}
        \item For any $\Phi \in C_c^\infty(A)$, we have $\hat{\Phi} \in C_c^\infty(A)$.
        \item For any $\psi: F \to \CC^\times$ with $\psi \neq 1$, there is a unique Haar measure $\mu_{\psi_A}$ on $A$ such that for the associated Fourier transform we have $$\hat{\hat{\Phi}}(x) = \Phi(-x)$$ for any $\Phi \in C_c^\infty(A)$ and $x \in A$.
    \end{itemize}
    
\end{prop}

\begin{notn}
    For the remainder of this subsection, $\psi \neq 1$ will be an additive character of $F$, $\psi_A = \psi \circ \mathrm{tr}$, and $\mu= \mu_{\psi_A}$ will denote the associated self-dual Haar measure on $A$.
\end{notn}

For $\chi:F^\times \to \CC^\times$ we defined for any $\Phi \in C_c^\infty(F)$ a zeta function $$\zeta(\Phi,\chi,s) = \int_{F^\times} \Phi(x)\chi(x) |x|^s d^*x.$$
To replicate this with $\pi : G \to \GL(V)$ we need to extract scalar values from $\pi(g) \in \GL(V)$. These will come from matrix coefficients.

\begin{defn}
Let $(\pi,V)$ be a smooth representation of $G$ with smooth dual $\check{V}$. For vectors $v\in V, \check{v} \in \check{V}$, define the smooth function $\gamma_{v \otimes \check{v}}: G \to \CC$ by 
$$\gamma_{\check{v} \otimes v} : g \mapsto \langle \check{v},\pi(g) v \rangle$$ where $\langle, \rangle$ denotes the natural pairing $\check{V} \otimes V \to \CC$. Let $\mathcal C(\pi)$ be the vector space spanned by the $\gamma_{\check{v} \otimes v}$. Elements of $\mathcal C(\pi)$ are called the matrix coefficients of $\pi$.
\end{defn}
\begin{rem}
    If $\pi=\chi:F^\times \to \CC^\times$ is a character, any matrix coefficient (defined in the analogous way for $F^\times$) of $\chi$ is some scalar multiple of $\chi$.

    If $V$ is the tautological representation of $G$ with basis $e_1,e_2$, then $\gamma_{\check{e_i} \otimes e_j}(g)$ is precisely the $(i,j)$-th entry of $g$ as a matrix with respect to the basis $e_1,e_2$.
\end{rem}
\begin{defn}
    Let $(\pi,V)$ be an irreducible smooth representation of $G$. The centre $Z$ of $G$ acts on $V$ via the central character $\omega_\pi : Z \to \CC^\times$.
\end{defn}
\begin{lemma}\label{central char}
    For any $f \in \mathcal C(\pi), z \in Z, g \in G$ we have $f(zg) = \omega_\pi(z) f(g)$.
\end{lemma}


Fix a smooth representation $\pi$ of $G$. We may now define zeta functions for any $f \in \mathcal C(\pi)$.

\begin{defn}
    For $\Phi \in C_c^\infty(A)$ and $f \in \mathcal C(\pi)$, define the zeta function $\zeta(\Phi,f,s)$ to be
    $$\zeta(\Phi,f,s) := \int_{G} \Phi(x)f(x)|\det x|^s d^*x,$$ in the formal variable $s$, where $d\mu^*(x) = d^*x$ denotes any choice of Haar measure on $G$.
\end{defn}

\begin{lemma}
    For any $\Phi \in C_c^\infty(A)$ and $f \in \mathcal C(\pi)$ we have $\zeta(\Phi,f,s) \in \CC((q^{-s}))$ in the formal variable $s$.
\end{lemma}
\begin{proof}
    This follows from \cite[Lemma 24.4.1]{BH1}.
\end{proof}

\begin{notn}
    Let $$\mathcal Z(\pi) = \{\zeta(\Phi,f,s+\frac{1}{2}) \mid \Phi \in C_c^\infty(A), f \in \mathcal C(\pi)\}.$$
\end{notn}
\begin{rem}
    The addition of $\frac{1}{2}$ will be explained in the case of principal series representations by the appearance of the modular character $\delta_B$.
\end{rem}

\begin{lemma}
    The space $\mathcal Z(\pi)$ is a $\CC[q^{-s},q^s]$-module, containing $\CC[q^{-s},q^s]$.
\end{lemma}
\begin{proof}
    \cite[Lemma 24.4.2]{BH1}.
\end{proof}


Consider now the situation where $\pi = \iota_B^G \chi$ is a parabolically induced representation, where $\chi = \chi_1 \otimes \chi_2$ is a character of $T$. We want to study the space $\mathcal Z(\pi)$ and prove an analogous result to Proposition \ref{prop:gl1factor}.





\begin{prop}\label{prop:gl2factor}
    Let $\chi=\chi_1\otimes \chi_2$ be a character of $T$ and let $(\pi,V)=\iota_B^G \chi$. Then, formally, we have
    $$\mathcal Z(\pi) = \mathcal Z(\chi_1) \mathcal Z(\chi_2) \subset \CC((q^{-s})).$$
    In particular, there is a unique polynomial $P_\pi \in \CC[X]$ with $P_\pi(0)=1$ such that 
    $$\mathcal Z(\pi) = P_\pi(q^{-s})^{-1} \cdot \CC[q^{-s},q^s].$$
    Moreover, $P_\pi(X) = P_{\chi_1}(X)P_{\chi_2}(X)$.
\end{prop}

We make some comments in preparation for the proof. The Proposition concerns the zeta integrals 
$$\zeta(\Phi,f,s+\frac{1}{2}) = \int_{G} \Phi(x)f(x)|\det x|^{s+\frac{1}{2}} d^*x.$$

The matrix coefficients $\mathcal C(\pi)$ are spanned by 
$$\gamma_{\tau \otimes \theta} : g \mapsto \langle \tau, \pi(g) \theta \rangle$$ over $\theta \in V, \tau \in \check{V}$. Here $\theta \in \iota_B^G \chi$ is viewed as a smooth function $\theta : G \to \CC$ satisfying 
$$\theta(ntg) = \delta_B^{-1/2}(t) \chi(t) \theta(g)$$
for any $t \in T, n \in N, g \in G$. The Duality Theorem [ADD REFERENCE] identifies $\check{V} \cong \iota_B^G \check{\chi}$. In this way we view $\tau$ as a smooth function $\tau: G \to \CC$ satisfying
$$\tau(ntg) = \delta_B^{-1/2}(t)\chi(t)^{-1}\tau(g)$$
for any $t \in T, n \in N, g \in G$. The proof of the Duality Theorem shows that the pairing between $V$ and $\check{V}$ gives
$$f(g) = \langle \tau, \pi(g)\theta \rangle = \int_{B\backslash G} \tau(x)\theta(xg) d\dot{x}$$ for a positive semi-invariant measure $d\dot{x}$ on $B \backslash G$. Let $K=\GL_2(\cO_F)$. Since we have a bijection $B \backslash G \leftrightarrow K \cap B \backslash K$ and $\delta_B(tn)=\delta_B(t) = |t_2/t_1|$ (Proposition \ref{prop:modularchar}) is trivial on $K\cap B$, we can rewrite this as 
$$f(g) = \int_K \tau(k)\theta(kg)dk$$ for some Haar measure $dk$ on $K$ (\cite[Corollary 7.6]{BH1}). Moreover, \cite[Equation 7.6.2]{BH1} tells us that there is a left Haar measure $db$ on $B$ such that
$$\int_G \phi(g) dg = \int_K \int_B \phi(bk) dbdk$$ for all $\phi \in C_c^\infty(G)$. Using this, our zeta integrals reduce to integrals over $B$ and $K$. Integration over $K$ is easier to handle using the smoothness of our representations. We can write $db = dn dt$ to view integration over $B$ as integration over $T$ and $N$. In order to relate $\zeta(\Phi,f,s+\frac{1}{2})$ to zeta functions coming from $\chi: T \to \CC^\times$, we want to express the integrals over $B$ solely in terms of integrals over $T$. To do so we use the following lemma. 

\begin{lemma}\label{lemma:phiT}
    Let $D$ be the algebra of diagonal matrices in $A$ so that $D^\times =T$. Let $\Phi \in C_c^\infty(A)$. There is a unique function $\Phi_T \in C_c^\infty(D)$ whose restriction to $T$ is given by 
    $$\Phi_T(t) = |t_1| \int_N \Phi(tn)dn, \hspace{1cm} t=\begin{psmallmatrix}
        t_1 & 0\\0&t_2
    \end{psmallmatrix}.$$
    The map $\Phi \mapsto \Phi_T$ is a linear surjection $C_c^\infty(A) \to C_c^\infty (D)$.
\end{lemma}
\begin{proof}
    The space $C_c^\infty(A)$ is spanned by functions of the form 
    $$\Phi = (\phi_{ij}): (a_{ij}) \mapsto \prod\limits_{i,j} \phi_{ij}(a_{ij})$$
    for $\phi_{ij} \in C_c^\infty (F)$. For such $\Phi$ we compute (identifying $N \cong F$)
    
    \begin{equation*}
        \begin{split}
            \Phi_T(t) &= |t_1| \int_F \phi_{11}(t_1)\phi_{12}(t_1n)\phi_{21}(0)\phi_{22}(t_2)dn \\
            &= \phi_{11}(t_1)\phi_{22}(t_2)\phi_{21}(0) |t_1|\int_F \phi_{12}(t_1n) dn \\
            &= \phi_{11}(t_1)\phi_{22}(t_2)\phi_{21}(0) \int_F \phi_{12}(n) dn
        \end{split}
    \end{equation*}
    which uniquely extends to a function in $C_c^\infty(D)$. Surjectivity is now clear.
\end{proof}
\begin{rem}
    The content of the lemma is that the function $\Phi_T$ is smooth, for which the introduction of the factor of $|t_1|$ is necessary.
\end{rem}

\begin{proof}[Proof of Proposition \ref{prop:gl2factor}]
    We first establish the containment $\mathcal Z(\pi) \subset \mathcal Z(\chi_1)\mathcal Z(\chi_2)$. We must show that for any $\Phi \in C_c^\infty(A)$ and $f \in \mathcal C(\pi)$ we have $\zeta(\Phi,f,s+\frac{1}{2}) \in \mathcal Z(\chi_2)\mathcal Z(\chi_2)$. Since $\mathcal C(\pi)$ is spanned by the coefficients $\gamma_{\tau \otimes \theta}$, for $\theta \in V, \tau \in \check{V}$, we assume $f$ is of this form.

    Formally expanding, for any $\Phi \in C_c^\infty(A)$
    \begin{equation*}
        \begin{split}
            \zeta(\Phi,f,s+\frac{1}{2}) &= \int_G \Phi(g)f(g) |\det g|^{s+\frac{1}{2}} dg \\
            &= \int_G \int_K \Phi(g) \tau(k) \theta(kg)|\det g|^{s+\frac{1}{2}} dk dg \\
            &= \int_K \int_G \Phi(k^{-1}g) \tau(k)\theta(g) |\det g|^{s+\frac{1}{2}} dg dk \\
            &= \int_K \int_K \int_B \Phi(k^{-1}bk') \tau(k)\theta(bk') |\det b|^{s+\frac{1}{2}} db dk' dk.
        \end{split}
    \end{equation*}
    Smoothness of $\Phi$ and $\theta$ imply there is some open normal subgroup $K_1$ of $K$ for which $\Phi$ is left and right translation invariant, and $\theta$ and $\tau$ are right translation invariant. Let $\{k_i\}$ be a finite set of coset representatives of $K/K_1$, and let $\Phi^{ij}(x) = \Phi(k_i^{-1}xk_j)$. Then $\zeta(\Phi,f,s+\frac{1}{2})$ can be expressed as a finite linear combination over $\CC$ of terms of the form 
    $$\int_B \Phi^{ij}(b) \tau(k_i)\theta(bk_j) |\det b|^{s+\frac{1}{2}} db.$$
    Using the formula $\theta(bk_j) = \delta_B^{-1/2}(t)\chi(t)\theta(k_j)$, we can express the above as
    $$\theta(k_j)\tau(k_i) \int_T\int_N \Phi^{ij}(tn) \chi(t)\delta_B^{-1/2}(t) |\det b|^{s+\frac{1}{2}} dt dn.$$
    We have $|\det b|=|\det t| = |t_1| |t_2|$ and $\delta_B^{-1/2}(t) = |t_2/t_1|^{-1/2}$. Combining with the previous lemma, we deduce that $\zeta(\Phi,f,s+\frac{1}{2})$ can be expressed as a linear combination of terms of the form 
    $$\theta(k_j)\tau(k_i) \int_T \Phi_T^{ij}(t) \chi(t) |\det t|^s dt.$$
    If $\Phi$ is of the form $(\phi_{ij})$ for $\phi_{ij} \in C_c^\infty(F)$, then the above term is a scalar multiple of $\zeta(\phi_{11},\chi_1,s)\zeta(\phi_{22},\chi_2,s)$ so that $\zeta(\Phi,f,s+\frac{1}{2}) \in \mathcal Z(\chi_1)\mathcal Z(\chi_2)$.

    In the other direction, we wish to find $\Phi \in C_c^\infty(A)$ and $f \in \mathcal C(\pi)$ such that $\zeta(\Phi,f,s+\frac{1}{2})$ is a constant multiple of $L(\chi_1,s)L(\chi_2,s)$. We will find $f$ of the form $\gamma_{\tau \otimes \theta}$ and reverse the above calculation. Suppose we were in the situation where $\Phi$ is left and right invariant under $K$, and $\theta$ and $\tau$ are right invariant under $K$. Then the above computation shows that 
    $$\zeta(\Phi,f,s+\frac{1}{2}) = \mu(K)^2 \theta(1)\tau(1) \int_T \Phi_T(t)\chi(t)|\det t|^s dt.$$
    Therefore, if we could choose $\Phi$ left and right invariant under $K$ with $\Phi_T = \phi_1\otimes \phi_2$, where $\phi_i \in C_c^\infty(F)$ satisfy $\zeta(\phi_i,\chi_i,s)=L(\chi_i,s)$, and also choose $\theta \in \iota_B^G \chi$, $\tau \in \iota_B^G \check{\chi}$, with $\theta(1), \tau(1) \neq 0$ and $\theta$, $\tau$ right invariant under $K$, then we would be done. Unfortunately, if this was the case then $$\theta(bk) = \chi(b) \delta_B^{-1/2}(b) \theta(1)$$ for all $b \in B, k \in K$. But this is not well defined - we would require $1=\chi(b)\delta_B^{-1/2}(b) = \chi(b)$ for all $b \in B \cap K$. This only occurs when $\chi_1$ and $\chi_2$ are both unramified.

    Instead, let $K_1$ be any open normal subgroup of $K$ such that $\chi$ is trivial on $B \cap K_1$, and let $k_i$ be a finite set of coset representatives of $K/K_1$. There are then unique $\theta \in \iota_B^G \chi$ and $\tau \in \iota_B^G \check{\chi}$, each supported on $BK_1$, invariant under right translation by $K_1$, and with $\theta(1)=1=\tau(1)$. Let $f=\gamma_{\tau \otimes \theta}$.
    
    For $\Phi \in C_c^\infty(A)$ left and right invariant under $K_1$, our previous computation gives us
    $$\zeta(\Phi,f,s+\frac{1}{2}) = \mu(K_1)^2 \sum\limits_{i,j}  \int_T \theta(k_j)\tau(k_i)\Phi_T^{ij}(t)\chi(t)|\det t|^s dt.
    $$
    To control the terms over all $i,j$, we would like to choose $\Phi$ such that 
    $$\theta(k_j)\tau(k_i)\Phi_T^{ij}(t) = \Phi_T(t)$$
    for all $t \in T$ and all $i,j$ such that $k_i,k_j \in BK_1$. Then, by construction of $\theta$ and $\tau$, each term $\theta(k_j)\tau(k_i)\Phi_T^{ij}(t)$ is either 0 or $\Phi_T(t)$, and at least one is $\Phi_T(t)$, so that
    $$\zeta(\Phi,f,s+\frac{1}{2}) = c \int_T \Phi_T(t) \chi(t) |\det t|^s dt$$ for some $c>0$. If $k_j = b_jk \in BK_1$, then $\theta(k_j) = \chi(b_j)\delta_B^{-1/2}(b_j)\theta(1) = \chi(b_j)$ because $\delta_B=1$ on $B \cap K$. Similarly, if $k_i=b_ik \in BK_1$, then $\tau(k_i)=\chi(b_i)^{-1}$. The condition $$\theta(k_j)\tau(k_i)\Phi_T^{ij}(t) = \Phi_T(t),$$ together with the $K_1$ invariance of $\Phi$, reduces to the condition
    $$\chi(b_j)\chi(b_i)^{-1} \int_N \Phi(b_i^{-1}tnb_j) dn = \int_N \Phi(tn)dn$$ for all $b_i,b_j \in B \cap K_1$, as functions of $t \in T$.

    To summarise, we want to construct $\Phi \in C_c^\infty(A)$ with the following properties:
    \begin{itemize}
        \item The function $\Phi$ is invariant under left and right translation by $K_1$.
        \item For all $b_i,b_j \in B \cap K_1$ and $b \in B$ we have $$\chi(b_j)\chi(b_i)^{-1}\Phi(b_i^{-1}bb_j) = \Phi(b).$$
        \item For our chosen $\phi_1,\phi_2 \in C_c^\infty(F)$ satisfying $\zeta(\phi_i,\chi_i,s)=L(\chi_i,s)$, we have $\Phi_T = c \cdot \phi_1 \otimes \phi_2 \in C_c^\infty(D)$ for some $c \neq 0$.
    \end{itemize}
    Since we may have chosen any open $K_1 \lhd K$, provided $\chi$ is trivial on $B \cap K_1$, we are free to shrink $K_1$ and adjust $\tau$ and $\theta$ accordingly. We can remove the dependence on $K_1$ by strengthening the second condition above, and now ask for $\Phi \in C_c^\infty(A)$ with the following properties:
    \begin{itemize}
        \item For all $x,y \in B \cap K$ and $b \in B$ we have $$\chi(xy)\Phi(xby) = \Phi(b).$$
        \item For some $\phi_1,\phi_2 \in C_c^\infty(F)$ satisfying $\zeta(\phi_i,\chi_i,s)=L(\chi_i,s)$, we have $\Phi_T = c \cdot \phi_1 \otimes \phi_2 \in C_c^\infty(D)$ for some $c \neq 0$.
    \end{itemize}
    If we take $\Phi$ of the form $\Phi=(\phi_{ij})$, and set $\phi_{12}=\phi_{21}=\mathbbm{1}_K$, then the computation of Lemma \ref{lemma:phiT} shows that for $t= \begin{psmallmatrix}
        t_1&0\\0&t_2
    \end{psmallmatrix}$,
    $$\Phi_T(t) = \mu(\cO_F)\phi_{11}(t_1)\phi_{22}(t_2).$$
    Taking $\phi_{ii}=\phi_i$, it suffices to find for each $i=1,2$ some $\phi_i \in C_c^\infty(F)$ such that
    \begin{itemize}
        \item For all $x,y \in \cO_F^\times$ and $a \in F^\times$ we have $$\chi_i(xy)\phi_i(xay) = \phi_i(a).$$
        \item We have $\zeta(\phi_i,\chi_i,s)=c \cdot L(\chi_i,s)$ for some $c \neq 0$.
    \end{itemize}
    Here we divide into cases. If $\chi_i$ is unramified, then we may take $\phi_i = \mathbbm{1}_{\cO_F}$ by the proof of Proposition \ref{prop:gl1factor}. If $\chi_i$ is ramified, and the restriction to $U_F^n$ is trivial, then we take 
    $$ \phi_i = \sum\limits_{u \in \cO_F^\times/U_F^n} \chi_i(u)^{-1} \mathbbm{1}_{uU_F^n}.$$ One sees that this satisfies the first condition. For the second we have 
    $$\zeta(\phi_i,\chi_i,s) = \sum\limits_u \int_{U_F^n} \chi_i(u)^{-1}\chi_i(ux)|x|^s d^*x = \mu(\cO_F^\times)$$ which is a constant (and $L(\chi_i,s)=1$ in the ramified case). We have proven $\mathcal Z(\chi_1)\mathcal Z(\chi_2) \subset \mathcal Z(\pi)$.

\end{proof}

\begin{rem}
    The computations of Proposition \ref{prop:gl2factor} show that each $\zeta(\Phi,f,s)$ converges absolutely and uniformly in vertical strips in some right half plane, and admit analytic continuation to a rational function in $q^{-s}$.
\end{rem}


\begin{defn}
    Define the $L$-function attached to $\pi = \iota_B^G \chi$, where $\chi=\chi_1\otimes \chi_2$ is a character of $T$, to be $$L(\pi,s) = P_\pi(q^{-s})^{-1} = L(\chi_1,s)L(\chi_2,s).$$
\end{defn}

We now turn to the functional equations satisfied by the zeta functions $\zeta(\Phi,f,s)$. This involves understanding these zeta functions when we replace $\Phi$ with its Fourier transform, $\hat{\Phi}$. From the computations of Proposition \ref{prop:gl2factor}, this boils down to relating the map $\Phi \mapsto \Phi_T$ to the various Fourier transforms over $A$ and $D$.

\begin{lemma}
    For $\Phi \in C_c^\infty(A)$, we have $(\hat{\Phi})_T = \widehat{\Phi_T}$.
\end{lemma}
\begin{proof}
    \cite[Lemma 26.3]{BH1}.
\end{proof}

\begin{lemma}\label{hat}
    For $k_i,k_j \in K$ let $\Phi^{ij}$ denote the function $x \mapsto \Phi(k_i^{-1}xk_j)$ for $\Phi \in C_c^\infty(A)$. Then $\hat\Phi^{ji} = \widehat{\Phi^{ij}}$. 
\end{lemma}
\begin{proof}
    We calculate 
    $$\hat\Phi^{ji}(x) = \int_A \Phi(y)\psi_A(k_j^{-1}xk_iy)dy$$
    and 
    $$\widehat{\Phi^{ij}}(x) = \int_A\Phi(k_i^{-1}yk_j)\psi_A(xy)dy = \int_A \Phi(y)\psi_A(xk_iyk_j^{-1})dy.$$
    Since $\psi_A = \psi \circ \mathrm{tr}$ and $\mathrm{tr}$ is invariant under conjugation, we have $\psi_A(k_j^{-1}xk_iy) = \psi_A(xk_iyk_j^{-1})$.
\end{proof}

\begin{notn}
    If $f \in \mathcal C(\pi)$ is a matrix coefficient, denote by $\check{f} \in \mathcal C(\check\pi)$ the matrix coefficient
    $\check{f}(g) = f(g^{-1})$.
\end{notn}

\begin{prop}\label{prop:gl2gamma}
    Let $\pi = \iota_B^G \chi$ where $\chi=\chi_1\otimes \chi_2$ is a character of $T$. There is a unique $\gamma(\pi,s,\psi) \in \CC(q^{-s})$, depending on the additive character $\psi \neq 1$ of $F$ defining the Fourier transform, such that 
    $$\zeta(\hat{\Phi},\check{f},(1-s)+\frac{1}{2}) = \gamma(\pi,s,\psi) \zeta(\Phi,f,s+\frac{1}{2})$$
    for all $\Phi \in C_c^\infty(A)$ and $f \in \mathcal C(\pi)$. Moreover, 
    $$\gamma(\pi,s,\psi) = \gamma(\chi_1,s,\psi)\gamma(\chi_2,s,\psi).$$
\end{prop}
\begin{proof}
    Since the zeta function is linear in the matrix coefficients, as is the operation $f \mapsto \check{f}$, it suffices to prove such $\gamma$ exists for all $\Phi \in C_c^\infty(A)$ and $f$ of the form $\gamma_{\tau \otimes \theta}$ as in the proof of Proposition \ref{prop:gl2factor}. We calculated that 
    $$f(g) = \int_{B \backslash G} \tau(x)\theta(xg) d\dot{x} = \int_K \tau(k)\theta(kg)dk,$$ for some Haar measure $dk$ on $K$, so that by right invariance of $d\dot{x}$ we have 
    $$\check{f}(g) = \int_{B \backslash G}\tau(xg)\theta(x) d\dot{x} = \int_K \tau(kg)\theta(k)dk.$$ The same computation as the proof of Proposition \ref{prop:gl2factor} gives (for the same $K_1$ and coset representatives $k_i$ of $K/K_1$)
    \begin{equation*}
        \begin{split}
            \zeta(\hat{\Phi},\check{f},(1-s)+\frac{1}{2}) &= \mu(K_1)^2 \sum\limits_{i,j} \theta(k_j)\tau(k_i) \int_T (\hat\Phi^{ji})_T(t) \chi(t)^{-1} |\det t|^{1-s} dt \\
            &= \mu(K_1)^2 \sum\limits_{i,j} \theta(k_j)\tau(k_i) \int_T \widehat{(\Phi_T^{ij})}(t) \chi(t)^{-1} |\det t|^{1-s} dt
        \end{split}
    \end{equation*}
    by Lemma \ref{hat}. Therefore, it suffices to show that 
    $$\int_{F^\times}\int_{F^\times} \widehat{(\Phi^{ij}_T)}(t)\chi_1(t_1)^{-1}\chi_2(t_2)^{-1}|t_1t_2|^{1-s} dt_2dt_1 =  \gamma(\chi_1,s,\psi)\gamma(\chi_2,s,\psi) \int_{F^\times} \int_{F^\times}\Phi^{ij}_T(t)\chi_1(t_1)\chi_2(t_2) |t_1t_2|^s dt_2dt_1$$
    where $t = \begin{psmallmatrix}
        t_1 &0\\0&t_2
    \end{psmallmatrix} \in T$. By Theorem \ref{thm:gl1gamma}, this equality holds whenever we replace $\Phi^{ij}_T \in C_c^\infty(D)$ by a function of the form $\phi_{11}(t_1) \otimes \phi_{22}(t_2) \in C_c^\infty(D)$. But such functions span $C_c^\infty(D)$, so we are done by linearity of the integrals.

\end{proof}

\begin{defn}
    Define the Godement-Jacquet local constant $\epsilon(\pi,s,\psi)$ of $\pi = \iota_B^G \chi$ by 
    $$\epsilon(\pi,s,\psi) = \gamma(\pi,s,\psi) \frac{L(\pi,s)}{L(\check{\pi},1-s)}.$$
\end{defn}

\begin{cor}
    For $\pi= \iota_B^G \chi$ we have
    $$\epsilon(\pi,s,\psi) = \epsilon(\chi_1,s,\psi)\epsilon(\chi_2,s,\psi).$$
\end{cor}
\begin{proof}
    This follows from Proposition \ref{prop:gl2gamma} and Proposition \ref{prop:gl2factor}.
\end{proof}

For context, we state more general versions of these results that hold for any irreducible smooth representation $\pi$ of $G$.

\begin{thm}\label{BHThm1}
    Let $\pi$ be an irreducible smooth representation of $G$. There is a unique polynomial $P_\pi(X) \in \CC[X]$, satsifying $P_\pi(0)=1$, and 
    $$\mathcal Z(\pi) = P_\pi(q^{-s})^{-1} \CC[q^{-s},q^s].$$
\end{thm}
\begin{proof}
    \cite[Theorem 24.2.1]{BH1}.
\end{proof}

\begin{notn}
    Set $L(\pi,s) = P_\pi(q^{-s})^{-1}$.
\end{notn}

\begin{thm}\label{BHThm2}
    Let $\pi$ be an irreducible smooth representation of $G$. There is a unique rational function $\gamma(\pi,s,\psi) \in \CC(q^{-s})$ such that 
    $$\zeta(\hat\Phi,\check{f},(1-s)+\frac{1}{2}) = \gamma(\pi,s,\psi) \zeta(\Phi,f,s+\frac{1}{2})$$ for all $\Phi \in C_c^\infty(A)$ and $f \in \mathcal C(\pi)$.
\end{thm}
\begin{proof}
    \cite[Theorem 24.2.2]{BH1}.
\end{proof}

\begin{defn}
    Define the Godement-Jacquet local constant $\epsilon(\pi,s,\psi)$ of an irreducible smooth representation $\pi$ of $G$ by 
    $$\epsilon(\pi,s,\psi) = \gamma(\pi,s,\psi) \frac{L(\pi,s)}{L(\check{\pi},1-s)}.$$
\end{defn}

\begin{cor}
    The local constant satisfies the functional equation
    $$\epsilon(\pi,s,\psi)\epsilon(\check{\pi},1-s,\psi) = \omega_\pi(-1).$$
    The local constant is of the form $$\epsilon(\pi,s,\psi) = aq^{bs}$$ for some $a \in \CC^\times$, $b \in \ZZ$. 
\end{cor}
\begin{proof}
    The first statement comes from the Fourier inversion formula and Theorem \ref{BHThm2}. The $\omega_\pi(-1)$ term comes from the minus sign in $\hat{\hat{\Phi}}(x)=\Phi(-x)$ and the observation that for a matrix coefficient $f \in \mathcal C(\pi)$ we have $f(-g)=\omega_\pi(-1)f(g)$. The functional equation and Theorem \ref{BHThm1} implies that $\epsilon$ is a unit in $\CC[q^{-s},q^s]$, and the units are precisely the elements of the form $aq^{bs}$ for $b \in \ZZ$.
\end{proof}

The Propositions \ref{prop:gl2factor} and \ref{prop:gl2gamma} prove the Theorems \ref{BHThm1} and \ref{BHThm2} in the case that $\pi = \iota_B^G \chi$ and $\pi$ is irreducible. As in Theorem \ref{classify}, the representations $\pi = \iota_B^G \chi$ are typically irreducible - they are only reducible when $\chi = \phi \delta_B^{\pm 1/2}$ for some character $\phi$ of $F^\times$. In this case the composition factors are characters $\phi \circ \det$, and twists of Steinberg $\phi \mathrm{St}_G$. We state without proof the $L$-functions and local constants in the case that $\pi$ is one of these composition factors. For more detail see Sections 26.5 - 26.8 of \cite{BH1}. The results for all principal series representations are summarised in the following table:

\begin{figure}[h!]
    \centering
    \begin{tabular}{ |c|c|c| }
        \hline
        Principal series representation $\pi$ & $L(\pi,s)$ & $\epsilon(\pi,s,\psi)$ \\ \hline
        $\iota_B^G \chi$, $\chi=\chi_1\otimes \chi_2$, $\chi \neq \phi \delta_B^{\pm 1/2}$ & $L(\chi_1,s)L(\chi_2,s)$ & $\epsilon(\chi_1,s,\psi)\epsilon(\chi_2,s,\psi)$ \\ 
        $\phi \circ \det$, $\phi :F^\times \to \CC^\times$ ramified & 1 & $\epsilon(\phi,s-\frac{1}{2},\psi)\epsilon(\phi,s+\frac{1}{2},\psi)$ \\ 
        $\phi \mathrm{St}_G$, $\phi :F^\times \to \CC^\times$ ramified & 1 & $\epsilon(\phi,s-\frac{1}{2},\psi)\epsilon(\phi,s+\frac{1}{2},\psi)$ \\  
        $\phi \circ \det$, $\phi :F^\times \to \CC^\times$ unramified & $L(\phi,s-\frac{1}{2})L(\phi,s+\frac{1}{2})$ & $\epsilon(\phi,s-\frac{1}{2},\psi)\epsilon(\phi,s+\frac{1}{2},\psi)$ \\ 
        $\phi \mathrm{St}_G$, $\phi :F^\times \to \CC^\times$ unramified & $L(\phi,s+\frac{1}{2})$ & $-\epsilon(\phi,s,\psi)$ \\     
        \hline
       \end{tabular}
       \caption{$L$-functions and local constants of principal series representations of $G$}
\end{figure}

In particular, if $\pi$ is a composition factor of $\iota_B^G \chi$ then $L(\pi,s) = L(\chi_1,s)L(\chi_2,s)$, unless $\pi = \phi \mathrm{St}_G$ for some unramified character $\phi : F^\times \to \CC^\times$.

\subsection{Converse Theorem}

Attached to any principal series representation $\pi$ of $G$ we have an associated $L$-function $L(\pi,s)$ and local constant $\epsilon(\pi,s,\psi)$. In some sense this is enough information to distinguish them as irreducible smooth representations of $G$. More precisely, one can also define $L$-functions and local constants for the cuspidal representations of $G$, and then we have

\begin{thm}[Converse Theorem]\label{thm:converse}
    Let $\psi:F \to \CC^\times$ be an additive character with $\psi \neq 1$. Let $\pi_1,\pi_2$ be irreducible smooth representations of $G=\GL_2(F)$. Suppose that 
    $$L(\chi\pi_1,s)=L(\chi\pi_2,s) \text{   and   } \epsilon(\chi\pi_1,s,\psi) = \epsilon(\chi\pi_2,s,\psi),$$ for all characters $\chi :F^\times \to \CC^\times$. Then $\pi_1 \cong \pi_2$.
\end{thm}

Recall that the twist $\chi\pi$ denotes the representation $g \mapsto \chi(\det(g))\pi(g)$.

We take as fact the following result for cuspidal representations.

\begin{prop}\label{prop:cuspL}
    Let $\pi$ be an irreducible cuspidal representation of $G$. Then $L(\pi,s)=1$.
\end{prop}
\begin{proof}
    \cite[Corollary 24.5]{BH1}.
\end{proof}

Then we can distinguish between cuspidal and principal series representations as follows.

\begin{prop}\label{prop:twistL}
    An irreducible smooth  representation $\pi$ of $G$ is cuspidal if and only if $L(\phi\pi,s)=1$ for all characters $\phi$ of $F^\times$.
\end{prop}
\begin{proof}
    Since twisting preserves principal series representations, it preserves cuspidal representations. Proposition \ref{prop:cuspL} implies that if $\pi$ is cuspidal then $L(\phi\pi,s)=1$ for all $\phi$. In the other direction, suppose that $\pi$ is a composition factor of $\iota_B^G \chi$ for $\chi= \chi_1\otimes \chi_2$ a character of $T$. Taking $\phi=\chi_2^{-1}$, $\phi\pi$ is a composition factor of $\iota_B^G \phi\chi$ with $\phi\chi = \chi_1\chi_2^{-1} \otimes 1$. Now, except for the case $\phi\pi$ is a twist of Steinberg by an unramified character, we have $L(\phi\pi,s) = L(\chi_1\chi_2^{-1},s)L(1,s)$, and then $L(1,s)=(1-q^{-s})^{-1}$ is nontrivial. In the case it is a twist of Steinberg by an unramified character, the $L$-function is still nontrivial as seen in Table 1.
\end{proof}

\begin{proof}[Proof of Theorem \ref{thm:converse} for principal series representations]
    Twisting $\pi$, we may assume that $L(\pi,s) \neq 1$ as in the proof of Proposition \ref{prop:twistL}. Then $L(\pi,s)$ has degree 2 (as a rational function of $q^{-s}$). 

    Suppose $L(\pi,s)$ has degree 2. From Table 1, $\pi$ is either $\iota_B^G \chi$ for some $\chi=\chi_1 \otimes \chi_2$, with $\chi_1\chi_2^{-1} \neq |-|^{\pm 1}$ and $\chi_i$ unramified, or $\pi = \phi \circ \det$ for some unramified character $\phi :F^\times \to \CC^\times$. In either case, we have $L(\pi,s)=L(\chi_1,s)L(\chi_2,s)$ for unramified characters $\chi_i$ of $F^\times$, where $\pi = \phi \circ \det$ corresponds to $\chi_i = \phi |-|^{\pm 1}$. But since an unramified character $\chi$ is determined by $\chi(\varpi)$, it is determined by $L(\chi,s)$. Since $\iota_B^G( \chi_1 \otimes \chi_2) \cong \iota_B^G (\chi_2 \otimes \chi_1)$, it follows that $L(\pi,s)$ is enough to distinguish all principal series representations $\pi$ for which $L(\pi,s)$ has degree 2.

    Suppose $L(\pi,s)$ has degree 1, and is $L(\theta,s)$ for some unramified character $\theta$ of $F^\times$. From Table 1, $\pi$ is either $\iota_B^G (\theta' \otimes \theta)$ for some ramified character $\theta'$, or $\pi = \theta' \mathrm{St}_G$ for $\theta' = \theta|-|^{-1/2}$. In the latter case, $\theta'$ is unramified and so for any ramified character $\phi$ we have $L(\phi\pi,s)=1$. This distinguishes it from the former case where if we take $\phi = (\theta')^{-1}$, a ramified character, we have $\phi\pi = \iota_B^G (1 \otimes \phi\theta)$ so that $L(\phi\pi,s) \neq 1$. To recover $\theta'$ in this case, we can choose some ramified character $\phi$ such that $L(\phi\pi,s) \neq 1$, say $L(\phi\pi,s) = L(\theta'',s)$ fo a unique unramified character $\theta''$ of $F^\times$. Since $\phi\pi = \iota_B^G (\phi\theta' \otimes \phi\theta,s)$, and $\phi\theta$ is ramified, we have $L(\phi\pi,s) = L(\phi\theta',s)$. Therefore $\theta' = \phi^{-1}\theta''$.
\end{proof}

\begin{rem}
    The proof of Theorem \ref{thm:converse} for principal series representations shows that the isomorphism class of $\pi$ is determined solely by the $L$-functions $L(\phi\pi,s)$ as we range over all characters $\phi :F^\times \to \CC^\times$. For cuspidal representations, all $L$-functions are 1 and they are instead distinguished solely by the local constants.
\end{rem}



