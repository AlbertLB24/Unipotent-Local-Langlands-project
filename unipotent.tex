The local Langlands correspondence for $G=\GL_2(F)$, where $F$ is a nonarchimedean local field, establishes a bijection between irreducible smooth representations of $G$, and 2-dimensional semisimple Weil-Deligne representations. In this report we have seen this bijection for the subsets of principal series representations of $G$ and reducible Weil-Deligne representations. The local Langlands correspondence is not merely a bijection of sets, but is compatible with various properties defined on either side. In the previous section, we saw that the correspondence for principal series representations is compatible with $L$-functions and local constants. 

On the Weil-Deligne side, we can distinguish the 2-dimensional semisimple representations $\rho$ which are trivial on the inertia subgroup $\mathcal I_F \leq \mathcal W_F$. These are the unramified Weil-Deligne representations. Since $\mathcal W_F/\mathcal I_F \cong \ZZ$ is abelian, $\rho$ is necessarily reducible, and we can ask what the corresponding principal series representations of $G$ are. The representations, as we shall see, are the unipotent representations of $G$.

Unipotent representations were first introduced by Deligne-Lusztig, in their famous paper \cite{DL1}, in the context of representations of reductive groups over finite fields. In particular, one can talk about unipotent representations of $\GL_2(\mathbb F_q)$, for a finite field $\mathbb F_q$. This case is particularly simple; only the trivial representation and Steinberg are unipotent. 

For a nonarchimedean local field $F$, there is a natural filtration of open compact subgroups of $F^\times$: $\cO_F^\times \supset 1+\varpi\cO_F \supset 1+\varpi^2 \cO_F \supset \cdots$. The quotient $\cO_F^\times/1+\varpi\cO_F$ is isomorphic to $\mathbb F_q^\times = \GL_1(\FF_q)$. There are similar filtrations of $G=\GL_2(F)$ by open compact subgroups, the Moy-Prasad filtrations, for which the first quotients are isomorphic to the group $\mathbf{G}$ of points of a reductive group over a finite field. Under certain conditions, smooth representations of $G$ descend to representations of $\mathbf G$. This will allow us to define unipotent representations of $G$.

In this section we first introduce the Moy--Prasad filtrations, specialised to the case of $G=\GL_2(F)$, and explain more precisely how smooth representations of $G$ descend to representations of some finite reductive group over $\FF_q$. We then briefly summarise the results of Deligne-Lusztig theory in our context and describe the unipotent representations over a finite field. Finally, we define the unipotent representations of $G$ and show that they correspond to the unramified Weil-Deligne representations.

\subsection{Moy--Prasad Filtrations of \texorpdfstring{$\GL_2(F)$}{TEXT}}

Fix a nonarchimedean local field $F$, with ring of integers $\cO_F$, uniformiser $\varpi$ and residue field $\FF_q$. Let $\nu$ be the standard discrete valuation on $F$ defined by $\nu(\varpi)=1$. Let $G=\GL_2(F)$ and $T=\left\{\begin{psmallmatrix}
    a&0\\0&d
\end{psmallmatrix} : a,d \in F^\times\right\}$ the split maximal torus of diagonal matrices.

Historically, for any reductive group over $F$, Bruhat and Tits introduced an associated Bruhat--Tits building in \cite{BT1} and \cite{BT2}. Moy and Prasad then associated to each point of the building a filtration of the reductive group by compact open subgroups in \cite{MP1} and \cite{MP2}. We will determine these explicitly in the setting of $G=\GL_2(F)$.

In order to define the Moy--Prasad filtrations of $G$ we must first recall some terminology from the general theory of reductive groups. Our presentation follows \cite{Fin1}. See also Chapter 1 of \cite{GH1} for a brief summary of the relevant theory of reductive groups.

Let $\mathfrak g$ denote the Lie algebra of $G$. It is isomorphic to the space $\mathrm{M}_2(F)$ of $2\times 2$ matrices over $F$, with Lie bracket given by $[A,B]=AB-BA$. There is a natural action of $G$ on $\mathfrak g$ given by conjugation, this is the \textit{adjoint action} $\mathrm{Ad}$. By restricting the adjoint action to $T$, we obtain the following \textit{root space decomposition of $\mathfrak g$},
$$\fg = \bigoplus\limits_{\alpha \in X^*(T)} \fg_\alpha(T),$$
where $\fg_\alpha = \{X \in \fg : \mathrm{Ad}(t)X = \alpha(t)X \text{ for all } t \in T\}$. Here $X^*(T) = \Hom(T,\GG_m)$ denotes the \textit{characters} of $T$; the \textit{cocharacters} of $T$ are denoted $X_*(T) = \Hom(\GG_m,T)$. By identifying $\Hom(\GG_m, \GG_m) = \ZZ$ (where $n$ corresponds to $t \mapsto t^n$), we have a pairing
$$\langle, \rangle : \left(X^*(T) \times X_*(T)\right) \otimes_\ZZ \RR \to \RR.$$ The characters $\alpha \neq 1$ for which $\fg_\alpha \neq 0$ in the root space decomposition are called the \textit{roots} of $G$. Denote the set of all roots of $G$ by $\Phi(G,T)$. For any root $\alpha \in \Phi(G,T)$, there is a unique connected subgroup $U_\alpha$ of $G$, stable under conjugation by $T$, whose Lie algebra is $\fg_\alpha \subset \fg$. The $U_\alpha$ are the \textit{root groups} of $G$ and are isomorphic as algebraic groups to $\GG_a$. We fix isomorphisms $x_\alpha: \GG_a(F)=F \cong U_\alpha$ by fixing (suitable) $0 \neq X_\alpha \in \fg_\alpha$, and characterising $x_\alpha$ by the property that its derivative sends $1 \in F=\mathrm{Lie(\GG_a)}(F)$ to $X_\alpha$. A (suitable) choice of $X_\alpha$ for all roots $\alpha$ is a \textit{Chevalley system}. 

\begin{example}
    The roots of $G=\GL_2(F)$ are $\Phi(G,T) = \{\alpha,-\alpha\}$, where the character $\alpha \in X^*(T)$ is defined by $\alpha(t) = t_1t_2^{-1}$, for $t=\begin{psmallmatrix}
        t_1 & 0 \\0&t_2
    \end{psmallmatrix} \in T$. Here $-\alpha$ is the character $t \mapsto t_2t_1^{-1}$. That these are the roots of $G$ follows from the calculation 
    $$\mathrm{Ad}(t)(X_{ij}) = t_it_j^{-1},$$
    where $X_{ij} \in \fg = M_2(F)$ is the matrix with entry 1 in the $(i,j)$-coordinate, and 0 elsewhere.

    The root groups are then 
    $$U_\alpha = \left\{\begin{pmatrix}
        1&b\\0&1
    \end{pmatrix} : b \in F\right\} \text{ and } U_{-\alpha} = \left\{\begin{pmatrix}
        1&0\\c&1
    \end{pmatrix} : c \in F\right\}.$$
    A Chevalley system for $G$ is given by $\left\{\begin{psmallmatrix}
        0&1\\0&0
    \end{psmallmatrix} \in \fg_\alpha, \begin{psmallmatrix}
        0&0\\1&0
    \end{psmallmatrix} \in \fg_{-\alpha}\right\}$. This determines isomorphisms $x_\alpha: F \cong U_\alpha$ and $x_{-\alpha}: F \cong U_{-\alpha}$ by sending $a \in F$ to $\begin{psmallmatrix}
        1&a//0&1
    \end{psmallmatrix}$ and $\begin{psmallmatrix}
        1&0//a&1
    \end{psmallmatrix}$ respectively.
\end{example}

In our example with $G=\GL_2(F)$, we see that $G$ is generated by $T$, $U_\alpha$ and $U_{-\alpha}$. One way to produce filtrations of $G$ is to take filtrations of $T$, $U_{\alpha}$ and $U_{-\alpha}$ and take their product. The filtrations on each of these will use the standard filtrations on $F$ and $F^\times$. In order to produce different filtrations, on the root groups $U_{\pm \alpha} \cong F$, we weight the filtration in some way depending on the roots $\pm \alpha$. This asymmetric weighting will come from fixing a cocharacter $x_{BT} \in X_*(T) \otimes \RR$, and using the pairing $\langle, \rangle$ between characters (including the roots) and cocharacters of the torus $T$.

\begin{notn}
    For $T$ the split maximal torus of diagonal matrices in $G$, let 
    $$T_0 := \begin{pmatrix}
        \cO^\times & 0 \\ 0 & \cO^\times
    \end{pmatrix} \text{ and } T_r := \begin{pmatrix}
        1+\varpi^{\lceil r \rceil}\cO & 0 \\ 0 & 1+ \varpi^{\lceil r\rceil}\cO
    \end{pmatrix},$$
    for $r>0$ real, where $\lceil r \rceil$ denotes the smallest integer $n$ with $n\geq r$.
\end{notn}

\begin{notn}
    Fix $x_{BT} \in X_*(T) \otimes \RR$ and a Chevalley system of isomorphisms $x_\alpha: F \cong U_\alpha$ over all $\alpha \in \Phi(G,T)$. Define for any root $\alpha$, and $r \geq 0$ real,
    $$U_{\alpha,x,r} := x_\alpha\left(\varpi^{\lceil r - \langle \alpha, x_{BT}\rangle \rceil} \cO \right).$$
\end{notn}

\begin{defn}
    Fix again $x_{BT}$ and $\{x_\alpha : \alpha \in \Phi(G,T)\}$. Define the \textit{Moy--Prasad filtration} $\{G_{x,r}: r \geq 0\}$ of $G$ by 
    $$G_{x,r} = \langle T_r, U_{\alpha,x,r} : \alpha \in \Phi(G,T)\rangle,$$
    the subgroup of $G$ generated by the $r$-th filtered pieces of the torus and root groups. 
\end{defn}

\begin{notn}
    We write
    $$G_{x,r+} := \bigcup\limits_{s>r} G_{x,s}.$$
\end{notn}

\begin{rem}
    It is a fact that, if we were to replace $G$ by a reductive group over $F$ and make the same definitions, the subgroup $G_{x,r}$ is normal in $G_{x,0}$ for any $r$, and the quotient $G_{x,0}/G_{x,0+}$ is isomorphic to the points of some reductive group over the residue field $\FF_q$ of $F$.
\end{rem}

To collect the notation that we have fixed, we make the following definition, following \cite{Fin1}.

\begin{defn}
    A \textit{BT triple} $(T,X_\alpha,x_{BT})$ of $G=\GL_2(F)$ consists of the following data. We fix a split maximal torus $T$ of $G$ (which for us will always be the group of diagonal matrices). We fix a Chevalley system of (suitable) $X_\alpha \in \fg_\alpha$ for each root $\alpha$, defining isomorphisms $x_\alpha: F \cong U_\alpha$. We fix some cocharacter $x_{BT} \in X_*(T) \otimes \RR$.
\end{defn}

\begin{defn}
    A \textit{parahoric subgroup of} $G$ is a subgroup of the form $G_{x,0}$ for some BT triple $x$.
\end{defn}

\begin{rem}
    The (underlying set of the) \textit{Bruhat--Tits building} $\mathcal B(G,F)$ of $G$ can be interpreted as the set of BT triples modulo the equivalence relation defined by saying two triples are equivalent if they define the same Moy--Prasad filtration.
\end{rem}

\begin{example}
    We compute the parahoric subgroups of $G=\GL_2(F)$. Recall that $\Phi(G,T)=\{\alpha,-\alpha\}$ where $\alpha$ is the root $t \mapsto t_1t_2^{-1}$. Let $\check\alpha$ denote the coroot $\GG_m \to T$ defined by $t \mapsto \begin{psmallmatrix}
        t&0\\0&t^{-1}
    \end{psmallmatrix}$ so that $\langle \alpha,\check \alpha\rangle=2$. We will consider BT triples of the form 
    $$x(c) = \left(T, \left\{\begin{pmatrix}
        0&1\\0&0
    \end{pmatrix}, \begin{pmatrix}
        0&0\\1&0
    \end{pmatrix}\right\}, x_{BT} = \frac{1}{2}c\check\alpha\right),$$ where $c \in \RR$. The filtered pieces of the root groups are then 
    $$U_{\alpha,x(c),r} = \begin{pmatrix}
        1& \varpi^{\lceil r-c\rceil}\cO \\ 0 & 1
    \end{pmatrix} \text{ and } U_{-\alpha,x(c),r} = \begin{pmatrix}
        1& 0 \\ \varpi^{\lceil r+c\rceil}\cO & 1
    \end{pmatrix}.$$
    The parahoric subgroups are then 
    $$
    G_{x(c),0} = 
    \begin{cases}
        \left\{
        g \in    
        \begin{pmatrix}
            \cO & \varpi^{-c}\cO \\ \varpi^{c} \cO & \cO
        \end{pmatrix} : \det g \in \cO^\times
        \right\} & c \in \ZZ, \\
        \begin{pmatrix}
            \cO^\times & \varpi^{\lceil -c\rceil}\cO \\ \varpi^{\lceil c\rceil} \cO & \cO^\times
        \end{pmatrix}  & c \not\in \ZZ.
    \end{cases}
    $$
    The conjugacy classes of parahoric subgroups of $G$ are represented by $\GL_2(\cO)$ and $\begin{psmallmatrix}
        \cO^\times & \cO \\ \varpi\cO & \cO^\times
    \end{psmallmatrix}$. The latter subgroup is the \textit{Iwahori subgroup of} $G$. We also compute 
    $$
    G_{x(c),0+} = 
    \begin{cases}  
        \begin{pmatrix}
            1+\varpi\cO & \varpi^{-c+1}\cO \\ \varpi^{c+1} \cO & 1+\varpi\cO
        \end{pmatrix} & c \in \ZZ, \\
        \begin{pmatrix}
            1+\varpi\cO & \varpi^{\lceil -c\rceil}\cO \\ \varpi^{\lceil c\rceil} \cO & 1+\varpi\cO
        \end{pmatrix}  & c \not\in \ZZ.
    \end{cases}
    $$
    Finally, the quotients are 
    $$
    G_{x(c),0}/G_{x(c),0+} \cong 
    \begin{cases}  
        \GL_2(\FF_q) & c \in \ZZ, \\
        T(\FF_q) := \begin{pmatrix}
            \FF_q^\times & 0 \\ 0 & \FF_q^\times
        \end{pmatrix}  & c \not\in \ZZ.
    \end{cases}
    $$
    The isomorphisms are obtained by viewing $G_{x(c),0+}$ as the kernels of reduction modulo $\begin{psmallmatrix}
        \varpi\cO & \varpi^{-c+1}\cO \\ \varpi^{c+1}\cO & \varpi \cO
    \end{psmallmatrix}$ and $\begin{psmallmatrix}
        \varpi\cO & \varpi^{\lceil -c\rceil}\cO \\ \varpi^{\lceil c\rceil} \cO & \varpi\cO
    \end{psmallmatrix}$, where one checks that these are closed under addition and multiplication.   
\end{example}

Finally, we record under what conditions smooth representations of $G$ descend to representations of these finite groups $\GL_2(\FF_q)$ and $T(\FF_q)$.

\begin{lemma}
    Fix $c \in \RR$ and let $K=G_{x(c),0}$ and $K^+ =G_{x(c),0+}$. Let $(\pi,V)$ be a smooth representation of $G$. Then the representation $\pi\mid_{K}$ on $V^{K^+}$ descends to a representation of $K/K^+$ on $V^{K^+}$. In particular, the representation of $K/K^+$ is nonzero if and only if $V$ has a fixed vector under $K^+$.
\end{lemma}

\begin{notn}
    We let $(\bar{\pi},V^{K^+})$ denote the representation of $K/K^+$ acting on $V^{K+}$.
\end{notn}


\subsection{Unipotent Representations over a Finite Field}

