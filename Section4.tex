In this section, we define the Hecke algebra $\mathcal H(G)$ associated to a locally profinite (unimodular) group $G$ and explain how to switch between smooth representations of $G$ and smooth modules of $\mathcal H(G)$. Under certain conditions on $G$ we consider a particular subalgebra of $\mathcal H(G)$; the unramified Hecke algebra $\mathcal H(G,K)$, which turns out to be commutative by the Satake isomorphism. We use as reference Chapter 4 of \cite{BH1} and Chapter 5 of \cite{GH1}.

If $G$ is a finite group, representations of $G$ are the same as $\CC[G]$-modules. We want to extend this notion to smooth representations of locally profinite groups, where we need to correctly interpret the group algebra.

Let $G$ be a locally profinite unimodular group and $K$ an open compact subgroup of $G$. Let $C_c^\infty(G)$ be the space of locally constant compactly supported functions $G \to \CC$ and $C_c^\infty(G//K)$ the $K$ bi-invariant subspace.

These are naturally $\CC$-vector spaces and we endow them with an associative (not necessarily unital) ring structure coming from convolution
$$f * h(g) := \int\limits_G f(x)h(x^{-1}g)dx$$
where we fix a Haar measure $\mu = dx$ on $G$.

When $G$ is discrete this is the usual product on $\CC[G]$. 

\begin{defn}
Let $\mathcal H(G)$ and $\mathcal H(G,K)$ denote $C_c^\infty(G)$ and $C_c^\infty(G//K)$ with the algebra structure specified above. We call $\mathcal H(G)$ the Hecke algebra of $G$.
\end{defn}

We study these algebras in more detail:

The element $e_K = \mu(K)^{-1} \mathbbm{1}_K \in \mathcal H(G)$ is idempotent and we have the property that 
$$e_K * f = f \Leftrightarrow f \text{ is $K$ left invariant }.$$
Thus $\mathcal H(G,K) = e_K * \mathcal H(G) * e_K$, and this subalgebra now has a unit $e_K$. The compactness of $K$ ensures $e_K \in C_c^\infty(G)$.

By Lemma 5.2.1 of \cite{GH1}, $\mathcal H(G)$ is spanned by indicator functions of $K'$-double cosets, where $K'$ ranges over all compact open subgroups of $G$. If we normalise these indicator functions by defining $$[K\alpha K] = \mu(K)^{-1} \mathbbm{1}_{K\alpha K},$$ then we have the formula
$$[K\alpha K] * [K\beta K] = \sum\limits_{i,j}[K\alpha_i \beta_j K]$$ where $K\alpha K = \sqcup K\alpha_i$ and $K\beta K = \sqcup \beta_j K$. This determines multiplication in the Hecke algebra.

\subsection{Smooth Representations and \texorpdfstring{$\mathcal H(G)$}{TEXT}-modules}

We now explain how the concepts of smooth representations of $G$ and smooth modules over $\mathcal H(G)$ are interchangeable. To define these smooth modules, we note that the Hecke algebra $\mathcal H(G)$ does not in general have a unit. Consequently, not every $\mathcal H(G)$-module $M$ satisfies $\mathcal H(G) M = M$. 

\begin{defn}
    We say that a $\mathcal H(G)$-module $M$ is smooth if $\mathcal H(G) M = M$. 
\end{defn}

\begin{defn}
From a representation $V$ of $G$ we define the action of $\mathcal H(G)$ on $V$ via 
$$f \cdot v := \int_G f(g) g \cdot v dg.$$
\end{defn}
This can be viewed as a weighted average of the action of $G$ on $v$, where the weighting is described by $f \in C_c^\infty(G)$. The integral defines an element of $V$ when $f \in C_c^\infty(G)$ as the integral reduces to a finite sum.

\begin{lemma}\label{project}
    Under this action, $e_K \in \mathcal H(G)$, for $K\leq G$ a compact open, is the projection $V \to V^K$ onto the $K$-invariants of $V$. In particular, $e_K$ is an idempotent element of $\mathcal H(G)$, it is the identity element of $\mathcal H(G,K)$, and $V^K$ is a $\mathcal H(G,K)$-module.
\end{lemma}
\begin{proof}
    Let $V(K) \leq V$ be the subspace spanned by vectors of the form $k \cdot v-v$. Since $e_K$ is invariant under $K$-translation, it is zero on $V(K)$. The normalisation of $e_K$ is such that $e_K$ is the identity on $V^K$, and this implies the result.
\end{proof}


\begin{prop}
    A representation $V$ of $G$ is smooth if and only if it is a smooth $\mathcal H(G)$-module.
\end{prop}
\begin{proof}
If $V$ is a smooth representation, then any $v \in V$ is $K$-invariant for some compact open $K$, and so $v=e_K \cdot v$. This implies that $V$ is a smooth $\mathcal H(G)$-module. Conversely, $\mathcal H(G)$ is the union of $e_K* \mathcal H(G) *e_K = \mathcal H(G,K)$ over all compact open $K$, and so if $V$ is a smooth $\mathcal H(G)$-module then any $v \in V$ is of the form $e_K * f * e_K \cdot v'$ for some $K, f, v'$. Then $e_K \cdot v=v$ and so $v \in V^K$.
\end{proof}

So we can view smooth representations of $G$ as smooth $\mathcal H(G)$-module. In the other direction, given $M$ a smooth $\mathcal H(G)$-module, we have $$\mathcal H(G) \otimes_{\mathcal H(G)} M = M$$ by smoothness. We can then view $M$ as a smooth $G$ representation by letting $G$ act on the first factor by left translation. Concretely, if $m \in M$ there exists $K$ such that $m \in \mathcal H(G,K)M = e_K*\mathcal H(G)*e_K M$. Then $m = e_K \cdot m'$ for some $m' \in M$ and therefore $e_K \cdot m = m$ because $e_K$ is idempotent in $\mathcal H(G)$. Then define $$g \cdot m := \mu(K)^{-1} \mathbbm{1}_{gK} \cdot m,$$ where this is independent of $K$ due to the normalisation factor $\mu(K)^{-1}$.

\subsection{Information in the \texorpdfstring{$K$}{TEXT}-invariants \texorpdfstring{$V^K$}{TEXT}}
For a smooth representation $V$ of $G$ it is often easier to study the $K$-invariants $V^K$ for compact open subgroups $K$ of $G$.



\begin{lemma}\label{K inv}
    A smooth representation $V$ of $G$ is irreducible if and only if each $V^K$ is either 0 or a simple $\mathcal H(G,K)$-module for all compact open $K \leq G$.
\end{lemma}
\begin{proof}
    Suppose $V$ is irreducible. If we had $0 \neq M \subset V^K$ a $\mathcal H(G,K)$-module, then $0 \neq \mathcal H(G) M \subset V$ as smooth $\mathcal H(G)$-modules. Since smooth $\mathcal H(G)$-modules are the same as smooth $G$-representations, and $V$ is irreducible, we deduce $\mathcal H(G)M = V$. So then $$V^K = e_K V = e_K * \mathcal H(G)M = e_K * \mathcal H(G) *e_K M = \mathcal H(G,K)M=M$$ which implies the result.

    If $V$ is not irreducible, and $W \neq 0$ is a proper subrepresentation, pick $v \in V-W$. By smoothness, there exists $K$ such that $v \in V^K$, and also $W^K \neq 0$, but then $v \not\in W^K$ so that $V^K$ is not 0 or simple.
\end{proof}

The next result tells us that for any $K$, any smooth representation $V$ of $G$ is determined by $V^K$ with its structure as a $\mathcal H(G,K)$-module, provided $V^K \neq 0$.

\begin{prop}\label{K bij}
    The map $V \mapsto V^K$ induces a bijection between
    \begin{itemize}
        \item equivalence classes of irreducible smooth representations $V$ of $G$ with $V^K \neq 0$;
        \item isomorphism classes of simple (by definition nonzero) $\mathcal H(G,K)$-modules.
    \end{itemize}
\end{prop}
\begin{proof}
    Proposition 4.3 of \cite{BH1}.
\end{proof}

\subsection{Unramified Representations of \texorpdfstring{$G$}{TEXT}}


It is interesting to study the smooth representations $V$ with $V^K \neq 0$ as above. For example, in an automorphic representation, Flath's theorem (\cite{GH1} Section 5.7) allows us to decompose into local factors, and furthermore tells us that almost all such local representations are unramified in the following sense:

\begin{defn}
    We consider the case $G=\GL_2(F)$. We say that an irreducible smooth representation $V$ of $G$ is unramified if $V^K \neq 0$ for $K= \GL_2(\cO_F)$. See Section 5.5 of \cite{GH1} for a more general definition for reductive groups.
\end{defn}

For the remainder of this subsection we work in the context of $G=\GL_2(F)$ and $K = \GL_2(\cO_F)$ for simplicity. The results generalise to reductive groups $G$ as in Sections 5.5 and 7.1 of \cite{GH1}.
\begin{defn}
    For $K$ as above, $\mathcal H(G,K)$ is called the unramified Hecke algebra of $G$.
\end{defn}

An application of the Satake isomorphism (\cite{GH1} Theorem 5.5.1) tells us that in this unramified case, the unramified Hecke algebra $\mathcal H(G,K)$ is commutative. It follows that if $V$ is $K$-unramified (in particular irreducible) then $V^K$ is 1-dimensional by Lemma \ref{K inv}. Thus $\mathcal H(G,K)$ acts on $V^K$ via scaling, called the Hecke character of $V$.

\begin{defn}
    The Hecke character (with respect to $K$) of a smooth representation $(\pi,V)$ of $G$ is the $\CC$-linear map
    \begin{equation*}
        \begin{split}
            \mathcal H(G,K) &\to \CC \\
            f &\mapsto \pi(f)
        \end{split}
    \end{equation*}
    defined by $f \cdot v =: \pi(f) v $ for any $v \in V^K$.
\end{defn}

We give an alternative proof of Proposition \ref{K bij}.

\begin{prop}
    Let $K \leq G$ be a compact open subgroup. If $V_1,V_2$ are irreducible smooth representations of $G$ such that $V_1^K$ and $V_2^K$ are nonzero and isomorphic as $\mathcal H(G,K)$-modules, then $V_1 \cong V_2$. In particular, unramified representations are determined by their Hecke characters.
\end{prop}
\begin{proof}
    This is Proposition 7.1.1 of \cite{GH1}. The idea is to extend an isomorphism $$I: V_1^K \to V_2^K$$ to a $G$-intertwining map $V_1\to V_2$ of $\mathcal H(G)$-modules. By irreducibility, $V_i = \mathcal H(G)V_i^K$. Take an element $\pi_1(f) \cdot \phi \in V_1$, with $f \in \mathcal H(G), \phi \in V_1^K$, then the obvious choice is to map this to $\pi_2(f) \cdot I(\phi)$. Provided this is well defined, this is a nonzero homomorphism of $\mathcal H(G)$-modules, so irreducibility of $V_1,V_2$ implies this is an isomorphism $V_1 \cong V_2$.

    We now check that this is well defined. Suppose $\pi_1(f_1)\phi_1 = \pi_1(f_2)\phi_2$. Since $V_1^K$ is irreducible, there exists $f_3 \in \mathcal H(G,K)$ such that $\pi_1(f_3)\phi_1 = \phi_2$, and so also $\pi_2(f_3)I(\phi_1) = I(\phi_2)$ since $I$ is $\mathcal H(G,K)$-intertwining. Then $\pi_1(f_1) \phi_1 = \pi_1(f_2 * f_3) \phi_1$. Thus it suffices to show that if $\pi_1(f)\phi =0$ then $\pi_2(f)I(\phi)=0$. We exploit the $\mathcal H(G,K)$-intertwining of $I$ (for the second implication below). For all $f_1 \in \mathcal H(G)$ we have:
    $$\pi_1(f)\phi = 0 \Rightarrow \pi_1(e_K*f_1*f*e_K)\phi = 0 \Rightarrow \pi_2(e_K*f_1*f*e_K)I(\phi)=\pi_2(e_K*f_1*f)I(\phi)=0.$$
    By Lemma \ref{project}, $e_K$ acts on $V_2$ by projection to $V_2^K$. If $\pi_2(f)I(\phi) \neq 0$, then $\pi_2(f_1)\pi_2(f)I(\phi)$, over all $f_1 \in \mathcal H(G)$, generates $V_2$ by irreducibility. The image under $\pi_2(e_K)$ is the exactly $V^K$, which is nonzero, contradicting the implication above.
\end{proof}

\subsection{Example Computation of Hecke Operators for \texorpdfstring{$\GL_2(F)$}{TEXT}}

[I haven't checked this subsection. Some parts might be more suitable for a section on modular forms. The computation of the modular character of $B$ will be needed in the main text. And the last proposition naturally goes with the unramified representations above.]

Let $G=\mathrm{GL}_2(F)$ and $K=\mathrm{GL}_2(\mathcal O)$ for $F$ a nonarchimedean local field with uniformiser $\varpi$. We have the Cartan decomposition $$G = \bigsqcup\limits_{a \geq b \in \ZZ} K \begin{pmatrix} \varpi^a & \\ & \varpi^b \end{pmatrix}K.$$ Let $S=K \begin{psmallmatrix} \varpi & \\ & \varpi \end{psmallmatrix}K$ and $T=K \begin{psmallmatrix} \varpi & \\ & 1 \end{psmallmatrix}K$, viewed as elements of $\mathcal H(G,K)$ via their indicator functions.

\begin{lemma}
    The unramified Hecke algebra is $\mathcal H(G,K) \cong \CC[S,S^{-1},T]$. In particular, this is commutative.
\end{lemma}
\begin{proof}
    This is some induction argument using the formula for convolutions of these indicator functions.
\end{proof}

\begin{rem}
    This fits into a general phenomenon - if $G$ is unramified and $K$ is a hyperspecial subgroup then the Satake isomorphism implies that the unramified Hecke algebra $\mathcal H(G,K)$ is always commutative.
\end{rem}

Later we will be interested in principal series representations, which are representations of $G$ coming from parabolic induction. So let $\chi = \begin{psmallmatrix}
    \chi_1 & \\ & \chi_2
\end{psmallmatrix}$ be a character of the torus $T$, and consider the normalised induced representation $$I(\chi) = \mathrm{Ind}_B^G \left( \chi \otimes \delta_B^{-1/2}\right)$$
where we recall that this is the space of functions $G \to \CC$ with $f(bg) = \chi(b)\delta_B^{-1/2}(b) f(g)$ for $b \in B$.

We briefly discuss the module character $\delta_B$. Although $G$ is unimodular (see Bushnell-Henniart Section 7.5), the Borel subgroup is not. We have $B=NT$ with $N\cong F$, $T \cong F^\times \times F^\times$ and $N$ normal in $B$. The failure of $B$ to be unimodular is a consequence of $T$ and $N$ not commuting. We can then define a linear function $I$ on $C_c^\infty(B) = C_c^\infty(T) \otimes C_c^\infty(N)$ by
$$I(\Phi) = \int_T\int_N \Phi(tn) dt dn$$ using Haar measures on $T$ and $N$.

\begin{prop}
    $I$ is a left Haar integral on $B$.
\end{prop}
\begin{proof}
    Let $b=sm \in TN$. By left invariance of $dt$ we have
    $$\int_T\int_N \Phi(smtn)dtdn = \int_T\int_N \Phi(mtn)dtdn = \int_T\int_N \Phi(tt^{-1}mtn)dtdn.$$
    Since we integrate $N$ first, we are integrating over fixed values of $t$ so that $t^{-1}mt \in N$ is just constant, so left invariance of $dn$ let's us pull out the $t^{-1}mt$ factor.
\end{proof}

\begin{prop}\label{prop:modularchar}
    The modular character $\delta_B$ of the group $B$ is
    $$\delta_B : tn \mapsto |t_2/t_1|, \quad n \in N, t = \begin{psmallmatrix}
        t_1 & 0 \\ 0 & t_2
    \end{psmallmatrix} \in T$$
\end{prop}
\begin{proof}
    By a similar argument as above, we have
    $$\int_T\int_N \Phi(tnsm) dtdn = \int_T\int_N \Phi(tss^{-1}nsm)dtdn = \int_T\int_N \Phi(ts^{-1}ns) dt dn.$$ Identifying $N \cong F$ this is
    $$\int_T\int_N \Phi(t \cdot \begin{psmallmatrix}
        1 & s_1^{-1}xs_2 \\0&1 
    \end{psmallmatrix}) d\mu_F(x) = |s_1/s_2|\int_T\int_N\Phi(tn)dtdn$$
    so by definition of the module character we have $\delta_B(sm) = |s_2/s_1|$.
\end{proof}

Going back to our principal series representation, the following proposition computes the action of the unramified Hecke algebra on the $K$-invariant subspace:

\begin{prop}
    Let $\chi:T \to \CC^\times$ be an unramified character of the torus (meaning trivial on $\begin{psmallmatrix}\mathcal O^\times & \\ & \mathcal O^\times \end{psmallmatrix}$) and consider the normalised parabolic induction $$I(\chi) = \mathrm{Ind}_B^G(\chi \otimes \delta_B^{-1/2}).$$ For $K=\mathrm{GL}_2(\mathcal O)$ as usual, the space $I(\chi)^K$ is 1-dimensional. As a $\mathcal H(G,K)$-module this is determined by the actions of $S$ and $T$. Since $\chi$ is unramified we know $\chi_1(z)=\alpha^{v_F(z)}$ and $\chi_2(z)=\beta^{v_F(z)}$ for some $\alpha,\beta \in \CC^\times$. Then $S$ acts on $I(\chi)^K$ by scaling by $\alpha\beta$ and $T$ acts by scaling by $q^{1/2}(\alpha +\beta)$.  
\end{prop}
\begin{proof}
    We have the Iwasawa decomposition $G=BK$ so that the functions $f \in I(\chi)^K$ satisfy
    $$f(bk)=f(b)=\chi(b)\delta_B^{-1/2}(b) \cdot f(1)$$ with $f(1) \in \CC$, so the space is 1-dimensional spanned by $\hat{f}(bk) = \chi(b)\delta_B^{-1/2}(b)$.

    The action of $S$ is given by:
    \begin{equation*}
        \begin{split}
            S\cdot f &= \mu(K)^{-1}\int_G \mathbbm{1}_{K\begin{psmallmatrix}\varpi & \\ & \varpi\end{psmallmatrix}K}(g) g \cdot f dg \\
            &= \mu(K)^{-1}\int_K \begin{psmallmatrix}\varpi & \\ & \varpi\end{psmallmatrix}k \cdot f dk \\
            &= \begin{psmallmatrix}\varpi & \\ & \varpi\end{psmallmatrix}\cdot f \\
            &= \chi\left(\begin{psmallmatrix}\varpi & \\ & \varpi\end{psmallmatrix}\right) \delta_B^{-1/2}\left(\begin{psmallmatrix}\varpi & \\ & \varpi\end{psmallmatrix}\right) f \\
            &= \alpha\beta f
        \end{split}
    \end{equation*}
    because $K\begin{psmallmatrix}\varpi & \\ & \varpi\end{psmallmatrix}K = \begin{psmallmatrix}\varpi & \\ & \varpi\end{psmallmatrix}K$.

    And for $T$ we pick coset representatives for $K\begin{psmallmatrix}\varpi & \\ & 1\end{psmallmatrix}K/K$ given by $\begin{psmallmatrix}\varpi &a \\ & 1\end{psmallmatrix}$ and $\begin{psmallmatrix}1 & \\ & \varpi\end{psmallmatrix}$, where $a$ ranges over representatives of $\mathcal O/\varpi$. Writing down the integral for the action of $T$ we decompose this into a sum over these left cosets and we deduce that $T$ acts by
    $$\chi_2(\varpi)|\varpi|^{-1/2}f + \sum\limits_{a \in \mathcal O/\varpi} \chi_1(\varpi)|\varpi|^{1/2}f = q^{1/2}(\alpha+\beta)$$
    since, for example, $\chi(\begin{psmallmatrix}\varpi & a\\ & 1\end{psmallmatrix})=\chi_1(\varpi)=\alpha$ and $\delta_B^{-1/2}(\begin{psmallmatrix}\varpi & a\\ & 1\end{psmallmatrix}) = |\varpi|^{1/2}$.
\end{proof}
\begin{rem}
    If we know the action of $S,T$ on $I(\chi)^K$ for some unramified character $\chi$ of the torus $T$, then we can recover $\alpha,\beta \in \CC^\times$ from the roots of the Satake polynomial $X^2-q^{-1/2}TX+S \in \mathcal H(G,K)[X]$.
\end{rem}

\newpage