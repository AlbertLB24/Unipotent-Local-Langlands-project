In the previous section, we classified the prinicipal series representations of $G=\GL_2(F)$ over a non-Archimedean local field $F$. For characters $\chi$ of $\GL_1(F)$, Tate's thesis \cite{Tate} associates a space $\mathcal Z(\chi)$ of zeta functions in a complex variable $s$. This space will, in a sense to be made precise, be generated by a single element, the $L$-function $L(\chi,s)$. The zeta functions will also satisfy a functional equation depending on the `local constant' $\epsilon(\chi,s,\psi)$. Here $\psi :F \to \CC^\times$ is a character whose purpose is to fix a form of Fourier transform on $F$. These definitions and results in Tate's thesis are intended to mimic the classical theory of $L$-functions due largely to Hecke, which encompass the Riemann zeta function. The $L$-function and local constant of a character $\chi:F^\times \to \CC^\times$ will turn out to carry the essential information of $\chi$. In the classical setting see, for example, the converse theorem of Weil reproduced in \cite[Theorem 1.5.1]{Bump}.

In the setting of irreducible smooth representations $\pi$ of $G$, in particular the principal series representations $\pi$, we want to again associate a space $\mathcal Z(\pi)$ of zeta functions, an $L$-function $L(\pi,s)$ and a local constant $\epsilon(\pi,s,\psi)$ determining a functional equation. 

We begin this section with a brief review of harmonic and Fourier analysis and the role it plays in representation theory. For more details, see \cite[Chapter 3.1]{Bump}. Following the presentation in \cite{GH1}, we define the $L$-functions and local constants of characters of $F^\times$. We explain how this theory generalises to irreducible smooth representations $\pi$ of $G$, culminating in the Theorems \ref{BHThm1} and \ref{BHThm2}, which determine the functional equations satisfied by the zeta functions associated to $\pi$. Propositions \ref{prop:gl2factor} and \ref{prop:gl2gamma} prove these in the case where $\pi = \iota_B^G \chi$ is a principal series representation. The case where $\iota_B^G \chi$ is reducible, so that $\pi$ is only a subquotient, requires slightly more work. The results are summarised in Table 1. Finally, we prove a converse theorem for principal series representations of $G$.

\subsection{Review of Harmonic Analysis}

Take as motivation the representation theory of a finite group $H$. Every irreducible representation of $H$ appears as a direct summand of the regular representation $\CC[H]$, with some multiplicity. For a locally compact topological group $\mathbb G$ with Haar measure $dg$, the correct generalisation of $\CC[H]$ is the space $L^2(\mathbb G)$ of measurable functions $f:\mathbb G \to \CC$ for which 
$$\int_{\mathbb G} |f(g)|^2 dg < \infty.$$
The action of $\mathbb G$ is by right translation. If $\mathbb G$ is additionally abelian, the group $\hat{\mathbb G}$ of (unitary) characters of $\mathbb G$ is also a locally compact abelian group, the Pontryagin dual of $\mathbb G$. 

\begin{example}
    The Pontryagin duals of $\mathbb G = \RR, \ZZ, \RR/\ZZ$ are $\RR, \RR/\ZZ, \ZZ$ respectively. The characters of $\RR$ are of the form $x \mapsto e^{-2\pi i xy}$ for $y \in \RR$. The characters of $\ZZ$ are of the form $n \mapsto e^{-2\pi i nx}$ for $x \in \RR/\ZZ \cong S^1$. The characters of $\RR/\ZZ$ are of the form $x \mapsto e^{-2\pi i nx}$ for $n \in \ZZ$. In particular, $\RR$ is self-dual.
\end{example}

On a suitable dense subset of $L^2(\mathbb G)$ (the Schwartz space), one can define the Fourier transform $\hat{f} \in L^2(\hat{\mathbb G})$ of $f$ by
$$\hat{f}(\chi) = \int_{\mathbb G} f(g)\chi(g) dg.$$
The Fourier transform uniquely extends to a map $L^2(\mathbb G) \to L^2(\hat{\mathbb G})$. For suitable choices of Haar measures there is then a Fourier inversion formula 
$$\hat{\hat{f}}(g)=f(-g),$$ so that the above map is a bijection.

\begin{example}
    For $\mathbb G=\RR$, the Fourier transform of $f$ is 
    $$\hat{f}(x) = \int_{\RR} f(y)e^{-2\pi i xy} dy$$
    which is the classical Fourier transform. Identifying $\hat{\RR} = \RR$, the Fourier transform gives an invertible map $L^2(\RR) \to L^2(\RR)$, so that any element of $L^2(\RR)$ can be expressed as an integral of elements of $\hat{\RR}$. 

    Inside the representation $L^2(\RR)$ of $\RR$ we therefore see this `continuous spectrum' of the irreducible unitary representations (characters) of $\RR$, parametrised by $\RR$. Note, however, that each such character can not be realised as a subrepresentation of $L^2(\RR)$; for $y \in \RR$ the character $x \mapsto e^{-2\pi i xy}$ is realised as the Fourier transform of a function on $\RR$ supported only at $y$, but such a function is not in $L^2(\RR)$.
\end{example}

\begin{example}
    For $\mathbb G = \ZZ$, the Fourier transform of $f$ is 
    $$\hat{f}(x) = \sum_{\ZZ} f(n)e^{-2\pi i nx}.$$
    So any element of $L^2(\RR/\ZZ)$ can be expressed as a sum of unitary characters of $\ZZ$; we have a `discrete spectrum'. 
\end{example}

\begin{rem}
    The terminology of discrete and continuous spectra comes from the analogy with the spectral theory of the Laplacian. Over $\RR$, the Laplacian is $\Delta = \frac{\partial^2}{\partial x^2}$, and the characters $x \mapsto e^{-2\pi i xy}$ are eigenfunctions. 
\end{rem}

The dichotomy in the above examples is reflected in the compactness of $S^1$ and non compactness of $\RR$. More generally,

\begin{thm}[Peter--Weyl]
    Let $K$ be a compact Hausdorff topological group. Any unitary representation of $K$ decomposes into a completed Hilbert space direct sum of irreducible unitary subrepresentations. There is a unitary equivalence
    $$L^2(K) \cong \widehat{\bigoplus}_{\pi \in \hat{K}} \mathrm{End}(V_\pi)$$
    of representations of $K\times K$, where $(\pi,V_\pi)$ ranges over the set $\hat{K}$ of equivalence classes of irreducible representations of $K$, and $\hat\oplus$ denotes the completed Hilbert space direct sum.
\end{thm}
\begin{proof}
    \cite[Theorem 7.3.2]{DE} and \cite[Theorem 7.2.3]{DE}.
\end{proof}

Even more generally, for so-called Type I groups one can decompose unitary representations through a combination of integrals and Hilbert space direct sums. See \cite[Section 3.10]{GH1} for further details.

Returning to $G=\GL_2(F)$, as this is not compact we would expect the regular representation $L^2(G)$ to decompose according to both a continuous spectra and a discrete spectra. This continous spectra is provided by the parabolically induced representations $\iota_B^G \chi$, where $\chi$ ranges over the characters of $T \cong F^\times \times F^\times$.

In order to compare representations of $G$ and Galois representations through the local Langlands correspondence, we would like to classify them according to some common language. This is provided by the zeta functions, $L$-functions and functional equations discussed in this section. 

The prototypical example of an $L$-function is the Riemann zeta function $\zeta(s) = \sum_{n \geq 1} n^{-s}$.

\begin{prop}
    The function $\zeta(s) = \sum_{n \geq 1} n^{-s}$ satisfies the following properties:
    \begin{itemize}
        \item (Analytic continuation) The Riemann zeta functions converges absolutely to a holomorphic function on $\mathrm{Re}(s)>1$. It has a unique analytic continuation to the complex plane, except the point $s=1$ where $\zeta(s)$ has a simple pole.
        \item (Euler product) We have the identity $$\sum\limits_{n=1}^\infty n^{-s} = \prod\limits_{p \text{ prime}} \frac{1}{1-p^{-s}},$$ convergent for $\mathrm{Re}(s)>1$.
        \item (Functional equation) There is an explicit function $\gamma(s)$ such that $\zeta(1-s)=\gamma(s)\zeta(s)$.
    \end{itemize}
\end{prop}

The approach of Tate in his thesis was to view the Riemann (And Dedekind) zeta functions from an adelic perspective. There the Euler product formulation is immediate and we only need to study the zeta functions locally. Attached to any character $\chi:F^\times \to \CC^\times$ there is an associated space $\mathcal Z(\chi)$ of zeta functions $\zeta(\Phi,\chi,s)$, where $\Phi \in C_c^\infty(F)$. The factor at the prime $p$ of the Riemann zeta function corresponds to the trivial character of $\QQ_p^\times$ and the function $\mathbbm{1}_{\ZZ_p} \in C_c^\infty(\QQ_p)$. A key ingredient in the proof of the functional equation of the Riemann zeta function is the Fourier transform over $\CC$. In general, the functional equation associated to $\chi$ relates zeta functions $\zeta(\hat{\Phi},\chi^{-1},1-s)$ and $\zeta(\Phi,\chi,s)$, where $\hat{\Phi}$ is the Fourier transform of $\Phi$ in $C_c^\infty(F)$. 

\subsection{Functional equation for \texorpdfstring{$\GL_1$}{TEXT}}

Let $F$ be a nonarchimedean local field, $\varpi$ be a uniformiser and $q$ be the size of the residue field. We will later define $L$-functions attached to an irreducible smooth representation of $\GL_2(F)$ and determine a functional equation they satisfy. First we explain this in the context of irreducible smooth representations $\chi$ of $\GL_1(F)$, necessarily a character $\chi: F^\times \to \CC^\times$.

Taking from the classical study of the Riemann zeta function and its functional equation, we want to introduce an analogue of the Fourier transform over $F$. We replace the additive character $e^{2\pi i -}: \RR \to \CC^\times$ with any choice of additive character $\psi: F \to \CC^\times$ with $\psi \neq 1$. In this way, all characters of $F$ are of the form $\psi(-y)$ for $y \in F$, by Additive Duality. The functions we will apply the Fourier transform to will be the algebra $C_c^\infty(F)$ of locally constant compactly supported functions $F \to \CC$. For any choice of Haar measure $\mu$ on $F$, we now define the Fourier transform.

\begin{defn}
    Let $\Phi \in C_c^\infty(F)$, $\psi:F \to \CC^\times$ be an additive character of $F$, and $\mu$ be a Haar measure on $F$. The Fourier transform of $\Phi$ (with respect to $\psi$ and $\mu$) is 
    $$\hat{\Phi}(x) := \int_F \Phi(y)\psi(xy) d\mu(y).$$
\end{defn}

To match the classical definition over $\RR$, we would like the Fourier transform to preserve $C_c^\infty(F)$, and to have a Fourier inversion formula. Indeed:

\begin{prop}
    
    \begin{itemize}
        \item For any $\Phi \in C_c^\infty(F)$, we have $\hat{\Phi} \in C_c^\infty(F)$.
        \item For any $\psi: F \to \CC^\times$ with $\psi \neq 1$, there is a unique Haar measure $\mu_\psi$ on $F$ such that for the associated Fourier transform we have $$\hat{\hat{\Phi}}(x) = \Phi(-x)$$ for any $\Phi \in C_c^\infty(F)$ and $x \in F$.
    \end{itemize}
    
\end{prop}
\begin{proof}
    \cite[Proposition 23.1]{BH1}
\end{proof}

\begin{notn}
    For the remainder of this subsection, $\psi \neq 1$ will be an additive character of $F$, and $\mu= \mu_\psi$ will denote the associated self-dual Haar measure on $F$.
\end{notn}


Now let $\chi: F^\times \to \CC^\times$ be a smooth character of $F^\times$. We want to attach to this character an $L$-function $L(\chi,s)$ in the formal variable $s$. This is defined to be $(1-\chi(\varpi)q^{-s})^{-1}$ when $\chi$ is unramified, and 1 otherwise. In order to generalise to $\GL_2$ it would be preferable to have a more intrinsic definition.

\begin{defn}
    For $\Phi \in C_c^\infty(F)$ and $\chi :F^\times \to \CC^\times$, define the zeta function $\zeta(\Phi,\chi,s)$ to be
    $$\zeta(\Phi,\chi,s) := \int_{F^\times} \Phi(x)\chi(x)|x|^s d^*x,$$ in the formal variable $s$, where $d\mu^*(x) = d^*x$ denotes any choice of Haar measure on $F^\times$.
\end{defn}

Equivalently, we have
$$\zeta(\Phi,\chi,s) = \sum\limits_{m \in \ZZ} z_m q^{-ms}$$
for $$z_m = \int\limits_{\varpi^m \mathcal O_F^\times} \Phi(x)\chi(x)d^*x.$$ In this way it is clear that $\zeta(\Phi,\chi,s) \in \CC((q^{-s}))$. The $z_m=z_m(\Phi,\chi)$ vanish for $m <<0$ because $\Phi$ is compactly supported on $F$.

The zeta function $\zeta(\Phi,\chi,s)$ only depends on $d^*x$ up to scaling. To remove this dependence we define the following notation.

\begin{notn}
    Let $$\mathcal Z(\chi) = \{\zeta(\Phi,\chi,s) \mid \Phi \in C_c^\infty(F)\}.$$
\end{notn}

\begin{notn}
    For $a \in F^\times$ and $\Phi \in C_c^\infty(F)$, denote by $a\Phi$ the function $x \mapsto \Phi(a^{-1}x)$.
\end{notn}

\begin{lemma}
    The space $\mathcal Z(\chi)$ is a $\CC[q^{-s},q^s]$-module, containing $\CC[q^{-s},q^s]$.
\end{lemma}
\begin{proof}
    Let $a \in F^\times$ of valuation $v_F(a)$. Then 
    $$\zeta(a\Phi,\chi,s) = \chi(a)q^{-v_f(a)s}\zeta(\Phi,\chi,s),$$ giving the desired module structure. To establish the containment we show that $\mathcal Z(\chi)$ contains a nonzero constant. Let $d$ be such that $\chi \mid_{U_F^{d+1}} = 1$. Taking $\Phi=\mathbbm{1}_{U_F^{d+1}}$, we see that 
    $$Z(\Phi,\chi,s) = \mu^*(U_F^{d+1}) \neq 0.$$
\end{proof}

\begin{prop}\label{prop:gl1factor}
    Let $\chi:F^\times \to \CC^\times$. There exists a unique polynomial $P_\chi \in \CC[X]$ with $P_\chi(0)=1$ such that
    $$\mathcal Z(\chi) = P_\chi(q^{-s})^{-1}\cdot \CC[q^{-s},q^s].$$
    Moreover, we have
    $$
    P_\chi(X) =
    \begin{cases}
        1-\chi(\varpi)X & \text{if $\chi$ is unramified} \\
        1 & \text{otherwise}
    \end{cases}
    $$
\end{prop}
\begin{proof}
    Suppose $\Phi(0)=0$. Then $\Phi|_{F^\times} \in C_c^\infty(F^\times)$, and so $\Phi$ is identically zero on $\varpi^m\mathcal O_F^\times$ for $|m| >>0$. Thus only finitely many of the coefficients $z_m$ are nonzero, so that $\Phi \in \CC[q^{-s},q^s]$.

    The space $C_c^\infty(F)$ is spanned by $C_c^\infty(F^\times)$ and $\mathbbm{1}_{\mathcal O_F}$. We compute
    $$\zeta(\mathbbm{1}_{\mathcal O_F},\chi,s) = \sum\limits_{m \geq 0} \chi(\varpi^m)q^{-ms} \int_{\mathcal O_F^\times} \chi(x)d^*x.$$
    If $\chi$ is unramified (trivial on $\mathcal O_F^\times$), this gives us 
    $$\sum\limits_{m \geq 0} \chi(\varpi)^mq^{-ms} \mu^*(\mathcal O_F^\times) = (1-\chi(\varpi)q^{-s})^{-1} \mu^*(\mathcal O_F^\times).$$
    When $\chi$ is ramified the integral is zero. Indeed, translation invariance of $d^*x$ implies
    $$\int_{\mathcal O_F^\times} \chi(x)d^*x = \int_{\mathcal O_F^\times} \chi(xy)d^*x = \chi(y)\int_{\cO_F^\times} \chi(x) d^*x$$ for any $y \in \cO_F^\times$, so that this is zero if there is some $y$ with $\chi(y) \neq 1$. This computation, together with the previous lemma, establish the result. 
\end{proof}

\begin{rem}
    The computation in the proof above shows, in the case $\chi = 1$, that $\zeta(\mathbbm{1}_{\cO_F},1,s) = (1-q^{-s})^{-1}$, provided we normalise $d^*x$ appropriately. If $F=K_v$ is the completion of a number field $K$ at a nonarchimedean place $v$, we recover the Euler factor of the Dedekind zeta function $\zeta_K(s)$ at the place $v$. This explains the naming of our zeta functions. 
\end{rem}

\begin{rem}
    The computations of Proposition \ref{prop:gl1factor} show that each $\zeta(\Phi,\chi,s)$ converges absolutely and uniformly in vertical strips in some right half plane, and admit analytic continuation to a rational function in $q^{-s}$.
\end{rem}

\begin{defn}
    Define the $L$-function attached to $\chi$ to be $L(\chi,s)=P_\chi(q^{-s})^{-1}$.
\end{defn}

As with the Riemann zeta function, we have functional equations for the zeta functions.

\begin{thm}\label{thm:gl1gamma}
    Let $\chi: F^\times \to \CC^\times$. There is a unique $\gamma(\chi,s,\psi) \in \CC(q^{-s})$ such that 
    $$\zeta(\hat{\Phi}, \check{\chi},1-s) = \gamma(\chi,s,\psi) \zeta(\Phi,\chi,s)$$ for all $\Phi \in C_c^\infty(F)$, where $\check{\chi}=1/\chi : F^\times \to \CC^\times$.
\end{thm}
\begin{proof}
    \cite[Theorem 23.3]{BH1}.
\end{proof}

Since $\mathcal Z(\chi) = L(\chi,s)\cdot \CC[q^{-s},q^s]$, it is natural to consider the terms $\frac{\zeta(\Phi,\chi,s)}{L(\chi,s)} \in \CC[q^{-s},q^s]$. This allows us to treat the case of $\chi$ ramified and unramified evenly. 

\begin{defn}
    Let $$\epsilon(\chi,s,\psi) := \gamma(\chi,s,\psi) \cdot \frac{L(\chi,s)}{L(\check{\chi},s)}.$$
\end{defn}
This is known as Tate's local constant.

The functional equation for $\zeta$ can be rewritten as
$$\frac{\zeta(\hat{\Phi},\check{\chi},1-s)}{L(\check{\chi},1-s)} = \epsilon(\chi,s,\psi) \frac{\zeta(\Phi,\chi,s)}{L(\chi,s)}.$$

\begin{cor}
    The local constant satisfies the functional equation
    $$\epsilon(\chi,s,\psi)\epsilon(\check{\chi},1-s,\psi) = \chi(-1).$$
    The local constant is of the form $$\epsilon(\chi,s,\psi) = aq^{bs}$$ for some $a \in \CC^\times$, $b \in \ZZ$.
\end{cor}
\begin{proof}
    The first statement comes from the Fourier inversion formula, where the $\chi(-1)$ term comes from the minus sign in $\hat{\hat{\Phi}}(x) = \Phi(-x)$. The functional equation implies that $\epsilon$ is a unit in $\CC[q^{-s},q^s]$, and the units are precisely the elements of the form $aq^{bs}$ for $b \in \ZZ$.
\end{proof}

\newpage