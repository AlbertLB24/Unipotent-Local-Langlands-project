\subsection{The L-function of a Character of \texorpdfstring{$F^\times$}{TEXT}}

Let $F$ be a non-Archimedean local field, $\varpi$ be a uniformiser and $q$ be the size of the residue field. We will later define $L$-functions attached to an irreducible smooth representation of $\GL_2(F)$ and determine a functional equation they satisfy. The ideas involved in the proof that such objects exists are similar (and highly dependent) on the ideas associated to the development of the theory discussed in the previous section for the local field $F$. 

Therefore, we explain these results first in the context of an irreducible smooth representation $\chi$ of $\GL_1(F)$, necessarily a character $\chi: F^\times \to \CC^\times$. To mimic the development we aim to follow for the case of irreducible representations of $\GL_2(F)$, we will first define the space $\mathcal Z(\chi)$ of zeta functions from which the definition of the L-function $L(\chi,s)$ arises naturally. Afterwards, we will introduce the analogue of the Fourier transform over $F$. This will lead us to the proof of the functional equation of $\GL_1$ and the construction of the local constant $\varepsilon(\chi,s,\psi)$. Before we start, we require one definition analogous to \ref{def:addlevel} in the multiplicative setting of $F^\times$.

\begin{defn}
    Let $\chi:F^\times\to\CC^\times$ be a character of $F^\times$. We say that $\chi$ is \textit{unramified} if $\mathcal{O}_F^\times\subseteq\ker\chi$ and \textit{ramified} otherwise. If $\chi$ is ramified, then the \textit{level} of $\chi$ is defined to be the least integer $d\geq0$ such that $U_F^{d+1}=1+\pp^{d+1}\subseteq\ker\chi$.
\end{defn}

Let $\chi: F^\times \to \CC^\times$ be a character of $F^\times$. We want to attach to this character an $L$-function $L(\chi,s)$ in the formal variable $s$. We remark that an unramified character $\chi$ of $F^\times$ is completely determined by the value of $\chi(\varpi)$ since $\chi(x)=\chi(\varpi)^m$ for each $x\in\varpi^m\mathcal{O}_F$. Classically, the L-function associated to $\chi$ is defined to be $(1-\chi(\varpi)q^{-s})^{-1}$ when $\chi$ is unramified, and $1$ otherwise. In order to generalise to $\GL_2$ it would be preferable to have a more intrinsic definition.

\begin{defn}
    For $\Phi \in C_c^\infty(F)$ and $\chi :F^\times \to \CC^\times$, define the zeta function $\zeta(\Phi,\chi,s)$ to be
    $$\zeta(\Phi,\chi,s) := \int_{F^\times} \Phi(x)\chi(x)|x|^s d^*x,$$ in the formal variable $s$, where $d\mu^*(x) = d^*x$ denotes any choice of Haar measure on $F^\times$.
\end{defn}

We remark that since $|x|$ is constant and equals $q^{-m}$ in $\varpi^m\mathcal{O}_F^\times$ for each $m\in\ZZ$, we can equivalently rewrite the zeta function as
$$\zeta(\Phi,\chi,s) = \sum\limits_{m \in \ZZ} z_m(\Phi,\chi) q^{-ms}$$
where $$z_m(\Phi,\chi) = \int\limits_{\varpi^m \mathcal O_F^\times} \Phi(x)\chi(x)d^*x.$$ 
Note that $z_m=z_m(\Phi,\chi)$ vanishes for $m \ll0$ because $\Phi$ is compactly supported on $F$, so $\supp\Phi\subseteq\pp^N$ for some sufficiently small $N$. In this way it is clear that $\zeta(\Phi,\chi,s) \in \CC((q^{-s}))$. 

\begin{example}\label{example:zetafnc}
    At this stage it is helpful and convenient to discuss some explicit examples of zeta functions that will be relevant in the future discussion.
    \begin{enumerate}[(1)]
        \item Let $\Phi=\mathbbm{1}_{\mathcal{O}_F}\in C_c^\infty(F)$ be the characteristic function of $\mathcal{O}_F$ and let $\chi$ is a character of $F^\times$. Since $\mathcal{O}_F=\cup_{m\geq0}\varpi^m\mathcal{O}_F^\times$, it follows that $z_m(\mathbbm{1}_{\mathcal{O}_F},\chi)=0$ for $m<0$. For $m\geq 0$, we have that 
        $$z_m(\mathbbm{1}_{\mathcal{O}_F},\chi)=\int_{\varpi^m \mathcal O_F^\times} \chi(x)d^*x=\chi(\varpi)^m\int_{\mathcal{O}_F^\times}\chi(x)d^*x$$
        
        We now consider the cases where $\chi$ is unramified or ramified separately.
        \begin{enumerate}
            \item If $\chi$ is unramified, then $z_m(\mathbbm{1}_{\mathcal{O}_F},\chi)=\mu^*(\mathcal{O}_F^\times)\chi(\varpi)^m$
            and therefore $$\zeta(\mathbbm{1}_{\mathcal{O}_F},\chi,s)=\mu^*(\mathcal{O}_F^\times)\sum_{m\geq0}\chi(\varpi)^mq^{-ms}=\mu^*(\mathcal{O}_F^\times)(1-\chi(\varpi)q^{-s})^{-1}.$$
            Note that, if $\chi(\varpi)=1$, then $\chi$ is the trivial character and $\zeta(\mathbbm{1}_{\mathcal{O}_F},\mathbbm{1}_{F^\times},s)=(1-q^{-s})^{-1}$ provided that $\mu^*(\mathcal{O}_F^\times)=1$. If $F=K_v$ is the completion of a number field $K$ at a non-Archimedean place $v$, we recover the Euler factor of the Dedekind zeta function $\zeta_K(s)$ at the place $v$. This explains the naming of our zeta functions. 
            \item If $\chi$ is ramified, then there is some $y\in\mathcal{O}_F^\times$ such that $\chi(y)\neq 1$. Then, by left translation invariance of $d^*x$ we have that
            $$\int_{\mathcal O_F^\times} \chi(x)d^*x = \int_{\mathcal O_F^\times} \chi(xy)d^*x = \chi(y)\int_{\cO_F^\times} \chi(x) d^*x.$$
            Since $y$ was chosen so that $\chi(y)\neq 1$, it follows that 
            the integral is equal to zero, so $z_m(\mathbbm{1}_{\mathcal{O}_F},\chi)=0$ for all $m\in\ZZ$. This implies that $\zeta(\mathbbm{1}_{\mathcal{O}_F},\chi,s)=0$.
        \end{enumerate}

        \item Let $\chi$ be a character of $F^\times$ and let $d$ be an integer such that $U_F^d\subseteq\ker\chi$. If $\Phi=\mathbbm{1}_{U_F^d}\in C_c^\infty(F)$, then $\supp(\Phi)\subseteq\mathcal{O}_F^\times$ and so $z_m(\mathbbm{1}_{U_F^d},\chi)=0$ for $m\neq0$. Finally, 
        $$z_0(\mathbbm{1}_{U_F^d},\chi)=\int_{U_F^d}\chi(x)d^*x=\mu^*(U_F^d)$$
        and therefore $\zeta(\mathbbm{1}_{U_F^d},\chi,s)=\mu^*(U_F^d)>0$ is a positive constant.       
    \end{enumerate}
\end{example}


It is clear from the examples that the zeta function $\zeta(\Phi,\chi,s)$ only depends on $d^*x$ up to scaling. To remove this dependence we define the following space.

\begin{defn}
    Let $\chi$ be a character of $F^\times$. Then we define the space of $\zeta$-functions associated to $\chi$ as
    $$\mathcal Z(\chi) = \{\zeta(\Phi,\chi,s) \mid \Phi \in C_c^\infty(F)\}.$$
\end{defn}

We note that $\mathcal{Z}(\chi)$ is a $\CC$ vector space. Indeed, $$\alpha\zeta(\Phi_1,\chi,s)+\zeta(\Phi_2,\chi,s)=\zeta(\alpha\Phi_1+\Phi_2,\chi,s)\in\mathcal{Z}(\chi)$$
for any $\Phi_1,\Phi_2\in C_c^\infty(F)$ and $\alpha\in\CC$. However, $\mathcal{Z}(\chi)$ has futher useful structure; it is also a $\CC[q^{-s},q^s]$-module. To see this, one needs to describe an invertible action of $q^{-s}$ on $\mathcal{Z}(\chi)$. To do this, we introduce some notation.

\begin{notn}
    For $a \in F^\times$ and $\Phi \in C_c^\infty(F)$, denote by $a\Phi$ the function $x \mapsto \Phi(a^{-1}x),x\in F$. This function also lies in $C_c^\infty(F)$.
\end{notn}

\begin{lemma}\label{lem:ZmodGL1}
    The space $\mathcal Z(\chi)$ is a $\CC[q^{-s},q^s]$-module, containing $\CC[q^{-s},q^s]$.
\end{lemma}
\begin{proof}
    Let $a \in F^\times$ of valuation $\nu(a)$. Then a short calculation shows that
    $$\zeta(a\Phi,\chi,s) = \chi(a)q^{-\nu(a)s}\zeta(\Phi,\chi,s).$$
    In other words, the action of $q^{-s}$ is given by 
    $$q^{-s}\zeta(\Phi,\chi,s)=\zeta(\chi(\varpi)^{-1}\varpi\Phi,\chi,s),$$
    which is an invertible action and thus gives the desired module structure. The containment follows directly from Example \ref{example:zetafnc}(2), where we showed that $\mathcal{Z}(\chi)$ contains a non-zero constant.
\end{proof}

\iffalse
\textbf{Question to potentially simplify the proof}

The ring $\CC[X,X^{-1}]$ of Laurent series with complex coefficients has very useful properties. It is a principal ideal domain and its group of units consists of the monomials $aX^b,a\in\CC^\times, b\in\ZZ$. Then $\mathcal Z(\chi)$, being a $\CC[q^{-s},q^s]$-module contained in its fraction field $\CC((q^{-s}))$, it of the form 
$$\mathcal{Z}(\chi)=P_\chi(q^{-s})\cdot\CC[q^{-s},q^s]$$ for some polynomial $P_\chi(X)\in\CC[X]$ that is unique under the assumption that $P_\chi(0)=1$.

\textbf{Is this correct?}
\fi

\begin{prop}\label{prop:gl1factor}
    Let $\chi:F^\times \to \CC^\times$ be a character. There exists a unique polynomial $P_\chi \in \CC[X]$ with $P_\chi(0)=1$ such that
    $$\mathcal Z(\chi) = P_\chi(q^{-s})^{-1}\cdot \CC[q^{-s},q^s].$$
    Moreover, we have
    $$
    P_\chi(X) =
    \begin{cases}
        1-\chi(\varpi)X & \text{if $\chi$ is unramified,} \\
        1 & \text{otherwise}
    \end{cases}
    $$
\end{prop}
\begin{proof}
    Suppose $\Phi(0)=0$. Then $\Phi|_{F^\times} \in C_c^\infty(F^\times)$, and so $\Phi$ is identically zero on $\varpi^m\mathcal O_F^\times$ for $|m| \gg0$. Thus only finitely many of the coefficients $z_m$ are nonzero, so that $\Phi \in \CC[q^{-s},q^s]$. \textbf{Maybe this needs slightly more explanation?}

    Any $\Phi\in C_c^\infty(F)$ can be expressed as $\Phi=\alpha\mathbbm{1}_{\mathcal{O}_F}+\Phi'$, where $\alpha=\Phi(0)\in\CC$ and $\Phi'\in C_c^\infty(F)$ satisfies $\Phi'(0)=0$. By linearity,
    \begin{equation}\tag{\dag}\label{eqn:linear}
        \zeta(\Phi,\chi,s)=\alpha\zeta(\mathbbm{1}_{\mathcal{O}_F},\chi,s)+\zeta(\Phi',\chi,s)
    \end{equation}
    and by the previous paragraph, $\zeta(\Phi',\chi,s)\in\CC[q^{-s},q^{s}]$. Hence, it only remains to compute $\zeta(\mathbbm{1}_{\mathcal O_F},\chi,s)$. This is precisely the contant of Example \ref{example:zetafnc}(1). If $\chi$ is unramified, then $\zeta(\mathbbm{1}_{\mathcal O_F},\chi,s)=\mu^*(\mathcal{O}_F^\times)(1-\chi(\varpi)q^{-s})^{-1}$ and if $\chi$ is ramified, then $\zeta(\mathbbm{1}_{\mathcal O_F},\chi,s)=0$. Together with \eqref{eqn:linear} and the previous lemma, this establishes the result. 
\end{proof}

\iffalse
\begin{rem}
    The computation in the proof above shows, in the case $\chi = 1$, that $\zeta(\mathbbm{1}_{\cO_F},1,s) = (1-q^{-s})^{-1}$, provided that $\mu^*(\mathcal{O}_F^\times)=1$. If $F=K_v$ is the completion of a number field $K$ at a non-Archimedean place $v$, we recover the Euler factor of the Dedekind zeta function $\zeta_K(s)$ at the place $v$. This explains the naming of our zeta functions. 
\end{rem}
\fi

\begin{rem}
    The computations of Proposition \ref{prop:gl1factor} show that all $\zeta(\Phi,\chi,s)$ converge absolutely and uniformly in vertical strips in some right half plane, and admit analytic continuation to a rational function in $q^{-s}$.
\end{rem}

\begin{defn}\label{def:lfunction}
    The $L$-function attached to a character $\chi$ of $F^\times$ is defined to be
    \begin{equation*}
        L(\chi,s)=P_\chi(q^{-s})^{-1}=
        \begin{cases}
            (1-\chi(\varpi)q^{-s})^{-1} & \text{if $\chi$ is unramified} \\
            1 & \text{otherwise},
        \end{cases}
    \end{equation*}
    which indeed coincides with the classical language.
\end{defn}

\subsection{The Functional Equation}

Next, taking from the classical study of the Riemann zeta function and its functional equation, we want to introduce an analogue of the Fourier transform over $F$. We replace the additive character $x\mapsto e^{2\pi i x},x\in\RR$ with any choice of additive character $\psi: F \to \CC^\times$ with $\psi \neq 1$. In this way, by Additive Duality, all characters of $F$ are of the form $y\mapsto\psi(ay),y\in F$ for some $a\in F$. The functions we will apply the Fourier transform to will be the algebra $C_c^\infty(F)$ of locally constant compactly supported functions $F \to \CC$. For any choice of Haar measure $\mu$ on $F$, we now define the Fourier transform.

\begin{defn}
    Let $\Phi \in C_c^\infty(F)$, $\psi:F \to \CC^\times$ be an additive character of $F$, and $\mu$ be a Haar measure on $F$. The Fourier transform of $\Phi$ (with respect to $\psi$ and $\mu$) is 
    $$\hat{\Phi}(x) := \int_F \Phi(y)\psi(xy) d\mu(y).$$
\end{defn}

To match the classical definition over $\RR$, we would like the Fourier transform to preserve $C_c^\infty(F)$, and to have a Fourier inversion formula. Indeed:

\begin{prop}
    The Fourier transform on $C_c^\infty(F)$ satisfies the following:
    \begin{itemize}
        \item For any $\Phi \in C_c^\infty(F)$, we have $\hat{\Phi} \in C_c^\infty(F)$.
        \item For any $\psi: F \to \CC^\times$ with $\psi \neq 1$, there is a unique Haar measure $\mu_\psi$ on $F$ such that for the associated Fourier transform we have $$\hat{\hat{\Phi}}(x) = \Phi(-x)$$ for any $\Phi \in C_c^\infty(F)$ and $x \in F$. This measure satisfies that $\mu_\psi(\mathfrak{o})=q^{l/2}$, where $l$ is the level of $\psi$.
    \end{itemize}
    
\end{prop}
\begin{proof}
    \cite[Proposition 23.1]{BH1}
\end{proof}

\begin{notn}
    For the remainder of this subsection, $\psi \neq 1$ will be an additive character of $F$, and $\mu= \mu_\psi$ will denote the associated self-dual Haar measure on $F$.
\end{notn}

As with the Riemann zeta function, we have a functional equation for the zeta functions.

\begin{thm}\label{thm:gl1gamma}
    Let $\chi: F^\times \to \CC^\times$. There is a unique $\gamma(\chi,s,\psi) \in \CC(q^{-s})$ such that 
    $$\zeta(\hat{\Phi}, \check{\chi},1-s) = \gamma(\chi,s,\psi) \zeta(\Phi,\chi,s)$$ for all $\Phi \in C_c^\infty(F)$, where $\check{\chi}=1/\chi : F^\times \to \CC^\times$.
\end{thm}
\begin{proof}
    \cite[Theorem 23.3]{BH1}.
\end{proof}

Since $\mathcal Z(\chi) = L(\chi,s)\cdot \CC[q^{-s},q^s]$, it is natural to consider the terms $\frac{\zeta(\Phi,\chi,s)}{L(\chi,s)} \in \CC[q^{-s},q^s]$. This allows us to treat the case of $\chi$ ramified and unramified evenly. 

\begin{defn}
    Given characters $\chi$ and $\psi$ of $F^\times$ and $F$ respectively,  we define %a rational function $\varepsilon(\chi,s,\psi)\in\CC(q^{-s})$ by
    $$\varepsilon(\chi,s,\psi) := \gamma(\chi,s,\psi)\frac{L(\chi,s)}{L(\check{\chi},1-s)}.$$
    Then $\varepsilon(\chi,s,\psi)\in\CC(q^{-s})$ is known as \textit{Tate's local constant}.
\end{defn}

\begin{rem}
    The functions $\gamma(\chi,s,\psi)$ and $\varepsilon(\chi,s,\psi)$ do indeed depend on the additive character $\psi$. By Additive Duality (Theorem \ref{thm:duality}), any other non-trivial additive character is of the form $a\psi$ for some $a\in F^\times$ and by carefully tracing back through the definitions, one can show that 
    \begin{align*}
        \gamma(\chi,s,a\psi)=\chi(a)|a|^{s-\frac{1}{2}}\gamma(\chi,s,\psi),\\
        \varepsilon(\chi,s,a\psi)=\chi(a)|a|^{s-\frac{1}{2}}\varepsilon(\chi,s,\psi).
    \end{align*}
\end{rem}

The local constant arises naturally in the functional equation for $\zeta$ since it can be rewritten as
$$\frac{\zeta(\hat{\Phi},\check{\chi},1-s)}{L(\check{\chi},1-s)} = \varepsilon(\chi,s,\psi) \frac{\zeta(\Phi,\chi,s)}{L(\chi,s)},$$
and it satisfies the following functional equation itself.

\begin{cor}
    The local constant satisfies the functional equation
    $$\varepsilon(\chi,s,\psi)\varepsilon(\check{\chi},1-s,\psi) = \chi(-1).$$
    The local constant is of the form $$\varepsilon(\chi,s,\psi) = aq^{bs}$$ for some $a \in \CC^\times$, $b \in \ZZ$.
\end{cor}
\begin{proof}
    The first statement comes from the Fourier inversion formula, where the $\chi(-1)$ term comes from the minus sign in $\hat{\hat{\Phi}}(x) = \Phi(-x)$. The functional equation implies that $\varepsilon(\chi,s,\psi)$ is a unit in $\CC[q^{-s},q^s]$, and the units are precisely the elements of the form $aq^{bs}$ for $a\in\CC^\times$ and $b \in \ZZ$.
\end{proof}

\textbf{The other thing to mention here is that $\varepsilon\in\CC[q^{-s},q^{s}]$, which is the non-trivial step of the proof. This crucially distinguishes between $\gamma$ and $\varepsilon$ right?}

