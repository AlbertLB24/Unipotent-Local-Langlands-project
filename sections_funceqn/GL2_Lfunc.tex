\subsection{The L-function of a Principal Series Representation}\label{sec:LfuncGL2}

Recall that for a character $\chi:F^\times \to \CC^\times$ we defined, for any $\Phi \in C_c^\infty(F)$, a zeta function $$\zeta(\Phi,\chi,s) = \int_{F^\times} \Phi(x)\chi(x) |x|^s d^*x.$$
To replicate this for smooth representations $\pi : G \to \GL(V)$ we need both an analogue to the space $C_c^\infty(F)$ and a way to extract scalar values from $\pi(g) \in \GL(V)$. The first is done in the obvious way; we let $A=M_2(F)$ be the additive group of $2\times2$ matrices with the product topology $A\cong F^4$ and we consider the space $C_c^\infty(A)$ of locally constant functions $\Phi:A\to\CC$ with compact support. Secondly, the scalar values of $\pi(g)$ will come from matrix coefficients.

\begin{defn}
Let $(\pi,V)$ be a smooth representation of $G$ with smooth dual $\check{V}$. For vectors $v\in V, \check{v} \in \check{V}$, define the smooth function $\gamma_{v \otimes \check{v}}: G \to \CC$ by 
$$\gamma_{\check{v} \otimes v} : g \mapsto \langle \check{v},\pi(g) v \rangle,$$ where $\langle, \rangle$ denotes the natural evaluation pairing $\check{V} \otimes V \to \CC$. Let $\mathcal C(\pi)$ be the vector space spanned by the functions $\gamma_{\check{v} \otimes v}$. Elements of $\mathcal C(\pi)$ are called the \textit{matrix coefficients} of $\pi$.
\end{defn}
\begin{rem}
    If $\pi=\chi:F^\times \to \CC^\times$ is a character, any matrix coefficient (defined in the analogous way for $F^\times$) of $\chi$ is some scalar multiple of $\chi$.

    Moreover, if $V$ is a finite-dimensional complex representation of $G$ and $\{v_1,\ldots,v_n\}$ is a basis of $V$ with dual basis $\{\check{v_1},\ldots,\check{v_n}\}$, then $\gamma_{\check{v_i}\otimes v_j}(g)$ for $g\in G$ is precisely the $(i,j)$-th entry of $\pi(g)$ as a matrix with respect to the basis $\{v_1,\ldots,v_n\}$.
\end{rem}

\iffalse
\begin{defn}
    Let $(\pi,V)$ be an irreducible smooth representation of $G$. The centre $Z$ of $G$ acts on $V$ via the central character $\omega_\pi : Z \to \CC^\times$.
\end{defn}
\fi

An important aspect of matrix coefficients is that they interact well with the action of the centre $Z$ of $G$ by left translation. 

\begin{lemma}\label{central char}
    Let $(\pi,V)$ be an irreducible smooth representation of $G$, and let $Z$ be its centre. For any $f \in \mathcal C(\pi), z \in Z$ and $g \in G$ we have $f(zg) = \omega_\pi(z) f(g)$, where $\omega_\pi:Z\to\CC^\times$ is the central character defined in Corollary \ref{cor:centralchar}.
\end{lemma}


Fix a smooth representation $\pi$ of $G$. We may now define zeta functions for any $f \in \mathcal C(\pi)$.

\begin{defn}
    For $\Phi \in C_c^\infty(A)$ and $f \in \mathcal C(\pi)$, define the \textit{zeta function} $\zeta(\Phi,f,s)$ to be
    $$\zeta(\Phi,f,s) := \int_{G} \Phi(x)f(x)|\det x|^s d^*x,$$ in the formal variable $s$, where $d\mu^*(x) = d^*x$ denotes any choice of Haar measure on $G$.
\end{defn}

Similarly to the zeta functions for $F^\times$, we can express 
$$\zeta(\Phi,f,s)=\sum_{m\in\ZZ}z_m(\Phi,f)q^{-ms},$$
where 
$$z_m(\Phi,f)=\int_{G_m}\Phi(x)f(x)d^*x$$
and $G_m=\{x\in G\mid\nu(\det x)=m\}$. 

The zeta functions associated to a smooth representation of $\pi$ of $G$ share many properties to the zeta functions associated to characters of $F^\times$. We state the relevant results now, and we give a sketch of the proof. For complete proofs, see \cite[\S 24.4]{BH1}.

\begin{lemma}
    For any $\Phi \in C_c^\infty(A)$ and $f \in \mathcal C(\pi)$ we have $\zeta(\Phi,f,s) \in \CC((q^{-s}))$ in the formal variable $s$.
\end{lemma}
\begin{proof}
    The result then follows from \cite[Lemma 24.4.1]{BH1}, where the Cartan decomposition (Section \ref{sec:measuregl2}) is used to prove that $z_m(\Phi,f)=0$ for $m\ll0$.
\end{proof}

\begin{example}\label{example:zetafncgl2}
    We now discuss the analogous computations to Example \ref{example:zetafnc} in the $\GL_2(F)$ setting. 
    \begin{enumerate}[(1)]
        \item Let $\chi$ be a character of $G$. A standard argument in group theory shows that the commutator subgroup of $G$ is $\mathrm{SL}_2(F)$, and therefore $\chi=\phi\circ\det$ for some character $\phi$ of $F^\times$. Let $H=M_2(\mathcal{O}_F)$ and let $\Phi=\mathbbm{1}_H\in C_c^\infty(A)$ be its characteristic function. The space of matrix coefficients $\mathcal{C}(\phi\circ\det)$ is one-dimensional, and it is generated by the function $f_\phi:g\mapsto\phi(\det g),g\in G$. We calculate explicitly 
        \begin{equation}\tag{\ddag}\label{eqn:zetagl2}
            \zeta(\mathbbm{1}_H,f_\phi,s)=\int_{G\cap H}\phi(\det g)|\det g|^sd^*g=\sum_{n\geq 0}q^{-ns}\int_{G_n\cap H}\phi(\det g)d^*g,
        \end{equation}
        a necessary step to find the $L$-function associated to $\phi\circ\det$. For each pair of integers $a\leq b$, let $m_{a,b}=\begin{psmallmatrix}
            \varpi^a & 0\\
            0 & \varpi^b\\
        \end{psmallmatrix}$, and by Cartan decomposition, we have that $G=\cup_{a\leq b}K m_{a,b}K$ where $K=\GL_2(\mathcal{O}_F)\subset H$. If $a\geq 0$, then $Km_{a,b}K\subset H$, while if $a<0$, then $Km_{a,b}K\cap H=\emptyset$. Therefore, we have shown 
        $$G\cap H=\bigcup_{0\leq a\leq b}Km_{a,b}K\quad\text{and}\quad G_n\cap H=\bigcup_{\substack{0\leq a\leq b \\ a+b=n}} Km_{a,b}K,$$ and so
        $$\zeta(\mathbbm{1}_H,f_\phi,s)=\sum_{0\leq a\leq b}\int_{Km_{a,b}K}\phi(\det g)|\det g|^sd^*g=\sum_{0\leq a\leq b}q^{-s(a+b)}\int_{Km_{a,b}K}\phi(\det g)d^*g.$$
        \begin{enumerate}
            \item If $\phi$ is unramified, then $\phi(\det g)=\phi(\varpi)^{a+b}$ for $g\in Km_{a,b}K$ and then
            $$\zeta(\mathbbm{1}_H,f_\phi,s)=\sum_{0\leq a\leq b}(\phi(\varpi)q^{-s})^{a+b}\mu^*(Km_{a,b}K).$$
            It remains to determine $\mu^*(Km_{a,b}K)$ in terms of $\mu^*(K)$ for each $0\leq a\leq b$. We can write 
            $Km_{a,b}K=\cup_{g\in K}Km_{a,b}g$
            and $Km_{a,b}g_1=Km_{a,b}g_2$ if and only if $g_2g_1^{-1}\in m_{a,b}^{-1}K m_{a,b}$. Hence,
            \[
            \mu^*(Km_{a,b}K)=\left|\frac{K}{K\cap m_{a,b}^{-1}K m_{a,b}}\right|\mu^*(K)=
            \begin{cases}
                \mu^*(K) &\text{ if } b=a,\\
                (q+1)q^{b-a-1}\mu^*(K) &\text{ if } b > a.\\
            \end{cases}
            \]
            Putting everything together, one obtains
            \begin{align*}
                \mu^*(K)^{-1}\zeta(\mathbbm{1}_H,f_\phi,s)&=\sum_{c=0}^{\infty}(\phi(\varpi)q^{-s})^{2c}+(q+1)\sum_{0\leq a< b}q^{b-a-1}(\phi(\varpi)q^{-s})^{a+b}\\
                &=\sum_{k=0}^\infty\left(\sum_{j=0}^k q^j\right)(\phi(\varpi)q^{-s})^k=(1-\phi(\varpi)q^{-s})^{-1}(1-\phi(\varpi)q^{1-s})^{-1},
            \end{align*}
            where the last two steps follow by carefully counting the coefficient of $(\phi(\varpi)q^{-s})^k$ for each $k\geq 0$.
            \item If $\phi$ is ramified instead, then there is some $h\in \GL_2(\mathcal{O}_F)$ such that $\phi(\det h)\neq 1$. By using an almost identical argument to Example \ref{example:zetafnc}(1)(b), one shows that
            $$\int_{G_m\cap H}\phi(\det g)d^*g=0,$$
            for each $m\geq 0$ and using equation \eqref{eqn:zetagl2}, we have $\zeta(\mathbbm{1}_H,f_\phi,s)=0$.
        \end{enumerate} 
        
        \item Let $(\pi,V)$ be any representation of $G$ and let $f=\gamma_{\check{v}\otimes v}$ be a matrix coefficient so that $\langle\check{v},v\rangle\neq 0$. In other words, $f(1)\neq 0$. Next, choose some open compact subgroup $K$ of $G$ fixing both $\check{v}$ and $v$, so that $f(k_1gk_2)=f(g)$ for all $g\in G$ and $k_1,k_2\in K$. In particular, $f$ is constant in $K$ so $f(g)=f(1)\neq 0$ for all $g\in K$. Recall that $\GL_2(\mathcal{O}_F)$ is an open and compact subgroup of $G$, so by intersecting with $\GL_2(\mathcal{O}_F)$ if necessary, we may assume that $K\leq\GL_2(\mathcal{O}_F)$. With these choices, it follows that 
        $$\zeta(\mathbbm{1}_K,f,s)=\int_{K}f(g)|\det g|^sd^*g=\int_K f(g)d^*g=f(1)\mu^*(K)$$ is a nonzero constant.    

    \end{enumerate}
\end{example}

\iffalse
\textbf{This would be a good point to introduce a few examples. Ideas:
\begin{enumerate}
    \item Explicit computation of some zeta functions associated to the character $\det$ or $\phi\circ\det$. This requires, however, the cartan decomposition, which is not here yet.
    \item Explicit computation that $\mathcal{Z}(\pi)$ contains a non-zero constant. This would be an explicit step of the proof of Lemma \ref{lem:ZmodGL2}. If we do this, should we just prove the lemma as a whole?
\end{enumerate}
}
\fi

%We now define the space of $\zeta$-functions associated to $\pi$.
\begin{defn}
    Let $(\pi,V)$ be a smooth representation of $G$. We define the space of $\zeta$-functions associated to $\pi$ as $$\mathcal Z(\pi) = \left\{\zeta\left(\Phi,f,s+\frac{1}{2}\right) \mid \Phi \in C_c^\infty(A), f \in \mathcal C(\pi)\right\}.$$
\end{defn}
\begin{rem}
    The addition of $1/2$ will be explained in the case of principal series representations by the appearance of the modular character $\delta_B$.
\end{rem}

The analogous result to Lemma \ref{lem:ZmodGL1} also holds in this case.

\begin{lemma}\label{lem:ZmodGL2}
    The space $\mathcal Z(\pi)$ is a $\CC[q^{-s},q^s]$-module, containing $\CC[q^{-s},q^s]$.
\end{lemma}
\begin{proof}
    The proof is a tedious generalization of Lemma \ref{lem:ZmodGL1}. Firstly, one needs to define appropriate actions of $G\times G$ on $C_c^\infty(A)$ and $\mathcal{C}(\pi)$ to describe the action of $q^{-s}$ on $\mathcal{Z}(\pi)$. Secondly, one needs to show that $\mathcal{Z}(\pi)$ contains non-zero constants; we have already done this in Example \ref{example:zetafncgl2}(2). For full details on the first step, see
    \cite[Lemma 24.4.2]{BH1}.
\end{proof}


Consider now the situation where $\pi = \iota_B^G \chi$ is a parabolically induced representation, where $\chi = \chi_1 \otimes \chi_2$ is a character of $T$. We want to study the space $\mathcal Z(\pi)$ and prove an analogous result to Proposition \ref{prop:gl1factor}. The following fundamental result provides a complete answer to this question.


\begin{prop}\label{prop:gl2factor}
    Let $\chi=\chi_1\otimes \chi_2$ be a character of $T$ and let $(\pi,V)=\iota_B^G \chi$. Then, formally, we have
    $$\mathcal Z(\pi) = \mathcal Z(\chi_1) \mathcal Z(\chi_2) \subset \CC((q^{-s})).$$
    In particular, there is a unique polynomial $P_\pi \in \CC[X]$ with $P_\pi(0)=1$ such that 
    $$\mathcal Z(\pi) = P_\pi(q^{-s})^{-1} \cdot \CC[q^{-s},q^s].$$
    Moreover, $P_\pi(X) = P_{\chi_1}(X)P_{\chi_2}(X)$.
\end{prop}

We make some comments in preparation for the proof. The proposition concerns the zeta integrals 
$$\zeta\left(\Phi,f,s+\frac{1}{2}\right) = \int_{G} \Phi(x)f(x)|\det x|^{s+\frac{1}{2}} d^*x.$$

The matrix coefficients $\mathcal C(\pi)$ are spanned by 
$$\gamma_{\tau \otimes \theta} : g \mapsto \langle \tau, \pi(g) \theta \rangle$$ over $\theta \in V, \tau \in \check{V}$. Here $\theta \in \iota_B^G \chi$ is viewed as a smooth function $\theta : G \to \CC$ satisfying 
$$\theta(ntg) = \delta_B^{-1/2}(t) \chi(t) \theta(g)$$
for any $t \in T, n \in N, g \in G$. The Duality Theorem \ref{thm:duality} identifies $\check{V} \cong \iota_B^G \check{\chi}$. In this way we view $\tau$ as a smooth function $\tau: G \to \CC$ satisfying
$$\tau(ntg) = \delta_B^{-1/2}(t)\chi(t)^{-1}\tau(g)$$
for any $t \in T, n \in N, g \in G$. The proof of the Duality Theorem \ref{thm:duality} shows that the pairing between $V$ and $\check{V}$ gives
$$\gamma_{\tau \otimes \theta}(g) = \langle \tau, \pi(g)\theta \rangle = \int_{B\backslash G} \tau(x)\theta(xg) d\dot{x}$$ for a positive semi-invariant measure $d\dot{x}$ on $B \backslash G$. Let $K=\GL_2(\cO_F)$. Since we have a bijection $B \backslash G \leftrightarrow K \cap B \backslash K$ and $\delta_B(tn)=\delta_B(t) = |t_2/t_1|$ (Proposition \ref{prop:modularchar}) is trivial on $K\cap B$, we can rewrite this as 
$$\gamma_{\tau \otimes \theta}(g) = \int_K \tau(k)\theta(kg)dk$$ for some Haar measure $dk$ on $K$ (\cite[Corollary 7.6]{BH1}). Moreover, \cite[Equation 7.6.2]{BH1} tells us that there is a left Haar measure $db$ on $B$ such that
$$\int_G \phi(g) dg = \int_K \int_B \phi(bk) dbdk$$ for all $\phi \in C_c^\infty(G)$. Using this, our zeta integrals reduce to integrals over $B$ and $K$. Integration over $K$ is easier to handle using the smoothness of our representations. We can write $db = dn dt$ to view integration over $B$ as integration over $T$ and $N$. In order to relate $\zeta(\Phi,f,s+\frac{1}{2})$ to zeta functions coming from $\chi: T \to \CC^\times$, we want to express the integrals over $B$ solely in terms of integrals over $T$. To do so we use the following lemma. 

\begin{lemma}\label{lemma:phiT}
    Let $D$ be the algebra of diagonal matrices in $A$ so that $D^\times =T$. Let $\Phi \in C_c^\infty(A)$. There is a unique function $\Phi_T \in C_c^\infty(D)$ whose restriction to $T$ is given by 
    $$\Phi_T(t) = |t_1| \int_N \Phi(tn)dn, \hspace{1cm} t=\begin{psmallmatrix}
        t_1 & 0\\0&t_2
    \end{psmallmatrix}.$$
    The map $\Phi \mapsto \Phi_T$ is a linear surjection $C_c^\infty(A) \to C_c^\infty (D)$.
\end{lemma}
\begin{proof}
    The space $C_c^\infty(A)$ is spanned by functions of the form 
    $$\Phi = (\phi_{uv}): (a_{uv}) \mapsto \prod\limits_{u,v} \phi_{uv}(a_{uv})$$
    for $\phi_{uv} \in C_c^\infty (F)$ and $1 \leq u,v \leq 2$. For such $\Phi$ we compute (identifying $N \cong F$)
    
    \begin{equation*}
        \begin{split}
            \Phi_T(t) &= |t_1| \int_F \phi_{11}(t_1)\phi_{12}(t_1n)\phi_{21}(0)\phi_{22}(t_2)dn \\
            &= \phi_{11}(t_1)\phi_{22}(t_2)\phi_{21}(0) |t_1|\int_F \phi_{12}(t_1n) dn \\
            &= \phi_{11}(t_1)\phi_{22}(t_2)\phi_{21}(0) \int_F \phi_{12}(n) dn
        \end{split}
    \end{equation*}
    which uniquely extends to a function in $C_c^\infty(D)$. Surjectivity is now clear because we are free to choose $\phi_{11}$ and $\phi_{22}$.
\end{proof}
\begin{rem}
    The content of the lemma is that the function $\Phi_T$ is compactly supported, for which the introduction of the factor of $|t_1|$ is necessary.
\end{rem}

\begin{proof}[Proof of Proposition \ref{prop:gl2factor}]
    We first establish the containment $\mathcal Z(\pi) \subset \mathcal Z(\chi_1)\mathcal Z(\chi_2)$. We must show that for any $\Phi \in C_c^\infty(A)$ and $f \in \mathcal C(\pi)$ we have $\zeta(\Phi,f,s+\frac{1}{2}) \in \mathcal Z(\chi_1)\mathcal Z(\chi_2)$. Since $\mathcal C(\pi)$ is spanned by the coefficients $\gamma_{\tau \otimes \theta}$, for $\theta \in V, \tau \in \check{V}$, we assume $f$ is of this form.

    Formally expanding, using the earlier formula for $f=\gamma_{\tau \otimes \theta}$, for any $\Phi \in C_c^\infty(A)$
    \begin{equation*}
        \begin{split}
            \zeta\left(\Phi,f,s+\frac{1}{2}\right) &= \int_G \Phi(g)f(g) |\det g|^{s+\frac{1}{2}} dg \\
            &= \int_G \int_K \Phi(g) \tau(k) \theta(kg)|\det g|^{s+\frac{1}{2}} dk dg \\
        \end{split}
    \end{equation*}
    Switching the order of integration, and translating $g$ by $k^{-1}$, this is 

    \begin{equation*}
        \begin{split}
            \zeta\left(\Phi,f,s+\frac{1}{2}\right) &= \int_K \int_G \Phi(k^{-1}g) \tau(k)\theta(g) |\det g|^{s+\frac{1}{2}} dg dk \\
            &= \int_K \int_K \int_B \Phi(k^{-1}bk') \tau(k)\theta(bk') |\det b|^{s+\frac{1}{2}} db dk' dk
        \end{split}
    \end{equation*}
    where we break up the integral over $G$ as integrals over $B$ and $K$ as earlier described. Smoothness of $\Phi$, $\theta$ and $\tau$ imply there is some open normal subgroup $K_1$ of $K$ for which $\Phi$ is left and right translation invariant, and $\theta$ and $\tau$ are right translation invariant. Let $\{k_i\}$ be a finite set of coset representatives of $K/K_1$, and let $\Phi^{ij}(x) = \Phi(k_i^{-1}xk_j)$. Then $\zeta(\Phi,f,s+\frac{1}{2})$ can be expressed as a finite linear combination over $\CC$ of terms of the form 
    $$\int_B \Phi^{ij}(b) \tau(k_i)\theta(bk_j) |\det b|^{s+\frac{1}{2}} db.$$
    Using the formula $\theta(bk_j) = \delta_B^{-1/2}(t)\chi(t)\theta(k_j)$ for $b=tn \in TN =B$, we can express the above as
    $$\theta(k_j)\tau(k_i) \int_T\int_N \Phi^{ij}(tn) \chi(t)\delta_B^{-1/2}(t) |\det b|^{s+\frac{1}{2}} dn dt.$$
    We have $|\det b|=|\det t| = |t_1| |t_2|$ and $\delta_B^{-1/2}(t) = |t_2/t_1|^{-1/2}$ where $t= \begin{psmallmatrix} t_1&0 \\0&t_2 \end{psmallmatrix}$. Combining with the previous lemma, we deduce that $\zeta(\Phi,f,s+\frac{1}{2})$ can be expressed as a linear combination of terms of the form 
    $$\theta(k_j)\tau(k_i) \int_T \Phi_T^{ij}(t) \chi(t) |\det t|^s dt.$$
    If $\Phi^{ij} \in C_c^\infty(A)$ is of the form $(\phi_{uv}): (a_{uv}) \mapsto \prod\limits_{u,v} \phi_{uv}(a_{uv})$ for $\phi_{uv} \in C_c^\infty(F)$, then the above term is a scalar multiple of $\zeta(\phi_{11},\chi_1,s)\zeta(\phi_{22},\chi_2,s)$ so that $\zeta(\Phi,f,s+\frac{1}{2}) \in \mathcal Z(\chi_1)\mathcal Z(\chi_2)$. In general, $\Phi^{ij}$ is a linear combination of terms of this form, so that we always have $\zeta(\Phi,f,s+\frac{1}{2}) \in \mathcal Z(\chi_1)\mathcal Z(\chi_2)$.

    In the other direction, to show $\mathcal Z(\chi_1)\mathcal Z(\chi_2) \subset \mathcal Z(\pi)$, we wish to find $\Phi \in C_c^\infty(A)$ and $f \in \mathcal C(\pi)$ such that $\zeta(\Phi,f,s+\frac{1}{2})$ is a constant multiple of $L(\chi_1,s)L(\chi_2,s)$. We will find $f$ of the form $\gamma_{\tau \otimes \theta}$ and reverse the above calculation. Suppose we were in the situation where $\Phi$ is left and right invariant under $K$, and $\theta$ and $\tau$ are right invariant under $K$. Then the above computation shows that 
    $$\zeta\left(\Phi,f,s+\frac{1}{2}\right) = \mu(K)^2 \theta(1)\tau(1) \int_T \Phi_T(t)\chi(t)|\det t|^s dt.$$
    Therefore, if we could choose $\Phi$ left and right invariant under $K$ with $\Phi_T = \phi_1\otimes \phi_2$, where $\phi_u \in C_c^\infty(F)$ satisfy $\zeta(\phi_u,\chi_u,s)=c_uL(\chi_u,s)$ for some nonzero $c_u \in \CC$, and also choose $\theta \in \iota_B^G \chi$, $\tau \in \iota_B^G \check{\chi}$, with $\theta(1), \tau(1) \neq 0$, and $\theta$, $\tau$ right invariant under $K$, then we would be done. Unfortunately, if this was the case then $$\theta(bk) = \chi(b) \delta_B^{-1/2}(b) \theta(1)$$ for all $b \in B, k \in K$. But this is not well defined - we would require $1=\chi(b)\delta_B^{-1/2}(b) = \chi(b)$ for all $b \in B \cap K$. This only occurs when $\chi_1$ and $\chi_2$ are both unramified.

    Instead, let $K_1$ be any open normal subgroup of $K$ such that $\chi$ is trivial on $B \cap K_1$, and let $\{k_i\}$ be a finite set of coset representatives of $K/K_1$. There are then unique $\theta \in \iota_B^G \chi$ and $\tau \in \iota_B^G \check{\chi}$, each supported on $BK_1$, invariant under right translation by $K_1$, and with $\theta(1)=1=\tau(1)$. Let $f=\gamma_{\tau \otimes \theta}$.
    
    For $\Phi \in C_c^\infty(A)$ left and right invariant under $K_1$, our previous computation gives us
    $$\zeta\left(\Phi,f,s+\frac{1}{2}\right) = \mu(K_1)^2 \sum\limits_{i,j}  \int_T \theta(k_j)\tau(k_i)\Phi_T^{ij}(t)\chi(t)|\det t|^s dt.
    $$
    To control the terms over all $i,j$, we would like to choose $\Phi$ such that 
    $$\theta(k_j)\tau(k_i)\Phi_T^{ij}(t) = \Phi_T(t)$$
    for all $t \in T$, and all $i,j$ such that $k_i,k_j \in BK_1$. Then, since $\theta$ and $\tau$ are supported on $BK_1$, each term $\theta(k_j)\tau(k_i)\Phi_T^{ij}(t)$ is $\Phi_T(t)$ when $k_i,k_j \in BK_1$, and 0 otherwise, so that
    $$\zeta\left(\Phi,f,s+\frac{1}{2}\right) = c \int_T \Phi_T(t) \chi(t) |\det t|^s dt$$ for some $c>0$. If $k_j = b_jk \in BK_1$, then $\theta(k_j) = \chi(b_j)\delta_B^{-1/2}(b_j)\theta(1) = \chi(b_j)$ because $\delta_B=1$ on $B \cap K$. Similarly, if $k_i=b_ik \in BK_1$, then $\tau(k_i)=\chi(b_i)^{-1}$. If we assume that $\Phi$ is left and right invariant under $K_1$, the condition $$\theta(k_j)\tau(k_i)\Phi_T^{ij}(t) = \Phi_T(t),$$ reduces to the condition
    $$\chi(b_j)\chi(b_i)^{-1} \int_N \Phi(b_i^{-1}tnb_j) dn = \int_N \Phi(tn)dn$$ for all $b_i,b_j \in B \cap K_1$, as functions of $t \in T$. In fact, we would like for the stronger condition
    $$\chi(b_j)\chi(b_i)^{-1}\Phi(b_i^{-1}bb_j) = \Phi(b)$$
    to hold, for any $b_i,b_j \in B\cap K_1$ and $b \in B$.

    To summarize, we want to construct a pair $(\Phi,K_1)$ with $\Phi \in C_c^\infty(A)$ and $K_1$ a sufficiently small (so that $\chi$ is trivial on $B \cap K_1$) open normal subgroup of $K$ with the following properties:
    \begin{itemize}
        \item The function $\Phi$ is invariant under left and right translation by $K_1$.
        \item For all $b_i,b_j \in B \cap K_1$ and $b \in B$ we have $$\chi(b_j)\chi(b_i)^{-1}\Phi(b_i^{-1}bb_j) = \Phi(b).$$
        \item For our chosen $\phi_1,\phi_2 \in C_c^\infty(F)$ satisfying $\zeta(\phi_u,\chi_u,s)=c_uL(\chi_u,s)$ for some $c_u \in \CC^\times$, we have $\Phi_T = c \cdot \phi_1 \otimes \phi_2 \in C_c^\infty(D)$ for some $c \in \CC^\times$.
    \end{itemize}
    We can remove the dependence on $K_1$ by strengthening the second condition above, and now ask for $\Phi \in C_c^\infty(A)$ with the following properties:
    \begin{itemize}
        \item For all $x,y \in B \cap K$ and $b \in B$ we have $$\chi(xy)\Phi(xby) = \Phi(b).$$
        \item For some $\phi_1,\phi_2 \in C_c^\infty(F)$ satisfying $\zeta(\phi_u,\chi_u,s)=c_uL(\chi_u,s)$ for some $c_u \in \CC^\times$, we have $\Phi_T = c \cdot \phi_1 \otimes \phi_2 \in C_c^\infty(D)$ for some $c \in \CC^\times$.
    \end{itemize}
    If we take $\Phi$ of the form $\Phi=(\phi_{uv})$, and set $\phi_{12}=\phi_{21}=\mathbbm{1}_{\mathcal O_F}$, then the computation of Lemma \ref{lemma:phiT} shows that for $t= \begin{psmallmatrix}
        t_1&0\\0&t_2
    \end{psmallmatrix}$,
    $$\Phi_T(t) = \mu(\cO_F)\phi_{11}(t_1)\phi_{22}(t_2).$$
    Taking $\phi_{uu}=\phi_u$, it suffices to find for each $u=1,2$ some $\phi_u \in C_c^\infty(F)$ such that
    \begin{itemize}
        \item For all $x,y \in \cO_F^\times$ and $a \in F^\times$ we have $$\chi_u(xy)\phi_u(xay) = \phi_u(a).$$
        \item We have $\zeta(\phi_u,\chi_u,s)=c_u \cdot L(\chi_u,s)$ for some $c_u \in \CC^\times$.
    \end{itemize}
    Here we divide into cases. If $\chi_u$ is unramified, then we may take $\phi_u = \mathbbm{1}_{\cO_F}$ by the proof of Proposition \ref{prop:gl1factor}. If $\chi_u$ is ramified, and the restriction to $U_F^n$ is trivial, then we take 
    $$ \phi_u = \sum\limits_{z \in \cO_F^\times/U_F^n} \chi_u(z)^{-1} \mathbbm{1}_{zU_F^n}.$$ One sees that this satisfies the first condition. For the second we have 
    $$\zeta(\phi_u,\chi_u,s) = \sum\limits_z \int_{U_F^n} \chi_i(z)^{-1}\chi_i(zx)|x|^s d^*x = \mu(\cO_F^\times)$$ which is a constant (and $L(\chi_u,s)=1$ in the ramified case). We have proven $\mathcal Z(\chi_1)\mathcal Z(\chi_2) \subset \mathcal Z(\pi)$.
\end{proof}

\begin{rem}
    The computations of Proposition \ref{prop:gl2factor} show that each $\zeta(\Phi,f,s)$ converges absolutely and uniformly in vertical strips in some right half plane, and admit analytic continuation to a rational function in $q^{-s}$.
\end{rem}


\begin{defn}
    Define the \textit{$L$-function} attached to $\pi = \iota_B^G \chi$, where $\chi=\chi_1\otimes \chi_2$ is a character of $T$, to be $$L(\pi,s) = P_\pi(q^{-s})^{-1} = L(\chi_1,s)L(\chi_2,s),$$
    where the $L(\chi_i,s)$ are the $L$-functions defined
    in \ref{def:lfunction}.
\end{defn}

\begin{rem}\label{BHThm1}
    As one might hope, it is also possible to attach an $L$-function to any irreducible representation of $G$ in a similar way. That is, for any irreducible smooth representation $\pi$ of $G$, there is a unique polynomial $P_\pi(X)\in\CC[X]$ such that $P_\pi(0)=1$ and $\mathcal Z(\pi) = P_\pi(q^{-s})^{-1} \CC[q^{-s},q^s],$ and one defines $L(\pi,s)=P_\pi(q^{-s})^{-1}$ (see \cite[Theorem 24.2.1]{BH1}).
\end{rem}

We have not studied cuspidal representations, so we will not prove this fact in general. However, we will give an explicit description of the $L$-functions associated to principal series representations. The following lemma is the next natural step towards this direction.

\begin{lemma}
    Let $\chi=\chi_1\otimes\chi_2$ be a character of $T$ and let $\pi$ be a $G$-composition factor of $\Sigma=\iota_B^G\chi$. Then there exits one unique polynomial $P_\pi(X)\in\CC[X]$ such that $P_\pi(0)=1$ and $\mathcal{Z}(\pi)=P_\pi(q^{-s})^{-1}\CC[q^{-s},q^s]$. Moreover, $P_\pi(X)$ divides $P_\Sigma(X)=P_{\chi_1}(X)P_{\chi_2}(X)$.
\end{lemma}
\begin{proof}
    By Lemma \ref{lem:ZmodGL2}, $\mathcal{Z}(\pi)$ is a $\CC[q^{-s},q^s]$-module containing $\CC[q^{-s},q^s]$. On the other hand, we can view matrix coefficients of $\pi$ as matrix coefficients of $\Sigma$, so it follows that $\CC[q^{-s},q^s]\subseteq\mathcal{Z}(\pi)\subseteq\mathcal{Z}(\Sigma)$. Since $\mathcal{Z}(\pi)$ is a $\CC[q^{-s},q^s]$-module, $\CC[X^{-1},X]$ is a principal ideal domain and $\mathcal{Z}(\Sigma)$ is generated by $P_\Sigma(q^{-s})$ as a $\CC[q^{-s},q^s]$-module, then $\mathcal{Z}(\pi)$ is also generated by one element $P_\pi(q^{-s})$, where $P_\pi(X)$ is a polynomial dividing $P_\Sigma(X)$ and unique under the condition that $P_\pi(0)=1$.
\end{proof}

We are finally ready to describe the $L$-functions associated to the remaining principal series representations. Recall that if $\Sigma=\iota_B^G(\chi_1\otimes\chi_2)$ is irreducible, then we already know that $L(\Sigma,s)=L(\chi_1,s)L(\chi_2,s)$. Therefore, we may assume that $\Sigma$ is reducible. By swapping $\chi_1$ and $\chi_2$ if necessary, we may assume that $\chi_1(x)\chi_2^{-1}(x)=|x|$, in which case $\chi_1(x)=\phi(x)|x|^{1/2}$ and $\chi_2(x)=\phi(x)|x|^{-1/2}$ for some character $\phi$ of $F^\times$. The $G$-composition factors of $\Sigma$ are precisely $\phi\circ\det$ and $\phi\St_G$. We distinguish $3$ cases.

\begin{enumerate}[(1)]
    \item If $\phi$ is ramified, then so are $\chi_1$ and $\chi_2$. In that case, $L(\chi_1,s)=L(\chi_2,s)=1$ and therefore $L(\phi\circ\det,s)=L(\phi\St_G,s)=1$.
    \item If $\phi$ is unramified and $\pi=\phi\circ\det$, then $\chi_1$ and $\chi_2$ are both unramified and so $P_\pi(X)$ has degree at most $2$. However, the calculation in Example \ref{example:zetafncgl2}(1)(a) shows that $P_\pi(X)$ has degree at least $2$. Hence,
    $$L(\phi\circ\det,s)=L(\chi_1,s)L(\chi_2,s)=L(\phi|\cdot|^{1/2},s)L(\phi|\cdot|^{-1/2},s)=L\left(\phi,s+\frac{1}{2}\right)L\left(\phi,s-\frac{1}{2}\right).$$
    \item If $\phi$ is unramified and $\pi=\phi\St_G$, then $P_\pi(X)\mid P_{\chi_1}(X)P_{\chi_2}(X)$. Since $P_{\chi_1}(X)$ and $P_{\chi_2}(X)$ both have degree $1$, there are $4$ distinct options for $P_\pi(X)$. The proof of this case is hard, and we only sketch the idea. Firstly, one needs to show that $P_{\chi_1}(X)$ divides $P_\pi(X)$. This reduces the number of possibilities to $L(\pi,s)=L(\phi,s+1/2)$ or $L(\pi,s)=L(\phi,s+1/2)L(\phi,s-1/2)$. To distinguish between two options, one needs to use complex analytic tools to prove the following.
    \begin{lemma}
        Let $f\in\mathcal{C}(\pi)$. Then then the zeta-integral $\zeta(\Phi,f,s)$ converges absolutely for $\mathrm{Re }s>1/2$ and any $\Phi\in C_c^\infty(A)$.
    \end{lemma}
    \begin{proof}
        \cite[Lemma 26.8]{BH1}
    \end{proof}
    Since the function $L(\phi,s-1/2)$ does not converge absolutely for $\mathrm{Re} s>1/2$, the above lemma implies that
    $$L(\phi\St_G,s)=L\left(\phi,s+\frac{1}{2}\right).$$
\end{enumerate}

\iffalse
For context, we state more general versions of these results that hold for any irreducible smooth representation $\pi$ of $G$.

\begin{thm}\label{}
    Let $\pi$ be an irreducible smooth representation of $G$. There is a unique polynomial $P_\pi(X) \in \CC[X]$, satisfying $P_\pi(0)=1$, and 
    $$\mathcal Z(\pi) = P_\pi(q^{-s})^{-1} \CC[q^{-s},q^s].$$
\end{thm}
\begin{proof}
    \cite[Theorem 24.2.1]{BH1}.
\end{proof}

\begin{notn}
    Set $L(\pi,s) = P_\pi(q^{-s})^{-1}$.
\end{notn}

\textbf{It remains to describe the L-functions of $\phi\circ\det$ and $\phi\St_G$. This could probably be done here, using the examples from earlier.}
\fi


