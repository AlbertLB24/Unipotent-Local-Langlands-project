

\documentclass{article}

%%%%%%%%%%%%%%%%%%%%%%%%%%%%%%%%%%%%%%%%%%%%%%%%%%%%%%%%%%%%%%%%%%%%%%%%%%%%%%%%
%% package setup
%%%%%%%%%%%%%%%%%%%%%%%%%%%%%%%%%%%%%%%%%%%%%%%%%%%%%%%%%%%%%%%%%%%%%%%%%%%%%%%%



\usepackage[shortlabels]{enumitem}
\usepackage{amsfonts,amsmath, soul, matlab-prettifier, bm, amsthm,enumitem,amssymb,multirow,float,mathtools,bbm,array,varwidth,hyperref}
\usepackage[all]{xy}
\usepackage[margin=0.9in]{geometry}
\usepackage{graphicx}
\usepackage{mathtools, caption, dsfont}

\mathchardef\mhyphen="2D


\raggedbottom

%%%%%%%%%%%%%%%%%%%%%%%%%%%%%%%%%%%%%%%%%%%%%%%%%%%%%%%%%%%%%%%%%%%%%%%%%%%%%%%%
%% operators and symbols
%%%%%%%%%%%%%%%%%%%%%%%%%%%%%%%%%%%%%%%%%%%%%%%%%%%%%%%%%%%%%%%%%%%%%%%%%%%%%%%%

% operators
\newcommand{\Hom}{\mathrm{Hom}}
\newcommand{\Ker}{\mathrm{Ker}}
\newcommand{\Pic}{\mathrm{Pic}}
\newcommand{\Proj}{\mathrm{Proj}}
\newcommand{\rk}{\mathrm{rk}}
\newcommand{\Spec}{\mathrm{Spec}}
\newcommand{\Sym}{\mathrm{Sym}}
\newcommand{\Frob}{\mathrm{Frob}}
\newcommand{\Gal}{\mathrm{Gal}}
\newcommand{\GL}{\mathrm{GL}}
\newcommand{\SL}{\mathrm{SL}}
\newcommand{\Ind}{\mathrm{Ind}}
\newcommand{\Rep}{\mathrm{Rep}}
\newcommand{\Aut}{\mathrm{Aut}}
\newcommand{\Res}{\mathrm{Res}}
\newcommand{\Smo}{\mathrm{Smo}}
\newcommand{\Span}{\mathrm{Span}}
\newcommand{\supp}{\mathrm{supp}}
\newcommand{\End}{\mathrm{End}}





% mathcal
\newcommand{\cO}{\mathcal{O}}

% mathbb
\newcommand{\CC}{\mathbb{C}}
\newcommand{\FF}{\mathbb{F}}
\newcommand{\NN}{\mathbb{N}}
\newcommand{\PP}{\mathbb{P}}
\newcommand{\QQ}{\mathbb{Q}}
\newcommand{\RR}{\mathbb{R}}
\newcommand{\ZZ}{\mathbb{Z}}
\newcommand{\GG}{\mathbb{G}}
\newcommand{\adele}{\mathbb{A}}
\newcommand{\pp}{\mathfrak{p}}

% mathfra
\newcommand{\fg}{\mathfrak{g}}
\newcommand{\fm}{\mathfrak{m}}

% shortcuts
\newcommand{\CG}{C_c^{\infty}(G)}
\newcommand{\cInd}{c\mhyphen\mathrm{Ind}}

\newcommand{\norm}[1]{\left\lVert#1\right\rVert}
\newcommand{\hatv}[1]{\overset{\vee}{\mathstrut#1}}

\DeclareMathOperator{\Ima}{Im}

\linespread{1.5}

\theoremstyle{plain}
\newtheorem{thm}{Theorem}[section]
\newtheorem{question}[thm]{Question}
\newtheorem{prop}[thm]{Proposition}
\newtheorem{convention}[thm]{Convention}
\newtheorem{lemma}[thm]{Lemma}
\newtheorem{cor}[thm]{Corollary}
\newtheorem{algo}[thm]{Algorithm}
\theoremstyle{definition}
\newtheorem{defn}[thm]{Definition}
\newtheorem{notn}[thm]{Notation}
\newtheorem{rem}[thm]{Remark}
\newtheorem{example}[thm]{Example}
\newtheorem{examples}[thm]{Examples}
\newtheorem{fact}[thm]{Fact}
\newtheorem*{hypothesis}{Hypothesis}


\title{Local Langlands for $\mathrm{GL}_2$}
\author{Yiannis Fam, Albert Lopez Bruch, Jakab Schrettner}

\begin{document}
	\maketitle
	\pagenumbering{arabic}

\section{Locally Profinite Groups and Smooth Representations}
The aim of this first section is to motivate the abstract notions of a \textit{locally profinite group} and of a \textit{smooth representation}, which will be the main objects of study during the later subsections. Most of the motivation for studying such groups comes from the study of representations of reductive groups over non-Archimedean local fields. Therefore, we begin this section by briefly recalling some basic facts about these fields and linear groups associated to them. For the sake of brevity, we will omit proofs, and therefore we assume familiarity with the subject. For the unfamiliar reader, we encourage them to read \textbf{(insert here Gouvea reference, or others)}, where a detailed explanation is provided. 


\subsection{Local Fields and Locally Profinite Groups}
We begin by recalling some basic objects from algebraic number theory. Given a field $F$, a \textit{discrete valuation} on $F$ is a surjective function $\nu: F\to\ZZ\cup\{\infty\}$ satisfying the conditions

\begin{enumerate}
    \item $\nu(xy)=\nu(x)+\nu(y)$ for any $x,y\in F$ 
    \item $\nu(x+y)\geq\min\{\nu(x),\nu(y)\}$ for any $x,y\in F$.
    \item $\nu(x)=\infty$ if and only if $x=0$.
\end{enumerate}

Any discrete valuation $\nu$ induces an absolute value on $F$ given by the formula 
$$|x|=c^{\nu(x)}$$ 
for any $c\in(0,1)$, and therefore it also induces a topology on $F$. This topology is independent of the choice of $c$. One easily checks that this absolute value satisfies $|x+y|\leq\max\{|x|,|y|\}$ for any $x,y\in F$. Absolute values with this property are called \textit{non-Archimedean}. 

A field $F$ with an absolute value $|\cdot|$ induced by a discrete valuation $\nu$ is the fraction field of the \textit{valuation ring}
$$\mathcal{O}_F:=\{x\in F:v(x)\geq 0\}=\{x\in F: |x|\leq1\},$$ 
which contains a unique maximal ideal
$$\pp:=\{x\in F:v(x)> 0\}=\{x\in F: |x|<1\},$$
the \textit{valuation ideal} or the \textit{ring of integers of $F$}. The valuation ideal is principal, and it is generated by any $\varpi\in F$ with $\nu(\varpi)=1$. Such an element is called a \textit{uniformiser} of $F$. Finally, the \textit{residue field} $\kappa$ of $F$ is the quotient $\mathcal{O}_F/\pp$. This motivates the following important definition.

\begin{defn}
    A field $F$ is a \textit{non-Archimedean local field} if it is complete with respect to a topology induced by a discrete valuation and the residue field is finite.
\end{defn}

\begin{rem}
    When the residue field is finite, it is conventional to define the absolute value on $F$ by 
    $|x|=q^{-\nu(x)},$
    where $q=|\kappa|$. From here onwards, we will follow this convention.
\end{rem}
\begin{rem}
    Local fields are ubiquitous in number theory. They arise as completions of number fields at non-Archimedean places in characteristic 0, or as completions of finite extensions of $\FF_p(t)$ at non-Archimedean places in positive characteristic.
\end{rem}

Let us now discuss important aspects of the topology on $F$ and $\mathcal{O}_F$ induced by the discrete valuation $\nu$. We have already seen that $\mathcal{O}_F$ is a local ring with maximal ideal $\pp$ and therefore $\mathcal{O}_F^\times=\mathcal{O}_F\setminus\pp$ is the set of units of $\mathcal{O}_F$. The ideals 
$$\pp^n=\{x\in F:\nu(x)\geq n\}=\{x\in F: |x|\leq q^{-n}\}=\varpi^n \mathcal{O}_F,\quad n\in\ZZ$$
are a complete set of fractional ideals of $\mathcal{O}_F$ and, since the valuation is assumed to be discrete, they are also open subsets of $F$.
Therefore, they are a fundamental system of neighbourhoods of the identity. A direct consequence of this fact implies that $F$ (and therefore $\mathcal{O}_F$) are totally disconnected topological rings.

Furthermore, the ring $\mathcal{O}_F$ is a closed subring of $F$, which is assumed to be complete. Hence, $\mathcal{O}_F$ is also complete, and a stardard topological argument shows that $\mathcal{O}_F$ is in fact compact. This proves that $\mathcal{O}_F$ (and therefore any $\pp^n$) is in fact a profinite group, and we have a topological isomorphism 
$$\mathcal{O}_F\longrightarrow\varprojlim_{n\geq 1} \mathcal{O}_F/\pp^n\quad x\mapsto (x\ (\textrm{mod }{\pp^n}))_{n\geq 1}$$
where the maps implicit in the right hand side are the obvious ones.

However, $F$ itself is clearly not compact, and therefore it is not profinite. Nevertheless, $F$ has the important property that any open neighbourhood of the identity contains an open compact (and therefore profinite) subgroup - some $\pp^n$ for a sufficiently large $n$.

We are now ready to give the main definition of this section, which encapsulates this last property in greater generality.

\begin{defn}\label{loc_prof_grp}
    A topological group $G$ (which we always assume to be Hausdorff) is a \textit{locally profinite group} if every open neighbourhood of the identity contains a compact open subgroup. 
\end{defn}

In this report we will be interested in studying the representation theory of many important groups and rings related to the local field $F$. The notion of a locally profinite group is an abstract one, but it has the great advantage of accomodating many important groups and rings associated to non-Archimedean local fields and their representation theory.

\begin{examples} \label{example_prof_groups}

    \begin{enumerate}[(1)]
        \item Trivially, any group equipped with the discrete topology is profinite, where $\{e\}$ is the fundamental neighbourhood.
        \item In the preceding discussion, we have shown that the local field $F$ is a locally profinite group, where $\pp^n$ for $n\geq1$ is a fundamental system of open compact subgroups. We remark that $F$ satisfies the rather special property of being the union of its open compact subgroups. %This fact has relevant consequences that will be discuss later.
        \item The multiplicative group $F^{\times}$ is also a locally profinite group, where the congruence unit groups $U_F^n=1+\pp^{n}$ for $n\geq1$ is a fundamental system of open compact subgroups. Unlike $F$, the group $F^{\times}$ is not the union of its open compact subgroups.
        \item Given $m\geq1$ an integer, the additive group $F^m=F\times\dots\times F$ is also a locally profinite group endowed with the product topology. A fundamental system of open compact subgroups is given by $\pp^{n}\times\dots\times\pp^{n}$ for $n\geq0$. More generally, any product of locally profinite groups is locally profinite.
        \item The matrix ring $M_m(F)$ is also locally profinite since it is isomorphic to $F^{m^2}$ as additive groups. The open compact subgroups $\pp^n M_m(\mathcal{O}_F)$ are a fundamental system of neighbourhood of the identity.
        \item The group $\GL_m(F)$ of invertible matrices is an open subset of $M_m(F)$ since $\det:M_m(F)\rightarrow F$ is continuous and $F^{\times}$ is an open subset of $F$. Furthermore, mutiplication by a matrix $A\in M_m(F)$ and inversion of matrices are continuous maps in $M_m(F)$, and therefore $\GL_m(F)$ is also a topological group. The subgroups
        $$K=\GL_m(\mathcal{O}_F),\quad K_n=1+\pp^{n}M_m(\mathcal{O}_F),\quad n\geq 1,$$
        are compact open, and a fundamental system of neighbourhoods of the identity.
        \item Let $G$ be a locally profinite group and $H\leq G$ be a closed subgroup. Then $H$ is also a locally profinite group. If in addition $H$ is assumed to be normal in $G$, then $G/H$ is locally profinite. 
        
        %If $U\subseteq H$ is a neighbourhood of the identity on $H$, then there is some $V$ open in $G$ such that $U=H\cap V$. Let $K\subseteq V$ be some open compact subgroup of $G$. Then $K\cap H$ is an open subgroup of $H$ and a closed subgroup of $K$. But since $K$ is compact and Hausdorff, $K\cap H$ is also compact. This shows that $H$ is also a locally profinite subgroup.
    \end{enumerate}
\end{examples}

We give some further insight into the terminology used. It is an easy exercise to prove that a profinite group is compact and locally profinite. Rather strikingly, the converse also holds. That is, if $K$ is a compact locally profinite group, then
$$K\longrightarrow\varprojlim_N K/N$$
is a topological isomorphism, where $N$ ranges over the normal open subgroups. Since $K$ is compact and $N$ is open, $K/N$ must be finite and discrete, showing that $K$ is profinite.


\subsection{Abstract Representations of Groups} \label{Abstract_Reps}
Before discussing the representation theory of locally profinite groups, we first review some general results and constructions of representations of arbitrary groups $G$. We begin by recalling the notion of a representation.

\begin{defn}
    A \textit{representation} of a group $G$ over a field $k$ is a pair $(\pi,V)$ where $V$ is a $k$-vector space and $\pi:G\rightarrow\GL(V)$ is a group homomorphism. We say that $\dim V$ is the \textit{dimension} of the representation.
\end{defn}

Equivalently, a representation of $G$ is a $k$-vector space $V$ equipped with a $k$-linear $G$-action. Whenever the representation is clear from the context, we will omit $\pi$ from the notation and write $g\cdot v$ for $\pi(g)v$. 

Throughout this document we will mostly be interested in complex representations, so from now on we will assume that $k=\CC$ unless otherwise stated.

We say that $U\leq V$ is a \textit{$G$-subspace} if $U$ is closed under the $G$-action; i.e., if $g\cdot U\subseteq U$ for every $g\in G$. When this happens, both $U$ and $V/U$ are naturally $G$-representations. We say that that a representation $(\pi,V)$ is \textit{irreducible} (or \textit{simple}) if $V$ has no non-trivial $G$-subspaces. These are the building blocks of more complicated representations, and thus we are often interested in classifying them.


\begin{defn}
    A representation $(\pi,V)$ of a group $G$ is \textit{semisimple} if it is the direct sum of simple subrepresentations. 
\end{defn}

\begin{rem}\label{rem_semisimple}
    If $G$ is a finite group, Maschke's Theorem shows that all finite dimensional complex representations of $G$ are semisimple. As a consequence, one can show that any complex irreducible representation of $G$ is finite dimensional, appearing as a subrepresentation of the regular representation $\CC G$. 
    
    As we shall see later in this chapter, continuous finite-dimensional representations of profinite groups also share these properties. However, it is easy to construct representations of locally profinite groups which are continuous yet not semisimple. For example,
    \begin{align*}
        \phi:\ZZ\longrightarrow \GL_2(\CC)\\
        n\mapsto 
        \begin{pmatrix}
            1 & n\\
            0 & 1\\
        \end{pmatrix}
    \end{align*}
    has a single one-dimensional invariant subspace. One can also construct irreducible representations that are infinite dimensional; we will meet some examples in Section \ref{sec:principal}.
\end{rem}


Naturally, we also define the notion of a morphism of representations.

\begin{defn}
    A morphism between two complex representations $(\pi,V)$, $(\sigma,W)$ of a group $G$ is a linear map $\phi:V\rightarrow W$ compatible with the $G$ action. That is, 
    $$\phi(\pi(g)v)=\sigma(g)\phi(v)\ \text{for all } g\in G,\ v\in V.$$
\end{defn}

This turns the set of complex representations of $G$ into a category, denoted $\Rep(G)$, which is an \textit{abelian category}.

We finish this subsection by introducing important constructions and functors between these categories that allow us to obtain new representations from old ones, which we will use heavily later on.

\begin{defn}
    Given $(\pi,V)\in\mathrm{Rep}_G$, define the dual space $V^*=\Hom(V,\CC)$, and denote by 
    \begin{align*}
        V^*\times V\longrightarrow \CC,\\
        (v^*,v)\longmapsto\langle v^*,v\rangle,
    \end{align*}
    the canonical evaluation homomorphism. Then $V^*$ carries a natural representation of $G$ defined by
    $$\langle\pi^*(g)v^*,v\rangle=\langle v^*,\pi(g^{-1})v\rangle.$$
    This is the \textit{dual representation} of $V$, and the functor 
    \begin{align*}
        (-)^*:\Rep(G)&\longrightarrow\Rep(G)\\
        (\pi,V)&\longrightarrow(\pi^*,V^*)
    \end{align*}
    is an additive and exact contravariant functor.
\end{defn}

One can also consider the composition of this functor with itself to obtain the \textit{double dual} $(\pi^{**},V^{**})$. There is a canonical $G$-homomorphism $\delta:V\rightarrow V^{**}$ such that $$\langle\delta(v),v^*\rangle_{V^*}=\langle v^*,v\rangle_{V}.$$
When $V$ is finite dimensional, $\delta$ is a $G$-isomorphism. For general representations of locally profinite groups, this is not always the case, but under additional assumptions it is possible to give a precise criterion that determines when $\delta$ is bijective (\cite[Corollary 2.8, Proposition 2.9]{BH1}).


\begin{defn}\label{def:absresind}
    Let $H\leq G$ be groups and let $(\pi,V)$ and $(\sigma,W)$ be representations of $G$ and $H$ respectively. The restriction of $\pi$ to $H$ gives a \textit{restriction} functor
    \begin{align*}
        \Res_H^G:\Rep(G)&\longrightarrow\Rep(H)\\
        (\pi,V)&\longmapsto(\pi|_H,V)
    \end{align*}
    On the other hand, given $(\sigma,W)\in\Rep(H)$, one can define the vector space
    $$X=\{f:G\to W:f(hg)=\sigma(h)f(g)\text{ for all }h\in H, g\in G\},$$
    equipped with the $G$-action $\Sigma:G\longrightarrow\Aut_\CC(X)$ defined by right translation:
    $$\Sigma(g)f:x\longmapsto f(xg),\ x,g\in G.$$
    This defines the \textit{induction} functor
    \begin{align*}
        \Ind_H^G:\Rep(H)&\longrightarrow\Rep(G)\\
        (\pi,V)&\longmapsto(\Sigma,X).
    \end{align*}
\end{defn}

As with the dual functor, both the restriction and induction functors are additive and exact, but are now covariant functors. To simplify notation, we will write $\Ind_H^G\sigma$ instead of $\Ind_H^G(\sigma,W)$, which is the usual convention in the literature.

We remark that one can construct the following canonical $H$-homomorphisms 
\begin{align*}
    a_\sigma: \Ind_H^G\sigma&\longrightarrow W\\
    f&\longmapsto f(1)
\end{align*}
and 
\begin{align*}
    a_\sigma^c: W&\longrightarrow \Ind_H^G\sigma,\\
    w&\longmapsto f_w
\end{align*}
where $f_w$ is supported in $H$ and $f_w(h)=\sigma(h)w$ for $h\in H$. The choice of notation will be understood later. These, in turn, induce the maps 
\begin{align*}
    \Psi:\Hom_G(\pi,\Ind_H^G\sigma)&\longrightarrow\Hom_H(\Res_H^G\pi,\sigma),\\
    \phi&\longmapsto a_\sigma\circ\phi,
\end{align*}
and
\begin{align*}
    \Psi^c:\Hom_G(\Ind_H^G\sigma,\pi)&\longrightarrow\Hom_H(\sigma,\Res_H^G\pi),\\
    f&\longmapsto f\circ a_\sigma^c.
\end{align*}
When $G$ is a finite group, we have the following result.
\begin{thm}[Frobenius reciprocity]
    Let $G$ be a finite group. Then the maps $\Psi$ and $\Psi^c$ are bijections that are natural in both variables $\sigma$ and $\pi$. In categorical terms, we have the adjunctions $$\Ind_H^G\dashv\Res_H^G\dashv\Ind_H^G.$$
\end{thm}

There is an analogue of Frobenius reciprocity for locally profinite groups; see Theorem \ref{thm:frob} and Theorem \ref{thm:frob2}.
\subsection{Characters of Local Fields}

Now we turn our attention to the representation theory of locally profinite groups. It turns out that there are too many representations, so we need to restrict our attention to those representations satisfying a certain smoothness condition. To motivate this condition, we will first describe the simplest case: one-dimensional representations of a local field $F$: that is, group homomorphisms $\phi:F\rightarrow\CC^{\times}$. Later in this section we will also study the one-dimensional representations of $F^{\times}$.

As we have discussed in the previous section, locally profinite groups carry a topology, so a natural condition to impose is \textit{continuity} with respect to the usual topologies in $\CC^{\times}$ and $G$. A \textit{character} of $G$ is a continuous homomorphism $\psi:G\rightarrow\CC^{\times}$.

Characters of a locally profinite group $G$ form a group $\hat{G}$ under multiplication. It turns out that for one-dimensional representations, continuity coincides with the smoothness condition we will introduce later. 

When $G$ is a finite group with the discrete topology, then any one-dimensional representation is a character, and we have the following simple description.

\begin{prop}
    If $G$ is a finite group with the discrete topology, then $\hat{G}\cong G^{ab}$. In particular, if $G$ is abelian then $\hat{G}\cong G$.
\end{prop}
\begin{proof}
    \textbf{Insert reference here}
\end{proof}

For general locally profinite results, we have this rather surprising result. 
\begin{lemma}\label{lem_cont_chars}
    Let $G$ be a locally profinite group and $\psi: G\rightarrow\CC^{\times}$ a homomorphism. Then $\psi$ is continuous if and only if $\ker\psi$ is open in $G$. Furthermore, if $G$ is the union of its compact open subgroups, then $$\psi(G)\subseteq\{z\in\CC^{\times}:|z|=1\}=S^1.$$
\end{lemma}
\begin{rem}
    Characters of locally profinite groups that have image in $S^1$ are called \textit{unitary}.
\end{rem}
\begin{proof}
    If $\ker\psi=\psi^{-1}(1)$ is open in $G$, then for any $z\in\Ima\psi$, the preimage $\psi^{-1}(z)=g\ker\psi$ is also open, for any $g \in G$ satisfying $\psi(g)=z$. Then for any $U\subseteq\CC^{\times}$, 
    $$\psi^{-1}(U)=\bigcup_{z\in U\cap\Ima\psi}\psi^{-1}(z),$$
    so that $\psi$ is continuous.
    Conversely, if $\psi$ is continuous, then for any open neighbourhood $\mathcal{N}$ of $1$, $\psi^{-1}(\mathcal{N})$ contains an open compact subgroup $K$ of $G$. But $\mathcal{N}$ can be chosen sufficiently small so that it does not contain any non-trivial subgroup of $\CC^{\times}$. Hence, $\psi(K)=1$, so $K\subseteq\ker\psi$, and since $K$ is open, so is $\ker\psi$.
    The last assertion is a direct consequence of the fact that the continuous image of a compact set is compact, and $S^1$ is the unique maximal compact subgroup of $\CC^{\times}$.
\end{proof}


Since the local field $F$ is the union of its open compact subgroups, all characters of $F$ are unitary. However, this is not the case for $F^{\times}$. Indeed, the map $x\mapsto|x|$ is a character of $F^{\times}$ that is not unitary. 

Before stating the classification theorem for characters of $F$, we need one last definition. 

\begin{defn}
    Let $\psi$ be a non-trivial character of $F$ (resp. of $F^{\times}$). The \textit{level} of $\psi$ is defined as the least integer $d\geq0$ such that $\pp^d\subseteq\ker\psi$ (resp. $U_F^{d+1}\subseteq\ker\psi$).
\end{defn}

\begin{lemma}
    Let $\psi\in\hat{F}$ be a character of level $d$ and let $a\in F$. Then the map $a\psi:x\mapsto\psi(ax)$ is a character of $F$, and if $a\neq0$ then $a\psi$ has level $d-\nu_F(a)$.
\end{lemma}
\begin{proof}
    It is clear that $a\psi$ is a character since if $x\in\pp^{d-\nu_F(a)}$, then $ax\in\pp^d$, so $a\psi(x)=1$, and therefore $\pp^{d-\nu_F(a)}\subseteq\ker(a\psi)$ and the kernel of $a\psi$ is open. Furthermore, there is some $y\in\pp^{d-1}$ such that $\psi(y)\neq1$, and so $a\psi(a^{-1}y)\neq1$. Since $a^{-1}y\in\pp^{d-1-\nu_F(a)}$, this indeed shows that the level of $a\psi$ is $d-\nu_F(a)$. 
\end{proof}

We are now ready to give the classfication theorem for $\hat{F}$.

\begin{thm}[Additive Duality]\label{add_dual}
    Let $\psi\in\hat{F}$ be a character of level $d$. The map $a\mapsto a\psi$ induces an isomorphism $F\cong\hat{F}$. 
\end{thm}

The proof of surjectivity of the theorem requires an inductive step, which heavily relies on the following results.

\begin{lemma}\label{lem_congruence}
    Let $\psi\in\hat{F}$ be a character of level $d$ and let $u,u'\in U_F$ be two units of $F$. Then $u\psi$ coincides with $u'\psi$ on $\pp^{d-n}$ if and only if $u'u^{-1}\in U_F^{n}$.
\end{lemma}
\begin{proof}
    Let $\alpha=\nu_F(u-u')$. A simple definition chase shows that $u\psi$ and $u'\psi$ agree on $\pp^{d-n}$ if and only if $\pp^{d-n+\alpha}=(u-u')\pp^{d-n}\subseteq\ker\psi$. By definition of level, this is the case if and only if $\alpha\geq n$; that is, if $u\equiv u'\pmod{\pp^n}$ or $u'u^{-1}\in U_F^{n}$.
\end{proof}

\begin{lemma}\label{lem_chars}
    Let $\theta:\pp^{n}\rightarrow\CC^{\times}$ be a character. Then there are exacty $q$ characters $\Theta$ of $\pp^{n-1}$ such that $\Theta|_{\pp^n}=\theta$.
\end{lemma}

\begin{proof}
    Since $\hat\kappa\cong\kappa$, where $\kappa$ is the residue field of $F$, it is enough to construct a bijection between $\mathcal{A}:=\{\Theta\in\widehat{\pp^{n-1}}:\Theta|_{\pp^n}=\theta\}$ and $\hat{\kappa}$. Let $\phi=\theta^{-1}$ and let $\Phi$ be \textit{any} lift of $\phi$ as a character of $\pp^{n-1}$. Now given $\Theta\in\mathcal{A}$, the character $\Theta\cdot\Phi$ is trivial on $\pp^{n}$ and thus it descends to a map 
    $$\overline{\Theta\cdot\Phi}:\kappa\cong\pp^{n-1}/\pp^n\longrightarrow\CC^{\times}.$$

    To construct an inverse to the map $\Theta\mapsto\overline{\Theta\cdot\Phi}$, choose some $\chi\in\hat\kappa$, view it as a character of $\pp^{n-1}/\pp^{n}$ and consider the map $\tilde\chi:\pp^{n-1}\rightarrow\CC^{\times}$ given by $\tilde{\chi}(u)=\chi(u+\pp^n)$. Then the map $\chi\mapsto\Phi^{-1}\cdot\tilde\chi$ is the required inverse map.
\end{proof}

We are now ready for the proof of Additive Duality.

\begin{proof}[Proof of Theorem \ref{add_dual}]
    The map $a\mapsto a\psi$ is clearly a homomorphism. To prove injectivity, suppose that $a\neq b$ but $a\psi=b\psi$. It follows that $x(a-b)\in\ker\psi$ for all $x\in F$. But since $a-b\neq 0$, we have $\ker\psi=F$, contradicting our assumption that $\psi$ is non-trivial.

    Let $\theta\in\hat{F}$ be any non-trivial character (if $\theta$ were trivial, then $0\psi=\theta$), and let $l$ be the level of $\theta$. By replacing $\theta$ with $\varpi^{l-d}\theta$, which has level $d$, we may assume without loss of generality that $\theta$ and $\psi$ have the same level $d$, and therefore they both agree on $\pp^d$. To show there is some $u\in F$ (in fact, $u\in U_F$ necessarily) such that $u\psi=\theta$,   we construct a sequence $\{u_n\}_{n\geq0}$ inductively such that $u_n\psi|_{\pp^{d-n}}=\theta|_{\pp^{d-n}}$ and $u_{n+1}\equiv u_n\pmod{\pp^n}$. Such a sequence is clearly Cauchy, and since $F$ is complete, it converges to some $u\in U_F$ such that $u\equiv u_n\pmod{\pp^n}$ for all $n\geq 1$ and thus $u\psi$ agrees with $\theta$ on $\cup_{n\in\ZZ}\ \pp^n=F$, which concludes the proof.

    Thus, it remains to construct the sequence above. To construct $u_1$ we note that by Lemma \ref{lem_chars}, there are exactly $q-1$ non-trivial characters on $\pp^{d-1}$ that are trivial on $\pp^d$. In addition, by Lemma \ref{lem_congruence}, as $u$ ranges over the cossets of $U_F/U_F^1$, the characters $u\psi|_{\pp^{d-1}}$ are distinct. Since $|U_F/U_F^1|=|\kappa^{\times}|=q-1$, there is some $u_1\in U_F$ such that $u_1\psi$ agrees with $\theta$ on $\pp^{d-1}$. 
    
    Assuming now we have constructed $u_1,\ldots,u_n$ in $U_F$ with the desired conditions, we note that by Lemma \ref*{lem_chars}, there are exactly $q$ characters of $\pp^{d-n-1}$ that coincide with $\theta|_{\pp^{d-n}}$ when they are restricted. Again by Lemma \ref*{lem_congruence}, as $\alpha$ ranges over the cossets of $U_F^n/U_F^{n+1}$ the characters $\alpha u_n\psi$ are distinct on $\pp^{d-n-1}$ but they all coincide on $\pp^{d-n}$. Since $|U_F^n/U_F^{n+1}|=|\kappa|=q$, there is some $\alpha_n$ such that $\alpha_n u_n\psi$ coincides with $\theta$ on $\pp^{d-n-1}$. Since $\alpha_n\in U_F^n$, $\alpha_n u_n\equiv u_n\pmod{\pp^n}$. Hence $u_{n+1}:=\alpha_n u_n$ has the requried properties.
\end{proof}
\subsection{Smooth Representations of Locally Profinite Groups} \label{sec:smoothrep}

We now turn our attention to representations of arbitrary dimension of locally profinite groups and we introduce the notion of \textit{smooth representations}, which form a full subcategory of $\Rep(G)$. For one-dimensional representations, we imposed a natural continuity condition, and Lemma \ref{lem_cont_chars} showed that characters have open kernel. This is a remarkable result, since this means that the homomorphism is continuous with respect to \textbf{any} topology on $\CC^{\times}$, not just the usual one. 

If $V$ is a finite-dimensional representation of a locally profinite group $G$, the group $\GL_\CC(V)$ has a natural topology as an open subspace of $M_n(\CC)\cong\CC^{n^2}$. Again, it is a natural requirement that finite dimensional representations should be continous with respect to these topologies. It is a fact, analogous to $\CC^{\times}$, that small neighbourhoods of the identity of $\GL_\CC(V)$ do not contain any non-trivial subgroups. Therefore, the same reasoning as in Lemma \ref{lem_cont_chars} shows that continuous finite-dimensional representations of $G$ have open kernel too. That is, the homomorphism is continuous with respect to any topology on $\GL_\CC(V)$.

However, for infinite-dimensional representations $V$, equipping $\GL_\CC(V)$ with a topology is not as straightforward, and the requirement of having an open kernel is too restrictive. Here is where the notion of smooth representation becomes relevant, for which we must first introduce the module of invariants and coinvariants.

\begin{defn}
    Let $H\leq G$ be groups and $(\pi,V)$ a representation of $G$. We define the $H$-invariants of $V$ to be 
    $$V^{H}:=\{v\in V:\pi(h)v=v\text{ for all }h\in H\},$$
    and the $H$-coinvariants to be 
    $$V_H:=V/V(H)\text{ where } V(H)=\textrm{Span}_\CC\{v-\pi(h)v:v\in V,h\in H\}.$$
    That is, $V^H$ (resp. $V_H$) is the largest subspace (resp. quotient) on which $H$ acts trivially.
\end{defn}


\begin{defn}
	A representation $V$ of $G$ is \textit{smooth} if for all $v\in V$ there exists a compact open subgroup $K\leq G$ such that $v\in V^K$. In other words,
    $$V=\bigcup_K V^K$$ as we range over all compact open subgroups $K$ of $G$. We say that $V$ is \textit{admissible} if $V^K$ is finite dimensional for all compact open $K$.
\end{defn}

Smooth representations of $G$ are a full abelian subcategory of $\Rep(G)$ denoted by $\Smo(G)$. 


\begin{rem}
    If $(\pi,V)$ is a finite-dimensional smooth representation and $\{v_1,\ldots,v_n\}$ is a $\CC$-basis such that $v_i\in V^{K_i}$ for some open compact subgroups $K_i$, then 
    $$K:=\bigcap_{i=1}^n K_i\subseteq\ker\pi$$
    is open and compact too, so the kernel is open. 
    Conversely, if $\ker\pi$ is open, then there is some open compact subgroup $K$ fixing all of $V$, so in this case smooth and continuous coincide. 
\end{rem}


As we hinted in Remark \ref{rem_semisimple}, smooth representations of locally profinite groups have remarkable algebraic structures, and they share many properties with representations of finite groups, particularly if the group is compact (and thus profinite). A direct application of Zorn's Lemma provides the following useful criterion to determine whether a representation is semisimple. 

\begin{prop}\label{prop_semisimple}
    Let $(\pi,V)$ be a smooth representation of a locally profinite group $G$. The following are equivalent:
    \begin{enumerate}
        \item $V$ is the sum of its irreducible $G$-subspaces.
        \item $V$ is the direct sum of a family of irreducible $G$-subspaces (i.e. $V$ is semisimple)
        \item any $G$-subspace of $V$ has a $G$-complement in $V$.
    \end{enumerate}
\end{prop}

\begin{proof}
    \cite[Lemma 2.2]{BH1}
\end{proof}

Using this proposition, we can now prove that smooth representations of profinite groups behave in a similar way to those of finite groups. We note that any open compact subgroup $K$ of a locally profinite group $G$ is profinite, and that any smooth $G$-representation is natually a smooth $K$-representation by restriction. Therefore, the following results apply for any open compact subgroup of $G$.

\begin{prop}\label{lem_profinite_smooth}
    Let $(\pi,V)$ be a representation of a profinite group $K$. If $V$ is irreducible then it is finite dimensional. If $V$ is finite dimensional, then it is semisimple.
\end{prop}

\begin{proof}
    The first statement is a matter of following the definitons. Fix any non-zero $v\in V$, and suppose $v\in V^{K_0}$ for some open compact $K_0 \leq K$. Then the subspace 
    $$U=\Span\{\pi(k)v:k\in K\}=\Span\{\pi(k)v:k\in K/K_0\}$$
    is clearly a $K$-subspace and it is also finite dimensional since $K_0$ is open and $K$ is compact, so $[K:K_0]$ is finite.   

    To prove the second statement, let $v$ and $K_0$ be as above. By replacing $K_0$ by $\cap_{g\in K/K_0}gK_0g^{-1}$ if needed, we may assume that $K_0$ is normal in $K$. As above, the subspace 
    $$W=\Span\{\pi(k)v:k\in K\}$$
    is finite dimensional and $K_0$ acts trivially on it.
    Thus $W$ factors through a finite dimensional representation of the finite group $K/K_0$. By Maschke's Theorem, $W$ is then the sum of its irreducible $K$ subspaces. Since $v$ was arbitrary this shows that condition $1.$ of Proposition \ref{prop_semisimple} is satisfied, so $V$ is semisimple.

\end{proof}

This proposition has important structural results. Let $\hat{K}$ denote the set of equivalence classes of irreducible smooth representations of $K$. As we shall see, this notation is consistent with $\hat{F}$ since all irreducible smooth representations of $F$ are one-dimensional.

Let $(\pi,V)$ be a smooth representation of a locally profinite group $G$ and let $K$ be an open compact subgroup. For each $\rho\in\hat{K}$, let $V^\rho$ be the sum of all irreducible $K$-subspaces of $V$ isomorphic to $\rho$, the \textit{$\rho$-isotypic component} of $V$. In particular, $V^{1_K}=V^K$.

\begin{prop}
    Let $G$ be a locally profinite group and $K$ a compact open subgroup of $G$. Let $(\tau,U),(\pi,V)$, $(\sigma,W)\in\Smo(G)$ and $a:U\rightarrow V$ and $b:V\rightarrow W$ be $G$-homomorphisms. 
    \begin{enumerate}
        \item The space $V$ is the sum of the $K$-isotypic components:
        $$V=\bigoplus_{\rho\in\hat{K}}V^\rho.$$
        \item The following holds:
        $$W^\rho\cap b(V)=b(V^\rho).$$
        \item The sequence
        $$U\xlongrightarrow{a} V\xlongrightarrow{b} W$$
        is exact if and only if 
        $$U^K\xlongrightarrow{a} V^K \xlongrightarrow{b} W^K$$
        is exact for every compact open subgroup $K$ of $G$.
        \item Denoting by $V(K)$ the span of the elements $v-\pi(k)v$ for $v\in V, k\in K$,
        $$V(K)=\bigoplus_{\substack{\rho\in\hat{K}\\\rho\neq 1}}V^\rho \text{ and } V=V^K\oplus V(K)$$
        and $V(K)$ is the unique $K$-complement of $V^K$ in $V$. 
    \end{enumerate}
\end{prop}

\begin{proof}
    \cite[Proposition 2.3 and Corollary 1.2]{BH1}
\end{proof}

As promised in \S\ref{Abstract_Reps}, we now discuss the dual, restriction and induction functors in the context of smooth representations of locally profinite groups. From our previous discussion, two major problems arise in this context. Firstly, given a locally profinite group $G$ and a subgroup $H$, there is no guarantee that $H$ is locally profinite, and thus $\Smo(H)$ may not be well-defined. Secondly, when we perform some construction on a smooth representation (e.g., constructing its dual, inducing to a bigger group) there is no gurarantee that the resulting representation is smooth. Thankfully, both of these problems can be resolved in a straightforward way.

To ensure that $H$ is locally profinite, we must add a condition on the topology of $H$. Based on Example \ref{example_prof_groups}(7), we just need to assume that $H$ is a closed subgroup of $G$. In some cases, we will need to assume that $H$ is also open, which is a more restrictive condition. To resolve the second problem, we construct a functor that associates, to each abstract representation, a smooth representation in a canonical way.

\begin{defn}
    Let $G$ be a locally profinite group. Define the \textit{smoothness functor}
    \begin{align*}
        (-)^\infty:\Rep(G)&\longrightarrow\Smo(G),\\
        (\pi,V)&\longmapsto(\pi^\infty,V^\infty)
    \end{align*}
    by defining 
    $$V^\infty:=\bigcup_K V^K \text{  and  } \pi^\infty(g):=\pi(g)|_{V^\infty} \text{  for each  } g\in G,$$ where $K$ ranges over the compact open subgroups of $G$.
\end{defn}

\begin{rem}
    One should check that the smoothness functor is well-defined. In other words, we should check that the space $V^\infty$ is preserved under the $G$-action, making it a $G$-representation. Let $v\in V^\infty$ and choose some open compact subgroup $K$ such that $v\in V^K$. For any $g\in G$, we have that $\pi^\infty(g)v=\pi(g)v\in V^{gKg^{-1}}\subseteq V^\infty$ since $\pi(gkg^{-1})\pi(g)v=\pi(gk)v=\pi(g)v$ for any $k\in K$.
\end{rem}

Furthermore, the functor $(-)^\infty$ is left-exact and it satisfies that
$$\Hom_G(V,W)=\Hom_G(V,W^\infty) \text{ for all } V\in\Smo(G), W\in\Rep(G).$$

Using these constructions, we can define the smooth dual, restriction and induction functors. If $H\leq G$ is a closed subgroup, the restriction functor $\Res_H^G:\Rep(G)\to\Rep(H)$ (Definition \ref{def:absresind}) sends smooth representations of $G$ to smooth representations of $H$. This is because the intersection of an open compact subgroup of $G$ with $H$ is still open compact in the subspace topology of $H$. The analogous statement does not hold for the dual and induction functors, so we must compose with the smoothness functor. 

\begin{defn}
    If $G$ is a locally profinite group, define the \textit{smooth dual functor} 
    \begin{align*}
        \check{(-)}:\Smo(G)&\longrightarrow\Smo(G),\\
        (\pi,V)&\longmapsto(\check{\pi},\check{V})
    \end{align*}
    by $(\check{\pi},\check{V})=(\pi^*,V^*)^\infty$.
\end{defn}

The smooth dual satisfies an important property: if $V$ is a smooth representation of $G$ and $v\in V, v\neq 0$, then there is some $\check{v}\in\check{V}$ such that $\langle\check{v},v\rangle\neq 0$. Consequently, the map $\delta:V\rightarrow\check{\check{V}}$ is injective, and the following proposition gives a criterion for surjectivity.

\begin{prop}
    If $G$ is a locally profinite group, and $V$ is a smooth representation of $G$, the canonical map $\delta:V\longrightarrow\check{\check{V}}$ is an isomorphism if and only if $(\pi,V)$ is admissible.
\end{prop}
\begin{proof}
    \cite[Proposition 2.9]{BH1}
\end{proof}

We also define the smooth induction functor as the composition of the induction and smoothness functor.

\begin{defn}\label{induction}
    Let $G$ be a locally profinite group and $H\leq G$ a closed subgroup. Define the \textit{smooth induction functor}
    \begin{align*}
        (\Ind_H^G(-))^\infty:\Smo(H)&\longrightarrow\Smo(G),\\
        (\sigma,W)&\longmapsto(\Sigma,X)^\infty
    \end{align*}
    where we recall that $X$ is the space of functions $f: G\to W$ satisfying $f(hg) = \sigma(h)f(g)$ for all $h\in H, g\in G$ and the action of $\Sigma$ on $X$ is given by right translation $\Sigma(g)f:x\mapsto f(xg)$.
\end{defn}

\begin{rem}
    Throghout this document, we will only be interested in studying the smooth induction of smooth representations. The idea is that smooth induction is the `right' construction in this setting, which coincides with the abstract induction from Definition \ref{def:absresind} when the group is finite with the discrete topology. Therefore, as it is common in the literature, we will use a slight abuse of notation and denote the smooth induction functor as $\Ind_H^G$. We will write 
    \begin{align*}
        \Ind_H^G:\Smo(H)&\longrightarrow\Smo(G),\\
        (\sigma,W)&\longmapsto(\Sigma,X)
    \end{align*}
    where $\Sigma$ is now the space of functions $f:G\to W$ satisfying:
    \begin{enumerate}
        \item For all $h\in H, g\in G$, we have $f(hg) = \sigma(h)f(g)$.
        \item There is some open compact subgroup $K$ of $G$ such that $f(xg)=f(x)$ for all $x\in G$ and $g\in K$,
    \end{enumerate}
    and $\Sigma$ is the action on $X$ by right translation.

    The second condition is precisely the smoothness condition that appears after composing the abstract induction functor with the smoothness functor.
    
\end{rem}

Since the action $\Sigma$ on $X$ is given by $\Sigma(g)f:x\mapsto f(xg)$, condition $2.$ is precisely the smoothness condition that $f\in X^K$ for some open compact subgroup $K$. As above, we will denote this representation of $G$ by $\Ind_H^G\sigma$. Under these conditions, the first half of Frobenius reciprocity holds:

\begin{thm}[Frobenius reciprocity]\label{thm:frob}
	Let $(\pi,V)$ be a smooth representation of $G$, and $(\sigma,W)$ a smooth representation of a closed subgroup $H$. Then the map
	\begin{align*}
		\Psi:\Hom_G(\pi, \Ind_H^G\sigma)&\longrightarrow \Hom_H(\Res_H^G \pi, \sigma),\\
		\varphi &\longmapsto a_\sigma \circ \varphi,
	\end{align*}
    is a bijection that is functorial in both variables $\pi,\sigma$. Here $a_\sigma:\Ind_H^G\sigma \to W$ is the canonical map $a_\sigma(f) = f(1)$. In categorical terms,
    $$\Res_H^G\dashv\Ind_H^G.$$
\end{thm}
\begin{proof}
    \cite[2.4 Frobenius Reciprocity]{BH1}
\end{proof}

However, in this context, it is not the case that $\Ind_H^G$ is left adjoint to $\Res_H^G$. With a small modification we can recover left exactness. Firstly, we note that to ensure that $a_\sigma^c$ (to be defined shortly) is a $H$-homomorphism, we need the stronger assumption that $H$ is open in $G$. Secondly, we observe that given representations $(\pi,V)$ and $(\sigma,W)$, of $G$ and $H$ respectively, $a_\sigma^c(w)$ is supported only in $H$ for any $w\in W$. Hence, one should not consider the entire representation $\Ind_H^G\sigma$, but rather a subrepresentation of it. Here is the precise construction.

\begin{defn}
	Let $G$ be a locally profinite group, $H$ a closed subgroup, and $(\sigma,W)$ a smooth representation of $H$. Define the \textit{compact induction functor} 
    \begin{align*}
        \cInd_H^G:\Smo(H)&\longrightarrow\Smo(G),\\
        (\sigma,W)&\longmapsto(\Sigma_c,X_c)
    \end{align*}
    where, if $\mathrm{Ind}_H^G(\sigma,W) = (\Sigma,X)$, then
    $$X_c:=\{f\in X: \supp f \text{ in } H\backslash G \text{ is compact}\},$$
    and $\Sigma_c$ acts on $X_c$ by right translation.
    We say that functions satisfying the later condition are \textit{compactly supported modulo $H$}, and this condition is equivalent to $\supp f\subseteq HC$ for some compact subset $C$ of $G$.
    The space $X_c$ is closed under the action by $\Sigma$, so the functor is well-defined.
\end{defn}

This construction is mainly of interest in the case when $H$ is open in $G$, in which case $a_\sigma^c$ is a $H$-homomorphism. This construction satisfies the second half of Frobenius reciprocity.

\begin{thm}\label{thm:frob2}
	Let $(\pi,V)$ be a smooth representation of $G$, and $(\sigma,W)$ a smooth representation of an open subgroup $H$. Then the map 
	\begin{align*}
		\Psi^c:\Hom_G(\cInd_H^G \sigma, \pi)&\longrightarrow \Hom_H(\sigma, \Res_H^G\pi)\\
		\varphi &\longmapsto \varphi \circ a^c_\sigma 
	\end{align*}
    is a bijection that is functorial in both variables $\pi,\sigma$. Here $a^c_\sigma: W\to c-\Ind_H^G \sigma$ is the map $w\mapsto f_w$, where $f_w$ is supported in $H$ and defined by $f_w(h) = hw$.
\end{thm}
\begin{proof}
    \cite[2.5 Theorem]{BH1}
\end{proof}
In categorical terms, under the assumptions of this theorem we have
$$\cInd_H^G\dashv\Res_H^G\dashv\Ind_H^G.$$
\subsection{Schur's Lemma}

We end this section by discussing a version of Schur's Lemma for smooth representations of locally profinite groups. Throughout, $G$ will denote a locally profinite group. We recall Schur's Lemma for finite groups.

\begin{thm}
    Let $\mathbf{G}$ be a finite group and let $(\pi,V)$ be a complex irreducible representation of $\mathbf G$. Then for any $\phi\in\End_{\mathbf{G}}(V)$, there is some $\lambda\in\CC$ such that $\phi(v)=\lambda v$ for all $v\in V$. In other words, $\End_{\mathbf{G}}(V)\cong\CC$.
\end{thm}

Schur's Lemma does not hold for complex smooth irreducible representations of a locally profinite group $G$. However, it is true under a mild hypothesis.

\begin{hypothesis}
    For any compact open subgroup $K$ of $G$, the set $K\backslash G$ is countable.
\end{hypothesis}

A short topological argument shows that if this hypothesis holds for one compact open subgroup $K$, then it holds for all of them.

\begin{example}
    This hypothesis is satisfied by all locally profinite groups in Examples \ref{example_prof_groups}, which are the groups of interest for us. For example, if $F$ is a local field of $0$ characteristic, then $F=K_\mathfrak{P}$ is the completion of a number field at some prime $\mathfrak{P}$. Then the composite map $K\hookrightarrow K_\mathfrak{P}=F\twoheadrightarrow F/R$ is surjective. Since $K$ is a number field, it is countable, which shows that $F/R$ is countable too. The other cases are proven using similar (yet tedious) reasonings.
\end{example}

 For the remainder of this section we assume the hypothesis.


\begin{lemma}
    Let $(\pi,V)$ be an irreducible smooth representation of $G$. Then the dimension $\dim_{\CC}V$ is countable.
\end{lemma}
\begin{proof}
    Let $v\in V$, $v\neq0$ and let $K \leq G$ be an open compact subgroup such that $v\in V^K$. The set $\{\pi(g)v:g\in G\}=\{\pi(g)v:g\in K\backslash G\}$ spans $V$, by irreducibility of $V$, and it is countable.
\end{proof}

We are now ready to state and prove Schur's Lemma in our context.

\begin{thm}[Schur's Lemma]\label{thm:schur}
    If $(\pi,V)$ is a smooth irreducible representation of $G$, then $\End_{\CC}V\cong \CC$.
\end{thm}
\begin{proof}
    \cite[2.6 Schur's Lemma]{BH1}
\end{proof}

This results has two important corollaries worth recalling. For the first one, we note that given a locally profinite group $G$, its centre $Z$ is a closed subgroup of $G$ and therefore a locally profinite group too.

\begin{cor}\label{cor:centralchar}
    Let $(\pi,V)$ be an irreducible smooth representation of $G$. The centre $Z$ of $G$ acts on $V$ via a character $\omega_{\pi}:Z\to\CC^{\times}$. In other words, $\pi(z)v=\omega_{\pi}(z)v$ for all $v\in V$ and $z\in Z$.
\end{cor}

\begin{proof}
    For any $z\in Z$, the automorphism $\pi(z):V\to V$ lies in $\End_G(V)\cong\CC$. Hence, the desired map $\omega_{\pi}:Z\to\CC^{\times}$ does indeed exist, and it is a group homomorphism. To prove smoothness, we note that if $K$ is an open compact subgroup such that $V^K\neq 0$, then $\omega_{\pi}$ is trivial on the open compact subgroup $K\cap Z$ of $Z$. So $\omega_{\pi}$ is indeed a character of $Z$.
\end{proof}

The character $\omega_{\pi}$ is called the \textit{central character} of $(\pi,V)$.

\begin{cor}
    If $G$ is abelian, any irreducible smooth represenation of $G$ is one dimensional.
\end{cor}

This justifies the notation $\hat{K}$ for the set of equivalence classes of irreducible smooth representations of a locally profinite group $K$, since this notation can now be seen to coincide with the set of characters $\hat{F}$ of $F$.

\newpage


\section{Measure and the Duality Theorem}
So far, we have introduced the central objects that we will study throughout: locally profinite groups and smooth representations. In addition, we have given a complete classification of the equivalence classes of irreducible smooth representations of a local field $F$. These are all $1$-dimensional by Schur's Lemma, and we have Additive Duality, $\hat{F}\cong F$, by Theorem \ref{add_dual}. 

Classifying irreducible smooth representations of other locally profinite groups is considerably harder. Even the structure of the group of characters of $F^{\times}$ is more subtle. To describe the local Langlands correspondence for $\GL_2$, we will need a classification theorem of irreducible smooth representations of $\GL_2(F)$. We will focus on a particular family of them, the so-called principal series representations. In order to study the group $\GL_2(F)$ and its subgroups, we study certain functions defined on them. To do so, we must first develop some measure theory on locally profinite groups. This is precisely the aim of this chapter, which follows a similar development to \cite[Chapter 3]{BH1}.

We finish this section by studying the relationship between induction and duality, which is encapsulated by the Duality Theorem \ref{thm:duality}.

\subsection{The Space \texorpdfstring{$C_c^{\infty}(G)$}{TEXT} and the Haar Measure}

Let $G$ be a locally profinite group. Denote by $\CG$ the space of functions $f:G\to\CC$ that are locally constant and compactly supported. 

\begin{exercise}
    Show that a function $f:G\to\CC$ lies in $\CG$ if and only if it is a finite linear combination of characteristic functions of double cosets $KgK$ for some open compact subgroup $K$ of $G$.
\end{exercise}

The space $\CG$ is a complex vector space and admits two natural actions of $G$ by left and right translation:

$$\lambda_g f:x\longmapsto f(g^{-1}x),\quad\text{and}\quad \rho_g f:x\longmapsto f(xg),$$
for $x,g\in G$ and $f\in\CG$. These actions endow $\CG$ with the structure of a smooth representation of $G$, because characteristic functions of double cosets of $K$ are invariant under translation by $K$. 

\begin{rem}
    The representation $(\CG,\rho)\in\Smo(G)$ is isomorphic to $\cInd_{\{1\}}^G\mathds{1}$, where $\{1\}$ is the trivial subgroup of $G$.  
\end{rem}

We are now ready to define the notion of a Haar integral and Haar measure.

\begin{defn}
    A \textit{left Haar integral} on $G$ is a non-zero linear functional 
    $$I:\CG\longrightarrow \CC$$
    such that
    \begin{enumerate}[(1)]
        \item $I(\lambda_g f)=I(f),\ g\in G,\ f\in\CG$, and
        \item $I(f)\geq 0$ for any $f\in\CG$ such that $\mathrm{Im}(f)\subseteq\RR_{\geq0}$.
    \end{enumerate}
    A \textit{right Haar integral} is defined analogously by replacing $\lambda_g$ with $\rho_g$.
\end{defn}

The usefulness of the Haar integral relies on the fact that locally profinite groups possess essentially one unique left Haar integral.

\begin{prop}\label{prop:haar}
    There exists a left Haar integral $I:\CG\to\CC$. Moreover, a linear functional $I':\CG\to\CC$ is a left Haar integral if and only if $I'=cI$ for some constant $c>0$.
\end{prop}
\begin{proof}
    %We give a proof sketch of this proposition. For each open compact subgroup $K$, we consider the space 
    
    %The main idea is to consider a descending chain of open compact subgroups $K_n$ of $G$ such that $\cap_nK_n=\{1\}$. 
    \cite[Proposition 3.1]{BH1}
\end{proof}

Whenever we have a left Haar integral $I$, we can define the associated \textit{left Haar measure} as follows. Let $S\subset G$ and let $\Gamma_S$ be its characteristic function. Then $\Gamma_S\in\CG$ if and only if $S$ is open and compact. In that case, we define $$\mu_G(S)=I(\Gamma_S)$$ to be the Haar measure of $S$. We note that $\mu_G(S)>0$ when $S$ is nonempty, and by left invariance, $\mu_G(gS)=\mu_G(S)$ for any $g\in G$. The relationship is commonly expressed by using the usual integral notation
\begin{equation}\tag{$\dagger$}\label{eq:integral}
    I(f)=\int_G f(g)d\mu_G(g),\quad f\in\CG.
\end{equation}

This choice of notation is motivated by the fact that one can also recover the left Haar integral from the left Haar measure. Indeed, since $f\in\CG$ is locally constant and has constant support, we can express $f=\sum_{i=1}^r\alpha_i\mathds{1}_{Kg_iK}$ for some open compact subgroup $K$, $g_i\in G, \alpha_i\in\CC$ and $r\geq 1$. Then, Equation \ref{eq:integral} represents the finite sum 
$$I(f)=\sum_{i=1}^r\alpha_i\mu_G(Kg_iK)$$
from which we can recover the left Haar integral from the left Haar measure. Therefore, both notions carry essentially the same information. During our discussion, we will often fix a left Haar measure on $G$ and then consider the Haar integral induced by the measure. This makes some arguments more natural to follow.

\begin{example}
    The notion of a left Haar measure is only determined up to a constant. In practice, to uniquely determine the measure, we associate a particular open compact subset with a value. For example, if $G=F$ is a local field, one commonly chooses $\mu_F$ so that $\mu_F(R)=1$, where $R$ is the valuation ring. Under this choice, we calculate that $\mu_F(\pp^n)=q^{-n}$.
\end{example}

Left Haar measures behave predictably under usual group constructions. For example, if $G_1,G_2$ are profinite groups, then $G=G_1\times G_2$ is also a profinite group, and we have an isomorphism 
\begin{align*}
    C_c^{\infty}(G_1)\otimes C_c^{\infty}(G_1)&\longrightarrow \CG\\
    \sum_{i=1}^{r}f_i^1\otimes f_i^2 &\longmapsto \left((g_1,g_2)\mapsto \sum_{i=1}^{r}f_i^1(g_1)f_i^2(g_2)\right).
\end{align*}
If $\mu_i$ is a left Haar measure on $G_i$ for $i=1,2$, then there is a unique left Haar measure $\mu_G$ on $G$ such that 
$$\int_G f_1\otimes f_2(g)d\mu_G(g)=\int_{G_1}f_1(g_1)d\mu_1(g_1)\int_{G_2}f_2(g_2)d\mu_2(g_2),$$
usually dentoted $\mu_G=\mu_1\otimes\mu_2$.

\subsection{The Modular Character of a Group}

Of course, the discussion from the previous subsection holds if we replace `right' by `left' throughout. At this point it is therefore natural to ask whether a left Haar integral $I$ on $G$ is also a right Haar integral. This important consideration motivates the following definition.

\begin{defn}
    A locally profinite group $G$ is \textit{unimodular} if any left Haar integral (resp. measure) on $G$ is also a right Haar integral (resp. measure). 
\end{defn}

As a first observation, we note that if the group $G$ is abelian, then $\lambda_g f=\rho_{g^{-1}} f$, and therefore $G$ is unimodular. However, for general locally profinite groups this is not always the case. 

To investigate this, choose some left Haar measure $\mu_G$ on a locally profinite group $G$ (not necessarily abelian), and consider the functional
\begin{align*}
    I_g:\CG&\longrightarrow\CC\\
    f\longmapsto\int_G& f(xg)d\mu_G(x).
\end{align*}
In other words, if $I$ is the associated left Haar integral of $\mu_G$, then $I_g(f)=I(\rho_g f)$. Since the actions of $G$ on $\CG$ by left and right translation commute, $$I_g(\lambda_h f)=I(\rho_g\lambda_h f)=I(\lambda_h\rho_g f)=I(\rho_g f)=I_g(f)$$
and so $I_g$ is also a left Haar integral. Therefore, there is a unique $\delta_G(g)\in \RR_+^{\times}$ such that $\delta_G(g)I_g(f)=I(f)$ for all $f\in\CG$. In the integral notation, this means that 
$$\delta_G(g)\int_G f(xg)d\mu_G(x)=\int_G f(x)d\mu_G(x)$$
for all $f\in\CG$. Moreover, the map $\delta_G$ also interacts predictably with the left Haar measure. If $S$ is an open compact subset of $G$ and $f=\Gamma_S$ is its characteristic function then one obtains that 
$$\delta_G(g)\mu_G(Sg)=\mu_G(S),$$
which also uniquely identifies $\delta_G(g)$.

\begin{lemma}
    The map $\delta_G:G\to\RR_+^{\times}$ is a homomorphism independent of the choice of left Haar integral $I$ and it is trivial on any open compact subgroup of $G$. In particular, $\delta_G$ is a character of $G$.
\end{lemma}
\begin{proof}
    By above, we have that 
    $$\delta_G(gh)I(\rho_{gh}f)=I(f)=\delta_G(g)I(\rho_g f)=\delta_G(g)\delta_G(h)I(\rho_h\rho_g f)$$
    for any $g,h\in G$ and $f\in\CG$.
    By uniqueness of $\delta_G$ and the fact that $\rho_{gh}=\rho_g\rho_h$, it follows that $\delta_G$ is a homomorphism. The fact that it is independent of the left Haar measure follows immediately from its definition and Proposition \ref{prop:haar}. 
    If $K$ is an open compact subgroup of $G$ and $k\in K$, then by choosing $f=\Gamma_K$ to be the characteristic function of $K$, it follows that $\rho_k f= f$ and therefore $\delta_G(k)=1$.
\end{proof}

\begin{defn}
    For $G$ a locally profinite group, the character $\delta_G:G\to\CC$ is called the \textit{modular character} of $G$
\end{defn}


\begin{lemma}
    Let $G$ be a locally profinite group let and $\delta_G:G\to\CC$ be its modular character. Then $G$ is unimodular if and only if $\delta_G$ is trivial. 
\end{lemma}
\begin{proof}
    Let $I$ be a left Haar integral on $G$. Then $G$ is unimodular if and only if $I$ is a right Haar integral. This is equivalent to $I(f)=I(\rho_g f)=I_g(f)=\delta(g)^{-1}I(f)$ for every $g\in G$. But this is clearly equivalent to $\delta_G$ being trivial. 
\end{proof}

Finally, when the group $G$ is not unimodular, the modular character $\delta_G$ gives a canonical relationship between left and right Haar integrals.

\begin{lemma}
    Let $I$ be a left Haar integral on $G$ with associated left Haar measure $\mu_G$. If $\delta_G$ is the modular character of $G$, then the functional
    \begin{align*}
        J:\CG&\longrightarrow\CC\\
        f\longmapsto\int_G& \delta_G(x)^{-1}f(x)d\mu_G(x)
    \end{align*}
    is a right Haar integral for $G$.
\end{lemma}
\begin{proof}
    The functional $J$ can also be expressed as $J(f)=I(\delta_G^{-1}f)$. We note that $\delta_G^{-1}\rho_g(f)=\delta_G(g)\rho_g\delta_G^{-1}f$ as elements of $\CG$ for all $g\in G$ and $f\in\CG$.  Hence, 
    $$J(\rho_g f)=I(\delta_G^{-1}\rho_g f)=\delta_G(g)I(\rho_g\delta_G^{-1}f)=I(\delta^{-1}_G f)=J(f)$$
    for every $g\in G$ and $f\in\CG$, as desired.
\end{proof}


\subsection{Positive Semi-invariant Measures and the Duality Theorem}

To classify the principal series representations of $\GL_2(F)$ in the following section, one needs to understand the interaction between the induction and the duality functor for smooth representations of locally profinite groups and their closed subgroups. To this aim, we need to develop one last bit of machinery from measure theory called positive semi-invariant measures, which generalise the notion of Haar measures.

Let $G$ be a locally profinite group and let $H$ be a closed subgroup. Fix some character $\theta$ of $H$ and consider the space of functions $f:G\to\CC$ that are $G$-smooth under right translation, are compactly supported modulo $H$ and satisfy $$f(hg)=\theta(h)f(g),\quad h\in H,g\in G.$$
This space is the compact induction $\cInd_H^G\theta$, but in analogy to $\CG=\cInd_{\{1\}}^G\mathds{1}$ we denote it as $C_c^{\infty}(H\backslash G,\theta)$. At this point it is natural to ask if there exists some non-zero linear functional $I_{\theta}:C_c^{\infty}(H\backslash G,\theta)\to\CC$ such that $I_{\theta}(\rho_g f)=I_{\theta}(f)$ for all $g\in G$. As it turns out, this is not always possible and there is a simple criterion to determine when it is.

\begin{prop}\label{prop:semiinvariant}
    Let $\theta:H\to\CC^{\times}$ be a character of $H$. Then there exists a non-zero linear functional $I_{\theta}:C_c^{\infty}(H\backslash G,\theta)\to\CC$ such that $I_{\theta}(\rho_g f)=I_{\theta}(f)$ for all $g\in G$ and $f\in C_c^{\infty}(H\backslash G,\theta)$ if and only if $\theta\delta_H=\delta_G|_H$.

    Furthermore, when this holds, the functional $I_\theta$ is uniquely determined up to a constant.
\end{prop}
\begin{proof}
    \cite[Proposition 3.4]{BH1}
\end{proof}

We remark that this is a generalisation of Proposition \ref{prop:haar}; indeed, by setting $H=\{1\}$ one recovers the usual right Haar integral on $G$. Similarly to the above case, when $\theta=\delta_H^{-1}\delta_G|_H$, one commonly expresses the functional $I_\theta$ with the integral notation 
$$I_\theta(f)=\int_{H\backslash G}f(g)d\mu_{H\backslash G}(g),\quad f\in C_c^{\infty}(H\backslash G,\theta),$$
where $\mu_{H\backslash G}$ is called a \textit{positive semi-invariant measure} on $H\backslash G$. Also, since such a $\theta$ for which Proposition \ref{prop:semiinvariant} holds is uniquely defined, it is common to write $\delta_{H\backslash G}$ for $\delta_H^{-1}\delta_G|_H$. We now have all the required machinery to describe the Duality Theorem.

\begin{thm}[Duality Theorem]\label{thm:duality}
    Let $H$ be a closed subgroup of a locally profinite group $G$ and let $\dot{\mu}$ be a positive semi-invariant measure on $H\backslash G$. Let $(\sigma,W)$ be a smooth representation of $H$. Then there is a natural isomorphism
    $$\left(\cInd_H^G\sigma \right)^\vee\cong \Ind_H^G(\delta_{H\backslash G}\otimes \check{\sigma}),$$
    which only depends on the choice of $\dot{\mu}$.
\end{thm}

\begin{proof}
    We sketch a proof to motivate why one would expect $\delta_{H\backslash G}$ to appear. For a detailed proof, see \cite[Theorem 3.5]{BH1}.
    Throughout, we view the action of $\delta_{H\backslash G}\otimes\check{\sigma}$ naturally on $\check{W}$ (where $\check{\sigma}$ acts). For $\phi\in \cInd_H^G\sigma$ and $\Phi\in\Ind_H^G\delta_{H\backslash G}\otimes\check{\sigma}$, 
    we have that $\phi(g)\in W$ and $\Phi(g)\in\check{W}$ for any $g\in G$. We can then consider the function $$f:g\longmapsto\langle\Phi(g),\phi(g)\rangle,\quad g\in G $$ 
    where $\langle\cdot,\cdot\rangle$ is the standard evaluation pairing on $\check{W}\times W$. This function satisfies 
    $$f(hg)=\langle\Phi(hg),\phi(hg)\rangle=\delta_{H\backslash G}(h)\langle\check{\sigma}(h)\Phi(g),\sigma(h)\phi(g)\rangle=\delta_{H\backslash G}(h)\langle\Phi(g),\phi(g)\rangle=\delta_{H\backslash G}(h)f(g)\quad h\in H, g\in G,$$
    so $f\in C_c^\infty(H\backslash G,\delta_{H\backslash G})$.
    Therefore, there is a well-defined pairing 
    \begin{align*}
        \Psi:\Ind_H^G(\delta_{H\backslash G}\otimes\check{\sigma})\times \cInd_H^G\sigma\longrightarrow\CC&,\\
        (\Phi,\phi)\longmapsto\int_{H\backslash G}\langle\Phi(x),\phi(x)\rangle d\dot{\mu}(x)&.
    \end{align*}
    Crucially, this pairing is $G$-invariant. Indeed, 
    $$\Psi(\rho_g\Phi,\rho_g\phi)=\int_{H\backslash G}\langle\Phi(xg),\phi(xg)\rangle d\dot{\mu}(x)=\int_{H\backslash G}\langle\Phi(x),\phi(x)\rangle d\dot{\mu}(x)=\Psi(\Phi,\phi)$$
    by right translation invariance of the positive semi-invariant measure on $H/G$. 
    This induces a $G$-homomorphism $\Ind_H^G(\delta_{H\backslash G}\otimes \check{\sigma})\to \left(\cInd_H^G\sigma \right)^\vee$. The remainder of the proof consists of proving that this is an isomoprhism.

    \begin{lemma}
        The above pairing identifies $\left(\Ind_H^G(\delta_{H\backslash G}\otimes \check{\sigma})\right)^K$ bijectively with the linear dual of $\left(\cInd_H^G\sigma \right)^K$.
    \end{lemma}
    \begin{proof}
        We omit the proof of this result. The advantage is that one can explicitly describe a canonical basis for each space, which are canonically identified by the pairing. For a complete description, check \cite[Lemma 3.5.2]{BH1}.
    \end{proof}
    This concludes the proof of the Duality Theorem.
\end{proof}

\subsection{Measure Theory on \texorpdfstring{$GL_2(F)$}{TEXT}}

We now focus on the group $G=\GL_2(F)$ over a non-Archimedean local field $F$. This group will be the main object of study of the next chapter, where we will study and classify a large family of irreducible representations of $G$. To that aim, we first need to develop some measure theory associated to the group, and we do this now.

We begin by introducing some notation. Let $$B=\left\{\begin{psmallmatrix} a&b\\0&d\end{psmallmatrix} \mid a,d \in F^\times, b \in F\right\},\quad T=\left\{\begin{psmallmatrix}
    a&0\\0&d
\end{psmallmatrix}\mid a,d \in F^\times\right\} \cong F^\times \times F^\times\quad\text{ and }\quad N=\left\{\begin{psmallmatrix}
    1&b\\0&1
\end{psmallmatrix}\mid b \in F\right\}\cong F$$ be the Borel subgroup $B$ of upper triangular matrices, the maximal torus $T$ and the subgroup of nilpotent elements $N$ of $B$, respectively. A simple calculation shows that $N$ is a normal subgroup of $B$ while $T$ is not. Furthermore, $B\cap N=\{1\}$ and $B=NT$ and so it follows that $B=N \rtimes T$.

Although $G$ is unimodular (\cite[Proposition 7.5]{BH1}), the Borel subgroup $B$ is not. The failure of $B$ to be unimodular is a consequence of the subgroups $T$ and $N$ not commuting. As $T$ and $N$ are abelian, they are unimodular, and so we may pick Haar measures $dt$ and $dn$ on $T$ and $N$ respectively. Define a linear function $I$ on $C_c^\infty(B) = C_c^\infty(T) \otimes C_c^\infty(N)$ by
$$I(\Phi) = \int_T\int_N \Phi(tn) dn dt.$$

\begin{prop}
    $I$ is a left Haar integral on $B$.
\end{prop}
\begin{proof}
    Let $b=sm \in TN$. By left invariance of $dt$ we have
    $$\int_T\int_N \Phi(smtn)dtdn = \int_T\int_N \Phi(mtn)dtdn = \int_T\int_N \Phi(tt^{-1}mtn)dtdn.$$
    Since we integrate $N$ first, we are integrating over fixed values of $t$ so that $t^{-1}mt \in N$ is just constant, so left invariance of $dn$ lets us pull out the $t^{-1}mt$ factor, and we recover $\int_T\int_N \Phi(tn) dn dt$.
\end{proof}

\begin{prop}\label{prop:modularchar}
    The modular character $\delta_B$ of the group $B$ is
    $$\delta_B : tn \mapsto |t_2/t_1|, \quad n \in N, t = \begin{psmallmatrix}
        t_1 & 0 \\ 0 & t_2
    \end{psmallmatrix} \in T$$
\end{prop}
\begin{proof}
    By a similar argument as above, we have
    $$\int_T\int_N \Phi(tnsm) dtdn = \int_T\int_N \Phi(tss^{-1}nsm)dtdn = \int_T\int_N \Phi(ts^{-1}ns) dt dn.$$ Identifying $N \cong F$ this is
    $$\int_T\int_N \Phi\left(t \cdot \begin{psmallmatrix}
        1 & s_1^{-1}xs_2 \\0&1 
    \end{psmallmatrix}\right) d\mu_F(x) = |s_1/s_2|\int_T\int_N\Phi(tn)dtdn$$
    so by definition of the modular character we have $\delta_B(sm) = |s_2/s_1|$.
\end{proof}

The family of irreducible representations that we will study in the next chapter arise as subrepresentations of $\Ind_B^G\sigma$ where $\sigma$ is a smooth representation of $T$. We remark that the Borel subgroup $B$ is a normal subgroup of $G$ and that $G/B$ is compact and therefore the functors $\Ind_B^G$ and $\cInd_B^G$ coincide.

Hence, by the Duality Theorem, it follows that 
$$(\Ind_B^G\sigma)^\vee\cong\Ind_B^G(\delta_B^{-1}\otimes\check{\sigma}),$$
for any smooth representation $\sigma$ of $T$. This is slightly impractical, so one introduces a related functor that interacts well with duality.

\begin{defn}
    Let $G$, $B$ and $T$ as above. Define the normalized induction functor 
    \begin{align*}
        \iota_B^G:\Smo(T)&\longrightarrow\Smo(G),\\
        \sigma&\longmapsto\Ind_B^G(\delta_B^{-1/2}\otimes\sigma).
    \end{align*}
\end{defn}
This functor is also additive and exact, and it gives the more natural formula
$$(\iota_B^G\sigma)^\vee\cong\iota_B^G\check{\sigma}.$$





%\section{Hecke Algebras}
%[Possibly put this whole section at the end with modular forms. I think we only need the computation of $\delta_B$ from here.]


%In this section, we define the Hecke algebra $\mathcal H(G)$ associated to a locally profinite (unimodular) group $G$ and explain how to switch between smooth representations of $G$ and smooth modules of $\mathcal H(G)$. Under certain conditions on $G$ we consider a particular subalgebra of $\mathcal H(G)$; the unramified Hecke algebra $\mathcal H(G,K)$, which turns out to be commutative by the Satake isomorphism. We use as reference Chapter 4 of \cite{BH1} and Chapter 5 of \cite{GH1}.

If $G$ is a finite group, representations of $G$ are the same as $\CC[G]$-modules. We want to extend this notion to smooth representations of locally profinite groups, where we need to correctly interpret the group algebra.

Let $G$ be a locally profinite unimodular group and $K$ an open compact subgroup of $G$. Let $C_c^\infty(G)$ be the space of locally constant compactly supported functions $G \to \CC$ and $C_c^\infty(G//K)$ the $K$ bi-invariant subspace.

These are naturally $\CC$-vector spaces and we endow them with an associative (not necessarily unital) ring structure coming from convolution
$$f * h(g) := \int\limits_G f(x)h(x^{-1}g)dx$$
where we fix a Haar measure $\mu = dx$ on $G$.

When $G$ is discrete this is the usual product on $\CC[G]$. 

\begin{defn}
Let $\mathcal H(G)$ and $\mathcal H(G,K)$ denote $C_c^\infty(G)$ and $C_c^\infty(G//K)$ with the algebra structure specified above. We call $\mathcal H(G)$ the Hecke algebra of $G$.
\end{defn}

We study these algebras in more detail:

The element $e_K = \mu(K)^{-1} \mathbbm{1}_K \in \mathcal H(G)$ is idempotent and we have the property that 
$$e_K * f = f \Leftrightarrow f \text{ is $K$ left invariant }.$$
Thus $\mathcal H(G,K) = e_K * \mathcal H(G) * e_K$, and this subalgebra now has a unit $e_K$. The compactness of $K$ ensures $e_K \in C_c^\infty(G)$.

By Lemma 5.2.1 of \cite{GH1}, $\mathcal H(G)$ is spanned by indicator functions of $K'$-double cosets, where $K'$ ranges over all compact open subgroups of $G$. If we normalise these indicator functions by defining $$[K\alpha K] = \mu(K)^{-1} \mathbbm{1}_{K\alpha K},$$ then we have the formula
$$[K\alpha K] * [K\beta K] = \sum\limits_{i,j}[K\alpha_i \beta_j K]$$ where $K\alpha K = \sqcup K\alpha_i$ and $K\beta K = \sqcup \beta_j K$. This determines multiplication in the Hecke algebra.

\subsection{Smooth representations and \texorpdfstring{$\mathcal H(G)$}{TEXT}-modules}

We now explain how the concepts of smooth representations of $G$ and smooth modules over $\mathcal H(G)$ are interchangeable. To define these smooth modules, we note that the Hecke algebra $\mathcal H(G)$ does not in general have a unit. Consequently, not every $\mathcal H(G)$-module $M$ satisfies $\mathcal H(G) M = M$. 

\begin{defn}
    We say that a $\mathcal H(G)$-module $M$ is smooth if $\mathcal H(G) M = M$. 
\end{defn}

\begin{defn}
From a representation $V$ of $G$ we define the action of $\mathcal H(G)$ on $V$ via 
$$f \cdot v := \int_G f(g) g \cdot v dg.$$
\end{defn}
This can be viewed as a weighted average of the action of $G$ on $v$, where the weighting is described by $f \in C_c^\infty(G)$. The integral defines an element of $V$ when $f \in C_c^\infty(G)$ as the integral reduces to a finite sum.

\begin{lemma}\label{project}
    Under this action, $e_K \in \mathcal H(G)$, for $K\leq G$ a compact open, is the projection $V \to V^K$ onto the $K$-invariants of $V$. In particular, $e_K$ is an idempotent element of $\mathcal H(G)$, it is the identity element of $\mathcal H(G,K)$, and $V^K$ is a $\mathcal H(G,K)$-module.
\end{lemma}
\begin{proof}
    Let $V(K) \leq V$ be the subspace spanned by vectors of the form $k \cdot v-v$. Since $e_K$ is invariant under $K$-translation, it is zero on $V(K)$. The normalisation of $e_K$ is such that $e_K$ is the identity on $V^K$, and this implies the result.
\end{proof}


\begin{prop}
    A representation $V$ of $G$ is smooth if and only if it is a smooth $\mathcal H(G)$-module.
\end{prop}
\begin{proof}
If $V$ is a smooth representation, then any $v \in V$ is $K$-invariant for some compact open $K$, and so $v=e_K \cdot v$. This implies that $V$ is a smooth $\mathcal H(G)$-module. Conversely, $\mathcal H(G)$ is the union of $e_K* \mathcal H(G) *e_K = \mathcal H(G,K)$ over all compact open $K$, and so if $V$ is a smooth $\mathcal H(G)$-module then any $v \in V$ is of the form $e_K * f * e_K \cdot v'$ for some $K, f, v'$. Then $e_K \cdot v=v$ and so $v \in V^K$.
\end{proof}

So we can view smooth representations of $G$ as smooth $\mathcal H(G)$-module. In the other direction, given $M$ a smooth $\mathcal H(G)$-module, we have $$\mathcal H(G) \otimes_{\mathcal H(G)} M = M$$ by smoothness. We can then view $M$ as a smooth $G$ representation by letting $G$ act on the first factor by left translation. Concretely, if $m \in M$ there exists $K$ such that $m \in \mathcal H(G,K)M = e_K*\mathcal H(G)*e_K M$. Then $m = e_K \cdot m'$ for some $m' \in M$ and therefore $e_K \cdot m = m$ because $e_K$ is idempotent in $\mathcal H(G)$. Then define $$g \cdot m := \mu(K)^{-1} \mathbbm{1}_{gK} \cdot m,$$ where this is independent of $K$ due to the normalisation factor $\mu(K)^{-1}$.

\subsection{Information in the \texorpdfstring{$K$}{TEXT}-invariants \texorpdfstring{$V^K$}{TEXT}}
For a smooth representation $V$ of $G$ it is often easier to study the $K$-invariants $V^K$ for compact open subgroups $K$ of $G$.



\begin{lemma}\label{K inv}
    A smooth representation $V$ of $G$ is irreducible if and only if each $V^K$ is either 0 or a simple $\mathcal H(G,K)$-module for all compact open $K \leq G$.
\end{lemma}
\begin{proof}
    Suppose $V$ is irreducible. If we had $0 \neq M \subset V^K$ a $\mathcal H(G,K)$-module, then $0 \neq \mathcal H(G) M \subset V$ as smooth $\mathcal H(G)$-modules. Since smooth $\mathcal H(G)$-modules are the same as smooth $G$-representations, and $V$ is irreducible, we deduce $\mathcal H(G)M = V$. So then $$V^K = e_K V = e_K * \mathcal H(G)M = e_K * \mathcal H(G) *e_K M = \mathcal H(G,K)M=M$$ which implies the result.

    If $V$ is not irreducible, and $W \neq 0$ is a proper subrepresentation, pick $v \in V-W$. By smoothness, there exists $K$ such that $v \in V^K$, and also $W^K \neq 0$, but then $v \not\in W^K$ so that $V^K$ is not 0 or simple.
\end{proof}

The next result tells us that for any $K$, any smooth representation $V$ of $G$ is determined by $V^K$ with its structure as a $\mathcal H(G,K)$-module, provided $V^K \neq 0$.

\begin{prop}\label{K bij}
    The map $V \mapsto V^K$ induces a bijection between
    \begin{itemize}
        \item equivalence classes of irreducible smooth representations $V$ of $G$ with $V^K \neq 0$;
        \item isomorphism classes of simple (by definition nonzero) $\mathcal H(G,K)$-modules.
    \end{itemize}
\end{prop}
\begin{proof}
    Proposition 4.3 of \cite{BH1}.
\end{proof}

\subsection{Unramified representations of \texorpdfstring{$G$}{TEXT}}


It is interesting to study the smooth representations $V$ with $V^K \neq 0$ as above. For example, in an automorphic representation, Flath's theorem (\cite{GH1} Section 5.7) allows us to decompose into local factors, and furthermore tells us that almost all such local representations are unramified in the following sense:

\begin{defn}
    We consider the case $G=\GL_2(F)$. We say that an irreducible smooth representation $V$ of $G$ is unramified if $V^K \neq 0$ for $K= \GL_2(\cO_F)$. See Section 5.5 of \cite{GH1} for a more general definition for reductive groups.
\end{defn}

For the remainder of this subsection we work in the context of $G=\GL_2(F)$ and $K = \GL_2(\cO_F)$ for simplicity. The results generalise to reductive groups $G$ as in Sections 5.5 and 7.1 of \cite{GH1}.
\begin{defn}
    For $K$ as above, $\mathcal H(G,K)$ is called the unramified Hecke algebra of $G$.
\end{defn}

An application of the Satake isomorphism (\cite{GH1} Theorem 5.5.1) tells us that in this unramified case, the unramified Hecke algebra $\mathcal H(G,K)$ is commutative. It follows that if $V$ is $K$-unramified (in particular irreducible) then $V^K$ is 1-dimensional by Lemma \ref{K inv}. Thus $\mathcal H(G,K)$ acts on $V^K$ via scaling, called the Hecke character of $V$.

\begin{defn}
    The Hecke character (with respect to $K$) of a smooth representation $(\pi,V)$ of $G$ is the $\CC$-linear map
    \begin{equation*}
        \begin{split}
            \mathcal H(G,K) &\to \CC \\
            f &\mapsto \pi(f)
        \end{split}
    \end{equation*}
    defined by $f \cdot v =: \pi(f) v $ for any $v \in V^K$.
\end{defn}

We give an alternative proof of Proposition \ref{K bij}.

\begin{prop}
    Let $K \leq G$ be a compact open subgroup. If $V_1,V_2$ are irreducible smooth representations of $G$ such that $V_1^K$ and $V_2^K$ are nonzero and isomorphic as $\mathcal H(G,K)$-modules, then $V_1 \cong V_2$. In particular, unramified representations are determined by their Hecke characters.
\end{prop}
\begin{proof}
    This is Proposition 7.1.1 of \cite{GH1}. The idea is to extend an isomorphism $$I: V_1^K \to V_2^K$$ to a $G$-intertwining map $V_1\to V_2$ of $\mathcal H(G)$-modules. By irreducibility, $V_i = \mathcal H(G)V_i^K$. Take an element $\pi_1(f) \cdot \phi \in V_1$, with $f \in \mathcal H(G), \phi \in V_1^K$, then the obvious choice is to map this to $\pi_2(f) \cdot I(\phi)$. Provided this is well defined, this is a nonzero homomorphism of $\mathcal H(G)$-modules, so irreducibility of $V_1,V_2$ implies this is an isomorphism $V_1 \cong V_2$.

    We now check that this is well defined. Suppose $\pi_1(f_1)\phi_1 = \pi_1(f_2)\phi_2$. Since $V_1^K$ is irreducible, there exists $f_3 \in \mathcal H(G,K)$ such that $\pi_1(f_3)\phi_1 = \phi_2$, and so also $\pi_2(f_3)I(\phi_1) = I(\phi_2)$ since $I$ is $\mathcal H(G,K)$-intertwining. Then $\pi_1(f_1) \phi_1 = \pi_1(f_2 * f_3) \phi_1$. Thus it suffices to show that if $\pi_1(f)\phi =0$ then $\pi_2(f)I(\phi)=0$. We exploit the $\mathcal H(G,K)$-intertwining of $I$ (for the second implication below). For all $f_1 \in \mathcal H(G)$ we have:
    $$\pi_1(f)\phi = 0 \Rightarrow \pi_1(e_K*f_1*f*e_K)\phi = 0 \Rightarrow \pi_2(e_K*f_1*f*e_K)I(\phi)=\pi_2(e_K*f_1*f)I(\phi)=0.$$
    By Lemma \ref{project}, $e_K$ acts on $V_2$ by projection to $V_2^K$. If $\pi_2(f)I(\phi) \neq 0$, then $\pi_2(f_1)\pi_2(f)I(\phi)$, over all $f_1 \in \mathcal H(G)$, generates $V_2$ by irreducibility. The image under $\pi_2(e_K)$ is the exactly $V^K$, which is nonzero, contradicting the implication above.
\end{proof}

\subsection{Example computation of Hecke operators for \texorpdfstring{$\GL_2(F)$}{TEXT}}

[I haven't checked this subsection. Some parts might be more suitable for a section on modular forms. The computation of the modular character of $B$ will be needed in the main text. And the last proposition naturally goes with the unramified representations above.]

Let $G=\mathrm{GL}_2(F)$ and $K=\mathrm{GL}_2(\mathcal O)$ for $F$ a nonarchimedean local field with uniformiser $\varpi$. We have the Cartan decomposition $$G = \bigsqcup\limits_{a \geq b \in \ZZ} K \begin{pmatrix} \varpi^a & \\ & \varpi^b \end{pmatrix}K.$$ Let $S=K \begin{psmallmatrix} \varpi & \\ & \varpi \end{psmallmatrix}K$ and $T=K \begin{psmallmatrix} \varpi & \\ & 1 \end{psmallmatrix}K$, viewed as elements of $\mathcal H(G,K)$ via their indicator functions.

\begin{lemma}
    The unramified Hecke algebra is $\mathcal H(G,K) \cong \CC[S,S^{-1},T]$. In particular, this is commutative.
\end{lemma}
\begin{proof}
    This is some induction argument using the formula for convolutions of these indicator functions.
\end{proof}

\begin{rem}
    This fits into a general phenomenon - if $G$ is unramified and $K$ is a hyperspecial subgroup then the Satake isomorphism implies that the unramified Hecke algebra $\mathcal H(G,K)$ is always commutative.
\end{rem}

Later we will be interested in principal series representations, which are representations of $G$ coming from parabolic induction. So let $\chi = \begin{psmallmatrix}
    \chi_1 & \\ & \chi_2
\end{psmallmatrix}$ be a character of the torus $T$, and consider the normalised induced representation $$I(\chi) = \mathrm{Ind}_B^G \left( \chi \otimes \delta_B^{-1/2}\right)$$
where we recall that this is the space of functions $G \to \CC$ with $f(bg) = \chi(b)\delta_B^{-1/2}(b) f(g)$ for $b \in B$.

We briefly discuss the module character $\delta_B$. Although $G$ is unimodular (see Bushnell-Henniart Section 7.5), the Borel subgroup is not. We have $B=NT$ with $N\cong F$, $T \cong F^\times \times F^\times$ and $N$ normal in $B$. The failure of $B$ to be unimodular is a consequence of $T$ and $N$ not commuting. We can then define a linear function $I$ on $C_c^\infty(B) = C_c^\infty(T) \otimes C_c^\infty(N)$ by
$$I(\Phi) = \int_T\int_N \Phi(tn) dt dn$$ using Haar measures on $T$ and $N$.

\begin{prop}
    $I$ is a left Haar integral on $B$.
\end{prop}
\begin{proof}
    Let $b=sm \in TN$. By left invariance of $dt$ we have
    $$\int_T\int_N \Phi(smtn)dtdn = \int_T\int_N \Phi(mtn)dtdn = \int_T\int_N \Phi(tt^{-1}mtn)dtdn.$$
    Since we integrate $N$ first, we are integrating over fixed values of $t$ so that $t^{-1}mt \in N$ is just constant, so left invariance of $dn$ let's us pull out the $t^{-1}mt$ factor.
\end{proof}

\begin{prop}\label{prop:modularchar}
    The modular character $\delta_B$ of the group $B$ is
    $$\delta_B : tn \mapsto |t_2/t_1|, \quad n \in N, t = \begin{psmallmatrix}
        t_1 & 0 \\ 0 & t_2
    \end{psmallmatrix} \in T$$
\end{prop}
\begin{proof}
    By a similar argument as above, we have
    $$\int_T\int_N \Phi(tnsm) dtdn = \int_T\int_N \Phi(tss^{-1}nsm)dtdn = \int_T\int_N \Phi(ts^{-1}ns) dt dn.$$ Identifying $N \cong F$ this is
    $$\int_T\int_N \Phi(t \cdot \begin{psmallmatrix}
        1 & s_1^{-1}xs_2 \\0&1 
    \end{psmallmatrix}) d\mu_F(x) = |s_1/s_2|\int_T\int_N\Phi(tn)dtdn$$
    so by definition of the module character we have $\delta_B(sm) = |s_2/s_1|$.
\end{proof}

Going back to our principal series representation, the following proposition computes the action of the unramified Hecke algebra on the $K$-invariant subspace:

\begin{prop}
    Let $\chi:T \to \CC^\times$ be an unramified character of the torus (meaning trivial on $\begin{psmallmatrix}\mathcal O^\times & \\ & \mathcal O^\times \end{psmallmatrix}$) and consider the normalised parabolic induction $$I(\chi) = \mathrm{Ind}_B^G(\chi \otimes \delta_B^{-1/2}).$$ For $K=\mathrm{GL}_2(\mathcal O)$ as usual, the space $I(\chi)^K$ is 1-dimensional. As a $\mathcal H(G,K)$-module this is determined by the actions of $S$ and $T$. Since $\chi$ is unramified we know $\chi_1(z)=\alpha^{v_F(z)}$ and $\chi_2(z)=\beta^{v_F(z)}$ for some $\alpha,\beta \in \CC^\times$. Then $S$ acts on $I(\chi)^K$ by scaling by $\alpha\beta$ and $T$ acts by scaling by $q^{1/2}(\alpha +\beta)$.  
\end{prop}
\begin{proof}
    We have the Iwasawa decomposition $G=BK$ so that the functions $f \in I(\chi)^K$ satisfy
    $$f(bk)=f(b)=\chi(b)\delta_B^{-1/2}(b) \cdot f(1)$$ with $f(1) \in \CC$, so the space is 1-dimensional spanned by $\hat{f}(bk) = \chi(b)\delta_B^{-1/2}(b)$.

    The action of $S$ is given by:
    \begin{equation*}
        \begin{split}
            S\cdot f &= \mu(K)^{-1}\int_G \mathbbm{1}_{K\begin{psmallmatrix}\varpi & \\ & \varpi\end{psmallmatrix}K}(g) g \cdot f dg \\
            &= \mu(K)^{-1}\int_K \begin{psmallmatrix}\varpi & \\ & \varpi\end{psmallmatrix}k \cdot f dk \\
            &= \begin{psmallmatrix}\varpi & \\ & \varpi\end{psmallmatrix}\cdot f \\
            &= \chi\left(\begin{psmallmatrix}\varpi & \\ & \varpi\end{psmallmatrix}\right) \delta_B^{-1/2}\left(\begin{psmallmatrix}\varpi & \\ & \varpi\end{psmallmatrix}\right) f \\
            &= \alpha\beta f
        \end{split}
    \end{equation*}
    because $K\begin{psmallmatrix}\varpi & \\ & \varpi\end{psmallmatrix}K = \begin{psmallmatrix}\varpi & \\ & \varpi\end{psmallmatrix}K$.

    And for $T$ we pick coset representatives for $K\begin{psmallmatrix}\varpi & \\ & 1\end{psmallmatrix}K/K$ given by $\begin{psmallmatrix}\varpi &a \\ & 1\end{psmallmatrix}$ and $\begin{psmallmatrix}1 & \\ & \varpi\end{psmallmatrix}$, where $a$ ranges over representatives of $\mathcal O/\varpi$. Writing down the integral for the action of $T$ we decompose this into a sum over these left cosets and we deduce that $T$ acts by
    $$\chi_2(\varpi)|\varpi|^{-1/2}f + \sum\limits_{a \in \mathcal O/\varpi} \chi_1(\varpi)|\varpi|^{1/2}f = q^{1/2}(\alpha+\beta)$$
    since, for example, $\chi(\begin{psmallmatrix}\varpi & a\\ & 1\end{psmallmatrix})=\chi_1(\varpi)=\alpha$ and $\delta_B^{-1/2}(\begin{psmallmatrix}\varpi & a\\ & 1\end{psmallmatrix}) = |\varpi|^{1/2}$.
\end{proof}
\begin{rem}
    If we know the action of $S,T$ on $I(\chi)^K$ for some unramified character $\chi$ of the torus $T$, then we can recover $\alpha,\beta \in \CC^\times$ from the roots of the Satake polynomial $X^2-q^{-1/2}TX+S \in \mathcal H(G,K)[X]$.
\end{rem}

\newpage

%\section{Modular forms as automorphic forms}
%This will be based on notes by Jeremy Booher, the LTCC notes, Getz-Hahn and Bump.

We will view modular forms as automorphic forms for $\GL_2$. We first recall the classical definition.

\begin{defn}
    For $\Gamma \leq \GL_2^+ (\QQ)$ a subgroup commensurable with $\SL_2 (\ZZ)$ (meaning the intersection has finite index with each), we define a modular form of level $\Gamma$ and weight $(k,t)$ for $t \in \RR$ to be a function $f: \mathcal H \to \CC$ such that
    \begin{itemize}
        \item $f$ is holomorphic;
        \item $f|_{(k,t)} \gamma = f$ for all $\gamma \in \Gamma$, where $$f|_{(k,t)} \gamma(\tau) = (\det \gamma)^t (c\tau+d)^{-k} f\left(\frac{a\tau+b }{c\tau +d}\right)$$
        \item $f_{(k,t)}\gamma(\tau)$ is bounded as $\tau \to i\infty$ for all $\gamma \in \GL_2^+ (\QQ)$.
    \end{itemize}
\end{defn}

\subsection{Adelic formulation of the modular curve}

The usual action of $\GL_2^+ (\RR)$ on the upper half plane $\mathcal H$ is transitive, and the stabiliser of $i$ is $\RR^+ \cdot \mathrm{SO}_2(\RR)$. Hence $\Gamma \backslash \mathcal H$ is in bijection with $\RR^+ \Gamma \backslash \GL_2^+ (\RR) / \mathrm{SO}_2 (\RR)$.

We want to make this adelic. If $K \leq \GL_2(\adele_f)$ is a compact open subgroup, we have the quotient
$$Y(K):= \GL_2^+(\QQ) \backslash \GL_2(\adele)/ \mathrm{SO}_2(\RR)\cdot K = \GL_2^+ (\QQ) \backslash \GL_2(\adele_f) \times \mathcal H/ K$$ where $\GL_2^+(\QQ)$ acts diagonally and $K$ acts by right translation on $\GL_2(\adele_f)$ only. Given $\Gamma$ as above we will associate an adelic modular curve $Y(K)$ for $K$ the closure of $\Gamma$ in the diagonal embedding of $\GL_2^+ (\QQ)$ in $\GL_2 (\adele_f)$. For example, the closure of $\Gamma_0(N)$ is the group $K_0(N)$ of matrices in $\GL_2(\hat{\ZZ})$ which are upper triangular mod $N$.

\begin{thm}
    Strong approximation holds for $\SL_2$, in the sense that $\SL_2(\QQ)$ is dense in $\SL_2(\adele_f)$.
\end{thm}
\begin{proof}
    Since $\SL_2(\ZZ)$ surjects onto $\SL_2(\ZZ/N\ZZ)$ for all $N$, we see that the closure of $\SL_2(\QQ)$ contains $\SL_2(\hat{\ZZ})$. The Cartan decomposition tells us that
    $$\SL_2(\adele_f) = \bigsqcup\limits_{m\geq 1} \SL_2(\hat{\ZZ})\begin{psmallmatrix}
        m& \\ & m^{-1}
    \end{psmallmatrix} \SL_2(\hat{\ZZ})$$ so that the closure of $\SL_2(\QQ)$ contains everything.
\end{proof}


\begin{prop}
    There is a bijection between $\GL_2^+(\QQ)\backslash\GL_2(\adele_f)/K$ for $K$ compact open in $\GL_2(\adele_f)$, and $\QQ^+ \backslash \adele_f^\times/\det(K)$. In particular, if $\det(K) = \hat{\ZZ}$ (for example when $K=K_0(N)$), both sides are in bijection with the class group of $\QQ$, which is trivial.
\end{prop}
\begin{proof}
The determinant map gives the exact sequence
$$\xymatrix{1 \ar[r] & \SL_2(\adele_f) \ar[r] & \GL_2(\adele_f) \ar[r] & \adele_f^\times \ar[r] & 1}$$ from which we get
$$\xymatrix{1 \ar[r] & \SL_2(\QQ)\backslash\SL_2(\adele_f)/K \cap \SL_2(\adele_f) \ar[r] & \GL_2^+(\QQ)\backslash\GL_2(\adele_f)/K \ar[r] & \QQ^+ \backslash \adele_f^\times/\det(K) \ar[r] & 1.}$$
Strong approximation for $\SL_2$ tells us the first term is trivial.
\end{proof}

\begin{thm}\label{adelic curve}
    For $K \leq \GL_2(\adele_f)$ compact open, $Y(K)$ is a manifold with finitely many connected components, each (non-canonically) isomorphic to a quotient of $\mathcal H$. More precisely, if $g_1,\dots,g_n \in \GL_2(\adele_f)$ are representatives of $\GL_2^+(\QQ)\backslash\GL_2(\adele_f)/K$ (by the above proposition this is equivalent to the determinants being representatives of $\adele_f^\times/\QQ^+\det(K)$), then defining $$\Gamma_i := \GL_2^+(\QQ) \cap g_i K g_i^{-1},$$ these $\Gamma_i$ are commensurable with $\SL_2(\ZZ)$ and 
    \begin{equation*}
        \begin{split}
        \bigsqcup \Gamma_i \backslash \mathcal H &\to Y(K) \\
        \tau &\mapsto (g_i,\tau)
        \end{split}
    \end{equation*}
    is the isomorphism.
\end{thm}
\begin{proof}
    Let $\gamma_i \in \Gamma_i$. To show the map is well defined we need that $(g_i,\gamma_i\tau) \sim (g_i,\tau)$. Certainly $(g_i,\tau) \sim (\gamma_ig_i,\gamma_i\tau)$ and then this is equivalent to $(g_i,\gamma_i\tau)$ since $\gamma_i \in g_iKg_i^{-1}$. The map is then well defined and is injective and surjective by construction (from the definition of the $g_i$). The $\Gamma_i$ are commensurable with $\SL_2(\ZZ)$ because $K$ is compact open and we can check this commensurability locally. 
\end{proof}
Note the $\Gamma_i$ are left quotients but $K$ is a right quotient.
\begin{example}
    When $K=K_0(N)$ (or $K_1(N)=\begin{psmallmatrix}
        *&*\\0&1
    \end{psmallmatrix}$) using $g_1=1$ we recover $\Gamma_0(N)\backslash \mathcal H \cong Y(K_0(N))$ via $\tau \mapsto (1,\tau)$.
\end{example}
\begin{defn}
    An adelic modular form of weight $(k,t)$ is a function 
    $$F: \GL_2(\adele_f) \times \mathcal H \to \CC$$ such that
    \begin{itemize}
        \item $F(g,\tau)$ is holomorphic in $\tau$ for every $g$.
        \item There exists open compact $K \leq \GL_2(\adele_f)$ such that $F$ is invariant under right translation by $K$ in the first factor.
        \item $F(\gamma g, -) = F(g, -)|_{(k,t)} \gamma^{-1}$ for all $\gamma \in \GL_2^+(\QQ)$ (the inverse is to go between left and right actions).
        \item For all $g \in \GL_2(\adele_f)$, $F(g,\tau)$ is bounded as $\tau \to i\infty$.
    \end{itemize}
\end{defn}
\begin{rem}
    For $\gamma \in \GL_2^+(\QQ)$ we compute $$F(\gamma g, \gamma \tau) = F(g,\gamma\tau)|_{(k,t)}\gamma^{-1} = F(g,\tau) \cdot C$$ where $C$ is some constant only depending on $\gamma$ (from the $j$-factor and the determinant). Under some appropriate renormalisation this should give a function
    $$\GL_2(\QQ) \backslash \GL_2(\adele_f)/\mathrm{SO}_2(\RR)K \to \CC$$ which resembles the usual definition of an automorphic form (the -1 determinant is absorbed in replacing $\GL_2^+(\RR)$ with $\GL_2(\RR)$). This possibly (probably) trades off $K$-invariance with $K$-finiteness.
\end{rem}
\begin{notn}
    Let $M_{k,t}$ and $S_{k,t}$ be the space of such (cuspidal) adelic modular forms.
\end{notn}
The spaces $M_{k,t}$ and $S_{k,t}$ are representations of $\GL_2(\adele_f)$ under right translation on the first factor, and by definition (invariance under some compact open $K$) this representation is smooth.

\begin{prop}
    Evaluation at the $g_1,\dots,g_n$ in Theorem \ref{adelic curve} gives an isomorphism 
    $$(S_{k,t})^K = \bigoplus\limits_{i=1}^n S_{k,t}(\Gamma_i)$$
    where the right hand side consists of cusp forms in the classical sense. A similar result holds for $M_{k,t}$. In particular, these gives admissible representations of $\GL_2(\adele_f)$.
\end{prop}
\begin{proof}
    Invert the isomorphism of Theorem \ref{adele curve} and check that the axioms defining a modular form match up.
\end{proof}

\begin{rem}
    It is convenient to work with the space of all (cuspidal) adelic modular forms without having to specify the level, instead incorporating the level through the fixed points.
\end{rem}
\begin{example}
    If $K=K_1(N) = \left\{\begin{psmallmatrix}
        * & * \\ & 1
    \end{psmallmatrix}\right\}$ we only have $g_1=1$ and we recover $S_{k,t}(\Gamma_1(N))$.
\end{example}

%maybe add more definitions and details later

\section{Principal series representations of \texorpdfstring{$\GL_2$}{TEXT}}


Let $F$ be a nonarchimedean local field, $G=\GL_2(F)$, and $B=\{\begin{psmallmatrix} a&b\\0&d\end{psmallmatrix} \mid a,d \in F^\times, b \in F\}$ the Borel subgroup of upper triangular matrices, so that $B=N \rtimes T$ for $T=\{\begin{psmallmatrix}
    a&0\\0&d
\end{psmallmatrix}\mid a,d \in F^\times\} \cong F^\times \times F^\times$ and $N=\{\begin{psmallmatrix}
    1&b\\0&1
\end{psmallmatrix}\mid b \in F\}\cong F$. Between $N$ and $B$ we also have the mirabolic subgroup $M=\{\begin{psmallmatrix}
    a&b\\0&1
\end{psmallmatrix} \mid a\in F^\times, b \in F\}$ with $M/N \cong F^\times$.

In studying the local Langlands correspondence, we want to understand all the irreducible smooth representations of $G$. One method for producing representations of $G$ is by induction from a subgroup of $G$. Typically one takes this subgroup to be `parabolic'; in our case there is one nontrivial parabolic, namely $B$. From our decomposition $B=N \rtimes T$ (more generally we have a so-called Levi decomposition) we see that we can produce representations of $B$ by inflating representations of the torus $T$. Since $T \cong F^\times \times F^\times$, the irreducible representations of $T$ are products of characters of $F^\times$, which are relatively easy to get a handle on.

\begin{defn}
    For $\chi:T \to \CC^\times$ a character of the torus, we say that the representation $\Ind_B^G \chi$ is a \textit{parabolically induced representation}. A \textit{principal series representation} is an irreducible subrepresentation of a parabolically induced representation.
\end{defn}

In this section, we will only concern ourselves with classifying the principal series representations of $G$. This means that we must understand how $\Ind_B^G \chi$ decomposes into irreducible representations of $G$, and also study the morphisms between them using Frobenius reciprocity.

To understand these decompositions, we want to study how they decompose into irreducibles over a less unwieldy subgroup of $G$, such as $B$. Note that restricting $\Ind_B^G \chi$ to $B$ is analogous to applying Mackey theory in the finite group context. It turns out that the $\Ind_B^G \chi$ do not decompose any further over $M$ than over $B$. On the other hand, the representation theory of $M$ is very easy to classify - the combination of these two observations is what makes the mirabolic subgroup so `miraculous'. To get representations of $M$ we can induce from characters of $N$, or inflate from $M/N\cong F^\times$. There are many characters of $N\cong F$, in fact these are in bijection with $F$ by Additive Duality \ref{add_dual}. The key property of $M$ is that conjugation by $M$ acts transitively on these characters $\psi$, which greatly simplifies the representation theory of $M$ coming via induction from $N$. The mirabolic $M$ is also small enough that this induction, together with the characters of $F^\times$, give all irreducible representations of $M$.

In this section, we begin by studying the representations of $N$ and introducing the Jacquet functor, before discussing representations of $M$. From there we determine that parabolically induced representations of $G$ decompose over $M$ with length at most 3. Theorem \ref{criterion} gives the decomposition of $\Ind_B^G \chi$ into irreducible representations of $G$, and then Theorem \ref{classify} lists the isomorphism classes of principal series representations. The presentation follows sections 8 and 9 of \cite{BH1}.

\subsection{Representations of \texorpdfstring{$N$}{TEXT}}

We first study the representation theory of $N \cong F$. This is an abelian group so, by Schur's lemma, all irreducible representations are characters (Corollary 2.6.2 \cite{BH1}). For finite abelian groups, any representation $V$ decomposes into a direct sum of characters. This is no longer true when $N\cong F$ is infinite, but it is still true that any vector in $V$ is nonzero in some quotient on which $N$ acts via a character. To formalise this, we define

\begin{notn}
    Let $V$ be a smooth representation of $N$ and $\theta$ a character of $N$. Let $V(\theta) \leq V$ be the subspace spanned by $\{n\cdot v - \theta(n)v \mid n \in N, v \in V\}$. Set $V_\theta = V/V(\theta)$ so that $N$ acts on $V_\theta$ by $\theta$. When $\theta$ is trivial we write $V(N)$ and $V_N$ respectively. 
\end{notn}

The following is a useful equivalent definition of $V(\theta)$:

\begin{lemma}\label{criteria N}
    The vector $v \in V$ lies in $V(\theta)$ if and only if 
    $$\int_{N_0} \theta(n)^{-1}  n \cdot v dn = 0$$
    for some compact open subgroup $N_0$ of $N$.
\end{lemma}
In the lemma we restrict to compact opens for the integral to be well defined.

\begin{proof}
    \cite[Lemma 8.1]{BH1}.
\end{proof}

\begin{cor}\label{exact}
    The functor $V \mapsto V_\theta$ from smooth representations of $N$ to complex vector spaces is exact.
\end{cor}
\begin{proof}
    One checks formally that the functor is right exact. For left exactness we need to show that if $f: V \hookrightarrow V'$ is injective then $V_\theta \hookrightarrow V'_\theta$ is injective. If $v \in V$ with $f(v) \in V'(\theta)$, then 
    $$\int_{N_0} \theta(n)^{-1}n \cdot f(v) dn = 0$$
    for some $N_0$ by the above lemma. Since $f$ is compatible with the action of $N$, we can pull $f$ out of the integral so that the injectivity of $f$ implies
    $$\int_{N_0} \theta(n)^{-1}n \cdot v dn = 0.$$
    We deduce that $v \in V(\theta)$ by the above lemma.
\end{proof}

\begin{prop}
    Let $V$ be a smooth representation of $N$. For any $v \neq 0$ in $V$, there exists a character $\theta$ of $N$ such that $v \not\in V(\theta)$.
\end{prop}
\begin{proof}
    \cite[Proposition 8.1]{BH1}.
\end{proof}

\begin{cor}\label{character}
    If $V$ is a smooth representation of $N$ such that $V_\theta=0$ for all $\theta$ then $V=0$.
\end{cor}


\subsection{Representations of \texorpdfstring{$M$}{TEXT}}

Now we consider $V$ an irreducible smooth representation of $M$. 

\begin{lemma}\label{coinvariants}
	The subspace $V(N) \leq V$ is a representation of $M$, and so $V_N$ is as well. Moreover, $S = \{\begin{psmallmatrix}
        a&0\\0&1
    \end{psmallmatrix} \mid a \in F^\times\}$ permutes the subspaces $V(\theta)$ with $\theta \not= 1$ transitively, and hence the $V_\theta$ are isomorphic as vector spaces.
\end{lemma}
\begin{proof}
	The first claim comes from the computation 
	$$mn\cdot v - m\cdot v = n'm\cdot v - m\cdot v$$ for some $n' \in N$, using the fact that $N \lhd M$. For the second claim we have the computation
	\[s(nv - \theta(n)v) = sns^{-1}\cdot sv - \theta (s^{-1}(sns^{-1})s)sv = n'\cdot sv - \theta(s^{-1}n's) sv\]
	where $n' = sns^{-1}\in N$. Hence $sV(\theta) = V(\theta')$ where $\theta'(n) := \theta(s^{-1}ns)$. Now the computation
	\[\begin{pmatrix}
		a & 0 \\ 0 & 1
	\end{pmatrix}\begin{pmatrix}
		1 & x \\ 0 & 1
	\end{pmatrix}\begin{pmatrix}
		a^{-1} & 1 \\ 0 & 1
	\end{pmatrix} = \begin{pmatrix}
		1 & ax \\ 0 & 1
	\end{pmatrix}\]
	together with Additive Duality \ref{add_dual} implies the claim.
\end{proof}



\begin{thm}\label{inf dim}
    Let $(\pi,V)$ be an irreducible smooth representation of $M$. Either 
    \begin{itemize}
        \item $\dim V=1$ and $\pi$ is the inflation of a character of $M/N \cong F^\times$, or
        \item $\dim V = \infty$ and $\pi \cong c\mathrm{-Ind}_N^M \theta$, for any nontrivial character $\theta$ of $N$.
    \end{itemize}
\end{thm}

This itself follows from the following theorems. To compare $V$ and $c\mathrm{-Ind}_N^M \theta$, it is more natural to compare $V$ and $\mathrm{Ind}_N^M V_\theta$. By Frobenius reciprocity,
$$\Hom_N(V,V_\theta) \cong \Hom_M(V,\mathrm{Ind}_N^M V_\theta).$$
Let $q_*: V \to \Ind_N^M(V_\theta)$ be the image of the quotient map $q: V \to V_\theta$.

\begin{thm}\label{mirabolic}
    The $M$-homomorphism $q_*: V \to \Ind_N^M V_\theta$ induces an isomorphism $V(N) \cong c\mathrm{-Ind}_N^M V_\theta$. 
\end{thm}
\begin{proof}
    \cite[Theorem 8.3]{BH1}.
\end{proof}

\begin{thm}\label{irred induction}
    For any nontrivial character $\theta$ of $N$, the smooth representation $c\mathrm{-Ind}_N^M \theta$ of $M$ is irreducible. 
\end{thm}
\begin{proof}
    \cite[Corollary 8.2]{BH1}
\end{proof}

\begin{proof}[Proof of Theorem \ref{inf dim}]
    If $V$ is an irreducible smooth representation of $M$, then either $V(N)=0$ or $V(N)=V$. In the former case $N$ acts trivially on $V$, so the action of $M$ factors through $M/N \cong F^\times$. Schur's lemma implies that $V$ is a character of $M$ factoring through $M/N$. 
    
    In the latter case, $V_N=0$, so we must have $V_\theta \neq 0$ for all nontrivial characters of $N$ by Lemma \ref{coinvariants} and Corollary \ref{character}. Thus the $M$-representation $V$ must have infinite dimension, since there are infinitely many characters $\theta$. Theorem \ref{mirabolic} implies that $V = V(N)$ is isomorphic to $c\mathrm{-Ind}_N^M V_\theta$, which is a direct sum of copies of $c\mathrm{-Ind}_N^M \theta$. Since $c\mathrm{-Ind}_N^M \theta$ is irreducible by Theorem \ref{irred induction}, we must have $V\cong c\mathrm{-Ind}_N^M \theta$. 
\end{proof}




\subsection{Irreducible principal series representations}

Let $V$ be a smooth representation of $G$. In the preceding subsections, we defined the quotient $V_N =V/V(N)$, called the $N$-coinvariants of $V$. As in Lemma \ref{coinvariants}, this is a representation of $B$ (as $N \lhd B$). As $N$ acts trivially on $V_N$, $V_N$ inherits the structure of a representation of $T=B/N$.

\begin{defn}
    Let $V$ be a smooth representation of $G$ (or $B$). The \textit{Jacquet module} of $V$ at $N$ is the space of $N$-coinvariants $V_N$ viewed as a representation of $T$. The \textit{Jacquet functor} is the functor sending the $G$-representation $(\pi,V)$ to the $T$-representation $(\pi_N,V_N)$. 
\end{defn}

By Corollary \ref{exact}, the Jacquet functor is exact.

If $V$ is a representation of $G$, and $\chi$ is a character of $T$, then we have by Frobenius Reciprocity that
\[\Hom_G(V, \Ind_B^G \chi) \cong \Hom_B(V, \chi)\]
But since $\chi$ as a character $B$ has trivial $N$-action, maps $V\to \chi$ factor through $V_N$, and we obtain a version of Frobenius reciprocity for the Jacquet module:
\[\Hom_G(V, \Ind_B^G \chi)\cong \Hom_T(V_N, \chi)\]
i.e. the Jacquet module is left adjoint to parabolic induction.

In the classical setting of representations of $\mathbf G=\GL_2(k)$ for a finite field $k$, we have the following dichotomy (where $\mathbf B,\mathbf T,\mathbf N$ are the appropriate subgroups of $\mathbf G$):
\begin{lemma}
    Let $(\pi,V)$ be an irreducible representation of $\mathbf G$. The following are equivalent:
    \begin{enumerate}
        \item $\pi$ contains the trivial character of $\mathbf N$
        \item $\pi$ is isomorphic to a $\mathbf G$-subrepresentation of $\Ind_{\mathbf B}^{\mathbf G} \chi$ for some character $\chi$ of $\mathbf T$ inflated to $\mathbf B$.
    \end{enumerate}
\end{lemma}
\begin{proof}
    \cite[Lemma 6.3]{BH1}.
\end{proof}

Returning to $G=\GL_2(F)$, if $(\pi,V)$ is a smooth representation, the restriction to $N$ is no longer necessarily semisimple because $F$ is of infinite order. We instead replace the condition that $\pi |_N$ contains the trivial character of $N$ with the condition that $N$ acts trivially on some nonzero quotient of $V$ (which is an equivalent condition in the finite field case). This is measured by the Jacquet module $V_N$. There is the analogous dichotomy which tells us that principal series representations can be identified as the irreducible smooth representations of $G$ with nonzero Jacquet module: 

\begin{prop}
    Let $(\pi,V)$ be an irreducible smooth representation of $G$. The following are equivalent:
    \begin{enumerate}
        \item $V_N \neq 0$
        \item $\pi$ is isomorphic to a $G$-subrepresentation of $\Ind_B^G \chi$ for some character $\chi$ of $T$ inflated to $B$.
    \end{enumerate}
\end{prop}
\begin{proof}[Proof sketch]
    (2) implies (1) is a consequence of Frobenius reciprocity:
    $$\Hom_G(\pi,\Ind \chi) = \Hom_T(\pi_N,\chi)$$

    Given (1), one shows by a technical argument that $V_N$ is finitely generated as a representation of $T$. An application of Zorn's lemma allows us to construct a maximal $T$-subspace $U$ of $V_N$, so that $V_N/U$ is a nonzero irreducible $T$-representation, and is thus a character $\chi$ by Schur's lemma. The above Frobenius reciprocity implies (2).
\end{proof}

\begin{rem}
    The same proof holds for the finite field case, where we bypass the technical details in showing (1) implies (2) because any representation of the finite group $T$ admits an irreducible quotient.
\end{rem}

\begin{rem}
    We ask for a nonzero Jacquet module $V_N$ rather than a trivial $N$-subrepresentation of $V$ because of the following fact:
\end{rem}

\begin{lemma}
    Let $(\pi,V)$ be an irreducible smooth representation of $G$ with a nonzero vector $v\in V$ fixed by $N$. Then $\pi = \phi \circ \det$, for some character $\phi$ of $F^\times$. In particular, $\pi$ is one dimensional.
\end{lemma}
\begin{proof}[Proof sketch]
    The vector $v$ is fixed by $N$, but also by a compact open subgroup $K$ of $G$ by smoothness. As we are working with $F$ a nonarchimedean local field (as opposed to a finite field), this implies $K$ contains a unipotent lower triangular matrix, and one shows that $v$ is fixed by $\mathrm{SL}_2(F)$. Thus $\pi$ factors through $\det$.
\end{proof}


Once again, let $\chi$ be a character of $T$ and let $(\Sigma,X)$ denote $\Ind_B^G \chi$. We want to study how $X$ decomposes into irreducible $G$-representations. As mentioned earlier, we will begin by studying their decompositions over $B$ or even $M$. 

To begin with, $X$ will never be irreducible over $B$ because we always have the canonical $B$-homomorphism $\Sigma \to \chi$, given by sending $f \mapsto f(1) \in \CC$. So we have an exact sequence of $B$-representations
$$\xymatrix{0 \ar[r] & V \ar[r] & X \ar[r] & \CC \ar[r] & 0,}$$
where $V=\{f \in X  \mid f(1)=0\}$, and $B$ acts on $\CC$ via $\chi$. Now we want to understand how $V$ decomposes over $B$. We have another exact sequence of $B$-representations,
$$\xymatrix{0 \ar[r] & V(N) \ar[r] & V \ar[r] & V_N \ar[r] & 0,}$$
so we reduce to studying $V(N)$ and $V_N$. We will show that $V(N)$ is irreducible over $B$ (and even over $M$), while $V_N$ will be determined by the Restriction-Induction lemma. 

%(which generally treats the exact sequence obtained by applying the Jacquet functor to the first exact sequence, where we may replace $\chi$ by any smooth representation $\sigma$ of $T$).

The following lemma makes the structure of $V$ more apparent.

\begin{lemma}\label{reduce to N}
    Let $V=\{f \in X:f(1)=0\}$. The map 
    \begin{equation*}
        \begin{split}
            V &\to C_c^\infty(N) \\
            f(-) &\mapsto f(w-) 
        \end{split}
    \end{equation*}
    is an $N$-isomorphism (with $N$ acting by right translation on either side), where $w=\begin{psmallmatrix}
        0&1\\1&0
    \end{psmallmatrix}$.
\end{lemma}
\begin{proof}
    We have the Bruhat decomposition $G=B \sqcup BwN$. Since $f(1)=0$, and $f$ is induced from $B$, we must have that $f$ is supported on $BwN$. $G$-smoothness of $f$ implies that $f$ is also zero on some compact open $K \leq G$. This will contain $\begin{psmallmatrix}
        1&0\\\varpi^n \mathcal O &1
    \end{psmallmatrix}$ for some $n$, so that $f$ vanishes on 
    $$\begin{pmatrix}
        1&0\\x&1
    \end{pmatrix} \in Bw \begin{pmatrix}
        1&x^{-1}\\0&1
    \end{pmatrix}$$
    for all $x \in \varpi^n \mathcal O$. Thus $f(w-)$ is supported on $\begin{psmallmatrix}
        1&y\\0&1
    \end{psmallmatrix} \in N$ with $v(y) > -n$ and so is compactly supported. $G$-smoothness of $f$ also implies that $f(w-)$ is $N$-smooth. Since $f$ is induced from $B$ and is supported on $BwN$, the map is injective. Conversely, any $g \in C_c^\infty(N)$ determines $f \in \Ind_B^G \chi$ such that $f(w-)=g$ and $f(B)=0$.
\end{proof}

\begin{prop}
    For $V$ as above, $V(N)$ is irreducible over $M$ (and hence over $B$). Moreover, $V(N)$ is infinite dimensional.
\end{prop}
\begin{proof}
    The idea will be to use Theorem \ref{mirabolic}, which tells us $V(N) \cong c\mathrm{-Ind}_N^M V_\theta$. This is irreducible over $M$ (and infinite dimensional) if we can show that $V_\theta$ is one dimensional, by Theorem \ref{irred induction}.

    By the above lemma we can identify $V \cong C_c^\infty(N)$ as $N$-representations. But $M$ also acts via right translation on $V$ (since $BwB=BwN=BwM$), which gives the structure of a $M$-representation on $C_c^\infty(N)$. We can calculate it explicitly (but we won't need it), where
    $$f\left(bw\begin{pmatrix}
        1&x\\0&1
    \end{pmatrix}\begin{pmatrix}
        a&0\\0&1
    \end{pmatrix}\right) = f\left(b \begin{pmatrix}
        1&0\\0&a
    \end{pmatrix}w\begin{pmatrix}
        1&a^{-1}x\\0&1
    \end{pmatrix}\right)$$
    tells us that the corresponding $M=F^\times N$ action on $C_c^\infty(N)$ is the composite of right translation by $N$ with the action 
    $$a\cdot \phi \begin{pmatrix}
        1&x\\0&1
    \end{pmatrix} = \chi_2(a) \phi \begin{pmatrix}
        1&a^{-1}x \\ 0&1
    \end{pmatrix}$$ of $a \in F^\times$.

    So now we may consider $V=C_c^\infty(N)$. The benefit is that for this representation, the spaces of coinvariants of characters $\theta$ of $N$ are very simple. In particular, the map $f \mapsto \theta f$ is a linear automorphism of $C_c^\infty(N)$ taking $V(N)$ to $V(\theta)$, since $$n \cdot f - f \mapsto \theta (n \cdot f) - \theta f = \theta(n)^{-1} n \cdot (\theta f) - \theta f \in V(\theta).$$
    Hence all the $V_\theta$ have the same dimension as $V_N=V/V(N)$, which has dimension 1 (we can see this from the characterisation of $V(N)$ as the zeros of some integral (Lemma \ref{criteria N}), or from the Restriction-Induction lemma to follow). The result follows from Theorem \ref{mirabolic} and Theorem \ref{irred induction}.
\end{proof}

We turn our attention to the Jacquet module $V_N$. Recall $V$ fits in the exact sequence
$$\xymatrix{0 \ar[r] & V \ar[r] & X = \mathrm{Ind}_B^G \chi \ar[rr]^{f \mapsto f(1)} && \CC \ar[r] & 0}$$ of smooth representations of $B$, where $B$ acts via $\chi$ on $\CC$. Since the Jacquet functor is exact, we get the exact sequence

$$\xymatrix{0 \ar[r] & V_N \ar[r] & X_N \ar[r] & \CC \ar[r] & 0}$$ of $T$-representations. The following lemma determines the structure of $V_N$ as a $T$-representation. This can be stated in more generality:

\begin{lemma}[Restriction-Induction lemma]
    Let $(\sigma, U)$ be a smooth representation of $T$ and $(\Sigma, X) = \mathrm{Ind}_B^G \sigma$. Then there is an exact sequence of smooth $T$ representations:
    $$\xymatrix{0 \ar[r] & \sigma^w \otimes \delta_B^{-1} \ar[r] & \Sigma_N \ar[r] & \sigma \ar[r] & 0.}$$
    Here, $\sigma^w(t) := \sigma(wtw)$ for $w=\begin{psmallmatrix}
        0&1\\1&0
    \end{psmallmatrix}$, so that if $\sigma$ is the character $\chi_1 \otimes \chi_2$ of $T$, then $\sigma^w = \chi_2\otimes \chi_1$.
\end{lemma}
\begin{proof}
    The proof of Lemma \ref{reduce to N} generalises to show that the vector space $V = \{f \in X \mid f(1)=0\}$ is isomorphic, as $N$-representations, to the space $\mathcal S$ of smooth compactly supported functions $N \to U$, by identifying $f$ with $f(w-)$.



    We can define a map $\mathcal S \to U$ by 
    $$g=f(w-) \mapsto \int_N f(wn) dn,$$ where this integral is finite since $g$ is compactly supported. By Lemma \ref{criteria N}, this induces an isomorphism $\mathcal S_N \cong U$.


    Now $V$ also carries the structure of a $B$-representation as well, since $BwB=BwN$. We can repeat the same calculation as in the previous proposition, replacing $F^\times$ with $T\cong F^\times \times F^\times$, to compute the action of $B=TN$ on $\mathcal S$. As usual, $N$ acts via right translation. If $t=\begin{psmallmatrix}
        t_1&0 \\0 & t_2
    \end{psmallmatrix} \in T$, then for $\phi \in \mathcal S$, 
    $$t\cdot \phi \begin{pmatrix}
        1&x\\0&1
    \end{pmatrix} = \sigma^w(t) \phi \begin{pmatrix}
        1&\frac{t_2}{t_1}x \\ 0&1
    \end{pmatrix}.$$

    Thus the $T$-representation structure on $U \cong \mathcal S_N \cong V_N$ is given by 

    $$t \cdot \int_N f(wn)dn = \sigma^w(t) \left| \frac{t_1}{t_2} \right| \int_N f(wn)dn,$$
    which is $\sigma^w \otimes \delta_B^{-1}$.
\end{proof}

\begin{cor}\label{length 3}
    As a representation of $B$ or $M$, $\mathrm{Ind}_B^G \chi$ has composition length 3. Two of the factors have dimension 1, and the other is infinite dimensional.
\end{cor}
\begin{proof}
    This follows from the exact sequences
    $$\xymatrix{0 \ar[r] & V \ar[r] & \mathrm{Ind}_B^G \CC \ar[r] & \chi \ar[r] & 0}$$
    and
    $$\xymatrix{0 \ar[r] & V(N) \ar[r] & V \ar[r] & V_N \ar[r] & 0}$$
    where we saw that $V(N)$ is irreducible and infinite dimensional, and $V_N \cong \chi^w \otimes \delta_B^{-1}$.
\end{proof}

So we understand how $\mathrm{Ind}_B^G \chi$ decomposes into irreducible $B$-representations, and we want to understand its decomposition into $G$-representations. Our goal is to prove the following:

\begin{thm}[Irreducibility Criterion]\label{criterion}
    Let $\chi = \chi_1 \otimes \chi_2$ be a character of $T$ and let $X = \mathrm{Ind}_B^G \chi$.
    \begin{enumerate}
        \item The representation $X$ of $G$ is irreducible if and only if $\chi_1\chi_2^{-1}$ is either the trivial character of $F^\times$, or the character $x \mapsto |x|^2$ of $F^\times$.
        \item Suppose $X$ is reducible. Then \begin{itemize}
            \item the $G$-composition length of $X$ is 2
            \item one factor has dimension 1, the other is infinite dimensional
            \item $X$ has a 1-dimensional $G$-subspace exactly when $\chi_1\chi_2^{-1}=1$
            \item $X$ has a 1-dimensional $G$-quotient exactly when $\chi_1\chi_2^{-1}(x) = |x|^2$.
        \end{itemize}
    \end{enumerate}
\end{thm}

We make some comments in preparation for the proof. Suppose $X$ is a reducible representation of $G$, and $X_0$ a nonzero proper subrepresentation. If $X_0$ is finite-dimensional, then its composition factors over $B$ can only consist of the 1-dimensional composition factors of $X$ over $B$ described in Corollary \ref{length 3}. If $X_0$ is infinite dimensional, then it contains the infinite-dimensional $B$-composition factor of Corollary \ref{length 3}, and so the quotient $X/X_0$ can only consist of the 1-dimensional factors. In all, if $X$ is reducible then it has a finite dimensional (dimension 1 or 2) $G$-subspace or $G$-quotient. By taking duals we can assume we are in the first case. In the Irreducibility Criterion, we want to show that this implies $\chi_1 = \chi_2$ and that $X$ has a 1-dimensional $G$-subspace.

\begin{defn}
    Let $\pi$ be a smooth representation of $G$ and $\phi$ a character of $F^\times$. The \text{twist of $\pi$ by $\phi$} is the representation $\phi\pi$ of $G$ defined by 
    $$\phi \pi(g) = \phi (\det g)\pi(g).$$
    In this way, for a character $\chi=\chi_1 \otimes \chi_2$ of $T$, we have $\phi\chi = \phi\chi_1 \otimes \phi\chi_2$. 
\end{defn}

\begin{lemma}
    For $\chi$ a character of $T$ and $\phi$ a character of $F^\times$, we have $\Ind_B^G(\phi\chi) = \phi\Ind_B^G \chi$.
\end{lemma}
\begin{proof}
    Since $\phi\chi(b) = \phi \circ \det(b) \chi(b)$ for any $b \in B$, where $\chi$ is inflated from $T$, we see that 
    $$(\phi \circ \det)(bg)f(bg) = \phi\chi(b)(\phi \circ \det)(g)f(g)$$ for any $f \in \Ind_B^G \chi$. Thus the map $f \mapsto (\phi \circ \det)f$ from $\Ind_B^G \chi \to \Ind_B^G(\phi\chi)$ is well defined on the underlying vector spaces. This induces an isomorphism of representations of $G$, $\phi\Ind_B^G \chi \cong \Ind_B^G(\phi\chi)$.
\end{proof}

\begin{prop}
    The following are equivalent:
    \begin{enumerate}
        \item $\chi_1=\chi_2$
        \item $X$ has a 1-dimensional $N$-subspace.
    \end{enumerate}
    If this holds then this subspace is also a $G$-subspace of $X$ not contained in $V$.
\end{prop}
\begin{proof}
    (1) implies (2): since induction commutes with twisting we may assume $\chi_1=\chi_2=1$. Then any nonzero constant function spans a 1-dimensional $G$-subspace (not just $N$-subspace) of $X = \mathrm{Ind}_B^G 1$.

    (2) implies (1): suppose this subspace is spanned by $f$. The group $N$ acts as a character on this subspace via right translation. We cannot have $f \in V$ (meaning $f(1)=0$) because we saw earlier that $f$ would then have support in some $BwN_0$ for $N_0 \leq N$ open compact, and this is not closed under multiplication by $N$.

    So $f \not\in V$ and therefore its image spans $X/V \cong \CC$, on which $B$ acts via $\chi$. On this quotient, $N$ acts trivially because $\chi$ was inflated from $B/N=T$. Thus $f$ is in fact fixed by $N$ under right translation. But $f$ is also fixed under right translation by some compact open of $G$, so for sufficiently large $|x|$ we have
    \begin{equation*}
        \begin{split}
            f(w) = f\left(w \begin{psmallmatrix}
                1&x \\0 & 1
            \end{psmallmatrix}\right) &= f\left( \begin{psmallmatrix}
                1&x^{-1} \\0 & 1
            \end{psmallmatrix}\begin{psmallmatrix}
                -x^{-1}&0 \\0 & x
            \end{psmallmatrix}\begin{psmallmatrix}
                1&0 \\x^{-1} & 1
            \end{psmallmatrix}\right) \\
            &= f\left( \begin{psmallmatrix}
                1&x^{-1} \\0 & 1
            \end{psmallmatrix}\begin{psmallmatrix}
                -x^{-1}&0 \\0 & x
            \end{psmallmatrix}\right) \\
            &= \chi_1(-1) \left( \chi_1^{-1}\chi_2(x)\right) f(1).
        \end{split}
    \end{equation*}

    The first equality comes from $f$ being fixed by $N$. The third equality comes from $f$ being fixed by a compact open subgroup of $G$.

    This tells us that $\chi_1^{-1}\chi_2(x)$ is constant for $|x|$ sufficiently large. In particular, for large $|x|$ we have $\chi_1^{-1}\chi_2(x) = \chi_1^{-1}\chi_2(x^2) = (\chi_1^{-1}\chi_2(x))^2$. We deduce that $\chi_1(x)=\chi_2(x)$ for $|x|$ sufficiently large. Now for any $y \in F^\times$, we can pick $|x|$ large enough so that $\chi_1(x)=\chi_2(x)$ and $\chi_1(xy)=\chi_2(xy)$, from which we deduce that $\chi_1(y)=\chi_2(y)$. 
    
    %The uniqueness of the 1-dimensional subspace comes from the fact that it must span $X/V \cong \CC$. 
    % if we had two then any linear combination must also not live in V, then the 2D space does not intersect the kernel of X to X/V, but this is 1D.
\end{proof}

\begin{proof}[Proof of Irreducibility Criterion]
    Assume that $X$ is reducible and we are in the case that $X$ has a finite dimensional $G$-subspace. It has a 1-dimensional $N$-subspace $L$ because $N$ is abelian. Then $L$ is also a $G$-subspace by the above proposition. Since $G$ must act via a character on $L$, it factors as $\phi \circ \det$, where $\chi_1=\phi=\chi_2$. 

    Let $Y$ be the $G$-representation $X/L$. Since $L$ spans the vector space $X/V$, the $B$-homomorphism \\${V \hookrightarrow X \to X/L}$ is surjective. It is injective since $L \cap V = 0$. Thus $Y \cong V$ as $B$-representations.

    We need to show that $X$ has $G$-length 2. By the Corollary \ref{length 3} it has length at most 3. We know that $V$ has $B$-length 2 with a 1-dimensional quotient $V_N$. If $Y$ had $G$-length 2, then the $B$-factors of $V$ are also $G$-factors, so that $G$ must act on $V_N$, necessarily by a character $\phi' \circ \det$. But this is impossible because $B \leq G$ acts on $V_N$ by $\phi \delta_B^{-1}$ by Restriction-Induction, and this does not factor through $\det$ on $B$. So we must have that $Y$ is irreducible over $G$ and so $X$ has $G$-length 2.

    In the other case we have a finite dimensional $G$-quotient. The smooth dual $X^\vee$ is then in the first case, where the Duality Theorem \ref{thm:duality} tells us that $X^\vee \cong \mathrm{Ind}_B^G \delta_B^{-1} \chi^\vee$. If we write $\delta_B^{-1} \chi^\vee = \psi_1 \otimes \psi_2$ then we must have $\psi_1 = \psi_2$. Computing $\psi_1(x) = |x|^{-1} \chi_1(x)$ and $\psi_2(x) = |x| \chi_2(x)$ gives $\chi_1\chi_2^{-1} = |\cdot|^2$.

    The converse direction to (1) follows from the previous proposition.
\end{proof}


\subsection{Classification of principal series representations}


Now that we've seen how parabolically induced representations decompose into irreducibles, we want to classify the isomorphism classes.

\begin{prop}
    Let $\chi, \xi$ be characters of $T$. The space $\Hom_G(\mathrm{Ind}_B^G \chi, \Ind_B^G \xi)$ is 1-dimensional if $\xi = \chi$ or $\chi^w \delta_B^{-1}$ and 0 otherwise.
\end{prop}
\begin{proof}
    Frobenius reciprocity tells us
    $$\Hom_G(\mathrm{Ind}_B^G \chi, \Ind_B^G \xi) \cong \Hom_T((\Ind \chi)_N, \xi).$$
    From the Restriction-Induction lemma we have the exact sequence of $T$-modules
    $$\xymatrix{0 \ar[r] & \chi^w \delta_B^{-1} \ar[r] & (\Ind \chi)_N \ar[r] & \chi \ar[r] & 0.}$$

    By taking duals of these finite dimensional $T$-modules, we see that both $\chi$ and $\chi^w \delta_B^{-1}$ are subrepresentations of $(\Ind \chi)_N$. In the case $\chi \neq \chi^w \delta_B^{-1}$ we must have $(\Ind \chi)_N = \chi \oplus \chi^w \delta_B^{-1}$ and the result follows. If $\chi = \chi^w \delta_B^{-1}$ then $\chi_1\chi_2^{-1} (x) = |x|$ so $\Ind \chi$ is irreducible and the result still follows from Schur's lemma.
\end{proof}

\begin{rem}
In the case that $\Ind \chi$ is irreducible, we deduce that $\Ind \chi \cong \Ind \chi^w \delta_B^{-1}$. And in the case $\Ind \chi$ is reducible, it is not semisimple, else $\Hom_G(\mathrm{Ind}_B^G \chi, \Ind_B^G \chi)$ would have dimension strictly greater than 1.
\end{rem}

We can be more explicit in the reducible case. One can check that the conditions for reducibility in the Irreducibility Criterion are equivalent to $\chi$ being of the form $\chi = \phi 1_T$ or $\chi =\phi \delta_B^{-1}$ for $\phi$ a character of $F^\times$. Untwisting, we may as well assume $\phi=1$ in what follows.

\begin{defn}
    The \textit{Steinberg representation} of $G$ is defined by the exact sequence
    $$\xymatrix{0 \ar[r] & 1_G \ar[r] & \Ind_B^G 1_T \ar[r] & \mathrm{St}_G \ar[r] & 0,}$$ and is an infinite-dimensional irreducible smooth representation. By Restriction-Induction, the Jacquet module is $(\mathrm{St}_G)_N \cong \delta_B^{-1}$. The representations $\phi \mathrm{St}_G$ are called `twists of Steinberg' or `special representations'.
\end{defn}

The case $\chi = \delta_B^{-1}$ can be dealt with by taking smooth duals (which is exact by \cite[Lemma 2.10]{BH1}) to get 
$$\xymatrix{0 \ar[r] & \mathrm{St}_G^\vee \ar[r] & \Ind_B^G \delta_B^{-1} \ar[r] & 1_G \ar[r] & 0,}$$ where we use the Duality Theorem \ref{thm:duality}. The Irreducibility Criterion implies that $\mathrm{St}_G^\vee$ is also irreducible, and in fact the previous proposition applied to $\chi=1, \xi = \delta_B^{-1}$ implies that
$$\mathrm{St}_G \cong \mathrm{St}_G^\vee.$$
% we can see irreducibility by prop 2.10 and using admissibility 
% add our own duality thm?

\begin{notn}
    Define normalised induction by
    $$\iota_B^G \sigma = \Ind_B^G (\delta_B^{-1/2} \otimes \sigma).$$
    This has the benefit that $(\iota_B^G \sigma)^\vee \cong \iota_B^G \sigma^\vee$ (Theorem \ref{thm:duality}).
\end{notn}

\begin{thm}[Classification Theorem]\label{classify}
    The following are all the isomorphism classes of principal series representations of $G$:
    \begin{itemize}
        \item the irreducible induced representations $\iota_B^G \chi$ when $\chi \neq \phi \delta_B^{\pm 1/2}$ for a character $\phi$ of $F^\times$.
        \item the one-dimensional representations $\phi \circ \det$ for $\phi$ a character of $F^\times$.
        \item the twists of Steinberg (special representations) $\phi \mathrm{St}_G$ for $\phi$ a character of $F^\times$.
    \end{itemize}
    These are all distinct isomorphism classes except in the first case where $\iota_B^G \chi \cong \iota_B^G \chi^w$.
\end{thm}

% Reference additive duality from section 1 when ready. Also duality thm end of 3.3
% explain characters of torus

\section{Functional equation for \texorpdfstring{$\GL_2$}{TEXT}}
In the previous section, we classified the principal series representations of $G=\GL_2(F)$ over a nonarchimedean local field $F$. For characters $\chi$ of $\GL_1(F)$, Tate's thesis \cite{Tate} associates a space $\mathcal Z(\chi)$ of zeta functions in a complex variable $s$. This space will, in a sense to be made precise, be generated by a single element, the $L$-function $L(\chi,s)$. The zeta functions will also satisfy a functional equation depending on the `local constant' $\epsilon(\chi,s,\psi)$. Here $\psi :F \to \CC^\times$ is a character whose purpose is to fix a form of Fourier transform on $F$. These definitions and results in Tate's thesis are intended to mimic the classical theory of $L$-functions, due largely to Hecke, which encompass the Riemann zeta function. The $L$-function and local constant of a character $\chi:F^\times \to \CC^\times$ will turn out to carry the essential information of $\chi$. In the classical setting see, for example, the converse theorem of Weil reproduced in \cite[Theorem 1.5.1]{Bump}.

In the setting of irreducible smooth representations of $G$, in particular the principal series representations $\pi$, we want to again associate a space $\mathcal Z(\pi)$ of zeta functions, an $L$-function $L(\pi,s)$ and a local constant $\epsilon(\pi,s,\psi)$ determining a functional equation. 

We begin this section with a brief review of harmonic and Fourier analysis and the role it plays in representation theory. For more details, see \cite[Chapter 3.1]{Bump}. Following the presentation in \cite{BH1}, we define the $L$-functions and local constants of characters of $F^\times$. We explain how this theory generalises to irreducible smooth representations $\pi$ of $G$, culminating in the Theorems \ref{BHThm1} and \ref{BHThm2}, which determine the functional equations satisfied by the zeta functions associated to $\pi$. Propositions \ref{prop:gl2factor} and \ref{prop:gl2gamma} prove these in the case where $\pi = \iota_B^G \chi$ is a principal series representation. The case where $\iota_B^G \chi$ is reducible, so that $\pi$ is only a subquotient, requires slightly more work. The results are summarised in Table 1. Finally, we prove a converse theorem for principal series representations of $G$.


\subsection{Review of harmonic analysis}

Take as motivation the representation theory of a finite group $H$. Every irreducible representation of $H$ appears as a direct summand of the regular representation $\CC[H]$, with some multiplicity. For a locally compact topological group $\mathbb G$ with Haar measure $dg$, the correct generalisation of $\CC[H]$ is the space $L^2(\mathbb G)$ of measurable functions $f:\mathbb G \to \CC$ for which 
$$\int_{\mathbb G} |f(g)|^2 dg < \infty.$$
The action of $\mathbb G$ is by right translation. If $\mathbb G$ is additionally abelian, the group $\hat{\mathbb G}$ of (unitary) characters of $\mathbb G$ is also a locally compact abelian group, the Pontryagin dual of $\mathbb G$. 

\begin{example}
    The Pontryagin duals of $\mathbb G = \RR, \ZZ, \RR/\ZZ$ are $\RR, \RR/\ZZ, \ZZ$ respectively. The characters of $\RR$ are of the form $x \mapsto e^{-2\pi i xy}$ for $y \in \RR$. The characters of $\ZZ$ are of the form $n \mapsto e^{-2\pi i nx}$ for $x \in \RR/\ZZ \cong S^1$. The characters of $\RR/\ZZ$ are of the form $x \mapsto e^{-2\pi i nx}$ for $n \in \ZZ$. In particular, $\RR$ is self-dual.
\end{example}

On a suitable dense subset of $L^2(\mathbb G)$ (the Schwartz space), one can define the Fourier transform $\hat{f} \in L^2(\hat{\mathbb G})$ of $f$ by
$$\hat{f}(\chi) = \int_{\mathbb G} f(g)\chi(g) dg.$$
The Fourier transform uniquely extends to a map $L^2(\mathbb G) \to L^2(\hat{\mathbb G})$. For suitable choices of Haar measures there is then a Fourier inversion formula 
$$\hat{\hat{f}}(g)=f(-g),$$ so that the above map is a bijection.

\begin{example}
    For $\mathbb G=\RR$, the Fourier transform of $f$ is 
    $$\hat{f}(x) = \int_{\RR} f(y)e^{-2\pi i xy} dy$$
    which is the classical Fourier transform. Identifying $\hat{\RR} = \RR$, the Fourier transform gives an invertible map $L^2(\RR) \to L^2(\RR)$, so that any element of $L^2(\RR)$ can be expressed as an integral of elements of $\hat{\RR}$. 

    Inside the representation $L^2(\RR)$ of $\RR$ we therefore see this `continuous spectrum' of the irreducible unitary representations (characters) of $\RR$, parametrised by $\RR$. Note, however, that each such character can not be realised as a subrepresentation of $L^2(\RR)$; for $y \in \RR$ the character $x \mapsto e^{-2\pi i xy}$ is realised as the Fourier transform of a function on $\RR$ supported only at $y$, but such a function is not in $L^2(\RR)$.
\end{example}

\begin{example}
    For $\mathbb G = \ZZ$, the Fourier transform of $f$ is 
    $$\hat{f}(x) = \sum_{\ZZ} f(n)e^{-2\pi i nx}.$$
    So any element of $L^2(\RR/\ZZ)$ can be expressed as a sum of unitary characters of $\ZZ$; we have a `discrete spectrum'. 
\end{example}

\begin{rem}
    The terminology of discrete and continuous spectra comes from the analogy with the spectral theory of the Laplacian. Over $\RR$, the Laplacian is $\Delta = \frac{\partial^2}{\partial x^2}$, and the characters $x \mapsto e^{-2\pi i xy}$ are eigenfunctions. 
\end{rem}

The dichotomy in the above examples is reflected in the compactness of $S^1$ and non compactness of $\RR$. More generally,

\begin{thm}[Peter-Weyl]
    Let $K$ be a compact Hausdorff topological group. Any unitary representation of $K$ decomposes into a completed Hilbert space direct sum of irreducible unitary subrepresentations. There is a unitary equivalence
    $$L^2(K) \cong \widehat{\bigoplus}_{\pi \in \hat{K}} \mathrm{End}(V_\pi)$$
    of representations of $K\times K$, where $(\pi,V_\pi)$ ranges over the set $\hat{K}$ of equivalence classes of irreducible representations of $K$, and $\hat\oplus$ denotes the completed Hilbert space direct sum.
\end{thm}
\begin{proof}
    \cite[Theorem 7.3.2]{DE} and \cite[Theorem 7.2.3]{DE}.
\end{proof}

Even more generally, for so-called Type I groups one can decompose unitary representations through a combination of integrals and Hilbert space direct sums. See \cite[Section 3.10]{GH1} for further details.

Returning to $G=\GL_2(F)$, as this is not compact we would expect the regular representation $L^2(G)$ to decompose according to both a continuous spectra and a discrete spectra. This continous spectra is provided by the parabolically induced representations $\iota_B^G \chi$, where $\chi$ ranges over the characters of $T \cong F^\times \times F^\times$.

In order to compare representations of $G$ and Galois representations through the local Langlands correspondence, we would like to classify them according to some common language. This is provided by the zeta functions, $L$-functions and functional equations discussed in this section. 

The prototypical example of an $L$-function is the Riemann zeta function $\zeta(s) = \sum_{n \geq 1} n^{-s}$.

\begin{prop}
    The function $\zeta(s) = \sum_{n \geq 1} n^{-s}$ satisfies the following properties:
    \begin{itemize}
        \item (Analytic continuation) The Riemann zeta function converges absolutely to a holomorphic function on $\mathrm{Re}(s)>1$. It has a unique analytic continuation to the complex plane, except the point $s=1$ where $\zeta(s)$ has a simple pole.
        \item (Euler product) We have the identity $$\sum\limits_{n=1}^\infty n^{-s} = \prod\limits_{p \text{ prime}} \frac{1}{1-p^{-s}},$$ convergent for $\mathrm{Re}(s)>1$.
        \item (Functional equation) There is an explicit function $\gamma(s)$ such that $\zeta(1-s)=\gamma(s)\zeta(s)$.
    \end{itemize}
\end{prop}

The approach of Tate in his thesis was to view the Riemann (and Dedekind) zeta functions from an adelic perspective. There the Euler product formulation is immediate and we only need to study the zeta functions locally. Attached to any character $\chi:F^\times \to \CC^\times$ there is an associated space $\mathcal Z(\chi)$ of zeta functions $\zeta(\Phi,\chi,s)$, where $\Phi \in C_c^\infty(F)$. The factor at the prime $p$ of the Riemann zeta function corresponds to the trivial character of $\QQ_p^\times$ and the function $\mathbbm{1}_{\ZZ_p} \in C_c^\infty(\QQ_p)$. A key ingredient in the proof of the functional equation of the Riemann zeta function is the Fourier transform over $\CC$. In general, the functional equation associated to $\chi$ relates zeta functions $\zeta(\hat{\Phi},\chi^{-1},1-s)$ and $\zeta(\Phi,\chi,s)$, where $\hat{\Phi}$ is the Fourier transform of $\Phi$ in $C_c^\infty(F)$. 







\subsection{Functional equation for \texorpdfstring{$\GL_1$}{TEXT}}

Let $F$ be a nonarchimedean local field, $\varpi$ be a uniformiser and $q$ be the size of the residue field. We will later define $L$-functions attached to an irreducible smooth representation of $\GL_2(F)$ and determine a functional equation they satisfy. First we explain this in the context of irreducible smooth representations $\chi$ of $\GL_1(F)$, necessarily a character $\chi: F^\times \to \CC^\times$.

Taking from the classical study of the Riemann zeta function and its functional equation, we want to introduce an analogue of the Fourier transform over $F$. We replace the additive character $e^{2\pi i -}: \RR \to \CC^\times$ with any choice of additive character $\psi: F \to \CC^\times$ with $\psi \neq 1$. In this way, all characters of $F$ are of the form $\psi(-y)$ for $y \in F$, by Additive Duality \ref{add_dual}. The functions we will apply the Fourier transform to will be the algebra $C_c^\infty(F)$ of locally constant compactly supported functions $F \to \CC$. For any choice of Haar measure $\mu$ on $F$, we now define the Fourier transform.

\begin{defn}
    Let $\Phi \in C_c^\infty(F)$, $\psi:F \to \CC^\times$ be a nontrivial additive character of $F$, and $\mu$ be a Haar measure on $F$. The \textit{Fourier transform} of $\Phi$ (with respect to $\psi$ and $\mu$) is 
    $$\hat{\Phi}(x) := \int_F \Phi(y)\psi(xy) d\mu(y).$$
\end{defn}

To match the classical definition over $\RR$, we would like the Fourier transform to preserve $C_c^\infty(F)$, and to have a Fourier inversion formula. Indeed:

\begin{prop}
    The Fourier transform on $C_c^\infty(F)$ satisfies the following:
    \begin{itemize}
        \item For any $\Phi \in C_c^\infty(F)$, we have $\hat{\Phi} \in C_c^\infty(F)$.
        \item For any $\psi: F \to \CC^\times$ with $\psi \neq 1$, there is a unique Haar measure $\mu_\psi$ on $F$ such that for the associated Fourier transform we have $$\hat{\hat{\Phi}}(x) = \Phi(-x)$$ for any $\Phi \in C_c^\infty(F)$ and $x \in F$.
    \end{itemize}
    
\end{prop}
\begin{proof}
    \cite[Proposition 23.1]{BH1}
\end{proof}

\begin{notn}
    For the remainder of this subsection, $\psi \neq 1$ will be an additive character of $F$, and $\mu= \mu_\psi$ will denote the associated self-dual Haar measure on $F$.
\end{notn}


Now let $\chi: F^\times \to \CC^\times$ be a smooth character of $F^\times$. We want to attach to this character an $L$-function $L(\chi,s)$ in the formal variable $s$. This is defined to be $(1-\chi(\varpi)q^{-s})^{-1}$ when $\chi$ is unramified, and 1 otherwise. In order to generalise to $\GL_2$ it would be preferable to have a more intrinsic definition.

\begin{defn}
    For $\Phi \in C_c^\infty(F)$ and $\chi :F^\times \to \CC^\times$, define the \textit{zeta function} $\zeta(\Phi,\chi,s)$ to be
    $$\zeta(\Phi,\chi,s) := \int_{F^\times} \Phi(x)\chi(x)|x|^s d^*x,$$ in the formal variable $s$, where $d\mu^*(x) = d^*x$ denotes any choice of Haar measure on $F^\times$.
\end{defn}

Equivalently, we have
$$\zeta(\Phi,\chi,s) = \sum\limits_{m \in \ZZ} z_m q^{-ms}$$
for $$z_m = \int\limits_{\varpi^m \mathcal O_F^\times} \Phi(x)\chi(x)d^*x.$$ In this way it is clear that $\zeta(\Phi,\chi,s) \in \CC((q^{-s}))$. The $z_m=z_m(\Phi,\chi)$ vanish for $m <<0$ because $\Phi$ is compactly supported on $F$.

The zeta function $\zeta(\Phi,\chi,s)$ only depends on $d^*x$ up to scaling. To remove this dependence we define the following notation.

\begin{notn}
    Let $$\mathcal Z(\chi) = \{\zeta(\Phi,\chi,s) \mid \Phi \in C_c^\infty(F)\}.$$
\end{notn}

\begin{notn}
    For $a \in F^\times$ and $\Phi \in C_c^\infty(F)$, denote by $a\Phi$ the function $x \mapsto \Phi(a^{-1}x)$.
\end{notn}

\begin{lemma}
    The space $\mathcal Z(\chi)$ is a $\CC[q^{-s},q^s]$-module, containing $\CC[q^{-s},q^s]$.
\end{lemma}
\begin{proof}
    Let $a \in F^\times$ of valuation $v_F(a)$. Then 
    $$\zeta(a\Phi,\chi,s) = \chi(a)q^{-v_f(a)s}\zeta(\Phi,\chi,s),$$ giving the desired module structure. To establish the containment, we show that $\mathcal Z(\chi)$ contains a nonzero constant. Let $d$ be such that $\chi \mid_{U_F^{d+1}} = 1$. Taking $\Phi=\mathbbm{1}_{U_F^{d+1}}$, we see that 
    $$\zeta(\Phi,\chi,s) = \mu^*(U_F^{d+1}) \neq 0.$$
\end{proof}

\begin{prop}\label{prop:gl1factor}
    Let $\chi:F^\times \to \CC^\times$. There exists a unique polynomial $P_\chi \in \CC[X]$ with $P_\chi(0)=1$ such that
    $$\mathcal Z(\chi) = P_\chi(q^{-s})^{-1}\cdot \CC[q^{-s},q^s].$$
    Moreover, we have
    $$
    P_\chi(X) =
    \begin{cases}
        1-\chi(\varpi)X & \text{if $\chi$ is unramified} \\
        1 & \text{otherwise}
    \end{cases}
    $$
\end{prop}
\begin{proof}
    Suppose $\Phi(0)=0$. Then $\Phi|_{F^\times} \in C_c^\infty(F^\times)$, and so $\Phi$ is identically zero on $\varpi^m\mathcal O_F^\times$ for $|m| >>0$. Thus only finitely many of the coefficients $z_m$ are nonzero, so that $\Phi \in \CC[q^{-s},q^s]$.

    The space $C_c^\infty(F)$ is spanned by $C_c^\infty(F^\times)$ and $\mathbbm{1}_{\mathcal O_F}$. We compute
    $$\zeta(\mathbbm{1}_{\mathcal O_F},\chi,s) = \sum\limits_{m \geq 0} \chi(\varpi^m)q^{-ms} \int_{\mathcal O_F^\times} \chi(x)d^*x.$$
    If $\chi$ is unramified (trivial on $\mathcal O_F^\times$), this gives us 
    $$\sum\limits_{m \geq 0} \chi(\varpi)^mq^{-ms} \mu^*(\mathcal O_F^\times) = (1-\chi(\varpi)q^{-s})^{-1} \mu^*(\mathcal O_F^\times).$$
    When $\chi$ is ramified the integral is zero. Indeed, translation invariance of $d^*x$ implies
    $$\int_{\mathcal O_F^\times} \chi(x)d^*x = \int_{\mathcal O_F^\times} \chi(xy)d^*x = \chi(y)\int_{\cO_F^\times} \chi(x) d^*x$$ for any $y \in \cO_F^\times$, so that this is zero if there is some $y$ with $\chi(y) \neq 1$. This computation, together with the previous lemma, establish the result. 
\end{proof}

\begin{rem}
    The computation in the proof above shows, in the case $\chi = 1$, that $\zeta(\mathbbm{1}_{\cO_F},1,s) = (1-q^{-s})^{-1}$, provided we normalise $d^*x$ appropriately. If $F=K_v$ is the completion of a number field $K$ at a nonarchimedean place $v$, we recover the Euler factor of the Dedekind zeta function $\zeta_K(s)$ at the place $v$. This explains the naming of our zeta functions. 
\end{rem}

\begin{rem}
    The computations of Proposition \ref{prop:gl1factor} show that each $\zeta(\Phi,\chi,s)$ converges absolutely and uniformly in vertical strips in some right half plane, and admit analytic continuation to a rational function in $q^{-s}$.
\end{rem}

\begin{defn}
    Define the \textit{$L$-function} attached to $\chi$ to be $L(\chi,s)=P_\chi(q^{-s})^{-1}$.
\end{defn}

As with the Riemann zeta function, we have functional equations for the zeta functions.

\begin{thm}\label{thm:gl1gamma}
    Let $\chi: F^\times \to \CC^\times$. There is a unique $\gamma(\chi,s,\psi) \in \CC(q^{-s})$ such that 
    $$\zeta(\hat{\Phi}, \check{\chi},1-s) = \gamma(\chi,s,\psi) \zeta(\Phi,\chi,s)$$ for all $\Phi \in C_c^\infty(F)$, where $\check{\chi}=1/\chi : F^\times \to \CC^\times$.
\end{thm}
\begin{proof}
    \cite[Theorem 23.3]{BH1}.
\end{proof}

Since $\mathcal Z(\chi) = L(\chi,s)\cdot \CC[q^{-s},q^s]$, it is natural to consider the terms $\frac{\zeta(\Phi,\chi,s)}{L(\chi,s)} \in \CC[q^{-s},q^s]$. This allows us to treat the case of $\chi$ ramified and unramified evenly. 

\begin{defn}
    Let $$\epsilon(\chi,s,\psi) := \gamma(\chi,s,\psi) \cdot \frac{L(\chi,s)}{L(\check{\chi},1-s)}.$$
\end{defn}
This is known as Tate's local constant.

The functional equation for $\zeta$ can be rewritten as
$$\frac{\zeta(\hat{\Phi},\check{\chi},1-s)}{L(\check{\chi},1-s)} = \epsilon(\chi,s,\psi) \frac{\zeta(\Phi,\chi,s)}{L(\chi,s)}.$$

\begin{cor}
    The local constant satisfies the functional equation
    $$\epsilon(\chi,s,\psi)\epsilon(\check{\chi},1-s,\psi) = \chi(-1).$$
    The local constant is of the form $$\epsilon(\chi,s,\psi) = aq^{bs}$$ for some $a \in \CC^\times$, $b \in \ZZ$.
\end{cor}
\begin{proof}
    The first statement comes from the Fourier inversion formula, where the $\chi(-1)$ term comes from the minus sign in $\hat{\hat{\Phi}}(x) = \Phi(-x)$. The functional equation implies that $\epsilon$ is a unit in $\CC[q^{-s},q^s]$, and the units are precisely the elements of the form $aq^{bs}$ for $b \in \ZZ$.
\end{proof}


\subsection{Functional equation for \texorpdfstring{$\GL_2$}{TEXT}}

We turn now to smooth representations $\pi$ of $G=\GL_2(F)$ and define the $L$-functions and local constants in an analogous manner to the characters $\chi :F^\times \to \CC^\times$.

In this context, we need an additive character of $A=M_2(F)$, which we will take to be $\psi_A = \psi \circ \mathrm{tr}$ for $\psi : F \to \CC^\times$ any nontrivial additive character of $F$. We will apply the Fourier transform to the $F$-algebra $\Phi \in C_c^\infty(A)$ of locally constant compactly supported functions on $M_2(F)$.

\begin{defn}
    With respect to a Haar measure $\mu$ in $A$, and $\psi_A=\psi \circ \mathrm{tr}$ an additive character of $A$, define for any $\Phi \in C_c^\infty(A)$ the \textit{Fourier transform}
    $$\hat{\Phi}(x) = \int_A\Phi(y) \psi_A(xy)d\mu(y).$$
\end{defn}

\begin{prop}
    The Fourier transform on $C_c^\infty(A)$ satisfies the following:    
    
    \begin{itemize}

        \item For any $\Phi \in C_c^\infty(A)$, we have $\hat{\Phi} \in C_c^\infty(A)$.
        \item For any $\psi: F \to \CC^\times$ with $\psi \neq 1$, there is a unique Haar measure $\mu_{\psi_A}$ on $A$ such that for the associated Fourier transform we have $$\hat{\hat{\Phi}}(x) = \Phi(-x)$$ for any $\Phi \in C_c^\infty(A)$ and $x \in A$.
    \end{itemize}
    
\end{prop}

\begin{notn}
    For the remainder of this subsection, $\psi \neq 1$ will be an additive character of $F$, $\psi_A = \psi \circ \mathrm{tr}$, and $\mu= \mu_{\psi_A}$ will denote the associated self-dual Haar measure on $A$.
\end{notn}

For $\chi:F^\times \to \CC^\times$ we defined for any $\Phi \in C_c^\infty(F)$ a zeta function $$\zeta(\Phi,\chi,s) = \int_{F^\times} \Phi(x)\chi(x) |x|^s d^*x.$$
To replicate this with $\pi : G \to \GL(V)$, we need to extract scalar values from $\pi(g) \in \GL(V)$. These will come from matrix coefficients.

\begin{defn}
Let $(\pi,V)$ be a smooth representation of $G$ with smooth dual $\check{V}$. For vectors $v\in V, \check{v} \in \check{V}$, define the smooth function $\gamma_{\check{v} \otimes v}: G \to \CC$ by 
$$\gamma_{\check{v} \otimes v} : g \mapsto \langle \check{v},\pi(g) v \rangle$$ where $\langle, \rangle$ denotes the natural pairing $\check{V} \otimes V \to \CC$. Let $\mathcal C(\pi)$ be the vector space spanned by the $\gamma_{\check{v} \otimes v}$. Elements of $\mathcal C(\pi)$ are called the \textit{matrix coefficients} of $\pi$.
\end{defn}
\begin{rem}
    If $\pi=\chi:F^\times \to \CC^\times$ is a character, any matrix coefficient (defined in the analogous way for $F^\times$) of $\chi$ is some scalar multiple of $\chi$.

    If $V$ is the tautological representation of $G$ with basis $e_1,e_2$, then $\gamma_{\check{e_i} \otimes e_j}(g)$ is precisely the $(i,j)$-th entry of $g$ as a matrix with respect to the basis $e_1,e_2$.
\end{rem}

    Recall that if $(\pi,V)$ is an irreducible smooth representation of $G$, the centre $Z$ of $G$ acts on $V$ via the central character $\omega_\pi : Z \to \CC^\times$.

\begin{lemma}\label{central char}
    For any $f \in \mathcal C(\pi), z \in Z, g \in G$ we have $f(zg) = \omega_\pi(z) f(g)$.
\end{lemma}


Fix a smooth representation $\pi$ of $G$. We may now define zeta functions for any $f \in \mathcal C(\pi)$.

\begin{defn}
    For $\Phi \in C_c^\infty(A)$ and $f \in \mathcal C(\pi)$, define the \textit{zeta function} $\zeta(\Phi,f,s)$ to be
    $$\zeta(\Phi,f,s) := \int_{G} \Phi(x)f(x)|\det x|^s d^*x,$$ in the formal variable $s$, where $d\mu^*(x) = d^*x$ denotes any choice of Haar measure on $G$.
\end{defn}

\begin{lemma}
    For any $\Phi \in C_c^\infty(A)$ and $f \in \mathcal C(\pi)$ we have $\zeta(\Phi,f,s) \in \CC((q^{-s}))$ in the formal variable $s$.
\end{lemma}
\begin{proof}
    This follows from \cite[Lemma 24.4.1]{BH1}.
\end{proof}

\begin{notn}
    Let $$\mathcal Z(\pi) = \left\{\zeta\left(\Phi,f,s+\frac{1}{2}\right) \mid \Phi \in C_c^\infty(A), f \in \mathcal C(\pi)\right\}.$$
\end{notn}
\begin{rem}
    The addition of $1/2$ will be explained in the case of principal series representations by the appearance of the modular character $\delta_B$.
\end{rem}

\begin{lemma}
    The space $\mathcal Z(\pi)$ is a $\CC[q^{-s},q^s]$-module, containing $\CC[q^{-s},q^s]$.
\end{lemma}
\begin{proof}
    \cite[Lemma 24.4.2]{BH1}.
\end{proof}


Consider now the situation where $\pi = \iota_B^G \chi$ is a parabolically induced representation, where $\chi = \chi_1 \otimes \chi_2$ is a character of $T$. We want to study the space $\mathcal Z(\pi)$ and prove an analogous result to Proposition \ref{prop:gl1factor}.





\begin{prop}\label{prop:gl2factor}
    Let $\chi=\chi_1\otimes \chi_2$ be a character of $T$ and let $(\pi,V)=\iota_B^G \chi$. Then, formally, we have
    $$\mathcal Z(\pi) = \mathcal Z(\chi_1) \mathcal Z(\chi_2) \subset \CC((q^{-s})).$$
    In particular, there is a unique polynomial $P_\pi \in \CC[X]$ with $P_\pi(0)=1$ such that 
    $$\mathcal Z(\pi) = P_\pi(q^{-s})^{-1} \cdot \CC[q^{-s},q^s].$$
    Moreover, $P_\pi(X) = P_{\chi_1}(X)P_{\chi_2}(X)$.
\end{prop}

We make some comments in preparation for the proof. The proposition concerns the zeta integrals 
$$\zeta\left(\Phi,f,s+\frac{1}{2}\right) = \int_{G} \Phi(x)f(x)|\det x|^{s+\frac{1}{2}} d^*x.$$

The matrix coefficients $\mathcal C(\pi)$ are spanned by 
$$\gamma_{\tau \otimes \theta} : g \mapsto \langle \tau, \pi(g) \theta \rangle$$ over $\theta \in V, \tau \in \check{V}$. Here $\theta \in \iota_B^G \chi$ is viewed as a smooth function $\theta : G \to \CC$ satisfying 
$$\theta(ntg) = \delta_B^{-1/2}(t) \chi(t) \theta(g)$$
for any $t \in T, n \in N, g \in G$. The Duality Theorem \ref{thm:duality} identifies $\check{V} \cong \iota_B^G \check{\chi}$. In this way we view $\tau$ as a smooth function $\tau: G \to \CC$ satisfying
$$\tau(ntg) = \delta_B^{-1/2}(t)\chi(t)^{-1}\tau(g)$$
for any $t \in T, n \in N, g \in G$. The proof of the Duality Theorem \ref{thm:duality} shows that the pairing between $V$ and $\check{V}$ gives
$$\gamma_{\tau \otimes \theta}(g) = \langle \tau, \pi(g)\theta \rangle = \int_{B\backslash G} \tau(x)\theta(xg) d\dot{x}$$ for a positive semi-invariant measure $d\dot{x}$ on $B \backslash G$. Let $K=\GL_2(\cO_F)$. Since we have a bijection $B \backslash G \leftrightarrow K \cap B \backslash K$ and $\delta_B(tn)=\delta_B(t) = |t_2/t_1|$ (Proposition \ref{prop:modularchar}) is trivial on $K\cap B$, we can rewrite this as 
$$\gamma_{\tau \otimes \theta}(g) = \int_K \tau(k)\theta(kg)dk$$ for some Haar measure $dk$ on $K$ (\cite[Corollary 7.6]{BH1}). Moreover, \cite[Equation 7.6.2]{BH1} tells us that there is a left Haar measure $db$ on $B$ such that
$$\int_G \phi(g) dg = \int_K \int_B \phi(bk) dbdk$$ for all $\phi \in C_c^\infty(G)$. Using this, our zeta integrals reduce to integrals over $B$ and $K$. Integration over $K$ is easier to handle using the smoothness of our representations. We can write $db = dn dt$ to view integration over $B$ as integration over $T$ and $N$. In order to relate $\zeta(\Phi,f,s+\frac{1}{2})$ to zeta functions coming from $\chi: T \to \CC^\times$, we want to express the integrals over $B$ solely in terms of integrals over $T$. To do so we use the following lemma. 

\begin{lemma}\label{lemma:phiT}
    Let $D$ be the algebra of diagonal matrices in $A$ so that $D^\times =T$. Let $\Phi \in C_c^\infty(A)$. There is a unique function $\Phi_T \in C_c^\infty(D)$ whose restriction to $T$ is given by 
    $$\Phi_T(t) = |t_1| \int_N \Phi(tn)dn, \hspace{1cm} t=\begin{psmallmatrix}
        t_1 & 0\\0&t_2
    \end{psmallmatrix}.$$
    The map $\Phi \mapsto \Phi_T$ is a linear surjection $C_c^\infty(A) \to C_c^\infty (D)$.
\end{lemma}
\begin{proof}
    The space $C_c^\infty(A)$ is spanned by functions of the form 
    $$\Phi = (\phi_{ij}): (a_{ij}) \mapsto \prod\limits_{i,j} \phi_{ij}(a_{ij})$$
    for $\phi_{ij} \in C_c^\infty (F)$. For such $\Phi$ we compute (identifying $N \cong F$)
    
    \begin{equation*}
        \begin{split}
            \Phi_T(t) &= |t_1| \int_F \phi_{11}(t_1)\phi_{12}(t_1n)\phi_{21}(0)\phi_{22}(t_2)dn \\
            &= \phi_{11}(t_1)\phi_{22}(t_2)\phi_{21}(0) |t_1|\int_F \phi_{12}(t_1n) dn \\
            &= \phi_{11}(t_1)\phi_{22}(t_2)\phi_{21}(0) \int_F \phi_{12}(n) dn
        \end{split}
    \end{equation*}
    which uniquely extends to a function in $C_c^\infty(D)$. Surjectivity is now clear.
\end{proof}
\begin{rem}
    The content of the lemma is that the function $\Phi_T$ is compactly supported, for which the introduction of the factor of $|t_1|$ is necessary.
\end{rem}

\begin{proof}[Proof of Proposition \ref{prop:gl2factor}]
    We first establish the containment $\mathcal Z(\pi) \subset \mathcal Z(\chi_1)\mathcal Z(\chi_2)$. We must show that for any $\Phi \in C_c^\infty(A)$ and $f \in \mathcal C(\pi)$ we have $\zeta(\Phi,f,s+\frac{1}{2}) \in \mathcal Z(\chi_1)\mathcal Z(\chi_2)$. Since $\mathcal C(\pi)$ is spanned by the coefficients $\gamma_{\tau \otimes \theta}$, for $\theta \in V, \tau \in \check{V}$, we assume $f$ is of this form.

    Formally expanding, for any $\Phi \in C_c^\infty(A)$
    \begin{equation*}
        \begin{split}
            \zeta\left(\Phi,f,s+\frac{1}{2}\right) &= \int_G \Phi(g)f(g) |\det g|^{s+\frac{1}{2}} dg \\
            &= \int_G \int_K \Phi(g) \tau(k) \theta(kg)|\det g|^{s+\frac{1}{2}} dk dg \\
            &= \int_K \int_G \Phi(k^{-1}g) \tau(k)\theta(g) |\det g|^{s+\frac{1}{2}} dg dk \\
            &= \int_K \int_K \int_B \Phi(k^{-1}bk') \tau(k)\theta(bk') |\det b|^{s+\frac{1}{2}} db dk' dk.
        \end{split}
    \end{equation*}
    Smoothness of $\Phi$, $\theta$ and $\tau$ imply there is some open normal subgroup $K_1$ of $K$ for which $\Phi$ is left and right translation invariant, and $\theta$ and $\tau$ are right translation invariant. Let $\{k_i\}$ be a finite set of coset representatives of $K/K_1$, and let $\Phi^{ij}(x) = \Phi(k_i^{-1}xk_j)$. Then $\zeta(\Phi,f,s+\frac{1}{2})$ can be expressed as a finite linear combination over $\CC$ of terms of the form 
    $$\int_B \Phi^{ij}(b) \tau(k_i)\theta(bk_j) |\det b|^{s+\frac{1}{2}} db.$$
    Using the formula $\theta(bk_j) = \delta_B^{-1/2}(t)\chi(t)\theta(k_j)$, we can express the above as
    $$\theta(k_j)\tau(k_i) \int_T\int_N \Phi^{ij}(tn) \chi(t)\delta_B^{-1/2}(t) |\det b|^{s+\frac{1}{2}} dn dt.$$
    We have $|\det b|=|\det t| = |t_1| |t_2|$ and $\delta_B^{-1/2}(t) = |t_2/t_1|^{-1/2}$. Combining with the previous lemma, we deduce that $\zeta(\Phi,f,s+\frac{1}{2})$ can be expressed as a linear combination of terms of the form 
    $$\theta(k_j)\tau(k_i) \int_T \Phi_T^{ij}(t) \chi(t) |\det t|^s dt.$$
    If $\Phi$ is of the form $(\phi_{ij})$ for $\phi_{ij} \in C_c^\infty(F)$, then the above term is a scalar multiple of $\zeta(\phi_{11},\chi_1,s)\zeta(\phi_{22},\chi_2,s)$ so that $\zeta(\Phi,f,s+\frac{1}{2}) \in \mathcal Z(\chi_1)\mathcal Z(\chi_2)$.

    In the other direction, we wish to find $\Phi \in C_c^\infty(A)$ and $f \in \mathcal C(\pi)$ such that $\zeta(\Phi,f,s+\frac{1}{2})$ is a constant multiple of $L(\chi_1,s)L(\chi_2,s)$. We will find $f$ of the form $\gamma_{\tau \otimes \theta}$ and reverse the above calculation. Suppose we were in the situation where $\Phi$ is left and right invariant under $K$, and $\theta$ and $\tau$ are right invariant under $K$. Then the above computation shows that 
    $$\zeta\left(\Phi,f,s+\frac{1}{2}\right) = \mu(K)^2 \theta(1)\tau(1) \int_T \Phi_T(t)\chi(t)|\det t|^s dt.$$
    Therefore, if we could choose $\Phi$ left and right invariant under $K$ with $\Phi_T = \phi_1\otimes \phi_2$, where $\phi_i \in C_c^\infty(F)$ satisfy $\zeta(\phi_i,\chi_i,s)=L(\chi_i,s)$, and also choose $\theta \in \iota_B^G \chi$, $\tau \in \iota_B^G \check{\chi}$, with $\theta(1), \tau(1) \neq 0$, and $\theta$, $\tau$ right invariant under $K$, then we would be done. Unfortunately, if this was the case then $$\theta(bk) = \chi(b) \delta_B^{-1/2}(b) \theta(1)$$ for all $b \in B, k \in K$. But this is not well defined - we would require $1=\chi(b)\delta_B^{-1/2}(b) = \chi(b)$ for all $b \in B \cap K$. This only occurs when $\chi_1$ and $\chi_2$ are both unramified.

    Instead, let $K_1$ be any open normal subgroup of $K$ such that $\chi$ is trivial on $B \cap K_1$, and let $\{k_i\}$ be a finite set of coset representatives of $K/K_1$. There are then unique $\theta \in \iota_B^G \chi$ and $\tau \in \iota_B^G \check{\chi}$, each supported on $BK_1$, invariant under right translation by $K_1$, and with $\theta(1)=1=\tau(1)$. Let $f=\gamma_{\tau \otimes \theta}$.
    
    For $\Phi \in C_c^\infty(A)$ left and right invariant under $K_1$, our previous computation gives us
    $$\zeta\left(\Phi,f,s+\frac{1}{2}\right) = \mu(K_1)^2 \sum\limits_{i,j}  \int_T \theta(k_j)\tau(k_i)\Phi_T^{ij}(t)\chi(t)|\det t|^s dt.
    $$
    To control the terms over all $i,j$, we would like to choose $\Phi$ such that 
    $$\theta(k_j)\tau(k_i)\Phi_T^{ij}(t) = \Phi_T(t)$$
    for all $t \in T$, and all $i,j$ such that $k_i,k_j \in BK_1$. Then, by construction of $\theta$ and $\tau$, each term $\theta(k_j)\tau(k_i)\Phi_T^{ij}(t)$ is either 0 or $\Phi_T(t)$, and at least one is $\Phi_T(t)$, so that
    $$\zeta(\Phi,f,s+\frac{1}{2}) = c \int_T \Phi_T(t) \chi(t) |\det t|^s dt$$ for some $c>0$. If $k_j = b_jk \in BK_1$, then $\theta(k_j) = \chi(b_j)\delta_B^{-1/2}(b_j)\theta(1) = \chi(b_j)$ because $\delta_B=1$ on $B \cap K$. Similarly, if $k_i=b_ik \in BK_1$, then $\tau(k_i)=\chi(b_i)^{-1}$. The condition $$\theta(k_j)\tau(k_i)\Phi_T^{ij}(t) = \Phi_T(t),$$ together with the $K_1$ invariance of $\Phi$, reduces to the condition
    $$\chi(b_j)\chi(b_i)^{-1} \int_N \Phi(b_i^{-1}tnb_j) dn = \int_N \Phi(tn)dn$$ for all $b_i,b_j \in B \cap K_1$, as functions of $t \in T$.

    To summarise, we want to construct $\Phi \in C_c^\infty(A)$ with the following properties:
    \begin{itemize}
        \item The function $\Phi$ is invariant under left and right translation by $K_1$.
        \item For all $b_i,b_j \in B \cap K_1$ and $b \in B$ we have $$\chi(b_j)\chi(b_i)^{-1}\Phi(b_i^{-1}bb_j) = \Phi(b).$$
        \item For our chosen $\phi_1,\phi_2 \in C_c^\infty(F)$ satisfying $\zeta(\phi_i,\chi_i,s)=L(\chi_i,s)$, we have $\Phi_T = c \cdot \phi_1 \otimes \phi_2 \in C_c^\infty(D)$ for some $c \neq 0$.
    \end{itemize}
    Since we may have chosen any open $K_1 \lhd K$, provided $\chi$ is trivial on $B \cap K_1$, we are free to shrink $K_1$ and adjust $\tau$ and $\theta$ accordingly. We can remove the dependence on $K_1$ by strengthening the second condition above, and now ask for $\Phi \in C_c^\infty(A)$ with the following properties:
    \begin{itemize}
        \item For all $x,y \in B \cap K$ and $b \in B$ we have $$\chi(xy)\Phi(xby) = \Phi(b).$$
        \item For some $\phi_1,\phi_2 \in C_c^\infty(F)$ satisfying $\zeta(\phi_i,\chi_i,s)=L(\chi_i,s)$, we have $\Phi_T = c \cdot \phi_1 \otimes \phi_2 \in C_c^\infty(D)$ for some $c \neq 0$.
    \end{itemize}
    If we take $\Phi$ of the form $\Phi=(\phi_{ij})$, and set $\phi_{12}=\phi_{21}=\mathbbm{1}_{\mathcal O_F}$, then the computation of Lemma \ref{lemma:phiT} shows that for $t= \begin{psmallmatrix}
        t_1&0\\0&t_2
    \end{psmallmatrix}$,
    $$\Phi_T(t) = \mu(\cO_F)\phi_{11}(t_1)\phi_{22}(t_2).$$
    Taking $\phi_{ii}=\phi_i$, it suffices to find for each $i=1,2$ some $\phi_i \in C_c^\infty(F)$ such that
    \begin{itemize}
        \item For all $x,y \in \cO_F^\times$ and $a \in F^\times$ we have $$\chi_i(xy)\phi_i(xay) = \phi_i(a).$$
        \item We have $\zeta(\phi_i,\chi_i,s)=c \cdot L(\chi_i,s)$ for some $c \neq 0$.
    \end{itemize}
    Here we divide into cases. If $\chi_i$ is unramified, then we may take $\phi_i = \mathbbm{1}_{\cO_F}$ by the proof of Proposition \ref{prop:gl1factor}. If $\chi_i$ is ramified, and the restriction to $U_F^n$ is trivial, then we take 
    $$ \phi_i = \sum\limits_{u \in \cO_F^\times/U_F^n} \chi_i(u)^{-1} \mathbbm{1}_{uU_F^n}.$$ One sees that this satisfies the first condition. For the second we have 
    $$\zeta(\phi_i,\chi_i,s) = \sum\limits_u \int_{U_F^n} \chi_i(u)^{-1}\chi_i(ux)|x|^s d^*x = \mu(\cO_F^\times)$$ which is a constant (and $L(\chi_i,s)=1$ in the ramified case). We have proven $\mathcal Z(\chi_1)\mathcal Z(\chi_2) \subset \mathcal Z(\pi)$.

\end{proof}

\begin{rem}
    The computations of Proposition \ref{prop:gl2factor} show that each $\zeta(\Phi,f,s)$ converges absolutely and uniformly in vertical strips in some right half plane, and admit analytic continuation to a rational function in $q^{-s}$.
\end{rem}


\begin{defn}
    Define the $L$-function attached to $\pi = \iota_B^G \chi$, where $\chi=\chi_1\otimes \chi_2$ is a character of $T$, to be $$L(\pi,s) = P_\pi(q^{-s})^{-1} = L(\chi_1,s)L(\chi_2,s).$$
\end{defn}

We now turn to the functional equations satisfied by the zeta functions $\zeta(\Phi,f,s)$. This involves understanding these zeta functions when we replace $\Phi$ with its Fourier transform, $\hat{\Phi}$. From the computations of Proposition \ref{prop:gl2factor}, this boils down to relating the map $\Phi \mapsto \Phi_T$ to the various Fourier transforms over $A$ and $D$.

\begin{lemma}
    For $\Phi \in C_c^\infty(A)$, we have $(\hat{\Phi})_T = \widehat{\Phi_T}$.
\end{lemma}
\begin{proof}
    \cite[Lemma 26.3]{BH1}.
\end{proof}

\begin{lemma}\label{hat}
    For $k_i,k_j \in K$ let $\Phi^{ij}$ denote the function $x \mapsto \Phi(k_i^{-1}xk_j)$ for $\Phi \in C_c^\infty(A)$. Then $\hat\Phi^{ji} = \widehat{\Phi^{ij}}$. 
\end{lemma}
\begin{proof}
    We calculate 
    $$\hat\Phi^{ji}(x) = \int_A \Phi(y)\psi_A(k_j^{-1}xk_iy)dy$$
    and 
    $$\widehat{\Phi^{ij}}(x) = \int_A\Phi(k_i^{-1}yk_j)\psi_A(xy)dy = \int_A \Phi(y)\psi_A(xk_iyk_j^{-1})dy.$$
    Since $\psi_A = \psi \circ \mathrm{tr}$ and $\mathrm{tr}$ is invariant under conjugation, we have $\psi_A(k_j^{-1}xk_iy) = \psi_A(xk_iyk_j^{-1})$.
\end{proof}

\begin{notn}
    If $f \in \mathcal C(\pi)$ is a matrix coefficient, denote by $\check{f} \in \mathcal C(\check\pi)$ the matrix coefficient
    $\check{f}(g) = f(g^{-1})$.
\end{notn}

\begin{prop}\label{prop:gl2gamma}
    Let $\pi = \iota_B^G \chi$ where $\chi=\chi_1\otimes \chi_2$ is a character of $T$. There is a unique $\gamma(\pi,s,\psi) \in \CC(q^{-s})$, depending on the additive character $\psi \neq 1$ of $F$ defining the Fourier transform, such that 
    $$\zeta(\hat{\Phi},\check{f},(1-s)+\frac{1}{2}) = \gamma(\pi,s,\psi) \zeta(\Phi,f,s+\frac{1}{2})$$
    for all $\Phi \in C_c^\infty(A)$ and $f \in \mathcal C(\pi)$. Moreover, 
    $$\gamma(\pi,s,\psi) = \gamma(\chi_1,s,\psi)\gamma(\chi_2,s,\psi).$$
\end{prop}
\begin{proof}
    Since the zeta function is linear in the matrix coefficients, as is the operation $f \mapsto \check{f}$, it suffices to prove such $\gamma$ exists for all $\Phi \in C_c^\infty(A)$ and $f$ of the form $\gamma_{\tau \otimes \theta}$ as in the proof of Proposition \ref{prop:gl2factor}. We calculated that 
    $$f(g) = \int_{B \backslash G} \tau(x)\theta(xg) d\dot{x} = \int_K \tau(k)\theta(kg)dk,$$ for some Haar measure $dk$ on $K$, so that by right invariance of $d\dot{x}$ we have 
    $$\check{f}(g) = \int_{B \backslash G}\tau(xg)\theta(x) d\dot{x} = \int_K \tau(kg)\theta(k)dk.$$ The same computation as the proof of Proposition \ref{prop:gl2factor} gives (for the same $K_1$ and coset representatives $k_i$ of $K/K_1$)
    \begin{equation*}
        \begin{split}
            \zeta(\hat{\Phi},\check{f},(1-s)+\frac{1}{2}) &= \mu(K_1)^2 \sum\limits_{i,j} \theta(k_j)\tau(k_i) \int_T (\hat\Phi^{ji})_T(t) \chi(t)^{-1} |\det t|^{1-s} dt \\
            &= \mu(K_1)^2 \sum\limits_{i,j} \theta(k_j)\tau(k_i) \int_T \widehat{(\Phi_T^{ij})}(t) \chi(t)^{-1} |\det t|^{1-s} dt
        \end{split}
    \end{equation*}
    by Lemma \ref{hat}. Therefore, it suffices to show that 
    $$\int_{F^\times}\int_{F^\times} \widehat{(\Phi^{ij}_T)}(t)\chi_1(t_1)^{-1}\chi_2(t_2)^{-1}|t_1t_2|^{1-s} dt_2dt_1 =  \gamma(\chi_1,s,\psi)\gamma(\chi_2,s,\psi) \int_{F^\times} \int_{F^\times}\Phi^{ij}_T(t)\chi_1(t_1)\chi_2(t_2) |t_1t_2|^s dt_2dt_1$$
    where $t = \begin{psmallmatrix}
        t_1 &0\\0&t_2
    \end{psmallmatrix} \in T$. By Theorem \ref{thm:gl1gamma}, this equality holds whenever we replace $\Phi^{ij}_T \in C_c^\infty(D)$ by a function of the form $\phi_{11}(t_1) \otimes \phi_{22}(t_2) \in C_c^\infty(D)$. But such functions span $C_c^\infty(D)$, so we are done by linearity of the integrals.

\end{proof}

\begin{defn}
    Define the Godement-Jacquet local constant $\epsilon(\pi,s,\psi)$ of $\pi = \iota_B^G \chi$ by 
    $$\epsilon(\pi,s,\psi) = \gamma(\pi,s,\psi) \frac{L(\pi,s)}{L(\check{\pi},1-s)}.$$
\end{defn}

\begin{cor}
    For $\pi= \iota_B^G \chi$ we have
    $$\epsilon(\pi,s,\psi) = \epsilon(\chi_1,s,\psi)\epsilon(\chi_2,s,\psi).$$
\end{cor}
\begin{proof}
    This follows from Proposition \ref{prop:gl2gamma} and Proposition \ref{prop:gl2factor}.
\end{proof}

For context, we state more general versions of these results that hold for any irreducible smooth representation $\pi$ of $G$.

\begin{thm}\label{BHThm1}
    Let $\pi$ be an irreducible smooth representation of $G$. There is a unique polynomial $P_\pi(X) \in \CC[X]$, satsifying $P_\pi(0)=1$, and 
    $$\mathcal Z(\pi) = P_\pi(q^{-s})^{-1} \CC[q^{-s},q^s].$$
\end{thm}
\begin{proof}
    \cite[Theorem 24.2.1]{BH1}.
\end{proof}

\begin{notn}
    Set $L(\pi,s) = P_\pi(q^{-s})^{-1}$.
\end{notn}

\begin{thm}\label{BHThm2}
    Let $\pi$ be an irreducible smooth representation of $G$. There is a unique rational function $\gamma(\pi,s,\psi) \in \CC(q^{-s})$ such that 
    $$\zeta(\hat\Phi,\check{f},(1-s)+\frac{1}{2}) = \gamma(\pi,s,\psi) \zeta(\Phi,f,s+\frac{1}{2})$$ for all $\Phi \in C_c^\infty(A)$ and $f \in \mathcal C(\pi)$.
\end{thm}
\begin{proof}
    \cite[Theorem 24.2.2]{BH1}.
\end{proof}

\begin{defn}
    Define the Godement-Jacquet local constant $\epsilon(\pi,s,\psi)$ of an irreducible smooth representation $\pi$ of $G$ by 
    $$\epsilon(\pi,s,\psi) = \gamma(\pi,s,\psi) \frac{L(\pi,s)}{L(\check{\pi},1-s)}.$$
\end{defn}

\begin{cor}
    The local constant satisfies the functional equation
    $$\epsilon(\pi,s,\psi)\epsilon(\check{\pi},1-s,\psi) = \omega_\pi(-1).$$
    The local constant is of the form $$\epsilon(\pi,s,\psi) = aq^{bs}$$ for some $a \in \CC^\times$, $b \in \ZZ$. 
\end{cor}
\begin{proof}
    The first statement comes from the Fourier inversion formula and Theorem \ref{BHThm2}. The $\omega_\pi(-1)$ term comes from the minus sign in $\hat{\hat{\Phi}}(x)=\Phi(-x)$ and the observation that for a matrix coefficient $f \in \mathcal C(\pi)$ we have $f(-g)=\omega_\pi(-1)f(g)$. The functional equation and Theorem \ref{BHThm1} implies that $\epsilon$ is a unit in $\CC[q^{-s},q^s]$, and the units are precisely the elements of the form $aq^{bs}$ for $b \in \ZZ$.
\end{proof}

The Propositions \ref{prop:gl2factor} and \ref{prop:gl2gamma} prove the Theorems \ref{BHThm1} and \ref{BHThm2} in the case that $\pi = \iota_B^G \chi$ and $\pi$ is irreducible. As in Theorem \ref{classify}, the representations $\pi = \iota_B^G \chi$ are typically irreducible - they are only reducible when $\chi = \phi \delta_B^{\pm 1/2}$ for some character $\phi$ of $F^\times$. In this case the composition factors are characters $\phi \circ \det$, and twists of Steinberg $\phi \mathrm{St}_G$. We state without proof the $L$-functions and local constants in the case that $\pi$ is one of these composition factors. For more detail see Sections 26.5 - 26.8 of \cite{BH1}. The results for all principal series representations are summarised in the following table:

\begin{figure}[h!]
    \centering
    \begin{tabular}{ |c|c|c| }
        \hline
        Principal series representation $\pi$ & $L(\pi,s)$ & $\epsilon(\pi,s,\psi)$ \\ \hline
        $\iota_B^G \chi$, $\chi=\chi_1\otimes \chi_2$, $\chi \neq \phi \delta_B^{\pm 1/2}$ & $L(\chi_1,s)L(\chi_2,s)$ & $\epsilon(\chi_1,s,\psi)\epsilon(\chi_2,s,\psi)$ \\ 
        $\phi \circ \det$, $\phi :F^\times \to \CC^\times$ ramified & 1 & $\epsilon(\phi,s-\frac{1}{2},\psi)\epsilon(\phi,s+\frac{1}{2},\psi)$ \\ 
        $\phi \mathrm{St}_G$, $\phi :F^\times \to \CC^\times$ ramified & 1 & $\epsilon(\phi,s-\frac{1}{2},\psi)\epsilon(\phi,s+\frac{1}{2},\psi)$ \\  
        $\phi \circ \det$, $\phi :F^\times \to \CC^\times$ unramified & $L(\phi,s-\frac{1}{2})L(\phi,s+\frac{1}{2})$ & $\epsilon(\phi,s-\frac{1}{2},\psi)\epsilon(\phi,s+\frac{1}{2},\psi)$ \\ 
        $\phi \mathrm{St}_G$, $\phi :F^\times \to \CC^\times$ unramified & $L(\phi,s+\frac{1}{2})$ & $-\epsilon(\phi,s,\psi)$ \\     
        \hline
       \end{tabular}
       \caption{$L$-functions and local constants of principal series representations of $G$}
\end{figure}

In particular, if $\pi$ is a composition factor of $\iota_B^G \chi$ then $L(\pi,s) = L(\chi_1,s)L(\chi_2,s)$, unless $\pi = \phi \mathrm{St}_G$ for some unramified character $\phi : F^\times \to \CC^\times$.

\subsection{Converse Theorem}

Attached to any principal series representation $\pi$ of $G$ we have an associated $L$-function $L(\pi,s)$ and local constant $\epsilon(\pi,s,\psi)$. In some sense this is enough information to distinguish them as irreducible smooth representations of $G$. More precisely, one can also define $L$-functions and local constants for the cuspidal representations of $G$, and then we have

\begin{thm}[Converse Theorem]\label{thm:converse}
    Let $\psi:F \to \CC^\times$ be an additive character with $\psi \neq 1$. Let $\pi_1,\pi_2$ be irreducible smooth representations of $G=\GL_2(F)$. Suppose that 
    $$L(\chi\pi_1,s)=L(\chi\pi_2,s) \text{   and   } \epsilon(\chi\pi_1,s,\psi) = \epsilon(\chi\pi_2,s,\psi),$$ for all characters $\chi :F^\times \to \CC^\times$. Then $\pi_1 \cong \pi_2$.
\end{thm}

Recall that the twist $\chi\pi$ denotes the representation $g \mapsto \chi(\det(g))\pi(g)$.

We take as fact the following result for cuspidal representations.

\begin{prop}\label{prop:cuspL}
    Let $\pi$ be an irreducible cuspidal representation of $G$. Then $L(\pi,s)=1$.
\end{prop}
\begin{proof}
    \cite[Corollary 24.5]{BH1}.
\end{proof}

Then we can distinguish between cuspidal and principal series representations as follows.

\begin{prop}\label{prop:twistL}
    An irreducible smooth  representation $\pi$ of $G$ is cuspidal if and only if $L(\phi\pi,s)=1$ for all characters $\phi$ of $F^\times$.
\end{prop}
\begin{proof}
    Since twisting preserves principal series representations, it preserves cuspidal representations. Proposition \ref{prop:cuspL} implies that if $\pi$ is cuspidal then $L(\phi\pi,s)=1$ for all $\phi$. In the other direction, suppose that $\pi$ is a composition factor of $\iota_B^G \chi$ for $\chi= \chi_1\otimes \chi_2$ a character of $T$. Taking $\phi=\chi_2^{-1}$, $\phi\pi$ is a composition factor of $\iota_B^G \phi\chi$ with $\phi\chi = \chi_1\chi_2^{-1} \otimes 1$. Now, except for the case $\phi\pi$ is a twist of Steinberg by an unramified character, we have $L(\phi\pi,s) = L(\chi_1\chi_2^{-1},s)L(1,s)$, and then $L(1,s)=(1-q^{-s})^{-1}$ is nontrivial. In the case it is a twist of Steinberg by an unramified character, the $L$-function is still nontrivial as seen in Table 1.
\end{proof}

\begin{proof}[Proof of Theorem \ref{thm:converse} for principal series representations]
    Twisting $\pi$, we may assume that $L(\pi,s) \neq 1$ as in the proof of Proposition \ref{prop:twistL}. Then $L(\pi,s)$ has degree 2 (as a rational function of $q^{-s}$). 

    Suppose $L(\pi,s)$ has degree 2. From Table 1, $\pi$ is either $\iota_B^G \chi$ for some $\chi=\chi_1 \otimes \chi_2$, with $\chi_1\chi_2^{-1} \neq |-|^{\pm 1}$ and $\chi_i$ unramified, or $\pi = \phi \circ \det$ for some unramified character $\phi :F^\times \to \CC^\times$. In either case, we have $L(\pi,s)=L(\chi_1,s)L(\chi_2,s)$ for unramified characters $\chi_i$ of $F^\times$, where $\pi = \phi \circ \det$ corresponds to $\chi_i = \phi |-|^{\pm 1}$. But since an unramified character $\chi$ is determined by $\chi(\varpi)$, it is determined by $L(\chi,s)$. Since $\iota_B^G( \chi_1 \otimes \chi_2) \cong \iota_B^G (\chi_2 \otimes \chi_1)$, it follows that $L(\pi,s)$ is enough to distinguish all principal series representations $\pi$ for which $L(\pi,s)$ has degree 2.

    Suppose $L(\pi,s)$ has degree 1, and is $L(\theta,s)$ for some unramified character $\theta$ of $F^\times$. From Table 1, $\pi$ is either $\iota_B^G (\theta' \otimes \theta)$ for some ramified character $\theta'$, or $\pi = \theta' \mathrm{St}_G$ for $\theta' = \theta|-|^{-1/2}$. In the latter case, $\theta'$ is unramified and so for any ramified character $\phi$ we have $L(\phi\pi,s)=1$. This distinguishes it from the former case where if we take $\phi = (\theta')^{-1}$, a ramified character, we have $\phi\pi = \iota_B^G (1 \otimes \phi\theta)$ so that $L(\phi\pi,s) \neq 1$. To recover $\theta'$ in this case, we can choose some ramified character $\phi$ such that $L(\phi\pi,s) \neq 1$, say $L(\phi\pi,s) = L(\theta'',s)$ fo a unique unramified character $\theta''$ of $F^\times$. Since $\phi\pi = \iota_B^G (\phi\theta' \otimes \phi\theta,s)$, and $\phi\theta$ is ramified, we have $L(\phi\pi,s) = L(\phi\theta',s)$. Therefore $\theta' = \phi^{-1}\theta''$.
\end{proof}

\begin{rem}
    The proof of Theorem \ref{thm:converse} for principal series representations shows that the isomorphism class of $\pi$ is determined solely by the $L$-functions $L(\phi\pi,s)$ as we range over all characters $\phi :F^\times \to \CC^\times$. For cuspidal representations, all $L$-functions are 1 and they are instead distinguished solely by the local constants.
\end{rem}





\newpage

\bibliography{references}
\bibliographystyle{amsalpha}


\end{document}