The aim of this first section is to motivate the notions of locally profinite groups and their smooth representations. Such groups arise in nature from taking the points of reductive groups over non-archimedean local fields. We begin this section by briefly recalling some basic facts about these fields and linear groups associated to them. For the sake of brevity, we will omit proofs. For more detail, the reader can consult, for example, \cite{Gou1}.

\subsection{Local Fields and Locally Profinite Groups}
We begin by recalling some basic objects from algebraic number theory. Given a field $F$, a \textit{discrete valuation} on $F$ is a surjective function $\nu: F\to\ZZ\cup\{\infty\}$ satisfying the conditions

\begin{enumerate}
    \item $\nu(xy)=\nu(x)+\nu(y)$ for any $x,y\in F$ 
    \item $\nu(x+y)\geq\min\{\nu(x),\nu(y)\}$ for any $x,y\in F$.
    \item $\nu(x)=\infty$ if and only if $x=0$.
\end{enumerate}

Any discrete valuation $\nu$ induces an absolute value on $F$ given by the formula 
$$|x|=c^{\nu(x)}$$ 
for any $c\in(0,1)$, and therefore it also induces a topology on $F$. This topology is independent of the choice of $c$. One easily checks that this absolute value satisfies $|x+y|\leq\max\{|x|,|y|\}$ for any $x,y\in F$. Absolute values with this property are called \textit{non-Archimedean}. 

A field $F$ with an absolute value $|\cdot|$ induced by a discrete valuation $\nu$ is the fraction field of the \textit{valuation ring}
$$R:=\{x\in F:v(x)\geq 0\}=\{x\in F: |x|\leq1\},$$ 
which contains a unique maximal ideal
$$\pp:=\{x\in F:v(x)> 0\}=\{x\in F: |x|<1\},$$
the \textit{valuation ideal} or the \textit{ring of integers of $F$}. The valuation ideal is principal, and it is generated by any $\varpi\in F$ with $\nu(\varpi)=1$. Such an element is called a \textit{uniformiser} of $F$. Finally, the \textit{residue field} $\kappa$ of $F$ is the quotient $R/\pp$. This motivates the following important definition.

\begin{defn}
    A field $F$ is a \textit{non-archimedean local field} if it is complete with respect to a topology induced by a discrete valuation and the residue field is finite.
\end{defn}

\begin{rem}
    When the residue field is finite, it is conventional to define the absolute value on $F$ by 
    $|x|=q^{-\nu(x)},$
    where $q=|\kappa|$. From here onwards, we will follow this convention.
\end{rem}
\begin{rem}
    Local fields are ubiquitous in number theory. They arise as completions of number fields at non-archimedean places in characteristic 0, or as completions of finite extensions of $\FF_p(t)$ at non-archimedean places in positive characteristic.
\end{rem}

Let us now discuss important aspects of the topology on $F$ and $R$ induced by the discrete valuation $\nu$. We have already seen that $R$ is a local ring with maximal ideal $\pp$ and therefore $U_F:=R\setminus\pp$ is the set of units of $R$. The ideals 
$$\pp^n=\{x\in F:\nu(x)\geq n\}=\{x\in F: |x|\leq q^{-n}\}=\varpi^n R,\quad n\in\ZZ$$
are a complete set of fractional ideals of $R$ and, since the valuation is assumed to be discrete, they are also open subsets of $F$.
Therefore, they are a fundamental system of neighbourhoods of the identity. A direct consequence of this fact implies that $F$ (and therefore $R$) are totally disconnected topological rings.

Furthermore, the ring $R$ is a closed subring of $F$, which is assumed to be complete. Hence, $R$ is also complete, and a stardard topological argument shows that $R$ is in fact compact. This proves that $R$ (and therefore any $\pp^n$) is in fact a profinite group, and we have a topological isomorphism 
$$R\longrightarrow\varprojlim_{n\geq 1} R/\pp^n\quad x\mapsto (x\ (\textrm{mod }{\pp^n}))_{n\geq 1}$$
where the maps implicit in the right hand side are the obvious ones.

However, $F$ itself is clearly not compact, and therefore it is not profinite. Nevertheless, $F$ has the important property that any open neighbourhood of the identity contains an open compact (and therefore profinite) subgroup - some $\pp^n$ for a sufficiently large $n$.

We are now ready to give the main definition of this section, which encapsulates this last property in greater generality.

\begin{defn}\label{loc_prof_grp}
    A topological group $G$ (which we always assume to be Hausdorff) is a \textit{locally profinite group} if every open neighbourhood of the identity contains a compact open subgroup. 
\end{defn}

In this report we will be interested in studying the representation theory of many important groups and rings related to the local field $F$. The notion of a locally profinite group is an abstract one, but it has the great advantage of accomodating many important groups and rings associated to non-Archimedean local fields and their representation theory.

\begin{examples} \label{example_prof_groups}

    \begin{enumerate}
        \item Trivially, any group equipped with the discrete topology is profinite, where $\{e\}$ is the fundamental neighbourhood.
        \item In the preceding discussion, we have shown that the local field $F$ is a locally profinite group, where $\pp^n$ for $n\geq1$ is a fundamental system of open compact subgrups. We remark that $F$ satisfies the rather special property of being the union of its open compact subgroups. %This fact has relevant consequences that will be discuss later.
        \item The multiplicative group $F^{\times}$ is also a locally profinite group, where the congruence unit groups $U_F^n=1+\pp^{n+1}$ for $n\geq1$ is a fundamental system of open compact subgroups. Unlike $F$, the group $F^{\times}$ is not the union of its open compact subgroups.
        \item Given $m\geq1$ an integer, the additive group $F^m=F\times\dots\times F$ is also a locally profinite group endowed with the product topology. A fundamental system of open compact subgroups is given by $\pp^{n}\times\dots\times\pp^{n}$ for $n\geq0$. More generally, any product of locally profinite groups is locally profinite.
        \item The matrix ring $M_m(F)$ is also locally profinite since it is isomorphic to $F^{m^2}$ as additive groups. The open compact subgroups $\pp^n M_m(R)$ are a fundamental system of neighbourhood of the identity.
        \item The group $\GL_m(F)$ of invertible matrices is an open subset of $M_m(F)$ since $\det:M_m(F)\rightarrow F$ is continuous and $F^{\times}$ is an open subset of $F$. Furthermore, mutiplication by a matrix $A\in M_m(F)$ and inversion of matrices are continous maps in $M_m(F)$, and therefore $\GL_m(F)$ is also a topological group. The subgroups
        $$K=\GL_m(R),\quad K_n=1+\pp^{n+1}M_m(R),\quad n\geq 0,$$
        are compact open, and a fundamental neighbourhood of the identity.
        \item Let $G$ be a locally profinite group and $H\leq G$ be a closed subgroup. Then $H$ is also a locally profinite group. If in addition $H$ is assumed to be normal in $G$, then $G/H$ is locally profinite. 
        
        %If $U\subseteq H$ is a neighbourhood of the identity on $H$, then there is some $V$ open in $G$ such that $U=H\cap V$. Let $K\subseteq V$ be some open compact subgroup of $G$. Then $K\cap H$ is an open subgroup of $H$ and a closed subgroup of $K$. But since $K$ is compact and Hausdorff, $K\cap H$ is also compact. This shows that $H$ is also a locally profinite subgroup.
    \end{enumerate}
\end{examples}

We give some further insight into the terminology used. It is an easy exercise to prove that a profinite group is compact and locally profinite. Rather strikingly, the converse also holds. That is, if $K$ is a compact locally profinite group, then
$$K\longrightarrow\varprojlim_N K/N$$
is a topological isomorphism, where $N$ ranges over the normal open subgroups. Since $K$ is compact and $N$ is open, $K/N$ must be finite and discrete, showing that $K$ is profinite.

\subsection{Abstract Representations of Groups} \label{Abstract_Reps}
Before discussing the representation theory of locally profinite groups, we first review some general results and constructions of representations of arbitrary groups $G$. We begin by recalling the notion of a representation.

\begin{defn}
    A \textit{representation} of a group $G$ over a field $k$ is a pair $(\pi,V)$ where $V$ is a $k$-vector space and $\pi:G\rightarrow\GL(V)$ is a group homomorphism. We say that $\dim V$ is the \textit{dimension} of the representation.
\end{defn}

Equivalently, a representation of $G$ is a $k$-vector space $V$ equipped with a $k$-linear $G$ action. Whenever the representation is clear from the context, we will omit $\pi$ from the notation and write $g\cdot v$ for $\pi(g)v$. 

Throughout this document, we will only be interested in complex representations, so for now on we will assume that $k=\CC$ unless otherwise stated, and hence we will omit the underlying field from the notation.

We say that $U\leq V$ is a $G$-subspace if $U$ is closed under the $G$-action; i.e. if $g\cdot U\subseteq U$ for every $g\in G$. When this happens, both $U$ and $V/U$ are naturally $G$-representations. Importantly, we say that that a representation $(\pi,V)$ is irreducible (or simple) if $V$ has no non-trivial $G$-subspaces. These are the \textit{building blocks} of more complicated representations, and thus we are often interested in classifying them.

This also motivates the following definition.

\begin{defn}
    A representation $(\pi,V)$ of a group $G$ is semisimple if it is the direct sum of simple subrepresentations. 
\end{defn}

\begin{rem}\label{rem_semisimple}
    If $G$ is a finite group, Maschke's Theorem shows that all finite dimensional complex representations of $G$ are semisimple. As a consequence, one can show that any complex irreducible representation of $G$ is finite dimensional, appearing as a subrepresentation of the canonical representation $\CC G$. 
    
    As we shall visit later in this chapter, continuous finite dimensional representations over profinite groups also share these properties. However, it is easy to construct representations of locally profinite groups which are continous yet not semisimple. For example,
    \begin{align*}
        \phi:\ZZ\longrightarrow \GL_2(\CC)\\
        n\mapsto 
        \begin{pmatrix}
            1 & n\\
            0 & 1\\
        \end{pmatrix}
    \end{align*}
    has a single one-dimensional invariant subspace. One can also construct irreducible representations that are infinite dimensional, but these are harder to construct. \textbf{Reference to a later section when we consider the reps of the Mirabolic subgroup or the steinberg reps of $\GL2(F)$}.
\end{rem}


Naturally, we also define the notion of a homomorphism between representations.

\begin{defn}
    A morphism between two complex representations $(\pi,V)$, $(\sigma,W)$ of a group $G$ is a linear map $\phi:V\rightarrow W$ compatible with the $G$ action. That is, 
    $$\phi(\pi(g)v)=\sigma(g)\phi(v)\ \text{for all } g\in G,\ v\in V.$$
\end{defn}

Therefore, the set of complex representations of $G$ form a category denoted by $\Rep_G$, which importantly is an \textbf{abelian category}.

We finish this section by introducing important constructions and functors between these categories that allow us to obtain new represenations from old ones, and which we will use heavily later on.

\begin{defn}
    Given $(\pi,V)\in\mathrm{Rep}_G$, define the dual space $V^*=\Hom(V,\CC)$, and denote by 
    \begin{align*}
        V^*\times V\longrightarrow \CC,\\
        (v^*,v)\longmapsto\langle v^*,v\rangle,
    \end{align*}
    the canonical evaluation homomorphism. Then $V^*$ carries a natural representation 
    $$\langle\pi^*(g)v^*,v\rangle=\langle v^*,\pi(g^{-1})v\rangle,$$
    denoted as the dual representation, and the functor 
    \begin{align*}
        (-)^*:\Rep_G&\longrightarrow\Rep_G\\
        (\pi,V)&\longrightarrow(\pi^*,V^*)
    \end{align*}
    is an additive and exact contravariant functor.
\end{defn}

One can also consider the composition of this functor with itself to obtain the \textbf{double dual} $(\pi^{**},V^{**})$. Then there is a canonical $G$-homomorphism $\delta:V\rightarrow V^{**}$ such that $$\langle\delta(v),v^*\rangle_{V^*}=\langle v^*,v\rangle_{V}.$$
When $V$ is finite dimensional, $\delta$ is a $G$-isomorphism. For general representations of locally profinite groups, this is not always the case, but under additional assumptions it is possible to give a precise criterion that determines when $\delta$ is bijective (\cite[2.8 Corollary, 2.9 Proposition]{BH1}).


\begin{defn}
    Let $H\leq G$ be groups and let $(\pi,V)$ and $(\sigma,W)$ be representations of $G$ and $H$ repectively. The restriction of $\pi$ to $H$ gives a \textbf{restriction} functor
    \begin{align*}
        \Res_H^G:\Rep_G&\longrightarrow\Rep_H\\
        (\pi,V)&\longmapsto(\pi|_H,V)
    \end{align*}
    On the other hand, given $(\sigma,W)\in\Rep_H$ one can construct the vector space
    $$X=\{f:G\to W:f(hg)=\sigma(h)f(g)\text{ for all }h\in H, g\in G\},$$
    equipped with the $G$-action $\Sigma:G\longrightarrow\Aut_\CC(X)$ such that
    $$\Sigma(g)f:x\longmapsto f(xg),\ x,g\in G.$$
    This gives the \textbf{induction} functor
    \begin{align*}
        \Ind_H^G:\Rep_H&\longrightarrow\Rep_G\\
        (\pi,V)&\longmapsto(\Sigma,X).
    \end{align*}
\end{defn}

As with the dual functor, both the restriction and induction functors are additive and exact covariant functors. To simplify notation, we will write $\Ind_H^G\sigma$ instead of $\Ind_H^G(\sigma,W)$, which is the usual convention in the literature.

We remark that one can construct the following canonical $H$-homomorphisms 
\begin{align*}
    a_\sigma: \Ind_H^G\sigma&\longrightarrow W\\
    f&\longmapsto f(1)
\end{align*}
and 
\begin{align*}
    a_\sigma^c: W&\longrightarrow \Ind_H^G\sigma\\
    w&\longmapsto f_w
\end{align*}
where $f_w$ is supported in $H$ and $f_w(h)=\sigma(h)w$ for $h\in H$. The choice of notation will be understood later. These, in turn, induce the maps 
\begin{align*}
    \Psi:\Hom_G(\pi,\Ind_H^G\sigma)&\longrightarrow\Hom_H(\Res_H^G\pi,\sigma),\\
    \phi&\longmapsto a_\sigma\circ\phi,
\end{align*}
and
\begin{align*}
    \Psi^c:\Hom_G(\Ind_H^G\sigma,\pi)&\longrightarrow\Hom_H(\sigma,\Res_H^G\pi),\\
    f&\longmapsto f\circ a_\sigma^c.
\end{align*}
When $G$ is a finite group, we have the following result.
\begin{thm}[Frobenius reciprocity]
    Let $G$ be a finite group. Then the maps $\Psi$ and $\Psi^c$ are bijections that are functorial in both variables $\sigma$ and $\pi$. In categorical terms, we have the adjunctions $$\Ind_H^G\dashv\Res_H^G\dashv\Ind_H^G.$$
\end{thm}

This theorem fails for general representations of locally profinite groups $G$ and subgroups $H\leq G$. However, the theorem does hold under certain additional assumptions on the topology of $H$ inside $G$, a smoothness condition on the representations and an adequate modification of the set $X$ above. This will all be discussed in Section \textbf{Insert here reference to the section in which this is discussed.}


\subsection{Characters of Local Fields}

Now we turn our attention to the representation theory of locally profinite groups. In studying their representations, it turns out that the entire set is too big, so we need to restrict our attention to those representations satisfying a certain ``smoothness" condition. To motivate this condition, we will first describe the simplest (yet very relevant!) case: one-dimensional representations of a local field $F$: that is, group homomorphisms $\phi:F\rightarrow\CC^{\times}$. Later in this section we will also study the one-dimensional representations of $F^{\times}$.

As we have discussed in the previous section, locally profinite groups carry a certain topology, so a natural condition to impose is \textbf{continuity} with respect to the usual topologies in $\CC^{\times}$ and $G$. A continuous homomorphism $\psi:G\rightarrow\CC^{\times}$ will be denoted as a \textit{character} of $G$.

Characters of a locally profinite group $G$ are a group under multiplication, denoted by $\hat{G}$. It turns out that for one-dimensional representations, continuity coincides with the smoothness condition we require which will be introduced later. 

When $G$ is a finite group with discrete topology, then any one-dimensional representation is a character, and we have the following simple description.

\begin{prop}
    If $G$ is a finite group with the discrete topology, then $\hat{G}\cong G^{ab}$. In particular, if $G$ is abelian then $\hat{G}\cong G$.
\end{prop}
\begin{proof}
    \textbf{Insert reference here}
\end{proof}

For general locally profinite results, we have this rather surprising result. 
\begin{lemma}\label{lem_cont_chars}
    Let $G$ be a locally profinite group and $\psi: G\rightarrow\CC^{\times}$ a homomorphism. Then $\psi$ is continous if and only if $\ker\psi$ is open in $G$. Furthermore, if $G$ is the union of its compact open subgroups, then\footnote{Characters satisfying this property are called \textit{unitary}} $$\psi(G)\subseteq\{z\in\CC^{\times}:|z|=1\}=S^1.$$
\end{lemma}
\begin{proof}
    If $\ker\psi=\psi^{-1}(1)$ is open in $G$, then for any $z\in\Ima\psi$, then $\psi^{-1}(z)=g\ker\psi$ is also open, where $\psi(g)=z$. So in fact, \textbf{for any} $U\subseteq\CC^{\times}$, 
    $$\psi^{-1}(U)=\bigcup_{z\in U\cap\Ima\psi}\psi^{-1}(z),$$
    and so in particular it is continous.
    Conversely, if $\psi$ is continous, then for any open neighbourhood $\mathcal{N}$ of $1$, $\psi^{-1}(\mathcal{N})$ contains an open compact subgroup $K$ of $G$. But $\mathcal{N}$ can be chosen sufficiently small so that it does not contain any non-trivial subgroup of $\CC^{\times}$. Hence, $\psi(K)=1$ so $K\subseteq\ker\psi$, and since $K$ is open, so is $\ker\psi$.
    The last assertion is a direct consequence of the fact that the continuous image of a compact set is compact and $S^1$ is the unique maximal compact subgroup of $\CC^{\times}$.
\end{proof}

Since $F$ is the union of its open compact subgroups, all characters of $F$ are unitary. However, this is not the case for $F^{\times}$. Indeed, the map $x\mapsto|x|$ is a character of $F^{\times}$, yet it is clearly not unitary. 

Before stating the classification theorem for characters of $F$, we need one last definition. 

\begin{defn}
    Let $\psi$ be a non-trivial character of $F$ (resp. of $F^{\times}$). The \textbf{level} of $\psi$ is defined as be the least $d\geq0$ such that $\pp^d\subseteq\ker\psi$ (resp. $U_F^{d+1}\subseteq\ker\psi$).
\end{defn}

\begin{lemma}
    Let $\psi\in\hat{F}$ be a character of level $d$ and let $a\in F$. Then the map $a\psi:x\mapsto\psi(ax)$ is a character of $F$ and if $a\neq0$ then $a\psi$ has level $d-\nu_F(a)$.
\end{lemma}
\begin{proof}
    It is clear that $a\psi$ is a character since if $x\in\pp^{d-\nu_F(a)}$, then $ax\in\pp^d$ so $a\psi(x)=1$ so $\pp^{d-\nu_F(a)}\subseteq\ker(a\psi)$ and the kernel is open. Furthermore, there is some $y\in\pp^{d-1}$ such that $\psi(y)\neq1$ so $a\psi(a^{-1}y)\neq1$. Since $a^{-1}y\in\pp^{d-1-\nu_F(a)}$, this indeed shows that the level of $a\psi$ is $d-\nu_F(a)$. 
\end{proof}

We are now ready to give the classfication theorem for $\hat{F}$.

\begin{thm}[Additive duality]\label{add_dual}
    Let $\psi\in\hat{F}$ be character with level $d$. The map $a\mapsto a\psi$ induces an isomorphism $F\cong\hat{F}$. 
\end{thm}

The proof of surjectivity of the theorem requires an inductive step, which heavily relies on the following results.

\begin{lemma}\label{lem_congruence}
    Let $\psi\in\hat{F}$ be a character of level $d$ and let $u,u'\in U_F$ be two units of $F$. Then $u\psi$ coincides with $u'\psi$ on $\pp^{d-n}$ if and only if $u'u^{-1}\in U_F^{n}$.
\end{lemma}
\begin{proof}
    Let $\alpha=\nu_F(u-u')$. A simple definition chase shows that $u\psi$ and $u'\psi$ agree on $\pp^{d-n}$ if and only if $\pp^{d-n+\alpha}=(u-u')\pp^{d-n}\subseteq\ker\psi$. By definition of level, this is the case if and only if $\alpha\geq n$; that is, if $u\equiv u'\pmod{\pp^n}$ or $u'u^{-1}\in U_F^{n}$.
\end{proof}

\begin{lemma}\label{lem_chars}
    Let $\theta:\pp^{n}\rightarrow\CC^{\times}$ be a character. Then there are exacty $q$ characters $\Theta$ of $\pp^{n-1}$ such that $\Theta|_{\pp^n}=\theta$.
\end{lemma}

\begin{proof}
    Since $\hat\kappa\cong\kappa$, it is enough to construct a bijection between $\mathcal{A}:=\{\Theta\in\widehat{\pp^{n-1}}:\Theta|_{\pp^n}=\theta\}$ and $\hat{\kappa}$. Let $\phi=\theta^{-1}$ and let $\Phi$ be \textbf{any} lift of $\phi$ as a character of $\pp^{n-1}$. Now given $\Theta\in\mathcal{A}$, the character $\Theta\cdot\Phi$ is trivial on $\pp^{n}$ and thus it descends to a map 
    $$\overline{\Theta\cdot\Phi}:\kappa\cong\pp^{n-1}/\pp^n\longrightarrow\CC^{\times}.$$

    To construct an inverse to the map $\Theta\mapsto\overline{\Theta\cdot\Phi}$, choose some $\chi\in\hat\kappa$, view it as a character of $\pp^{n-1}/\pp^{n}$ and consider the map $\tilde\chi:\pp^{n-1}\rightarrow\CC^{\times}$ given by $\tilde{\chi}(u)=\chi(u+\pp^n)$. Then the map $\chi\mapsto\Phi^{-1}\cdot\tilde\chi$ is the required inverse map.
\end{proof}

We are now ready for the proof of Additive duality.

\begin{proof}[Proof of Theorem \ref{add_dual}]
    The map $a\mapsto a\psi$ is clearly a homomorphism. To prove injectivity, suppose that $a\neq b$ but $a\psi=b\psi$. Then it follows that $x(a-b)\in\ker\psi$ for all $x\in F$. But since $a-b\neq 0$, then $\ker\psi=F$, a contradiction.

    Let $\theta\in\hat{F}$ be any non-trivial character (if $\theta$ were trivial, then $0\psi=\theta$), and let $l$ be the level of $\theta$. By replacing $\theta$ with $\varpi^{l-d}\theta$, which has level $d$, we may assume without loss of generality that $\theta$ and $\psi$ have the same level $d$, and therefore they both agree on $\pp^d$. To show there is some $u\in F$ (in fact, $u\in U_F$ necessarily) such that $u\psi=\theta$,   we construct a sequence $\{u_n\}_{n\geq0}$ inductively such that $u_n\psi|_{\pp^{d-n}}=\theta|_{\pp^{d-n}}$ and $u_{n+1}\equiv u_n\pmod{\pp^n}$. Such a sequence is clearly Cauchy, and since $F$ is complete, it converges to some $u\in U_F$ such that $u\equiv u_n\pmod{\pp^n}$ for all $n\geq 1$ and thus $u\psi$ agrees with $\theta$ on $\cup_{n\in\ZZ}\ \pp^n=F$, which concludes the proof.

    Thus, it remains to construct the sequence above. To construct $u_1$ we note that by Lemma \ref{lem_chars}, there are exactly $q-1$ non-trivial characters on $\pp^{d-1}$ that are trivial on $\pp^d$. In addition, by Lemma \ref{lem_congruence}, as $u$ ranges over the cossets of $U_F/U_F^1$, the characters $u\psi|_{\pp^{d-1}}$ are distinct. Since $|U_F/U_F^1|=|\kappa^{\times}|=q-1$, there is some $u_1\in U_F$ such that $u_1\psi$ agrees with $\theta$ on $\pp^{d-1}$. 
    
    Assuming now we have constructed $u_1,\ldots,u_n$ in $U_F$ with the desired conditions, we note that by Lemma \ref*{lem_chars}, there are exactly $q$ characters of $\pp^{d-n-1}$ that coincide with $\theta|_{\pp^{d-n}}$ when they are restricted. Again by Lemma \ref*{lem_congruence}, as $\alpha$ ranges over the cossets of $U_F^n/U_F^{n+1}$ the characters $\alpha u_n\psi$ are distinct on $\pp^{d-n-1}$ but they all coincide on $\pp^{d-n}$. Since $|U_F^n/U_F^{n+1}|=|\kappa|=q$, there is some $\alpha_n$ such that $\alpha_n u_n\psi$ coincides with $\theta$ on $\pp^{d-n-1}$. Since $\alpha_n\in U_F^n$, $\alpha_n u_n\equiv u_n\pmod{\pp^n}$. Hence $u_{n+1}:=\alpha_n u_n$ has the requried properties.
\end{proof}


\subsection{Smooth Representations of Locally Profinite Groups}

We now turn our attention representations of locally profinite groups of arbitrary dimension. For one-dimensional representations, we imposed a natural continuity condition, and Lemma \ref{lem_cont_chars} showed that characters have open kernel. This is a remarkable result, since this means that the homomorphism is continuous with respect to \textbf{any} topology on $\CC^{\times}$, not just the usual one.

If $V$ is a finite dimensional representation of a locally profinite group $G$, the group $\Aut_\CC(V)$ has a natural topology as an open subspace of $M_n(\CC)\cong\CC^{n^2}$. It turns out that, analogously to $C^{\times}$, small neighbourhoods of the identity of $\Aut_\CC(V)$ do not contain any non-trivial subgroups. The same reasoning as in Lemma \ref{lem_cont_chars} shows that continous finite dimensional representations have open kernel too. That is, the homomorphism is continous with respect to any topology on $\Aut_\CC(V)$.

However, for infinite dimensional representations $V$, equipping $\Aut_\CC(V)$ with a topology is not as straightforward and the requirement of having an open kernel is too restrictive. The natural idea behind a \textit{smooth representation} is the requirement that any finite dimensional subrepresentation $V$ (if any) should have an open kernel. To give the precise definition, we first introduce the module of invariants and coinvariants.

\begin{defn}
    Let $H\leq G$ be groups and $(\pi,V)$ a representation of $G$. We define the $H$-invariants to be 
    $$V^{H}:=\{v\in V:\pi(h)v=v\text{ for all }h\in H\},$$
    and the $H$-coinvariants to be 
    $$V_H:=V/V(H)\text{ where } V(H)=\textrm{Span}_\CC\{v-\pi(h)v:v\in V,h\in H\}.$$
    That is, $V^H$ (resp. $V_H$) is the largest subspace (resp. quotient) on which $H$ acts trivially.
\end{defn}

We are now ready to define a \textit{smooth representation} of a locally profinite group $G$.

\begin{defn}
	A representation $V$ of $G$ is \textbf{smooth} if for $v\in V$ there exists a compact-open subgroup $K\subseteq G$ such that $v\in V^K$. In other words,
    $$V=\bigcup_K V^K.$$
    Moreover, we say $V$ is \textbf{admissible} if $V^K$ is finite-dimensional for all compact-open $K$.
\end{defn}

Smooth representations of $G$ are a full, abelian subcategory of $\Rep_G$, and this category is denoted by $\Smo_G$. 

\begin{rem}
    If $(\pi,V)$ is a finite dimensional smooth representation and $\{v_1,\ldots,v_n\}$ is a $\CC$-basis such that $v_i\in V^{K_i}$ for some open compact subgroups $K_i$, then 
    $$K:=\bigcap_{i=1}^n K_i\subseteq\ker\pi$$
    and it is open and compact too, so the kernel is also open. 
    Conversely, if $\ker\pi$ is open, then there is some open compact subgroup $K$ fixing all of $V$, so in this case smooth and continuous coincide. 
\end{rem}


As we hinted in Remark \ref{rem_semisimple}, smooth representations of locally profinite groups have remarkable algebraic structures, and they share many properties with representations of finite groups, particularly if the group is compact (and thus profinite), which we briefly recall. A direct application (yet technical) of Zorn's Lemma provides the follwing useful criterion determine whether a representation is semisimple. 

\begin{prop}\label{prop_semisimple}
    Let $(\pi,V)$ be a smooth representation of a locally profinite group $G$. The following are equivalent:
    \begin{enumerate}
        \item $V$ is the sum of its irreducible $G$-subspaces.
        \item $V$ is the direct sum of a family of irreducible $G$-subspaces (i.e. $V$ is semisimple)
        \item any $G$-subspace of $V$ has a $G$-complement in $V$.
    \end{enumerate}
\end{prop}

\begin{proof}
    \cite[2.2 Lemma]{BH1}
\end{proof}

Using this Proposition, we can now prove that smooth representations of profinite groups behave in a similar way to those of finite groups. We note that any open compact subgroup $K$ of a locally profinite group $G$ is profinite, and that any smooth $G$-representation is natually a smooth $K$-representation by restriction. Therefore, the following results apply for any open compact subgroup of $G$.

\begin{prop}\label{lem_profinite_smooth}
    Let $(\pi,V)$ be a representation of a profinite group $K$. If $V$ is irreducible then it is finite dimensional. Conversely, if $V$ is finite dimensional, then it is semisimple.
\end{prop}

\begin{proof}
    The first statement is a matter of following the definitons. Fix any non-zero $v\in V$, and suppose $v\in V^{K_0}$. Then the subspace 
    $$U=\Span\{\pi(k)v:k\in K\}=\Span\{\pi(k)v:k\in K/K_0\}$$
    is clearly a $K$-subspace and it is also finite dimensional since $K_0$ is open and $K$ is compact, so $(K:K_0)$ is finite.   

    To prove the second statement, let $v$ and $K_0$ be as above. By replacing $K_0$ by $\cap_{g\in K/K_0}gK_0g^{-1}$ if needed, we may assume that $K_0$ is normal. As above, the subspace 
    $$W=\Span\{\pi(k)v:k\in K\}$$
    is finite dimensional and $K_0$ acts trivially on it.
    Thus $W$ is effectively a finite dimensional representation of the finite group $K/K_0$ and thus by Maschke's Theorem, $W$ is the sum of its irreducible $K$ subspaces. Since $v$ was arbitrary this shows that condition $1.$ of Proposition \ref{prop_semisimple} is satisfied, so $V$ is semisimple.

\end{proof}

This proposition has important structure results. Let $\hat{K}$ denote the set of equivalence classes of irreducible smooth representations of $K$. As we shall see, this notation is consistent with $\hat{F}$ since all irreducible smooth representations of $F$ are one-dimensional.

Let $(\pi,V)$ be a smooth representation of a locally profinite group $G$ and let $K$ be an open compact subgroup. For each $\rho\in\hat{K}$, let $V^\rho$ be the sum of all irreducible $K$-subspaces of $V$ isomorphic to $K$, which is denoted as the $\rho$-isotypic component of $V$. Note also that $V^{1_K}=V^K$.

\begin{prop}
    Let $G$ be a locally profinite group and $K$ a compact open subgroup of $G$. Let $(\tau,U),(\pi,V)$, $(\sigma,W)\in\Smo_G$ and $a:U\rightarrow V$ and $b:V\rightarrow W$ be $G$-homomorphisms. 
    \begin{enumerate}
        \item The space $V$ is the sum of the $K$-isotypic components:
        $$V=\bigoplus_{\rho\in\hat{K}}V^\rho.$$
        \item The following holds:
        $$W^\rho\cap b(V)=b(V^\rho).$$
        \item The sequence
        $$U\xlongrightarrow{a} V\xlongrightarrow{b} W$$
        is exact if and only if 
        $$U^K\xlongrightarrow{a} V^K \xlongrightarrow{b} W^K$$
        is exact for every compact open subgroup $K$ of $G$.
        \item If $V(K)$ is the span of the elements $v-\pi(k)v$ for $v\in V, k\in K$, then
        $$V(K)=\bigoplus_{\substack{\rho\in\hat{K}\\\rho\neq 1}}V^\rho \text{ and thus } V=V^K\oplus V(K)$$
        and $V(K)$ is the unique $K$-compement of $V^K$ in $V$. 
    \end{enumerate}
\end{prop}

\begin{proof}
    \cite[2.3 Proposition and Corollary 1,2]{BH1}
\end{proof}

As promised in \S\ref{Abstract_Reps}, we now discuss the dual, restriction and induction functors in the context of smooth representations of locally profinite groups. From our previous discussion, two major problems arise in this context. Firstly, given a locally profinite group $G$ and a subgroup $H$, there is no guarantee that $H$ is locally profinite and thus $\Smo_H$ may not be well-defined. Secondly, when we perform some construction on a smooth representation (constructing its dual, inducing to a bigger group,...) there is no gurarantee that the resulting representation is smooth. Thankfully, both of this problems can be resolved in a straightforward way.

To ensure that $H$ is locally profinite, we must add a condition on the topology on $H$. Based on Example \ref{example_prof_groups}(7), we just need to assume that $H$ is a closed subgroup of $G$. In some cases, we will need to assume that $H$ is also open, which is a more restrictive condition. To ensure that the functors give smooth representations of $G$, we simply compose them with the \textbf{smoothness functor}
\begin{align*}
    (-)^\infty:\Rep_G&\longrightarrow\Smo_G,\\
    (\pi,V)&\longmapsto(\pi^\infty,V^\infty)
\end{align*}
where $$V^\infty=\bigcup_K V^K \text{  and  } \pi^\infty(g)=\pi(g)|_{V^\infty} \text{  for each  } g\in G,$$ and $K$ ranges over the compact open subgroups of $G$. By chasing definitions, one can show that $(-)^\infty$ is a well-defined left-exact functor such that 
$$\Hom_G(V,W)=\Hom_G(V,W^\infty) \text{ for all } V\in\Smo_G, W\in\Rep_G.$$

Using this constructions, we can define the smooth dual, restriction and induction functors. We remark that as long as $H\leq G$ is closed, the usual restriction sends smooth representations of $G$ to smooth representations of $H$. This is not the case for the dual and induction functors, so we apply to construction above.

\begin{defn}
    If $G$ is a locally profinite group, define the smooth dual functor 
    \begin{align*}
        \check{(-)}:\Smo_G&\longrightarrow\Smo_G,\\
        (\pi,V)&\longmapsto(\check{\pi},\check{V})
    \end{align*}
    where $(\check{\pi},\check{V})=(\pi^*,V^*)^\infty$.
\end{defn}

The smooth dual satisfies an important property: if $V$ is a smooth representation of $G$ and $v\in V, v\neq 0$, then there is some $\check{v}\in\check{V}$ such that $\langle\check{v},v\rangle\neq 0$. Consequently, the map $\delta:V\rightarrow\check{\check{V}}$ is injective, and the following proposition gives a criterion for surjectivity.

\begin{prop}
    With the notation as above, the canonical map $\delta:V\longrightarrow\check\check{V}$ is an isomorphism if and only if $(\pi,V)$ is admissible.
\end{prop}
\begin{proof}
    \cite[2.9 Proposition]{BH1}
\end{proof}

We also define the smooth induction functor as the composition of the induction and smoothness functor.

\begin{defn}\label{induction}
    Let $G$ be a locally profinite group and $H\leq G$ a closed subgroup. Define the smooth induction functor
    \begin{align*}
        \Ind_H^G:\Smo_H&\longrightarrow\Smo_G,\\
        (\sigma,W)&\longmapsto(\Sigma,X)
    \end{align*}
    where $X$ is the space of functions $f: G\to W$ satisfying
	\begin{enumerate}
		\item $f(hg) = \sigma(h)f(g)$ for all $h\in H, g\in G$.
		\item There is a compact open subgroup $K\subseteq G$ (depending on $f$) such that $f(gk) = f(g)$ for all $g\in G, k\in K$.
	\end{enumerate}
\end{defn}

Since the action $\Sigma$ on $X$ is given by $\Sigma(g)f:x\mapsto f(xg)$, condition $2.$ is precisely the smoothness condition that $f\in X^K$ for some open compact subgroup $K$. As above, we will denote this representation of $G$ as $\Ind_H^G\sigma$. Under these conditions, the first half of Frobenius Reciprocity holds:

\begin{thm}[Frobenius reciprocity]
	Let $(\pi,V)$ be a smooth representation of $G$, and $(\sigma,W)$ a smooth representation of a closed subgroup $H$. Then the map
	\begin{align*}
		\Psi:\Hom_G(\pi, \Ind_H^G\sigma)&\longrightarrow \Hom_H(\Res_H^G \pi, \sigma),\\
		\varphi &\longmapsto \alpha_\sigma \circ \varphi,
	\end{align*}
    is a bijection that is functorial in both variables $\pi,\sigma$, \textit{and where $\alpha_\sigma:\Ind_H^G\sigma \to W$ is the canonical map $\alpha_\sigma(f) = f(1)$}. In categorical terms,
    $$\Res_H^G\dashv\Ind_H^G.$$
\end{thm}

However, in this context, it is not the case that $\Ind_H^G$ is left adjoint to $\Res_H^G$. With a small modification we can however obtain an analogous result. Firstly, we note that to ensure that $a_\sigma^c$ is a $H$-homomorphism, we need the stronger assumption that $H$ is open in $G$. Secondly, we observe that given representations $(\pi,V)$ and $(\sigma,W)$, of $G$ and $H$ respectively, $a_\sigma^c(w)$ is supported only in $H$ for any $w\in W$. Hence, one should not consider the entire representation $\Ind_H^G\sigma$, but rather a subrepresentation of it. Here is the precise construction.

\begin{defn}
	Let $G$ be a locally profinite group, $H$ a closed subgroup, and $(\sigma,W)$ a smooth representation of $H$. Consider the functor 
    \begin{align*}
        c-\Ind_H^G:\Smo_H&\longrightarrow\Smo_G,\\
        (\sigma,W)&\longmapsto(\Sigma,X_c)
    \end{align*}
    where 
    $$X_c=\{f\in X: \supp f\subseteq H\backslash G \text{ is compact}\}.$$
    We denote functions satisfying the later condition as compactly supported modulo $H$, and this condition is equivalent to $\supp f\subseteq HC$ for some compact set $C$.
    The action by $\Sigma$ is closed in $\Sigma_c$, so the functor is well-defined.
\end{defn}

This construction is mainly of interest in the case when $H$ is open in $G$, in which case $a_\sigma^c$ is a $H$-homomorphism. This construction satisfies the second half of Frobenius Reciprocity.

\begin{thm}
	Let $(\pi,V)$ be a smooth representation of $G$, and $(\sigma,W)$ a smooth representation of an open subgroup $H$. Then the map 
	\begin{align*}
		\Psi^c:\Hom_G(c-\Ind_H^G \sigma, \pi)&\longrightarrow \Hom_H(\sigma, \Res_H^G\pi)\\
		\varphi &\longmapsto \varphi \circ \alpha^c_\sigma 
	\end{align*}
    is a bijection that is functorial in both variables $\pi,\sigma$, and \textbf{where $\alpha^c_\sigma: W\to c-\Ind_H^G \sigma$ is the map $w\mapsto f_w$ where $f_w$ is supported in $H$ and satisfies $f_w(h) = hw$.}
\end{thm}
In categorial terms, we have the important property
$$c-\Ind_H^G\dashv\Res_H^G\dashv\Ind_H^G.$$

\newpage

